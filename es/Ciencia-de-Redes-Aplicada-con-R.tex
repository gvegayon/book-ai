% Options for packages loaded elsewhere
\PassOptionsToPackage{unicode}{hyperref}
\PassOptionsToPackage{hyphens}{url}
\PassOptionsToPackage{dvipsnames,svgnames,x11names}{xcolor}
%
\documentclass[
  12pt,
  letterpaper,
  DIV=11,
  numbers=noendperiod]{scrreprt}

\usepackage{amsmath,amssymb}
\usepackage{iftex}
\ifPDFTeX
  \usepackage[T1]{fontenc}
  \usepackage[utf8]{inputenc}
  \usepackage{textcomp} % provide euro and other symbols
\else % if luatex or xetex
  \usepackage{unicode-math}
  \defaultfontfeatures{Scale=MatchLowercase}
  \defaultfontfeatures[\rmfamily]{Ligatures=TeX,Scale=1}
\fi
\usepackage{lmodern}
\ifPDFTeX\else  
    % xetex/luatex font selection
\fi
% Use upquote if available, for straight quotes in verbatim environments
\IfFileExists{upquote.sty}{\usepackage{upquote}}{}
\IfFileExists{microtype.sty}{% use microtype if available
  \usepackage[]{microtype}
  \UseMicrotypeSet[protrusion]{basicmath} % disable protrusion for tt fonts
}{}
\makeatletter
\@ifundefined{KOMAClassName}{% if non-KOMA class
  \IfFileExists{parskip.sty}{%
    \usepackage{parskip}
  }{% else
    \setlength{\parindent}{0pt}
    \setlength{\parskip}{6pt plus 2pt minus 1pt}}
}{% if KOMA class
  \KOMAoptions{parskip=half}}
\makeatother
\usepackage{xcolor}
\usepackage[top=1in,bottom=1in,left=1in,right=1in]{geometry}
\ifLuaTeX
  \usepackage{luacolor}
  \usepackage[soul]{lua-ul}
\else
  \usepackage{soul}
  
\fi
\setlength{\emergencystretch}{3em} % prevent overfull lines
\setcounter{secnumdepth}{5}
% Make \paragraph and \subparagraph free-standing
\makeatletter
\ifx\paragraph\undefined\else
  \let\oldparagraph\paragraph
  \renewcommand{\paragraph}{
    \@ifstar
      \xxxParagraphStar
      \xxxParagraphNoStar
  }
  \newcommand{\xxxParagraphStar}[1]{\oldparagraph*{#1}\mbox{}}
  \newcommand{\xxxParagraphNoStar}[1]{\oldparagraph{#1}\mbox{}}
\fi
\ifx\subparagraph\undefined\else
  \let\oldsubparagraph\subparagraph
  \renewcommand{\subparagraph}{
    \@ifstar
      \xxxSubParagraphStar
      \xxxSubParagraphNoStar
  }
  \newcommand{\xxxSubParagraphStar}[1]{\oldsubparagraph*{#1}\mbox{}}
  \newcommand{\xxxSubParagraphNoStar}[1]{\oldsubparagraph{#1}\mbox{}}
\fi
\makeatother

\usepackage{color}
\usepackage{fancyvrb}
\newcommand{\VerbBar}{|}
\newcommand{\VERB}{\Verb[commandchars=\\\{\}]}
\DefineVerbatimEnvironment{Highlighting}{Verbatim}{commandchars=\\\{\}}
% Add ',fontsize=\small' for more characters per line
\usepackage{framed}
\definecolor{shadecolor}{RGB}{241,243,245}
\newenvironment{Shaded}{\begin{snugshade}}{\end{snugshade}}
\newcommand{\AlertTok}[1]{\textcolor[rgb]{0.68,0.00,0.00}{#1}}
\newcommand{\AnnotationTok}[1]{\textcolor[rgb]{0.37,0.37,0.37}{#1}}
\newcommand{\AttributeTok}[1]{\textcolor[rgb]{0.40,0.45,0.13}{#1}}
\newcommand{\BaseNTok}[1]{\textcolor[rgb]{0.68,0.00,0.00}{#1}}
\newcommand{\BuiltInTok}[1]{\textcolor[rgb]{0.00,0.23,0.31}{#1}}
\newcommand{\CharTok}[1]{\textcolor[rgb]{0.13,0.47,0.30}{#1}}
\newcommand{\CommentTok}[1]{\textcolor[rgb]{0.37,0.37,0.37}{#1}}
\newcommand{\CommentVarTok}[1]{\textcolor[rgb]{0.37,0.37,0.37}{\textit{#1}}}
\newcommand{\ConstantTok}[1]{\textcolor[rgb]{0.56,0.35,0.01}{#1}}
\newcommand{\ControlFlowTok}[1]{\textcolor[rgb]{0.00,0.23,0.31}{\textbf{#1}}}
\newcommand{\DataTypeTok}[1]{\textcolor[rgb]{0.68,0.00,0.00}{#1}}
\newcommand{\DecValTok}[1]{\textcolor[rgb]{0.68,0.00,0.00}{#1}}
\newcommand{\DocumentationTok}[1]{\textcolor[rgb]{0.37,0.37,0.37}{\textit{#1}}}
\newcommand{\ErrorTok}[1]{\textcolor[rgb]{0.68,0.00,0.00}{#1}}
\newcommand{\ExtensionTok}[1]{\textcolor[rgb]{0.00,0.23,0.31}{#1}}
\newcommand{\FloatTok}[1]{\textcolor[rgb]{0.68,0.00,0.00}{#1}}
\newcommand{\FunctionTok}[1]{\textcolor[rgb]{0.28,0.35,0.67}{#1}}
\newcommand{\ImportTok}[1]{\textcolor[rgb]{0.00,0.46,0.62}{#1}}
\newcommand{\InformationTok}[1]{\textcolor[rgb]{0.37,0.37,0.37}{#1}}
\newcommand{\KeywordTok}[1]{\textcolor[rgb]{0.00,0.23,0.31}{\textbf{#1}}}
\newcommand{\NormalTok}[1]{\textcolor[rgb]{0.00,0.23,0.31}{#1}}
\newcommand{\OperatorTok}[1]{\textcolor[rgb]{0.37,0.37,0.37}{#1}}
\newcommand{\OtherTok}[1]{\textcolor[rgb]{0.00,0.23,0.31}{#1}}
\newcommand{\PreprocessorTok}[1]{\textcolor[rgb]{0.68,0.00,0.00}{#1}}
\newcommand{\RegionMarkerTok}[1]{\textcolor[rgb]{0.00,0.23,0.31}{#1}}
\newcommand{\SpecialCharTok}[1]{\textcolor[rgb]{0.37,0.37,0.37}{#1}}
\newcommand{\SpecialStringTok}[1]{\textcolor[rgb]{0.13,0.47,0.30}{#1}}
\newcommand{\StringTok}[1]{\textcolor[rgb]{0.13,0.47,0.30}{#1}}
\newcommand{\VariableTok}[1]{\textcolor[rgb]{0.07,0.07,0.07}{#1}}
\newcommand{\VerbatimStringTok}[1]{\textcolor[rgb]{0.13,0.47,0.30}{#1}}
\newcommand{\WarningTok}[1]{\textcolor[rgb]{0.37,0.37,0.37}{\textit{#1}}}

\providecommand{\tightlist}{%
  \setlength{\itemsep}{0pt}\setlength{\parskip}{0pt}}\usepackage{longtable,booktabs,array}
\usepackage{calc} % for calculating minipage widths
% Correct order of tables after \paragraph or \subparagraph
\usepackage{etoolbox}
\makeatletter
\patchcmd\longtable{\par}{\if@noskipsec\mbox{}\fi\par}{}{}
\makeatother
% Allow footnotes in longtable head/foot
\IfFileExists{footnotehyper.sty}{\usepackage{footnotehyper}}{\usepackage{footnote}}
\makesavenoteenv{longtable}
\usepackage{graphicx}
\makeatletter
\newsavebox\pandoc@box
\newcommand*\pandocbounded[1]{% scales image to fit in text height/width
  \sbox\pandoc@box{#1}%
  \Gscale@div\@tempa{\textheight}{\dimexpr\ht\pandoc@box+\dp\pandoc@box\relax}%
  \Gscale@div\@tempb{\linewidth}{\wd\pandoc@box}%
  \ifdim\@tempb\p@<\@tempa\p@\let\@tempa\@tempb\fi% select the smaller of both
  \ifdim\@tempa\p@<\p@\scalebox{\@tempa}{\usebox\pandoc@box}%
  \else\usebox{\pandoc@box}%
  \fi%
}
% Set default figure placement to htbp
\def\fps@figure{htbp}
\makeatother
% definitions for citeproc citations
\NewDocumentCommand\citeproctext{}{}
\NewDocumentCommand\citeproc{mm}{%
  \begingroup\def\citeproctext{#2}\cite{#1}\endgroup}
\makeatletter
 % allow citations to break across lines
 \let\@cite@ofmt\@firstofone
 % avoid brackets around text for \cite:
 \def\@biblabel#1{}
 \def\@cite#1#2{{#1\if@tempswa , #2\fi}}
\makeatother
\newlength{\cslhangindent}
\setlength{\cslhangindent}{1.5em}
\newlength{\csllabelwidth}
\setlength{\csllabelwidth}{3em}
\newenvironment{CSLReferences}[2] % #1 hanging-indent, #2 entry-spacing
 {\begin{list}{}{%
  \setlength{\itemindent}{0pt}
  \setlength{\leftmargin}{0pt}
  \setlength{\parsep}{0pt}
  % turn on hanging indent if param 1 is 1
  \ifodd #1
   \setlength{\leftmargin}{\cslhangindent}
   \setlength{\itemindent}{-1\cslhangindent}
  \fi
  % set entry spacing
  \setlength{\itemsep}{#2\baselineskip}}}
 {\end{list}}
\usepackage{calc}
\newcommand{\CSLBlock}[1]{\hfill\break\parbox[t]{\linewidth}{\strut\ignorespaces#1\strut}}
\newcommand{\CSLLeftMargin}[1]{\parbox[t]{\csllabelwidth}{\strut#1\strut}}
\newcommand{\CSLRightInline}[1]{\parbox[t]{\linewidth - \csllabelwidth}{\strut#1\strut}}
\newcommand{\CSLIndent}[1]{\hspace{\cslhangindent}#1}

\usepackage{booktabs}
\usepackage{hyperref}
  \hypersetup{allcolors=blue, colorlinks=true}

\usepackage{setspace}
\usepackage{bm}
\onehalfspacing

\allowdisplaybreaks
\KOMAoption{captions}{tableheading}
\makeatletter
\@ifpackageloaded{tcolorbox}{}{\usepackage[skins,breakable]{tcolorbox}}
\@ifpackageloaded{fontawesome5}{}{\usepackage{fontawesome5}}
\definecolor{quarto-callout-color}{HTML}{909090}
\definecolor{quarto-callout-note-color}{HTML}{0758E5}
\definecolor{quarto-callout-important-color}{HTML}{CC1914}
\definecolor{quarto-callout-warning-color}{HTML}{EB9113}
\definecolor{quarto-callout-tip-color}{HTML}{00A047}
\definecolor{quarto-callout-caution-color}{HTML}{FC5300}
\definecolor{quarto-callout-color-frame}{HTML}{acacac}
\definecolor{quarto-callout-note-color-frame}{HTML}{4582ec}
\definecolor{quarto-callout-important-color-frame}{HTML}{d9534f}
\definecolor{quarto-callout-warning-color-frame}{HTML}{f0ad4e}
\definecolor{quarto-callout-tip-color-frame}{HTML}{02b875}
\definecolor{quarto-callout-caution-color-frame}{HTML}{fd7e14}
\makeatother
\makeatletter
\@ifpackageloaded{bookmark}{}{\usepackage{bookmark}}
\makeatother
\makeatletter
\@ifpackageloaded{caption}{}{\usepackage{caption}}
\AtBeginDocument{%
\ifdefined\contentsname
  \renewcommand*\contentsname{Table of contents}
\else
  \newcommand\contentsname{Table of contents}
\fi
\ifdefined\listfigurename
  \renewcommand*\listfigurename{List of Figures}
\else
  \newcommand\listfigurename{List of Figures}
\fi
\ifdefined\listtablename
  \renewcommand*\listtablename{List of Tables}
\else
  \newcommand\listtablename{List of Tables}
\fi
\ifdefined\figurename
  \renewcommand*\figurename{Figure}
\else
  \newcommand\figurename{Figure}
\fi
\ifdefined\tablename
  \renewcommand*\tablename{Table}
\else
  \newcommand\tablename{Table}
\fi
}
\@ifpackageloaded{float}{}{\usepackage{float}}
\floatstyle{ruled}
\@ifundefined{c@chapter}{\newfloat{codelisting}{h}{lop}}{\newfloat{codelisting}{h}{lop}[chapter]}
\floatname{codelisting}{Listing}
\newcommand*\listoflistings{\listof{codelisting}{List of Listings}}
\makeatother
\makeatletter
\makeatother
\makeatletter
\@ifpackageloaded{caption}{}{\usepackage{caption}}
\@ifpackageloaded{subcaption}{}{\usepackage{subcaption}}
\makeatother

\usepackage{bookmark}

\IfFileExists{xurl.sty}{\usepackage{xurl}}{} % add URL line breaks if available
\urlstyle{same} % disable monospaced font for URLs
\hypersetup{
  pdftitle={Ciencia de Redes Aplicada con R},
  pdfauthor={George G. Vega Yon, Ph.D.},
  colorlinks=true,
  linkcolor={blue},
  filecolor={Maroon},
  citecolor={Blue},
  urlcolor={Blue},
  pdfcreator={LaTeX via pandoc}}


\title{Ciencia de Redes Aplicada con R}
\author{George G. Vega Yon, Ph.D.}
\date{2025-08-23}

\begin{document}
\maketitle

\renewcommand*\contentsname{Table of contents}
{
\hypersetup{linkcolor=}
\setcounter{tocdepth}{2}
\tableofcontents
}

\bookmarksetup{startatroot}

\chapter{Prefacio}\label{prefacio}

\renewcommand{\Pr}[1]{\text{Pr}\left(#1\right)}
\renewcommand{\exp}[1]{\text{exp}\left\{#1\right\}}

\begin{tcolorbox}[enhanced jigsaw, colback=white, opacityback=0, coltitle=black, title=\textcolor{quarto-callout-warning-color}{\faExclamationTriangle}\hspace{0.5em}{Nota de Traducción}, bottomrule=.15mm, colbacktitle=quarto-callout-warning-color!10!white, toptitle=1mm, colframe=quarto-callout-warning-color-frame, titlerule=0mm, rightrule=.15mm, leftrule=.75mm, breakable, bottomtitle=1mm, left=2mm, arc=.35mm, toprule=.15mm, opacitybacktitle=0.6]

Esta versión del capítulo fue traducida de manera automática utilizando
IA. El capítulo aún no ha sido revisado por un humano.

\end{tcolorbox}

Los métodos estadísticos para sistemas en red están presentes en la
mayoría de las disciplinas. A pesar de las diferencias de lenguaje entre
áreas, muchos métodos desarrollados para estudiar problemas específicos
pueden ser útiles fuera de su contexto original; esta es la premisa de
este libro. \textbf{Ciencia de Redes Aplicada con R} proporciona
ejemplos utilizando el lenguaje de programación R para estudiar sistemas
en red. Aunque la mayoría de los casos tratan sobre análisis de redes
sociales, los métodos presentados aquí pueden aplicarse a contextos como
redes biológicas, redes de transporte y muchos otros.

Todo el libro fue escrito utilizando
\href{https://quarto.org}{quarto}--un sistema de
\href{https://en.wikipedia.org/w/index.php?title=Literate_programming&oldid=1219921237}{programación
literaria} que permite mezclar texto y código--lo que significa que todo
el código presentado es 100\% ejecutable y, por tanto, reproducible. El
código fuente está disponible en GitHub en
\url{https://github.com/gvegayon/appliedsnar}. Se anima a los lectores a
descargar el código y ejecutarlo en sus máquinas utilizando
\href{https://posit.co}{RStudio} o
\href{https://code.visualstudio.com/}{VScode}.

Además de la programación en R, estaremos utilizando RStudio. Para el
manejo de datos, utilizaremos \texttt{dplyr} y \texttt{data.table}. Los
paquetes de manejo y visualización de datos de redes que utilizaremos
son \texttt{igraph}, netdiffuseR, la suite statnet, y \texttt{netplot}.

\section{Sobre el proyecto}\label{sobre-el-proyecto}

Este proyecto comenzó hace más de seis años como parte de una serie de
talleres y tutoriales que impartí en el \textbf{Centro de Análisis de
Redes Aplicadas} de USC. Hoy, lo uso para recopilar y estudiar métodos
estadísticos para analizar redes, con énfasis en sistemas sociales y
biológicos. Además, el libro utilizará métodos de computación
estadística como componente central.

\section{Sobre el Autor}\label{sobre-el-autor}

Soy Profesor Asistente de Investigación en la \textbf{División de
Epidemiología de la Universidad de Utah}, donde trabajo estudiando
Sistemas Complejos utilizando Computación Estadística. Nací y crecí en
Chile. Tengo más de diez años de experiencia desarrollando software
científico con enfoque en computación de alto rendimiento, visualización
de datos y análisis de redes sociales. Mi formación es en Políticas
Públicas (M.A.~UAI, 2011), Economía (M.Sc. Caltech, 2015), y
Bioestadística (Ph.D.~USC, 2020).

Obtuve mi Ph.D.~en Bioestadística bajo la supervisión del
\textbf{Prof.~Paul Marjoram} y la \textbf{Prof.~Kayla de la Haye}, con
mi disertación titulada ``\emph{Essays on Bioinformatics and Social
Network Analysis: Statistical and Computational Methods for Complex
Systems.}''

Si desea aprender más sobre mí, por favor visite mi sitio web en
https://ggvy.cl.

\section{Sobre la versión en
Español}\label{sobre-la-versiuxf3n-en-espauxf1ol}

Esta versión en español del libro fue creada utilizando traducción
automática con inteligencia artificial. Aunque se ha hecho un esfuerzo
por mantener la precisión técnica y el contexto, algunos términos
especializados y conceptos pueden requerir revisión adicional. La
versión original en inglés permanece como la referencia autorizada.

Los lectores que encuentren errores de traducción o áreas que requieran
clarificación son bienvenidos a contribuir reportando problemas en el
repositorio de GitHub del proyecto.

\section{Divulgación sobre IA}\label{divulgaciuxf3n-sobre-ia}

A partir de mediados de 2023, he estado utilizando IA para ayudarme a
escribir este libro. Principalmente, uso una combinación de
\href{https://github.com/features/copilot}{GitHub co-pilot}, que ayuda
con código y texto, y \href{https://grammarly.com}{Grammarly}, que ayuda
con gramática y estilo. El papel de la IA ha sido ayudarme a escribir
más rápido y con mayor precisión, pero no ha estado involucrada en la
conceptualización del libro o el desarrollo de los métodos presentados
aquí.

\bookmarksetup{startatroot}

\chapter{Introducción}\label{introducciuxf3n}

\begin{tcolorbox}[enhanced jigsaw, colback=white, opacityback=0, coltitle=black, title=\textcolor{quarto-callout-warning-color}{\faExclamationTriangle}\hspace{0.5em}{Nota de Traducción}, bottomrule=.15mm, colbacktitle=quarto-callout-warning-color!10!white, toptitle=1mm, colframe=quarto-callout-warning-color-frame, titlerule=0mm, rightrule=.15mm, leftrule=.75mm, breakable, bottomtitle=1mm, left=2mm, arc=.35mm, toprule=.15mm, opacitybacktitle=0.6]

Esta versión del capítulo fue traducida de manera automática utilizando
IA. El capítulo aún no ha sido revisado por un humano.

\end{tcolorbox}

El Análisis de Redes Sociales y la Ciencia de Redes tienen una larga
tradición académica. Desde modelos de difusión social hasta redes de
interacción de proteínas, estas disciplinas de sistemas complejos cubren
una variedad de problemas a través de campos científicos. Sin embargo,
aunque estos podrían verse como ampliamente diferentes, el objeto bajo
el microscopio es el mismo: las redes.

Con una larga historia (e insuficiente colaboración interdisciplinaria,
si me permiten decir) de avances científicos que ocurren de manera algo
aislada, el potencial para la polinización cruzada entre disciplinas
dentro de la ciencia de redes es inmenso.

Este libro intenta compilar los muchos métodos disponibles en el ámbito
de las ciencias de la complejidad, proporcionar un examen matemático
profundo--cuando sea posible--y proporcionar algunos ejemplos que
ilustren su uso.

\bookmarksetup{startatroot}

\chapter{Fundamentos de R}\label{fundamentos-de-r}

\begin{tcolorbox}[enhanced jigsaw, colback=white, opacityback=0, coltitle=black, title=\textcolor{quarto-callout-warning-color}{\faExclamationTriangle}\hspace{0.5em}{Nota de Traducción}, bottomrule=.15mm, colbacktitle=quarto-callout-warning-color!10!white, toptitle=1mm, colframe=quarto-callout-warning-color-frame, titlerule=0mm, rightrule=.15mm, leftrule=.75mm, breakable, bottomtitle=1mm, left=2mm, arc=.35mm, toprule=.15mm, opacitybacktitle=0.6]

Esta versión del capítulo fue traducida de manera automática utilizando
IA. El capítulo aún no ha sido revisado por un humano.

\end{tcolorbox}

R (R Core Team 2024) es un lenguaje de programación orientado a la
computación estadística. R se ha convertido en el lenguaje de
programación \emph{de facto} en la comunidad de redes sociales debido al
gran número de paquetes disponibles para análisis de redes. Los paquetes
de R son colecciones de funciones, datos y documentación que extienden
R. Un buen libro de referencia tanto para usuarios novatos como
avanzados es \href{https://nostarch.com/artofr.htm}{``The Art of R
programming''} Matloff (2011)\footnote{\href{http://heather.cs.ucdavis.edu/~matloff/145/PLN/RMaterials/NSPpart.pdf}{Aquí}
  una versión pdf gratuita distribuida por el autor.}.

\section{Obtener R}\label{obtener-r}

Puedes obtener R desde el sitio web de Comprehensive R Archive Network
{[}CRAN{]} (\href{https://cran.r-project.org/}{enlace}). CRAN es una red
de servidores en todo el mundo que almacenan versiones idénticas y
actualizadas de código y documentación para R. El sitio web de CRAN
también tiene mucha información sobre R, incluyendo manuales, FAQs y
listas de correo.

Aunque R viene con una Interfaz Gráfica de Usuario {[}GUI{]}, recomiendo
obtener una alternativa como \href{https://posit.com}{RStudio} o
\href{https://code.visualstudio.com/}{VSCode}. RStudio y VSCode son
excelentes compañeros para programar en R. Mientras que RStudio es más
común entre los usuarios de R, VSCode es un IDE de propósito más general
que puede usarse para muchos otros lenguajes de programación, incluyendo
Python y C++.

\section{Cómo instalar paquetes}\label{cuxf3mo-instalar-paquetes}

Hoy en día, hay dos formas de instalar paquetes de R (que yo conozca),
ya sea usando \texttt{install.packages}, que es una función que viene
con R, o usando el paquete de R
\href{https://cran.r-project.org/package=devtools}{\texttt{devtools}}
para instalar un paquete desde algún repositorio remoto que no sea CRAN,
aquí hay algunos ejemplos:

\begin{Shaded}
\begin{Highlighting}[]
\CommentTok{\# Esto instalará el paquete igraph desde CRAN}
\SpecialCharTok{\textgreater{}} \FunctionTok{install.packages}\NormalTok{(}\StringTok{"netdiffuseR"}\NormalTok{)}

\CommentTok{\# ¡Esto instalará la versión más reciente desde el repositorio GitHub del proyecto!}
\SpecialCharTok{\textgreater{}}\NormalTok{ devtools}\SpecialCharTok{::}\FunctionTok{install\_github}\NormalTok{(}\StringTok{"USCCANA/netdiffuseR"}\NormalTok{)}
\end{Highlighting}
\end{Shaded}

El primero, usando \texttt{install.packages}, instala la versión de CRAN
de
\href{https://r-project.org/package=netdiffuseR}{\texttt{netdiffuseR}},
mientras que la línea de código instala cualquier versión que esté
publicada en https://github.com/USCCANA/netdiffuseR, que usualmente se
llama la versión de desarrollo.

En algunos casos, los usuarios pueden querer/necesitar instalar paquetes
desde la línea de comandos ya que algunos paquetes necesitan
configuración extra para ser instalados. Pero no necesitaremos ver eso
ahora.

\section{\texorpdfstring{Una Introducción \st{suave} Rápida y Sucia a
R}{Una Introducción suave Rápida y Sucia a R}}\label{una-introducciuxf3n-suave-ruxe1pida-y-sucia-a-r}

Algunas tareas comunes en R

\begin{enumerate}
\def\labelenumi{\arabic{enumi}.}
\setcounter{enumi}{-1}
\item
  Obtener ayuda (y leer el manual) es \emph{LO MÁS IMPORTANTE} que
  deberías saber. Por ejemplo, si quieres leer el manual (archivo de
  ayuda) de la función \texttt{read.csv}, puedes escribir cualquiera de
  estos:

\begin{Shaded}
\begin{Highlighting}[]
\NormalTok{?read.csv}
\NormalTok{?}\StringTok{"read.csv"}
\FunctionTok{help}\NormalTok{(read.csv)}
\FunctionTok{help}\NormalTok{(}\StringTok{"read.csv"}\NormalTok{)}
\end{Highlighting}
\end{Shaded}

  Si no estás completamente seguro de cuál es el nombre de la función,
  siempre puedes usar la \emph{búsqueda difusa}

\begin{Shaded}
\begin{Highlighting}[]
\FunctionTok{help.search}\NormalTok{(}\StringTok{"linear regression"}\NormalTok{)}
\NormalTok{??}\StringTok{"linear regression"}
\end{Highlighting}
\end{Shaded}
\item
  En R, puedes crear nuevos objetos usando el operador de asignación
  (\texttt{\textless{}-}) o el signo igual \texttt{=}, por ejemplo, los
  siguientes dos son equivalentes:
  \texttt{r\ \ \ \ \ a\ \textless{}-\ 1\ \ \ \ \ a\ =\ \ 1}
  Históricamente, el operador de asignación es el más comúnmente usado.
\item
  R tiene varios tipos de objetos. Las estructuras más básicas en R son
  \texttt{vectors}, \texttt{matrix}, \texttt{list}, \texttt{data.frame}.
  Aquí hay un ejemplo de creación de varios de estos (cada línea está
  encerrada con paréntesis para que R imprima el elemento resultante):

\begin{Shaded}
\begin{Highlighting}[]
\NormalTok{(a\_vector     }\OtherTok{\textless{}{-}} \DecValTok{1}\SpecialCharTok{:}\DecValTok{9}\NormalTok{)}
\end{Highlighting}
\end{Shaded}

\begin{verbatim}
[1] 1 2 3 4 5 6 7 8 9
\end{verbatim}

\begin{Shaded}
\begin{Highlighting}[]
\NormalTok{(another\_vect }\OtherTok{\textless{}{-}} \FunctionTok{c}\NormalTok{(}\DecValTok{1}\NormalTok{, }\DecValTok{2}\NormalTok{, }\DecValTok{3}\NormalTok{, }\DecValTok{4}\NormalTok{, }\DecValTok{5}\NormalTok{, }\DecValTok{6}\NormalTok{, }\DecValTok{7}\NormalTok{, }\DecValTok{8}\NormalTok{, }\DecValTok{9}\NormalTok{))}
\end{Highlighting}
\end{Shaded}

\begin{verbatim}
[1] 1 2 3 4 5 6 7 8 9
\end{verbatim}

\begin{Shaded}
\begin{Highlighting}[]
\NormalTok{(a\_string\_vec }\OtherTok{\textless{}{-}} \FunctionTok{c}\NormalTok{(}\StringTok{"I"}\NormalTok{, }\StringTok{"like"}\NormalTok{, }\StringTok{"netdiffuseR"}\NormalTok{))}
\end{Highlighting}
\end{Shaded}

\begin{verbatim}
[1] "I"           "like"        "netdiffuseR"
\end{verbatim}

\begin{Shaded}
\begin{Highlighting}[]
\NormalTok{(a\_matrix     }\OtherTok{\textless{}{-}} \FunctionTok{matrix}\NormalTok{(a\_vector, }\AttributeTok{ncol =} \DecValTok{3}\NormalTok{))}
\end{Highlighting}
\end{Shaded}

\begin{verbatim}
     [,1] [,2] [,3]
[1,]    1    4    7
[2,]    2    5    8
[3,]    3    6    9
\end{verbatim}

\begin{Shaded}
\begin{Highlighting}[]
\CommentTok{\# Las matrices también pueden ser de strings}
\NormalTok{(a\_string\_mat }\OtherTok{\textless{}{-}} \FunctionTok{matrix}\NormalTok{(letters[}\DecValTok{1}\SpecialCharTok{:}\DecValTok{9}\NormalTok{], }\AttributeTok{ncol=}\DecValTok{3}\NormalTok{)) }
\end{Highlighting}
\end{Shaded}

\begin{verbatim}
     [,1] [,2] [,3]
[1,] "a"  "d"  "g" 
[2,] "b"  "e"  "h" 
[3,] "c"  "f"  "i" 
\end{verbatim}

\begin{Shaded}
\begin{Highlighting}[]
\CommentTok{\# El operador \textasciigrave{}cbind\textasciigrave{} hace "column bind"}
\NormalTok{(another\_mat  }\OtherTok{\textless{}{-}} \FunctionTok{cbind}\NormalTok{(}\DecValTok{1}\SpecialCharTok{:}\DecValTok{4}\NormalTok{, }\DecValTok{11}\SpecialCharTok{:}\DecValTok{14}\NormalTok{)) }
\end{Highlighting}
\end{Shaded}

\begin{verbatim}
     [,1] [,2]
[1,]    1   11
[2,]    2   12
[3,]    3   13
[4,]    4   14
\end{verbatim}

\begin{Shaded}
\begin{Highlighting}[]
\CommentTok{\# El operador \textasciigrave{}rbind\textasciigrave{} hace "row bind"}
\NormalTok{(another\_mat2 }\OtherTok{\textless{}{-}} \FunctionTok{rbind}\NormalTok{(}\DecValTok{1}\SpecialCharTok{:}\DecValTok{4}\NormalTok{, }\DecValTok{11}\SpecialCharTok{:}\DecValTok{14}\NormalTok{))}
\end{Highlighting}
\end{Shaded}

\begin{verbatim}
     [,1] [,2] [,3] [,4]
[1,]    1    2    3    4
[2,]   11   12   13   14
\end{verbatim}

\begin{Shaded}
\begin{Highlighting}[]
\NormalTok{(a\_string\_mat }\OtherTok{\textless{}{-}} \FunctionTok{matrix}\NormalTok{(letters[}\DecValTok{1}\SpecialCharTok{:}\DecValTok{9}\NormalTok{], }\AttributeTok{ncol =} \DecValTok{3}\NormalTok{))}
\end{Highlighting}
\end{Shaded}

\begin{verbatim}
     [,1] [,2] [,3]
[1,] "a"  "d"  "g" 
[2,] "b"  "e"  "h" 
[3,] "c"  "f"  "i" 
\end{verbatim}

\begin{Shaded}
\begin{Highlighting}[]
\NormalTok{(a\_list       }\OtherTok{\textless{}{-}} \FunctionTok{list}\NormalTok{(a\_vector, a\_matrix))}
\end{Highlighting}
\end{Shaded}

\begin{verbatim}
[[1]]
[1] 1 2 3 4 5 6 7 8 9

[[2]]
     [,1] [,2] [,3]
[1,]    1    4    7
[2,]    2    5    8
[3,]    3    6    9
\end{verbatim}

\begin{Shaded}
\begin{Highlighting}[]
\CommentTok{\# ¡igual pero con nombres!}
\NormalTok{(another\_list }\OtherTok{\textless{}{-}} \FunctionTok{list}\NormalTok{(}\AttributeTok{my\_vec =}\NormalTok{ a\_vector, }\AttributeTok{my\_mat =}\NormalTok{ a\_matrix)) }
\end{Highlighting}
\end{Shaded}

\begin{verbatim}
$my_vec
[1] 1 2 3 4 5 6 7 8 9

$my_mat
     [,1] [,2] [,3]
[1,]    1    4    7
[2,]    2    5    8
[3,]    3    6    9
\end{verbatim}

\begin{Shaded}
\begin{Highlighting}[]
\CommentTok{\# Los data frames pueden tener múltiples tipos de elementos; es}
\CommentTok{\# una colección de listas}
\NormalTok{(a\_data\_frame }\OtherTok{\textless{}{-}} \FunctionTok{data.frame}\NormalTok{(}\AttributeTok{x =} \DecValTok{1}\SpecialCharTok{:}\DecValTok{10}\NormalTok{, }\AttributeTok{y =}\NormalTok{ letters[}\DecValTok{1}\SpecialCharTok{:}\DecValTok{10}\NormalTok{]))}
\end{Highlighting}
\end{Shaded}

\begin{verbatim}
    x y
1   1 a
2   2 b
3   3 c
4   4 d
5   5 e
6   6 f
7   7 g
8   8 h
9   9 i
10 10 j
\end{verbatim}
\item
  Dependiendo del tipo de objeto, podemos acceder a sus componentes
  usando indexación:

\begin{Shaded}
\begin{Highlighting}[]
\CommentTok{\# Primeros 3 elementos}
\NormalTok{a\_vector[}\DecValTok{1}\SpecialCharTok{:}\DecValTok{3}\NormalTok{]}
\end{Highlighting}
\end{Shaded}

\begin{verbatim}
[1] 1 2 3
\end{verbatim}

\begin{Shaded}
\begin{Highlighting}[]
\CommentTok{\# Tercer elemento}
\NormalTok{a\_string\_vec[}\DecValTok{3}\NormalTok{]}
\end{Highlighting}
\end{Shaded}

\begin{verbatim}
[1] "netdiffuseR"
\end{verbatim}

\begin{Shaded}
\begin{Highlighting}[]
\CommentTok{\# Una sub matriz}
\NormalTok{a\_matrix[}\DecValTok{1}\SpecialCharTok{:}\DecValTok{2}\NormalTok{, }\DecValTok{1}\SpecialCharTok{:}\DecValTok{2}\NormalTok{]}
\end{Highlighting}
\end{Shaded}

\begin{verbatim}
     [,1] [,2]
[1,]    1    4
[2,]    2    5
\end{verbatim}

\begin{Shaded}
\begin{Highlighting}[]
\CommentTok{\# Tercera columna}
\NormalTok{a\_matrix[,}\DecValTok{3}\NormalTok{]}
\end{Highlighting}
\end{Shaded}

\begin{verbatim}
[1] 7 8 9
\end{verbatim}

\begin{Shaded}
\begin{Highlighting}[]
\CommentTok{\# Tercera fila}
\NormalTok{a\_matrix[}\DecValTok{3}\NormalTok{,]}
\end{Highlighting}
\end{Shaded}

\begin{verbatim}
[1] 3 6 9
\end{verbatim}

\begin{Shaded}
\begin{Highlighting}[]
\CommentTok{\# Primeros 6 elementos de la matriz. R almacena matrices}
\CommentTok{\# por columna.}
\NormalTok{a\_string\_mat[}\DecValTok{1}\SpecialCharTok{:}\DecValTok{6}\NormalTok{]}
\end{Highlighting}
\end{Shaded}

\begin{verbatim}
[1] "a" "b" "c" "d" "e" "f"
\end{verbatim}

\begin{Shaded}
\begin{Highlighting}[]
\CommentTok{\# Estos tres son equivalentes}
\NormalTok{another\_list[[}\DecValTok{1}\NormalTok{]]}
\end{Highlighting}
\end{Shaded}

\begin{verbatim}
[1] 1 2 3 4 5 6 7 8 9
\end{verbatim}

\begin{Shaded}
\begin{Highlighting}[]
\NormalTok{another\_list}\SpecialCharTok{$}\NormalTok{my\_vec}
\end{Highlighting}
\end{Shaded}

\begin{verbatim}
[1] 1 2 3 4 5 6 7 8 9
\end{verbatim}

\begin{Shaded}
\begin{Highlighting}[]
\NormalTok{another\_list[[}\StringTok{"my\_vec"}\NormalTok{]]}
\end{Highlighting}
\end{Shaded}

\begin{verbatim}
[1] 1 2 3 4 5 6 7 8 9
\end{verbatim}

\begin{Shaded}
\begin{Highlighting}[]
\CommentTok{\# Los data frames son como listas}
\NormalTok{a\_data\_frame[[}\DecValTok{1}\NormalTok{]]}
\end{Highlighting}
\end{Shaded}

\begin{verbatim}
 [1]  1  2  3  4  5  6  7  8  9 10
\end{verbatim}

\begin{Shaded}
\begin{Highlighting}[]
\NormalTok{a\_data\_frame[,}\DecValTok{1}\NormalTok{]}
\end{Highlighting}
\end{Shaded}

\begin{verbatim}
 [1]  1  2  3  4  5  6  7  8  9 10
\end{verbatim}

\begin{Shaded}
\begin{Highlighting}[]
\NormalTok{a\_data\_frame[[}\StringTok{"x"}\NormalTok{]]}
\end{Highlighting}
\end{Shaded}

\begin{verbatim}
 [1]  1  2  3  4  5  6  7  8  9 10
\end{verbatim}

\begin{Shaded}
\begin{Highlighting}[]
\NormalTok{a\_data\_frame}\SpecialCharTok{$}\NormalTok{x}
\end{Highlighting}
\end{Shaded}

\begin{verbatim}
 [1]  1  2  3  4  5  6  7  8  9 10
\end{verbatim}
\item
  Declaraciones de flujo de control

\begin{Shaded}
\begin{Highlighting}[]
\CommentTok{\# El bucle for de toda la vida}
\ControlFlowTok{for}\NormalTok{ (i }\ControlFlowTok{in} \DecValTok{1}\SpecialCharTok{:}\DecValTok{10}\NormalTok{) \{}
  \FunctionTok{print}\NormalTok{(}\FunctionTok{paste}\NormalTok{(}\StringTok{"Estoy en el paso"}\NormalTok{, i, }\StringTok{"/"}\NormalTok{, }\DecValTok{10}\NormalTok{))}
\NormalTok{\}}
\end{Highlighting}
\end{Shaded}

\begin{verbatim}
[1] "Estoy en el paso 1 / 10"
[1] "Estoy en el paso 2 / 10"
[1] "Estoy en el paso 3 / 10"
[1] "Estoy en el paso 4 / 10"
[1] "Estoy en el paso 5 / 10"
[1] "Estoy en el paso 6 / 10"
[1] "Estoy en el paso 7 / 10"
[1] "Estoy en el paso 8 / 10"
[1] "Estoy en el paso 9 / 10"
[1] "Estoy en el paso 10 / 10"
\end{verbatim}

\begin{Shaded}
\begin{Highlighting}[]
\CommentTok{\# Un buen ifelse}

\ControlFlowTok{for}\NormalTok{ (i }\ControlFlowTok{in} \DecValTok{1}\SpecialCharTok{:}\DecValTok{10}\NormalTok{) \{}

  \ControlFlowTok{if}\NormalTok{ (i }\SpecialCharTok{\%\%} \DecValTok{2}\NormalTok{) }\CommentTok{\# Operando módulo}
    \FunctionTok{print}\NormalTok{(}\FunctionTok{paste}\NormalTok{(}\StringTok{"Estoy en el paso"}\NormalTok{, i, }\StringTok{"/"}\NormalTok{, }\DecValTok{10}\NormalTok{, }\StringTok{"(y soy impar)"}\NormalTok{))}
  \ControlFlowTok{else}
    \FunctionTok{print}\NormalTok{(}\FunctionTok{paste}\NormalTok{(}\StringTok{"Estoy en el paso"}\NormalTok{, i, }\StringTok{"/"}\NormalTok{, }\DecValTok{10}\NormalTok{, }\StringTok{"(y soy par)"}\NormalTok{))}

\NormalTok{\}}
\end{Highlighting}
\end{Shaded}

\begin{verbatim}
[1] "Estoy en el paso 1 / 10 (y soy impar)"
[1] "Estoy en el paso 2 / 10 (y soy par)"
[1] "Estoy en el paso 3 / 10 (y soy impar)"
[1] "Estoy en el paso 4 / 10 (y soy par)"
[1] "Estoy en el paso 5 / 10 (y soy impar)"
[1] "Estoy en el paso 6 / 10 (y soy par)"
[1] "Estoy en el paso 7 / 10 (y soy impar)"
[1] "Estoy en el paso 8 / 10 (y soy par)"
[1] "Estoy en el paso 9 / 10 (y soy impar)"
[1] "Estoy en el paso 10 / 10 (y soy par)"
\end{verbatim}

\begin{Shaded}
\begin{Highlighting}[]
\CommentTok{\# Un while}
\NormalTok{i }\OtherTok{\textless{}{-}} \DecValTok{10}
\ControlFlowTok{while}\NormalTok{ (i }\SpecialCharTok{\textgreater{}} \DecValTok{0}\NormalTok{) \{}
  \FunctionTok{print}\NormalTok{(}\FunctionTok{paste}\NormalTok{(}\StringTok{"Estoy en el paso"}\NormalTok{, i, }\StringTok{"/"}\NormalTok{, }\DecValTok{10}\NormalTok{))}
\NormalTok{  i }\OtherTok{\textless{}{-}}\NormalTok{ i }\SpecialCharTok{{-}} \DecValTok{1}
\NormalTok{\}}
\end{Highlighting}
\end{Shaded}

\begin{verbatim}
[1] "Estoy en el paso 10 / 10"
[1] "Estoy en el paso 9 / 10"
[1] "Estoy en el paso 8 / 10"
[1] "Estoy en el paso 7 / 10"
[1] "Estoy en el paso 6 / 10"
[1] "Estoy en el paso 5 / 10"
[1] "Estoy en el paso 4 / 10"
[1] "Estoy en el paso 3 / 10"
[1] "Estoy en el paso 2 / 10"
[1] "Estoy en el paso 1 / 10"
\end{verbatim}
\item
  R tiene un conjunto convincente de funciones de generación de números
  pseudo-aleatorios. En general, las funciones de distribución tienen la
  siguiente estructura de nombres:

  \begin{enumerate}
  \def\labelenumii{\alph{enumii}.}
  \tightlist
  \item
    Generación de Números Aleatorios:
    \texttt{r{[}nombre-de-la-distribución{]}}, \emph{ej.},
    \texttt{rnorm} para normal, \texttt{runif} para uniforme.
  \item
    Función de densidad: \texttt{d{[}nombre-de-la-distribución{]}}, ej.
    \texttt{dnorm} para normal, \texttt{dunif} para uniforme.
  \item
    Función de Distribución Acumulativa (CDF):
    \texttt{p{[}nombre-de-la-distribución{]}}, \emph{ej.},
    \texttt{pnorm} para normal, \texttt{punif} para uniforme.
  \item
    Función inversa (cuantil):
    \texttt{q{[}nombre-de-la-distribución{]}}, ej. \texttt{qnorm} para
    la normal, \texttt{qunif} para la uniforme.
  \end{enumerate}

  Aquí hay algunos ejemplos:

\begin{Shaded}
\begin{Highlighting}[]
\CommentTok{\# Para asegurar reproducibilidad}
\FunctionTok{set.seed}\NormalTok{(}\DecValTok{1231}\NormalTok{)}

\CommentTok{\# 100,000 números Unif(0,1)}
\NormalTok{x }\OtherTok{\textless{}{-}} \FunctionTok{runif}\NormalTok{(}\FloatTok{1e5}\NormalTok{)}
\FunctionTok{hist}\NormalTok{(x)}
\end{Highlighting}
\end{Shaded}

  \pandocbounded{\includegraphics[keepaspectratio]{part-01-02-the-basics_files/figure-pdf/random-numbers-1.pdf}}

\begin{Shaded}
\begin{Highlighting}[]
\CommentTok{\# 100,000 números N(0,1)}
\NormalTok{x }\OtherTok{\textless{}{-}} \FunctionTok{rnorm}\NormalTok{(}\FloatTok{1e5}\NormalTok{)}
\FunctionTok{hist}\NormalTok{(x)}
\end{Highlighting}
\end{Shaded}

  \pandocbounded{\includegraphics[keepaspectratio]{part-01-02-the-basics_files/figure-pdf/random-numbers-2.pdf}}

\begin{Shaded}
\begin{Highlighting}[]
\CommentTok{\# 100,000 números N(10,25)}
\NormalTok{x }\OtherTok{\textless{}{-}} \FunctionTok{rnorm}\NormalTok{(}\FloatTok{1e5}\NormalTok{, }\AttributeTok{mean =} \DecValTok{10}\NormalTok{, }\AttributeTok{sd =} \DecValTok{5}\NormalTok{)}
\FunctionTok{hist}\NormalTok{(x)}
\end{Highlighting}
\end{Shaded}

  \pandocbounded{\includegraphics[keepaspectratio]{part-01-02-the-basics_files/figure-pdf/random-numbers-3.pdf}}

\begin{Shaded}
\begin{Highlighting}[]
\CommentTok{\# 100,000 números Poisson(5)}
\NormalTok{x }\OtherTok{\textless{}{-}} \FunctionTok{rpois}\NormalTok{(}\FloatTok{1e5}\NormalTok{, }\AttributeTok{lambda =} \DecValTok{5}\NormalTok{)}
\FunctionTok{hist}\NormalTok{(x)}
\end{Highlighting}
\end{Shaded}

  \pandocbounded{\includegraphics[keepaspectratio]{part-01-02-the-basics_files/figure-pdf/random-numbers-4.pdf}}

\begin{Shaded}
\begin{Highlighting}[]
\CommentTok{\# 100,000 números rexp(5)}
\NormalTok{x }\OtherTok{\textless{}{-}} \FunctionTok{rexp}\NormalTok{(}\FloatTok{1e5}\NormalTok{, }\DecValTok{5}\NormalTok{)}
\FunctionTok{hist}\NormalTok{(x)}
\end{Highlighting}
\end{Shaded}

  \pandocbounded{\includegraphics[keepaspectratio]{part-01-02-the-basics_files/figure-pdf/random-numbers-5.pdf}}

  Más distribuciones están disponibles en \texttt{??Distributions}.
\end{enumerate}

Para una buena introducción a R, echa un vistazo a
\href{https://nostarch.com/artofr.htm}{``The Art of R Programming'' por
Norman Matloff}. Para usuarios más avanzados, echa un vistazo a
\href{http://adv-r.had.co.nz/}{``Advanced R'' por Hadley Wickham}.

\part{\textbf{Aplicaciones}}

\chapter{Redes escolares}\label{redes-escolares}

\begin{tcolorbox}[enhanced jigsaw, colback=white, opacityback=0, coltitle=black, title=\textcolor{quarto-callout-warning-color}{\faExclamationTriangle}\hspace{0.5em}{Nota de Traducción}, bottomrule=.15mm, colbacktitle=quarto-callout-warning-color!10!white, toptitle=1mm, colframe=quarto-callout-warning-color-frame, titlerule=0mm, rightrule=.15mm, leftrule=.75mm, breakable, bottomtitle=1mm, left=2mm, arc=.35mm, toprule=.15mm, opacitybacktitle=0.6]

Esta versión del capítulo fue traducida de manera automática utilizando
IA. El capítulo aún no ha sido revisado por un humano.

\end{tcolorbox}

Este capítulo proporciona un ejemplo de principio a fin para procesar
datos tipo encuesta en R. El capítulo presenta el conjunto de datos del
Estudio de Redes Sociales {[}SNS{]}. Puedes descargar los datos para
este capítulo
\href{https://cdn.rawgit.com/gvegayon/appliedsnar/fdc0d26f/03-sns.dta}{aquí},
y el libro de códigos para los datos proporcionados aquí está en
\hyperref[sns-data]{el apéndice}.

Los objetivos para este capítulo son:

\begin{enumerate}
\def\labelenumi{\arabic{enumi}.}
\item
  Leer los datos en R.
\item
  Crear una red con ellos.
\item
  Calcular estadísticas descriptivas.
\item
  Visualizar la red.
\end{enumerate}

\section{Preprocesamiento de datos}\label{preprocesamiento-de-datos}

\subsection{Leyendo los datos en R}\label{leyendo-los-datos-en-r}

R tiene varias formas de leer datos. Tus datos pueden ser archivos de
texto plano como CSV, delimitados por tabulaciones, o especificados por
ancho de columna. Para leer datos de texto plano, puedes usar el paquete
\href{https://cran.r-project.org/package=readr}{\texttt{readr}}
(Wickham, Hester, and Bryan 2024). En el caso de archivos binarios, como
archivos de Stata, Octave, o SPSS, puedes usar el paquete de R
\href{https://cran.r-project.org/package=readr}{\texttt{foreign}} (R
Core Team 2023). Si tus datos están formateados como hojas de cálculo de
Microsoft, el paquete de R
\href{https://cran.r-project.org/package=readxl}{\texttt{readxl}}
(Wickham and Bryan 2023) es la alternativa a usar. En nuestro caso, los
datos para esta sesión están en formato Stata:

\begin{Shaded}
\begin{Highlighting}[]
\FunctionTok{library}\NormalTok{(foreign)}

\CommentTok{\# Leyendo los datos}
\NormalTok{dat }\OtherTok{\textless{}{-}}\NormalTok{ foreign}\SpecialCharTok{::}\FunctionTok{read.dta}\NormalTok{(}\StringTok{"03{-}sns.dta"}\NormalTok{)}

\CommentTok{\# Echando un vistazo a las primeras 5 columnas y 5 filas de los datos}
\NormalTok{dat[}\DecValTok{1}\SpecialCharTok{:}\DecValTok{5}\NormalTok{, }\DecValTok{1}\SpecialCharTok{:}\DecValTok{10}\NormalTok{]}
\end{Highlighting}
\end{Shaded}

\begin{verbatim}
  photoid school hispanic female1 female2 female3 female4 grades1 grades2
1       1    111        1      NA      NA       0       0      NA      NA
2       2    111        1       0      NA      NA       0     3.0      NA
3       7    111        0       1       1       1       1     5.0     4.5
4      13    111        1       1       1       1       1     2.5     2.5
5      14    111        1       1       1       1      NA     3.0     3.5
  grades3
1     3.5
2      NA
3     4.0
4     2.5
5     3.5
\end{verbatim}

\subsection{Creando un id único para cada
participante}\label{creando-un-id-uxfanico-para-cada-participante}

Debemos crear un \texttt{id} único usando la escuela y el \texttt{id} de
foto. Dado que ambas variables son numéricas, codificar el id es una
buena forma de hacer esto. Por ejemplo, los últimos tres números son el
\texttt{photoid}, y los primeros números son el \texttt{id} de la
escuela. Para hacer esto, necesitamos tomar en cuenta el rango de las
variables:

\begin{Shaded}
\begin{Highlighting}[]
\NormalTok{(photo\_id\_ran }\OtherTok{\textless{}{-}} \FunctionTok{range}\NormalTok{(dat}\SpecialCharTok{$}\NormalTok{photoid))}
\end{Highlighting}
\end{Shaded}

\begin{verbatim}
[1]    1 2074
\end{verbatim}

Como la variable se extiende hasta 2074, necesitamos establecer las
últimas 4 unidades de la variable para almacenar el \texttt{photoid}.
Usaremos \texttt{dplyr} (Wickham et al. 2023) para crear esta variable y
la llamaremos \texttt{id}:

\begin{Shaded}
\begin{Highlighting}[]
\FunctionTok{library}\NormalTok{(dplyr)}
\end{Highlighting}
\end{Shaded}

\begin{verbatim}

Attaching package: 'dplyr'
\end{verbatim}

\begin{verbatim}
The following objects are masked from 'package:stats':

    filter, lag
\end{verbatim}

\begin{verbatim}
The following objects are masked from 'package:base':

    intersect, setdiff, setequal, union
\end{verbatim}

\begin{Shaded}
\begin{Highlighting}[]
\CommentTok{\# Creando la variable}
\NormalTok{dat }\OtherTok{\textless{}{-}}\NormalTok{ dat }\SpecialCharTok{|\textgreater{}}
  \FunctionTok{mutate}\NormalTok{(}\AttributeTok{id =}\NormalTok{ school}\SpecialCharTok{*}\DecValTok{10000} \SpecialCharTok{+}\NormalTok{ photoid)}

\CommentTok{\# Primeras filas}
\NormalTok{dat }\SpecialCharTok{|\textgreater{}}
  \FunctionTok{head}\NormalTok{() }\SpecialCharTok{|\textgreater{}}
  \FunctionTok{select}\NormalTok{(school, photoid, id)}
\end{Highlighting}
\end{Shaded}

\begin{verbatim}
  school photoid      id
1    111       1 1110001
2    111       2 1110002
3    111       7 1110007
4    111      13 1110013
5    111      14 1110014
6    111      15 1110015
\end{verbatim}

¡Vaya, qué pasó en las últimas líneas de código! ¿Qué es ese
\texttt{\textbar{}\textgreater{}}? Bueno, ese es el operador
pipe\footnote{Introducido en R versión 4.1.0, el operador pipe de R base
  \texttt{\textbar{}\textgreater{}} funciona de manera similar al pipe
  de \texttt{magrittr} \texttt{\%\textgreater{}\%}. Las diferencias
  clave entre estos dos se explican en
  \url{https://www.tidyverse.org/blog/2023/04/base-vs-magrittr-pipe/}.},
y es una forma atractiva de escribir llamadas a funciones anidadas. En
este caso, en lugar de escribir algo como:

\begin{Shaded}
\begin{Highlighting}[]
\NormalTok{dat\_filtered}\SpecialCharTok{$}\NormalTok{id }\OtherTok{\textless{}{-}}\NormalTok{ dat\_filtered}\SpecialCharTok{$}\NormalTok{school}\SpecialCharTok{*}\DecValTok{10000} \SpecialCharTok{+}\NormalTok{ dat\_filtered}\SpecialCharTok{$}\NormalTok{photoid}
\FunctionTok{subset}\NormalTok{(}\FunctionTok{head}\NormalTok{(dat\_filtered), }\AttributeTok{select =} \FunctionTok{c}\NormalTok{(school, photoid, id))}
\end{Highlighting}
\end{Shaded}

\section{Creando una red}\label{creando-una-red}

\begin{itemize}
\item
  Queremos construir una red social. Para eso, usamos una matriz de
  adyacencia o una lista de enlaces.
\item
  Cada individuo de los datos SNS nominó 19 amigos de la escuela.
  Usaremos esas nominaciones para crear la red social.
\item
  En este caso, crearemos la red coercionando el conjunto de datos en
  una lista de enlaces.
\end{itemize}

\subsection{De encuesta a lista de
enlaces}\label{de-encuesta-a-lista-de-enlaces}

Comencemos cargando un par de paquetes útiles de R. Cargaremos
\texttt{tidyr} (Wickham, Vaughan, and Girlich 2024) y \texttt{stringr}
(Wickham 2023). Usaremos el primero, \texttt{tidyr}, para remodelar los
datos. El segundo, \texttt{stringr}, nos ayudará a procesar cadenas
usando \emph{expresiones regulares}\footnote{Por favor, consulta el
  archivo de ayuda
  \texttt{?\textquotesingle{}regular\ expression\textquotesingle{}} en
  R. El paquete de R \texttt{rex} (Ushey, Hester, and Krzyzanowski 2021)
  es un compañero amigable para escribir expresiones regulares. También
  hay un complemento de RStudio ordenado (pero experimental) que puede
  ser muy útil para entender cómo funcionan las expresiones regulares,
  el complemento
  \href{https://github.com/gadenbuie/regexplain}{regexplain}.}.

\begin{Shaded}
\begin{Highlighting}[]
\FunctionTok{library}\NormalTok{(tidyr)}
\FunctionTok{library}\NormalTok{(stringr)}
\end{Highlighting}
\end{Shaded}

Opcionalmente, podemos usar el tipo de objeto \texttt{tibble}, una
alternativa al \texttt{data.frame} actual. Este objeto proporciona
\emph{métodos más eficientes para matrices y marcos de datos}.

\begin{Shaded}
\begin{Highlighting}[]
\NormalTok{dat }\OtherTok{\textless{}{-}} \FunctionTok{as\_tibble}\NormalTok{(dat)}
\end{Highlighting}
\end{Shaded}

Lo que me gusta de los \texttt{tibbles} es que cuando los imprimes en la
consola, estos se ven bien:

\begin{Shaded}
\begin{Highlighting}[]
\NormalTok{dat}
\end{Highlighting}
\end{Shaded}

\begin{verbatim}
# A tibble: 2,164 x 100
   photoid school hispanic female1 female2 female3 female4 grades1 grades2
     <int>  <int>    <dbl>   <int>   <int>   <int>   <int>   <dbl>   <dbl>
 1       1    111        1      NA      NA       0       0    NA      NA  
 2       2    111        1       0      NA      NA       0     3      NA  
 3       7    111        0       1       1       1       1     5       4.5
 4      13    111        1       1       1       1       1     2.5     2.5
 5      14    111        1       1       1       1      NA     3       3.5
 6      15    111        1       0       0       0       0     2.5     2.5
 7      20    111        1       1       1       1       1     2.5     2.5
 8      22    111        1      NA      NA       0       0    NA      NA  
 9      25    111        0       1       1      NA       1     4.5     3.5
10      27    111        1       0      NA       0       0     3.5    NA  
# i 2,154 more rows
# i 91 more variables: grades3 <dbl>, grades4 <dbl>, eversmk1 <int>,
#   eversmk2 <int>, eversmk3 <int>, eversmk4 <int>, everdrk1 <int>,
#   everdrk2 <int>, everdrk3 <int>, everdrk4 <int>, home1 <int>, home2 <int>,
#   home3 <int>, home4 <int>, sch_friend11 <int>, sch_friend12 <int>,
#   sch_friend13 <int>, sch_friend14 <int>, sch_friend15 <int>,
#   sch_friend16 <int>, sch_friend17 <int>, sch_friend18 <int>, ...
\end{verbatim}

\begin{Shaded}
\begin{Highlighting}[]
\CommentTok{\# Tal vez demasiados pipes... ¡pero es genial!}
\NormalTok{net }\OtherTok{\textless{}{-}}\NormalTok{ dat }\SpecialCharTok{|\textgreater{}} 
  \FunctionTok{select}\NormalTok{(id, school, }\FunctionTok{starts\_with}\NormalTok{(}\StringTok{"sch\_friend"}\NormalTok{)) }\SpecialCharTok{|\textgreater{}}
  \FunctionTok{gather}\NormalTok{(}\AttributeTok{key =} \StringTok{"varname"}\NormalTok{, }\AttributeTok{value =} \StringTok{"content"}\NormalTok{, }\SpecialCharTok{{-}}\NormalTok{id, }\SpecialCharTok{{-}}\NormalTok{school) }\SpecialCharTok{|\textgreater{}}
  \FunctionTok{filter}\NormalTok{(}\SpecialCharTok{!}\FunctionTok{is.na}\NormalTok{(content)) }\SpecialCharTok{|\textgreater{}}
  \FunctionTok{mutate}\NormalTok{(}
    \AttributeTok{friendid =}\NormalTok{ school}\SpecialCharTok{*}\DecValTok{10000} \SpecialCharTok{+}\NormalTok{ content,}
    \AttributeTok{year     =} \FunctionTok{as.integer}\NormalTok{(}\FunctionTok{str\_extract}\NormalTok{(varname, }\StringTok{"(?\textless{}=[a{-}z])[0{-}9]"}\NormalTok{)),}
    \AttributeTok{nnom     =} \FunctionTok{as.integer}\NormalTok{(}\FunctionTok{str\_extract}\NormalTok{(varname, }\StringTok{"(?\textless{}=[a{-}z][0{-}9])[0{-}9]+"}\NormalTok{))}
\NormalTok{  )}
\end{Highlighting}
\end{Shaded}

Veamos esto paso a paso:

\begin{enumerate}
\def\labelenumi{\arabic{enumi}.}
\item
  Primero, subconjuntamos los datos: Queremos mantener
  \texttt{id,\ school,\ sch\_friend*.} Para este último, usamos la
  función \texttt{starts\_with} (del paquete \texttt{tidyselect}). Este
  último nos permite seleccionar todas las variables que comienzan con
  la palabra ``\texttt{sch\_friend}'', lo que significa que
  \texttt{sch\_friend11,\ sch\_friend12,\ ...} serán seleccionadas.

\begin{Shaded}
\begin{Highlighting}[]
\NormalTok{dat }\SpecialCharTok{|\textgreater{}} 
  \FunctionTok{select}\NormalTok{(id, school, }\FunctionTok{starts\_with}\NormalTok{(}\StringTok{"sch\_friend"}\NormalTok{))}
\end{Highlighting}
\end{Shaded}

\begin{verbatim}
# A tibble: 2,164 x 78
        id school sch_friend11 sch_friend12 sch_friend13 sch_friend14
     <dbl>  <int>        <int>        <int>        <int>        <int>
 1 1110001    111           NA           NA           NA           NA
 2 1110002    111          424          423          426          289
 3 1110007    111          629          505           NA           NA
 4 1110013    111          232          569           NA           NA
 5 1110014    111          582          134           41          592
 6 1110015    111           26          488           81          138
 7 1110020    111          528           NA          492          395
 8 1110022    111           NA           NA           NA           NA
 9 1110025    111          135          185          553           84
10 1110027    111          346          168          559            5
# i 2,154 more rows
# i 72 more variables: sch_friend15 <int>, sch_friend16 <int>,
#   sch_friend17 <int>, sch_friend18 <int>, sch_friend19 <int>,
#   sch_friend110 <int>, sch_friend111 <int>, sch_friend112 <int>,
#   sch_friend113 <int>, sch_friend114 <int>, sch_friend115 <int>,
#   sch_friend116 <int>, sch_friend117 <int>, sch_friend118 <int>,
#   sch_friend119 <int>, sch_friend21 <int>, sch_friend22 <int>, ...
\end{verbatim}
\item
  Luego, lo remodelamos a formato \emph{largo}: Transponiendo todos los
  \texttt{sch\_friend*} a formato largo. Hacemos esto usando la función
  \texttt{gather} (del paquete \texttt{tidyr}); una alternativa a la
  función \texttt{reshape}, que encuentro más fácil de usar. Veamos cómo
  funciona:

\begin{Shaded}
\begin{Highlighting}[]
\NormalTok{dat }\SpecialCharTok{|\textgreater{}} 
  \FunctionTok{select}\NormalTok{(id, school, }\FunctionTok{starts\_with}\NormalTok{(}\StringTok{"sch\_friend"}\NormalTok{)) }\SpecialCharTok{|\textgreater{}}
  \FunctionTok{gather}\NormalTok{(}\AttributeTok{key =} \StringTok{"varname"}\NormalTok{, }\AttributeTok{value =} \StringTok{"content"}\NormalTok{, }\SpecialCharTok{{-}}\NormalTok{id, }\SpecialCharTok{{-}}\NormalTok{school)}
\end{Highlighting}
\end{Shaded}

\begin{verbatim}
# A tibble: 164,464 x 4
        id school varname      content
     <dbl>  <int> <chr>          <int>
 1 1110001    111 sch_friend11      NA
 2 1110002    111 sch_friend11     424
 3 1110007    111 sch_friend11     629
 4 1110013    111 sch_friend11     232
 5 1110014    111 sch_friend11     582
 6 1110015    111 sch_friend11      26
 7 1110020    111 sch_friend11     528
 8 1110022    111 sch_friend11      NA
 9 1110025    111 sch_friend11     135
10 1110027    111 sch_friend11     346
# i 164,454 more rows
\end{verbatim}

  En este caso, el parámetro \texttt{key} establece el nombre de la
  variable que contendrá el nombre de la variable que fue remodelada,
  mientras que \texttt{value} es el nombre de la variable que contendrá
  el contenido de los datos (por eso los nombré así). El bit
  \texttt{-id,\ -school} le dice a la función que ``elimine'' esas
  variables antes de remodelar. En otras palabras, ``remodela todo
  excepto \texttt{id} y \texttt{school.}''

  También, nota que pasamos de 2164 filas a 19 (nominaciones) * 2164
  (sujetos) * 4 (ondas) = 164464 filas, como se esperaba.
\item
  Como los datos de nominación pueden estar vacíos para algunas celdas,
  necesitamos cuidar esos casos, los \texttt{NA}s, así que filtramos los
  datos:

\begin{Shaded}
\begin{Highlighting}[]
\NormalTok{dat }\SpecialCharTok{|\textgreater{}} 
  \FunctionTok{select}\NormalTok{(id, school, }\FunctionTok{starts\_with}\NormalTok{(}\StringTok{"sch\_friend"}\NormalTok{)) }\SpecialCharTok{|\textgreater{}}
  \FunctionTok{gather}\NormalTok{(}\AttributeTok{key =} \StringTok{"varname"}\NormalTok{, }\AttributeTok{value =} \StringTok{"content"}\NormalTok{, }\SpecialCharTok{{-}}\NormalTok{id, }\SpecialCharTok{{-}}\NormalTok{school) }\SpecialCharTok{|\textgreater{}}
  \FunctionTok{filter}\NormalTok{(}\SpecialCharTok{!}\FunctionTok{is.na}\NormalTok{(content))}
\end{Highlighting}
\end{Shaded}

\begin{verbatim}
# A tibble: 39,561 x 4
        id school varname      content
     <dbl>  <int> <chr>          <int>
 1 1110002    111 sch_friend11     424
 2 1110007    111 sch_friend11     629
 3 1110013    111 sch_friend11     232
 4 1110014    111 sch_friend11     582
 5 1110015    111 sch_friend11      26
 6 1110020    111 sch_friend11     528
 7 1110025    111 sch_friend11     135
 8 1110027    111 sch_friend11     346
 9 1110029    111 sch_friend11     369
10 1110030    111 sch_friend11     462
# i 39,551 more rows
\end{verbatim}
\item
  Y finalmente, creamos tres nuevas variables de este conjunto de datos:
  \texttt{friendid,}, \texttt{year}, y \texttt{nom\_num} (número de
  nominación). Todo usando expresiones regulares:

\begin{Shaded}
\begin{Highlighting}[]
\NormalTok{dat }\SpecialCharTok{|\textgreater{}} 
  \FunctionTok{select}\NormalTok{(id, school, }\FunctionTok{starts\_with}\NormalTok{(}\StringTok{"sch\_friend"}\NormalTok{)) }\SpecialCharTok{|\textgreater{}}
  \FunctionTok{gather}\NormalTok{(}\AttributeTok{key =} \StringTok{"varname"}\NormalTok{, }\AttributeTok{value =} \StringTok{"content"}\NormalTok{, }\SpecialCharTok{{-}}\NormalTok{id, }\SpecialCharTok{{-}}\NormalTok{school) }\SpecialCharTok{|\textgreater{}}
  \FunctionTok{filter}\NormalTok{(}\SpecialCharTok{!}\FunctionTok{is.na}\NormalTok{(content)) }\SpecialCharTok{|\textgreater{}}
  \FunctionTok{mutate}\NormalTok{(}
    \AttributeTok{friendid =}\NormalTok{ school}\SpecialCharTok{*}\DecValTok{10000} \SpecialCharTok{+}\NormalTok{ content,}
    \AttributeTok{year     =} \FunctionTok{as.integer}\NormalTok{(}\FunctionTok{str\_extract}\NormalTok{(varname, }\StringTok{"(?\textless{}=[a{-}z])[0{-}9]"}\NormalTok{)),}
    \AttributeTok{nnom     =} \FunctionTok{as.integer}\NormalTok{(}\FunctionTok{str\_extract}\NormalTok{(varname, }\StringTok{"(?\textless{}=[a{-}z][0{-}9])[0{-}9]+"}\NormalTok{))}
\NormalTok{    )}
\end{Highlighting}
\end{Shaded}

\begin{verbatim}
# A tibble: 39,561 x 7
        id school varname      content friendid  year  nnom
     <dbl>  <int> <chr>          <int>    <dbl> <int> <int>
 1 1110002    111 sch_friend11     424  1110424     1     1
 2 1110007    111 sch_friend11     629  1110629     1     1
 3 1110013    111 sch_friend11     232  1110232     1     1
 4 1110014    111 sch_friend11     582  1110582     1     1
 5 1110015    111 sch_friend11      26  1110026     1     1
 6 1110020    111 sch_friend11     528  1110528     1     1
 7 1110025    111 sch_friend11     135  1110135     1     1
 8 1110027    111 sch_friend11     346  1110346     1     1
 9 1110029    111 sch_friend11     369  1110369     1     1
10 1110030    111 sch_friend11     462  1110462     1     1
# i 39,551 more rows
\end{verbatim}

  La expresión regular \texttt{(?\textless{}={[}a-z{]})} coincide con
  una cadena precedida por cualquier letra de \emph{a} a \emph{z}. En
  contraste, la expresión \texttt{{[}0-9{]}} coincide con un solo
  número. Por lo tanto, de la cadena \texttt{"sch\_friend12"}, la
  expresión regular solo coincidirá con el \texttt{1}, ya que es el
  único número seguido por una letra. La expresión
  \texttt{(?\textless{}={[}a-z{]}{[}0-9{]})} coincide con una cadena
  precedida por una letra minúscula y un número de un dígito.
  Finalmente, la expresión \texttt{{[}0-9{]}+} coincide con una cadena
  de números--así que podría ser más de uno. Por lo tanto, de la cadena
  \texttt{"sch\_friend12"}, obtendremos \texttt{2}:

\begin{Shaded}
\begin{Highlighting}[]
\FunctionTok{str\_extract}\NormalTok{(}\StringTok{"sch\_friend12"}\NormalTok{, }\StringTok{"(?\textless{}=[a{-}z])[0{-}9]"}\NormalTok{)}
\end{Highlighting}
\end{Shaded}

\begin{verbatim}
[1] "1"
\end{verbatim}

\begin{Shaded}
\begin{Highlighting}[]
\FunctionTok{str\_extract}\NormalTok{(}\StringTok{"sch\_friend12"}\NormalTok{, }\StringTok{"(?\textless{}=[a{-}z][0{-}9])[0{-}9]+"}\NormalTok{)}
\end{Highlighting}
\end{Shaded}

\begin{verbatim}
[1] "2"
\end{verbatim}
\end{enumerate}

Y finalmente, la función \texttt{as.integer} coerciona el valor de
retorno de la función \texttt{str\_extract} de \texttt{character} a
\texttt{integer}. Ahora que tenemos esta lista de enlaces, podemos crear
un objeto igraph

\subsection{Red igraph}\label{red-igraph}

Para coercionar la lista de enlaces en un objeto igraph, usaremos la
función \texttt{graph\_from\_data\_frame} en igraph (Csárdi et al.
2024). Esta función recibe los siguientes argumentos: un marco de datos
donde las dos primeras columnas son ``source'' (ego) y ``target''
(alter), un indicador de si la red es dirigida o no, y un marco de datos
opcional con vértices, en cuya primera columna debería contener los ids
de vértice.

Usar el argumento opcional \texttt{vertices} es una buena práctica: Le
dice a la función qué \texttt{id}s debería esperar. Usando el conjunto
de datos original, crearemos un marco de datos con vértices de nombre:

\begin{Shaded}
\begin{Highlighting}[]
\NormalTok{vertex\_attrs }\OtherTok{\textless{}{-}}\NormalTok{ dat }\SpecialCharTok{|\textgreater{}} 
  \FunctionTok{select}\NormalTok{(id, school, hispanic, female1, }\FunctionTok{starts\_with}\NormalTok{(}\StringTok{"eversmk"}\NormalTok{))}
\end{Highlighting}
\end{Shaded}

Ahora, usemos la función \texttt{graph\_from\_data\_frame} para crear un
objeto \texttt{igraph}:

\begin{Shaded}
\begin{Highlighting}[]
\FunctionTok{library}\NormalTok{(igraph)}

\NormalTok{ig\_year1 }\OtherTok{\textless{}{-}}\NormalTok{ net }\SpecialCharTok{|\textgreater{}}
  \FunctionTok{filter}\NormalTok{(year }\SpecialCharTok{==} \StringTok{"1"}\NormalTok{) }\SpecialCharTok{|\textgreater{}} 
  \FunctionTok{select}\NormalTok{(id, friendid, nnom) }\SpecialCharTok{|\textgreater{}}
  \FunctionTok{graph\_from\_data\_frame}\NormalTok{(}
    \AttributeTok{vertices =}\NormalTok{ vertex\_attrs}
\NormalTok{  )}
\end{Highlighting}
\end{Shaded}

\begin{verbatim}
Error in graph_from_data_frame(select(filter(net, year == "1"), id, friendid, : Some vertex names in edge list are not listed in vertex data frame
\end{verbatim}

¡Ups! Parece que los individuos están nominando a otros estudiantes no
incluidos en la encuesta. ¿Cómo resolver eso? Bueno, ¡todo depende de lo
que necesites hacer! En este caso, iremos por la estrategia de
\emph{elimínalos-silenciosamente-y-no-digas-nada}:

\begin{Shaded}
\begin{Highlighting}[]
\FunctionTok{library}\NormalTok{(igraph)}

\NormalTok{ig\_year1 }\OtherTok{\textless{}{-}}\NormalTok{ net }\SpecialCharTok{|\textgreater{}}
  \FunctionTok{filter}\NormalTok{(year }\SpecialCharTok{==} \StringTok{"1"}\NormalTok{) }\SpecialCharTok{|\textgreater{}}
  
  \CommentTok{\# Línea extra, todas las nominaciones deben estar en ego también.}
  \FunctionTok{filter}\NormalTok{(friendid }\SpecialCharTok{\%in\%}\NormalTok{ id) }\SpecialCharTok{|\textgreater{}} 
  
  \FunctionTok{select}\NormalTok{(id, friendid, nnom) }\SpecialCharTok{|\textgreater{}}
  \FunctionTok{graph\_from\_data\_frame}\NormalTok{(}
    \AttributeTok{vertices =}\NormalTok{ vertex\_attrs}
\NormalTok{    )}

\NormalTok{ig\_year1}
\end{Highlighting}
\end{Shaded}

\begin{verbatim}
IGRAPH 2cd1263 DN-- 2164 9514 -- 
+ attr: name (v/c), school (v/n), hispanic (v/n), female1 (v/n),
| eversmk1 (v/n), eversmk2 (v/n), eversmk3 (v/n), eversmk4 (v/n), nnom
| (e/n)
+ edges from 2cd1263 (vertex names):
 [1] 1110007->1110629 1110013->1110232 1110014->1110582 1110015->1110026
 [5] 1110025->1110135 1110027->1110346 1110029->1110369 1110035->1110034
 [9] 1110040->1110390 1110041->1110557 1110044->1110027 1110046->1110030
[13] 1110050->1110086 1110057->1110263 1110069->1110544 1110071->1110167
[17] 1110072->1110289 1110073->1110014 1110075->1110352 1110084->1110305
[21] 1110086->1110206 1110093->1110040 1110094->1110483 1110095->1110043
+ ... omitted several edges
\end{verbatim}

Así que tenemos nuestra red con 2164 nodos y 9514 enlaces. Los
siguientes pasos: obtener algunas estadísticas descriptivas y visualizar
nuestra red.

\section{Estadísticas descriptivas de
red}\label{estaduxedsticas-descriptivas-de-red}

Aunque podríamos hacer todas las redes a la vez, en esta parte, nos
enfocaremos en calcular algunas estadísticas de red para una sola
escuela. Comenzamos por la escuela 111. La primera pregunta que deberías
estar haciéndote ahora es, ``¿cómo puedo obtener esa información del
objeto igraph?.'' Los atributos de vértices y enlaces se pueden acceder
a través de las funciones \texttt{V} y \texttt{E}, respectivamente;
además, podemos listar qué atributos de vértice/enlace están
disponibles:

\begin{Shaded}
\begin{Highlighting}[]
\FunctionTok{vertex\_attr\_names}\NormalTok{(ig\_year1)}
\end{Highlighting}
\end{Shaded}

\begin{verbatim}
[1] "name"     "school"   "hispanic" "female1"  "eversmk1" "eversmk2" "eversmk3"
[8] "eversmk4"
\end{verbatim}

\begin{Shaded}
\begin{Highlighting}[]
\FunctionTok{edge\_attr\_names}\NormalTok{(ig\_year1) }
\end{Highlighting}
\end{Shaded}

\begin{verbatim}
[1] "nnom"
\end{verbatim}

Tal como haríamos con marcos de datos, acceder a atributos de vértice se
hace a través del operador signo de dólar \texttt{\$}. Junto con la
función \texttt{V}; por ejemplo, acceder a los primeros diez elementos
de la variable \texttt{hispanic} se puede hacer de la siguiente manera:

\begin{Shaded}
\begin{Highlighting}[]
\FunctionTok{V}\NormalTok{(ig\_year1)}\SpecialCharTok{$}\NormalTok{hispanic[}\DecValTok{1}\SpecialCharTok{:}\DecValTok{10}\NormalTok{]}
\end{Highlighting}
\end{Shaded}

\begin{verbatim}
 [1] 1 1 0 1 1 1 1 1 0 1
\end{verbatim}

Ahora que sabes cómo acceder a atributos de vértice, podemos obtener la
red correspondiente a la escuela 111 identificando qué vértices son
parte de ella y pasar esa información a la función
\texttt{induced\_subgraph}:

\begin{Shaded}
\begin{Highlighting}[]
\CommentTok{\# ¿Qué ids son de la escuela 111?}
\NormalTok{school111ids }\OtherTok{\textless{}{-}} \FunctionTok{which}\NormalTok{(}\FunctionTok{V}\NormalTok{(ig\_year1)}\SpecialCharTok{$}\NormalTok{school }\SpecialCharTok{==} \DecValTok{111}\NormalTok{)}

\CommentTok{\# Creando un subgrafo}
\NormalTok{ig\_year1\_111 }\OtherTok{\textless{}{-}} \FunctionTok{induced\_subgraph}\NormalTok{(}
  \AttributeTok{graph =}\NormalTok{ ig\_year1,}
  \AttributeTok{vids  =}\NormalTok{ school111ids}
\NormalTok{)}
\end{Highlighting}
\end{Shaded}

La función \texttt{which} en R devuelve un vector de índices indicando
qué elementos pasan la prueba, devolviendo verdadero y falso, de lo
contrario. En nuestro caso, resultará en un vector de índices de los
vértices que tienen el atributo \texttt{school} igual a 111. Con el
subgrafo, podemos calcular diferentes medidas de centralidad\footnote{Para
  más información sobre las diferentes medidas de centralidad, por favor
  echa un vistazo al artículo ``Centrality'' en
  \href{https://en.wikipedia.org/wiki/Centrality}{Wikipedia}.} para cada
vértice y almacenarlas en el objeto igraph mismo:

\begin{Shaded}
\begin{Highlighting}[]
\CommentTok{\# Calculando medidas de centralidad para cada vértice}
\FunctionTok{V}\NormalTok{(ig\_year1\_111)}\SpecialCharTok{$}\NormalTok{indegree   }\OtherTok{\textless{}{-}} \FunctionTok{degree}\NormalTok{(ig\_year1\_111, }\AttributeTok{mode =} \StringTok{"in"}\NormalTok{)}
\FunctionTok{V}\NormalTok{(ig\_year1\_111)}\SpecialCharTok{$}\NormalTok{outdegree  }\OtherTok{\textless{}{-}} \FunctionTok{degree}\NormalTok{(ig\_year1\_111, }\AttributeTok{mode =} \StringTok{"out"}\NormalTok{)}
\FunctionTok{V}\NormalTok{(ig\_year1\_111)}\SpecialCharTok{$}\NormalTok{closeness  }\OtherTok{\textless{}{-}} \FunctionTok{closeness}\NormalTok{(ig\_year1\_111, }\AttributeTok{mode =} \StringTok{"total"}\NormalTok{)}
\FunctionTok{V}\NormalTok{(ig\_year1\_111)}\SpecialCharTok{$}\NormalTok{betweeness }\OtherTok{\textless{}{-}} \FunctionTok{betweenness}\NormalTok{(ig\_year1\_111, }\AttributeTok{normalized =} \ConstantTok{TRUE}\NormalTok{)}
\end{Highlighting}
\end{Shaded}

Desde aquí, podemos \emph{volver} a nuestros viejos hábitos y obtener el
conjunto de atributos de vértice como un marco de datos para que podamos
calcular algunas estadísticas de resumen sobre las medidas de
centralidad que acabamos de obtener

\begin{Shaded}
\begin{Highlighting}[]
\CommentTok{\# Extrayendo cada característica de vértice como un data.frame}
\NormalTok{stats }\OtherTok{\textless{}{-}} \FunctionTok{as\_data\_frame}\NormalTok{(ig\_year1\_111, }\AttributeTok{what =} \StringTok{"vertices"}\NormalTok{)}

\CommentTok{\# Calculando cuantiles para cada variable}
\NormalTok{stats\_degree }\OtherTok{\textless{}{-}} \FunctionTok{with}\NormalTok{(stats, \{}
 \FunctionTok{cbind}\NormalTok{(}
   \AttributeTok{indegree   =} \FunctionTok{quantile}\NormalTok{(indegree, }\FunctionTok{c}\NormalTok{(.}\DecValTok{025}\NormalTok{, .}\DecValTok{5}\NormalTok{, .}\DecValTok{975}\NormalTok{), }\AttributeTok{na.rm =} \ConstantTok{TRUE}\NormalTok{),}
   \AttributeTok{outdegree  =} \FunctionTok{quantile}\NormalTok{(outdegree, }\FunctionTok{c}\NormalTok{(.}\DecValTok{025}\NormalTok{, .}\DecValTok{5}\NormalTok{, .}\DecValTok{975}\NormalTok{), }\AttributeTok{na.rm =} \ConstantTok{TRUE}\NormalTok{),}
   \AttributeTok{closeness  =} \FunctionTok{quantile}\NormalTok{(closeness, }\FunctionTok{c}\NormalTok{(.}\DecValTok{025}\NormalTok{, .}\DecValTok{5}\NormalTok{, .}\DecValTok{975}\NormalTok{), }\AttributeTok{na.rm =} \ConstantTok{TRUE}\NormalTok{),}
   \AttributeTok{betweeness =} \FunctionTok{quantile}\NormalTok{(betweeness, }\FunctionTok{c}\NormalTok{(.}\DecValTok{025}\NormalTok{, .}\DecValTok{5}\NormalTok{, .}\DecValTok{975}\NormalTok{), }\AttributeTok{na.rm =} \ConstantTok{TRUE}\NormalTok{)}
\NormalTok{ )}
\NormalTok{\})}

\NormalTok{stats\_degree}
\end{Highlighting}
\end{Shaded}

\begin{verbatim}
      indegree outdegree    closeness  betweeness
2.5%         0         0 0.0005915148 0.000000000
50%          4         4 0.0007487833 0.001879006
97.5%       16        16 0.0008838413 0.016591048
\end{verbatim}

La función \texttt{with} es algo similar a lo que \texttt{dplyr} nos
permite hacer cuando queremos trabajar con el conjunto de datos pero sin
mencionar su nombre cada vez que pedimos una variable. Sin usar la
función \texttt{with}, lo anterior podría haberse hecho de la siguiente
manera:

\begin{Shaded}
\begin{Highlighting}[]
\NormalTok{stats\_degree }\OtherTok{\textless{}{-}} 
 \FunctionTok{cbind}\NormalTok{(}
   \AttributeTok{indegree   =} \FunctionTok{quantile}\NormalTok{(stats}\SpecialCharTok{$}\NormalTok{indegree, }\FunctionTok{c}\NormalTok{(.}\DecValTok{025}\NormalTok{, .}\DecValTok{5}\NormalTok{, .}\DecValTok{975}\NormalTok{), }\AttributeTok{na.rm =} \ConstantTok{TRUE}\NormalTok{),}
   \AttributeTok{outdegree  =} \FunctionTok{quantile}\NormalTok{(stats}\SpecialCharTok{$}\NormalTok{outdegree, }\FunctionTok{c}\NormalTok{(.}\DecValTok{025}\NormalTok{, .}\DecValTok{5}\NormalTok{, .}\DecValTok{975}\NormalTok{), }\AttributeTok{na.rm =} \ConstantTok{TRUE}\NormalTok{),}
   \AttributeTok{closeness  =} \FunctionTok{quantile}\NormalTok{(stats}\SpecialCharTok{$}\NormalTok{closeness, }\FunctionTok{c}\NormalTok{(.}\DecValTok{025}\NormalTok{, .}\DecValTok{5}\NormalTok{, .}\DecValTok{975}\NormalTok{), }\AttributeTok{na.rm =} \ConstantTok{TRUE}\NormalTok{),}
   \AttributeTok{betweeness =} \FunctionTok{quantile}\NormalTok{(stats}\SpecialCharTok{$}\NormalTok{betweeness, }\FunctionTok{c}\NormalTok{(.}\DecValTok{025}\NormalTok{, .}\DecValTok{5}\NormalTok{, .}\DecValTok{975}\NormalTok{), }\AttributeTok{na.rm =} \ConstantTok{TRUE}\NormalTok{)}
\NormalTok{ )}
\end{Highlighting}
\end{Shaded}

A continuación, calcularemos algunas estadísticas a nivel de grafo:

\begin{Shaded}
\begin{Highlighting}[]
\FunctionTok{cbind}\NormalTok{(}
  \AttributeTok{size    =} \FunctionTok{vcount}\NormalTok{(ig\_year1\_111),}
  \AttributeTok{nedges  =} \FunctionTok{ecount}\NormalTok{(ig\_year1\_111),}
  \AttributeTok{density =} \FunctionTok{edge\_density}\NormalTok{(ig\_year1\_111),}
  \AttributeTok{recip   =} \FunctionTok{reciprocity}\NormalTok{(ig\_year1\_111),}
  \AttributeTok{centr   =} \FunctionTok{centr\_betw}\NormalTok{(ig\_year1\_111)}\SpecialCharTok{$}\NormalTok{centralization,}
  \AttributeTok{pathLen =} \FunctionTok{mean\_distance}\NormalTok{(ig\_year1\_111)}
\NormalTok{  )}
\end{Highlighting}
\end{Shaded}

\begin{verbatim}
     size nedges     density     recip      centr pathLen
[1,]  533   2638 0.009303277 0.3731513 0.02179154 4.23678
\end{verbatim}

Censo triádico

\begin{Shaded}
\begin{Highlighting}[]
\NormalTok{triadic }\OtherTok{\textless{}{-}} \FunctionTok{triad\_census}\NormalTok{(ig\_year1\_111)}
\NormalTok{triadic}
\end{Highlighting}
\end{Shaded}

\begin{verbatim}
 [1] 24059676   724389   290849     3619     3383     4401     3219     2997
 [9]      407       33      836      235      163      137      277       85
\end{verbatim}

Para obtener una vista más agradable de esto, podemos usar una tabla que
recuperé de \texttt{?triad\_census}. Además, podemos normalizar el
objeto \texttt{triadic} por su suma en lugar de mirar conteos en bruto.
De esa manera, obtenemos proporciones en su lugar\footnote{Durante
  nuestro taller, la Prof.~De la Haye sugirió usar \({n \choose 3}\)
  como una constante normalizadora. ¡Resulta que
  \texttt{sum(triadic)\ =\ choose(n,\ 3)}! Así que cualquier enfoque es
  correcto.}

\begin{Shaded}
\begin{Highlighting}[]
\NormalTok{knitr}\SpecialCharTok{::}\FunctionTok{kable}\NormalTok{(}\FunctionTok{cbind}\NormalTok{(}
  \AttributeTok{Pcent =}\NormalTok{ triadic}\SpecialCharTok{/}\FunctionTok{sum}\NormalTok{(triadic)}\SpecialCharTok{*}\DecValTok{100}\NormalTok{,}
  \FunctionTok{read.csv}\NormalTok{(}\StringTok{"triadic\_census.csv"}\NormalTok{)}
\NormalTok{  ), }\AttributeTok{digits =} \DecValTok{2}\NormalTok{)}
\end{Highlighting}
\end{Shaded}

\begin{longtable}[]{@{}
  >{\raggedleft\arraybackslash}p{(\linewidth - 4\tabcolsep) * \real{0.0769}}
  >{\raggedright\arraybackslash}p{(\linewidth - 4\tabcolsep) * \real{0.0641}}
  >{\raggedright\arraybackslash}p{(\linewidth - 4\tabcolsep) * \real{0.8590}}@{}}
\toprule\noalign{}
\begin{minipage}[b]{\linewidth}\raggedleft
Pcent
\end{minipage} & \begin{minipage}[b]{\linewidth}\raggedright
code
\end{minipage} & \begin{minipage}[b]{\linewidth}\raggedright
description
\end{minipage} \\
\midrule\noalign{}
\endhead
\bottomrule\noalign{}
\endlastfoot
95.88 & 003 & A,B,C, the empty graph. \\
2.89 & 012 & A-\textgreater B, C, the graph with a single directed
edge. \\
1.16 & 102 & A\textless-\textgreater B, C, the graph with a mutual
connection between two vertices. \\
0.01 & 021D & A\textless-B-\textgreater C, the out-star. \\
0.01 & 021U & A-\textgreater B\textless-C, the in-star. \\
0.02 & 021C & A-\textgreater B-\textgreater C, directed line. \\
0.01 & 111D & A\textless-\textgreater B\textless-C. \\
0.01 & 111U & A\textless-\textgreater B-\textgreater C. \\
0.00 & 030T & A-\textgreater B\textless-C, A-\textgreater C. \\
0.00 & 030C & A\textless-B\textless-C, A-\textgreater C. \\
0.00 & 201 & A\textless-\textgreater B\textless-\textgreater C. \\
0.00 & 120D & A\textless-B-\textgreater C, A\textless-\textgreater C. \\
0.00 & 120U & A-\textgreater B\textless-C, A\textless-\textgreater C. \\
0.00 & 120C & A-\textgreater B-\textgreater C,
A\textless-\textgreater C. \\
0.00 & 210 & A-\textgreater B\textless-\textgreater C,
A\textless-\textgreater C. \\
0.00 & 300 & A\textless-\textgreater B\textless-\textgreater C,
A\textless-\textgreater C, the complete graph. \\
\end{longtable}

\section{Graficando la red en igraph}\label{graficando-la-red-en-igraph}

\subsection{Gráfico único}\label{gruxe1fico-uxfanico}

Echemos un vistazo a cómo se ve nuestra red cuando usamos los parámetros
predeterminados en el método plot del objeto igraph:

\begin{Shaded}
\begin{Highlighting}[]
\FunctionTok{plot}\NormalTok{(ig\_year1)}
\end{Highlighting}
\end{Shaded}

\begin{figure}[H]

{\centering \pandocbounded{\includegraphics[keepaspectratio]{part-01-03-week-1-sns-study_files/figure-pdf/03-plot-raw-1.pdf}}

}

\caption{Un gráfico de red no muy agradable. Esto es lo que obtenemos
con los parámetros predeterminados en igraph.}

\end{figure}%

No muy agradable, ¿verdad? Un par de cosas con este gráfico:

\begin{enumerate}
\def\labelenumi{\arabic{enumi}.}
\item
  Estamos viendo todas las escuelas simultáneamente, lo que no tiene
  sentido. Así que, en lugar de graficar \texttt{ig\_year1}, nos
  enfocaremos en \texttt{ig\_year1\_111}.
\item
  Todos los vértices tienen el mismo tamaño y se están solapando. En
  lugar de usar el tamaño predeterminado, dimensionaremos los vértices
  por indegree usando la función \texttt{degree} y pasando el vector de
  grados a \texttt{vertex.size}.\footnote{Descubrir cuál es el tamaño de
    vértice óptimo es un poco complicado. Sin ponerse demasiado técnico,
    no hay otra forma de obtener un tamaño de vértice \emph{agradable}
    que no sea simplemente jugar con diferentes valores de él. Una
    solución agradable a esto es usar
    \href{https://www.rdocumentation.org/packages/netdiffuseR/versions/1.17.0/topics/rescale_vertex_igraph}{\texttt{netdiffuseR::igraph\_vertex\_rescale}}
    que reescala los vértices para que estos mantengan su relación de
    aspecto a una proporción predefinida de la pantalla.}
\item
  Dado el número de vértices en estas redes, las etiquetas no son útiles
  aquí. Así que las eliminaremos estableciendo
  \texttt{vertex.label\ =\ NA}. Además, reduciremos el tamaño de la
  punta de las flechas estableciendo \texttt{edge.arrow.size\ =\ 0.25}.
\item
  Y finalmente, estableceremos el color de cada vértice para que sea una
  función de si el individuo es hispano o no. Para esta última parte
  necesitamos ir un poco más de programación:
\end{enumerate}

\begin{Shaded}
\begin{Highlighting}[]
\NormalTok{col\_hispanic }\OtherTok{\textless{}{-}} \FunctionTok{V}\NormalTok{(ig\_year1\_111)}\SpecialCharTok{$}\NormalTok{hispanic }\SpecialCharTok{+} \DecValTok{1}
\NormalTok{col\_hispanic }\OtherTok{\textless{}{-}} \FunctionTok{coalesce}\NormalTok{(col\_hispanic, }\DecValTok{3}\NormalTok{) }
\NormalTok{col\_hispanic }\OtherTok{\textless{}{-}} \FunctionTok{c}\NormalTok{(}\StringTok{"steelblue"}\NormalTok{, }\StringTok{"tomato"}\NormalTok{, }\StringTok{"white"}\NormalTok{)[col\_hispanic]}
\end{Highlighting}
\end{Shaded}

Línea por línea, hicimos lo siguiente:

\begin{enumerate}
\def\labelenumi{\arabic{enumi}.}
\item
  La primera línea agregó uno a todos los valores no \texttt{NA} para
  que los 0s (no hispanos) se convirtieran en 1s y los 1s (hispanos) se
  convirtieran en 2s.
\item
  La segunda línea reemplazó todos los \texttt{NA}s con el número tres
  para que nuestro vector \texttt{col\_hispanic} ahora vaya de uno a
  tres sin \texttt{NA}s en él.
\item
  En la última línea, creamos un vector de colores. Esencialmente, lo
  que estamos haciendo aquí es decirle a R que cree un vector de
  longitud \texttt{length(col\_hispanic)} seleccionando elementos por
  índice del vector \texttt{c("steelblue",\ "tomato",\ "white")}. De
  esta manera, si, por ejemplo, el primer elemento del vector
  \texttt{col\_hispanic} fuera un 3, nuestro nuevo vector de colores
  tendría un \texttt{"white"} en él.
\end{enumerate}

Para asegurarnos de que sabemos que estamos en lo correcto, imprimamos
los primeros 10 elementos de nuestro nuevo vector de colores junto con
la columna original \texttt{hispanic}:

\begin{Shaded}
\begin{Highlighting}[]
\FunctionTok{cbind}\NormalTok{(}
  \AttributeTok{original =} \FunctionTok{V}\NormalTok{(ig\_year1\_111)}\SpecialCharTok{$}\NormalTok{hispanic[}\DecValTok{1}\SpecialCharTok{:}\DecValTok{10}\NormalTok{],}
  \AttributeTok{colors   =}\NormalTok{ col\_hispanic[}\DecValTok{1}\SpecialCharTok{:}\DecValTok{10}\NormalTok{]}
\NormalTok{  )}
\end{Highlighting}
\end{Shaded}

\begin{verbatim}
      original colors     
 [1,] "1"      "tomato"   
 [2,] "1"      "tomato"   
 [3,] "0"      "steelblue"
 [4,] "1"      "tomato"   
 [5,] "1"      "tomato"   
 [6,] "1"      "tomato"   
 [7,] "1"      "tomato"   
 [8,] "1"      "tomato"   
 [9,] "0"      "steelblue"
[10,] "1"      "tomato"   
\end{verbatim}

Con nuestro agradable vector de colores, ahora podemos pasarlo a
\texttt{plot.igraph} (que llamamos implícitamente simplemente llamando
\texttt{plot}), a través del argumento \texttt{vertex.color}:

\begin{Shaded}
\begin{Highlighting}[]
\CommentTok{\# Gráfico elegante}
\FunctionTok{set.seed}\NormalTok{(}\DecValTok{1}\NormalTok{)}
\FunctionTok{plot}\NormalTok{(}
\NormalTok{  ig\_year1\_111,}
  \AttributeTok{vertex.size     =} \FunctionTok{degree}\NormalTok{(ig\_year1\_111)}\SpecialCharTok{/}\DecValTok{10} \SpecialCharTok{+}\DecValTok{1}\NormalTok{,}
  \AttributeTok{vertex.label    =} \ConstantTok{NA}\NormalTok{,}
  \AttributeTok{edge.arrow.size =}\NormalTok{ .}\DecValTok{25}\NormalTok{,}
  \AttributeTok{vertex.color    =}\NormalTok{ col\_hispanic}
\NormalTok{  )}
\end{Highlighting}
\end{Shaded}

\begin{figure}[H]

{\centering \pandocbounded{\includegraphics[keepaspectratio]{part-01-03-week-1-sns-study_files/figure-pdf/03-plot-neat1-1.pdf}}

}

\caption{Red de amigos en tiempo 1 para la escuela 111.}

\end{figure}%

¡Agradable! Así que se ve mejor. El único problema es que tenemos muchos
aislados. Intentemos de nuevo dibujando el mismo gráfico sin aislados.
Para hacer eso, necesitamos filtrar el grafo, para lo cual usaremos la
función \texttt{induced\_subgraph}

\begin{Shaded}
\begin{Highlighting}[]
\CommentTok{\# ¿Qué vértices no son aislados?}
\NormalTok{which\_ids }\OtherTok{\textless{}{-}} \FunctionTok{which}\NormalTok{(}\FunctionTok{degree}\NormalTok{(ig\_year1\_111, }\AttributeTok{mode =} \StringTok{"total"}\NormalTok{) }\SpecialCharTok{\textgreater{}} \DecValTok{0}\NormalTok{)}

\CommentTok{\# Obteniendo el subgrafo}
\NormalTok{ig\_year1\_111\_sub }\OtherTok{\textless{}{-}} \FunctionTok{induced\_subgraph}\NormalTok{(ig\_year1\_111, which\_ids)}

\CommentTok{\# Necesitamos obtener el mismo subconjunto en col\_hispanic}
\NormalTok{col\_hispanic }\OtherTok{\textless{}{-}}\NormalTok{ col\_hispanic[which\_ids]}
\end{Highlighting}
\end{Shaded}

\begin{Shaded}
\begin{Highlighting}[]
\CommentTok{\# Gráfico elegante}
\FunctionTok{set.seed}\NormalTok{(}\DecValTok{1}\NormalTok{)}
\FunctionTok{plot}\NormalTok{(}
\NormalTok{  ig\_year1\_111\_sub,}
  \AttributeTok{vertex.size     =} \FunctionTok{degree}\NormalTok{(ig\_year1\_111\_sub)}\SpecialCharTok{/}\DecValTok{5} \SpecialCharTok{+}\DecValTok{1}\NormalTok{,}
  \AttributeTok{vertex.label    =} \ConstantTok{NA}\NormalTok{,}
  \AttributeTok{edge.arrow.size =}\NormalTok{ .}\DecValTok{25}\NormalTok{,}
  \AttributeTok{vertex.color    =}\NormalTok{ col\_hispanic}
\NormalTok{  )}
\end{Highlighting}
\end{Shaded}

\begin{figure}[H]

{\centering \pandocbounded{\includegraphics[keepaspectratio]{part-01-03-week-1-sns-study_files/figure-pdf/03-plot-neat2-1.pdf}}

}

\caption{Red de amigos en tiempo 1 para la escuela 111. El grafo excluye
aislados.}

\end{figure}%

¡Ahora eso está mejor! Un patrón interesante que surge es que los
individuos parecen agruparse por si son hispanos o no.

Podemos escribir esto como una función para evitar copiar y pegar el
código \(n\) veces (suponiendo que queremos crear un gráfico similar a
este \(n\) veces). Hacemos esto último en la siguiente subsección.

\subsection{Múltiples gráficos}\label{muxfaltiples-gruxe1ficos}

Cuando te estás repitiendo repetidamente, es una buena idea escribir una
secuencia de comandos como una función. En este caso, dado que
ejecutaremos el mismo tipo de gráfico para todas las escuelas/ondas,
escribimos una función en la que las únicas cosas que cambian son: (a)
el id de la escuela, y (b) el color de los nodos.

\begin{Shaded}
\begin{Highlighting}[]
\NormalTok{myplot }\OtherTok{\textless{}{-}} \ControlFlowTok{function}\NormalTok{(}
\NormalTok{  net,}
\NormalTok{  schoolid,}
  \AttributeTok{mindgr =} \DecValTok{1}\NormalTok{,}
  \AttributeTok{vcol   =} \StringTok{"tomato"}\NormalTok{,}
\NormalTok{  ...) \{}
  
  \CommentTok{\# Creando un subgrafo}
\NormalTok{  subnet }\OtherTok{\textless{}{-}} \FunctionTok{induced\_subgraph}\NormalTok{(}
\NormalTok{    net,}
    \FunctionTok{which}\NormalTok{(}\FunctionTok{degree}\NormalTok{(net, }\AttributeTok{mode =} \StringTok{"all"}\NormalTok{) }\SpecialCharTok{\textgreater{}=}\NormalTok{ mindgr }\SpecialCharTok{\&} \FunctionTok{V}\NormalTok{(net)}\SpecialCharTok{$}\NormalTok{school }\SpecialCharTok{==}\NormalTok{ schoolid)}
\NormalTok{  )}
  
  \CommentTok{\# Gráfico elegante}
  \FunctionTok{set.seed}\NormalTok{(}\DecValTok{1}\NormalTok{)}
  \FunctionTok{plot}\NormalTok{(}
\NormalTok{    subnet,}
    \AttributeTok{vertex.size     =} \FunctionTok{degree}\NormalTok{(subnet)}\SpecialCharTok{/}\DecValTok{5}\NormalTok{,}
    \AttributeTok{vertex.label    =} \ConstantTok{NA}\NormalTok{,}
    \AttributeTok{edge.arrow.size =}\NormalTok{ .}\DecValTok{25}\NormalTok{,}
    \AttributeTok{vertex.color    =}\NormalTok{ vcol,}
\NormalTok{    ...}
\NormalTok{    )}
\NormalTok{\}}
\end{Highlighting}
\end{Shaded}

La definición de la función:

\begin{enumerate}
\def\labelenumi{\arabic{enumi}.}
\item
  El
  \texttt{myplot\ \textless{}-\ function({[}argumentos{]})\ \{{[}cuerpo\ de\ la\ función{]}\}}
  le dice a R que vamos a crear una función llamada \texttt{myplot}.
\item
  Declaramos cuatro argumentos específicos: \texttt{net},
  \texttt{schoolid}, \texttt{mindgr}, y \texttt{vcol}. Estos son un
  objeto igraph, el id de la escuela, el grado mínimo que los vértices
  deben tener para ser incluidos en la figura, y el color de los
  vértices. Observa que, comparado con otros lenguajes de programación,
  R no requiere declarar los tipos de datos.
\item
  El objeto de puntos suspensivos, \texttt{...}, es un objeto especial
  en R que nos permite pasar otros argumentos sin especificar cuáles. Si
  echas un vistazo al bit \texttt{plot} en el cuerpo de la función,
  verás que también agregamos \texttt{...}. Usamos los puntos
  suspensivos para pasar argumentos extra (diferentes de los que
  definimos explícitamente) directamente a \texttt{plot}. En la
  práctica, esto implica que podemos, por ejemplo, establecer el
  argumento \texttt{edge.arrow.size} al llamar \texttt{myplot}, ¡incluso
  aunque no lo incluimos en la definición de la función! (Ver
  \texttt{?dotsMethods} en R para más detalles).
\end{enumerate}

En las siguientes líneas de código, usando nuestra nueva función,
graficaremos la red de cada escuela en el mismo dispositivo de graficado
(ventana) con la ayuda de la función \texttt{par}, y agregaremos leyenda
con el \texttt{legend}:

\begin{Shaded}
\begin{Highlighting}[]
\CommentTok{\# Graficando todos juntos}
\NormalTok{oldpar }\OtherTok{\textless{}{-}} \FunctionTok{par}\NormalTok{(}\AttributeTok{no.readonly =} \ConstantTok{TRUE}\NormalTok{)}
\FunctionTok{par}\NormalTok{(}\AttributeTok{mfrow =} \FunctionTok{c}\NormalTok{(}\DecValTok{2}\NormalTok{, }\DecValTok{3}\NormalTok{), }\AttributeTok{mai =} \FunctionTok{rep}\NormalTok{(}\DecValTok{0}\NormalTok{, }\DecValTok{4}\NormalTok{), }\AttributeTok{oma=} \FunctionTok{c}\NormalTok{(}\DecValTok{1}\NormalTok{, }\DecValTok{0}\NormalTok{, }\DecValTok{0}\NormalTok{, }\DecValTok{0}\NormalTok{))}
\FunctionTok{myplot}\NormalTok{(ig\_year1, }\DecValTok{111}\NormalTok{, }\AttributeTok{vcol =} \StringTok{"tomato"}\NormalTok{)}
\FunctionTok{myplot}\NormalTok{(ig\_year1, }\DecValTok{112}\NormalTok{, }\AttributeTok{vcol =} \StringTok{"steelblue"}\NormalTok{)}
\FunctionTok{myplot}\NormalTok{(ig\_year1, }\DecValTok{113}\NormalTok{, }\AttributeTok{vcol =} \StringTok{"black"}\NormalTok{)}
\FunctionTok{myplot}\NormalTok{(ig\_year1, }\DecValTok{114}\NormalTok{, }\AttributeTok{vcol =} \StringTok{"gold"}\NormalTok{)}
\FunctionTok{myplot}\NormalTok{(ig\_year1, }\DecValTok{115}\NormalTok{, }\AttributeTok{vcol =} \StringTok{"white"}\NormalTok{)}
\FunctionTok{par}\NormalTok{(oldpar)}

\CommentTok{\# Una leyenda elegante}
\FunctionTok{legend}\NormalTok{(}
  \StringTok{"bottomright"}\NormalTok{,}
  \AttributeTok{legend =} \FunctionTok{c}\NormalTok{(}\DecValTok{111}\NormalTok{, }\DecValTok{112}\NormalTok{, }\DecValTok{113}\NormalTok{, }\DecValTok{114}\NormalTok{, }\DecValTok{115}\NormalTok{),}
  \AttributeTok{pt.bg  =} \FunctionTok{c}\NormalTok{(}\StringTok{"tomato"}\NormalTok{, }\StringTok{"steelblue"}\NormalTok{, }\StringTok{"black"}\NormalTok{, }\StringTok{"gold"}\NormalTok{, }\StringTok{"white"}\NormalTok{),}
  \AttributeTok{pch    =} \DecValTok{21}\NormalTok{,}
  \AttributeTok{cex    =} \DecValTok{1}\NormalTok{,}
  \AttributeTok{bty    =} \StringTok{"n"}\NormalTok{,}
  \AttributeTok{title  =} \StringTok{"Escuela"}
\NormalTok{  )}
\end{Highlighting}
\end{Shaded}

\begin{figure}[H]

{\centering \pandocbounded{\includegraphics[keepaspectratio]{part-01-03-week-1-sns-study_files/figure-pdf/03-myplot-call-1.pdf}}

}

\caption{Las 5 escuelas en tiempo 1. Nuevamente, los grafos excluyen
aislados.}

\end{figure}%

Entonces, ¿qué pasó aquí?

\begin{itemize}
\item
  \texttt{oldpar\ \textless{}-\ par(no.readonly\ =\ TRUE)} Esta línea
  almacena los parámetros actuales para graficar. Dado que vamos a estar
  cambiándolos, ¡más vale asegurarnos de que podemos volver!.
\item
  \texttt{par(mfrow\ =\ c(2,\ 3),\ mai\ =\ rep(0,\ 4),\ oma=rep(0,\ 4))}
  Aquí estamos estableciendo varias cosas al mismo tiempo.
  \texttt{mfrow} especifica cuántas \emph{figuras} se dibujarán, y en
  qué orden. En particular, estamos pidiendo al dispositivo de graficado
  que haga espacio para 2*3 = 6 figuras organizadas en dos filas y tres
  columnas dibujadas por fila.

  \texttt{mai} especifica el tamaño de los márgenes en pulgadas,
  establecer todos los márgenes iguales a cero (que es lo que estamos
  haciendo ahora) da más espacio al gráfico. Lo mismo es cierto para
  \texttt{oma}. Ver \texttt{?par} para más información.
\item
  \texttt{myplot(ig\_year1,\ ...)} Esto es simplemente llamar nuestra
  función de graficado. La parte elegante de esto es que, dado que
  establecimos \texttt{mfrow\ =\ c(2,\ 3)}, R se encarga de
  \emph{distribuir} los gráficos en el dispositivo.
\item
  \texttt{par(oldpar)} Esta línea nos permite restaurar los parámetros
  de graficado.
\end{itemize}

\section{Pruebas estadísticas}\label{pruebas-estaduxedsticas}

\subsection{¿Está correlacionado el número de nominación con el
indegree?}\label{estuxe1-correlacionado-el-nuxfamero-de-nominaciuxf3n-con-el-indegree}

Hipótesis: Los individuos que, en promedio, están entre las primeras
nominaciones de sus pares son más populares

\begin{Shaded}
\begin{Highlighting}[]
\CommentTok{\# Obteniendo todos los datos en formato largo}
\NormalTok{edgelist }\OtherTok{\textless{}{-}} \FunctionTok{as\_long\_data\_frame}\NormalTok{(ig\_year1) }\SpecialCharTok{|\textgreater{}}
  \FunctionTok{as\_tibble}\NormalTok{()}

\CommentTok{\# Calculando indegree (de nuevo) y número promedio de nominación}
\CommentTok{\# Incluir "En una escala del uno al cinco qué tan cerca te sientes"}
\CommentTok{\# También para amigos egocéntricos (A. Amigos)}
\NormalTok{indeg\_nom\_cor }\OtherTok{\textless{}{-}} \FunctionTok{group\_by}\NormalTok{(edgelist, to, to\_name, to\_school) }\SpecialCharTok{|\textgreater{}}
  \FunctionTok{summarise}\NormalTok{(}
    \AttributeTok{indeg   =} \FunctionTok{length}\NormalTok{(nnom),}
    \AttributeTok{nom\_avg =} \DecValTok{1}\SpecialCharTok{/}\FunctionTok{mean}\NormalTok{(nnom)}
\NormalTok{  ) }\SpecialCharTok{|\textgreater{}}
  \FunctionTok{rename}\NormalTok{(}
    \AttributeTok{school =}\NormalTok{ to\_school}
\NormalTok{  )}
\end{Highlighting}
\end{Shaded}

\begin{verbatim}
`summarise()` has grouped output by 'to', 'to_name'. You can override using the
`.groups` argument.
\end{verbatim}

\begin{Shaded}
\begin{Highlighting}[]
\NormalTok{indeg\_nom\_cor}
\end{Highlighting}
\end{Shaded}

\begin{verbatim}
# A tibble: 1,561 x 5
# Groups:   to, to_name [1,561]
      to to_name school indeg nom_avg
   <dbl> <chr>    <int> <int>   <dbl>
 1     2 1110002    111    22   0.222
 2     3 1110007    111     7   0.175
 3     4 1110013    111     6   0.171
 4     5 1110014    111    19   0.134
 5     6 1110015    111     3   0.15 
 6     7 1110020    111     6   0.154
 7     9 1110025    111     6   0.214
 8    10 1110027    111    13   0.220
 9    11 1110029    111    14   0.131
10    12 1110030    111     6   0.222
# i 1,551 more rows
\end{verbatim}

\begin{Shaded}
\begin{Highlighting}[]
\CommentTok{\# Usando correlación de Pearson}
\FunctionTok{with}\NormalTok{(indeg\_nom\_cor, }\FunctionTok{cor.test}\NormalTok{(indeg, nom\_avg))}
\end{Highlighting}
\end{Shaded}

\begin{verbatim}

    Pearson's product-moment correlation

data:  indeg and nom_avg
t = -12.254, df = 1559, p-value < 2.2e-16
alternative hypothesis: true correlation is not equal to 0
95 percent confidence interval:
 -0.3409964 -0.2504653
sample estimates:
       cor 
-0.2963965 
\end{verbatim}

\begin{Shaded}
\begin{Highlighting}[]
\FunctionTok{save.image}\NormalTok{(}\StringTok{"03.rda"}\NormalTok{)}
\end{Highlighting}
\end{Shaded}

\chapter{TBD}\label{tbd}

\begin{tcolorbox}[enhanced jigsaw, colback=white, opacityback=0, coltitle=black, title=\textcolor{quarto-callout-warning-color}{\faExclamationTriangle}\hspace{0.5em}{Nota de Traducción}, bottomrule=.15mm, colbacktitle=quarto-callout-warning-color!10!white, toptitle=1mm, colframe=quarto-callout-warning-color-frame, titlerule=0mm, rightrule=.15mm, leftrule=.75mm, breakable, bottomtitle=1mm, left=2mm, arc=.35mm, toprule=.15mm, opacitybacktitle=0.6]

Esta versión del capítulo fue traducida de manera automática utilizando
IA. El capítulo aún no ha sido revisado por un humano.

\end{tcolorbox}

\emph{Este capítulo está pendiente de traducción.}

\chapter{Redes egocéntricas}\label{redes-egocuxe9ntricas}

\begin{tcolorbox}[enhanced jigsaw, colback=white, opacityback=0, coltitle=black, title=\textcolor{quarto-callout-warning-color}{\faExclamationTriangle}\hspace{0.5em}{Nota de Traducción}, bottomrule=.15mm, colbacktitle=quarto-callout-warning-color!10!white, toptitle=1mm, colframe=quarto-callout-warning-color-frame, titlerule=0mm, rightrule=.15mm, leftrule=.75mm, breakable, bottomtitle=1mm, left=2mm, arc=.35mm, toprule=.15mm, opacitybacktitle=0.6]

Esta versión del capítulo fue traducida de manera automática utilizando
IA. El capítulo aún no ha sido revisado por un humano.

\end{tcolorbox}

En el análisis de redes sociales egocéntricas (ESNA, para nuestro
libro,) en lugar de tratar con una sola red, tenemos tantas redes como
participantes en el estudio. Los egos--los sujetos principales del
estudio--se analizan desde la perspectiva de su red social local. Para
una vista más extendida de ESNA, revisa ``\emph{Análisis de redes
egocéntricas con R}'' de Raffaele Vacca.

En este capítulo, muestro cómo trabajar con un tipo particular de datos
ESNA: información generada por la herramienta Network Canvas. Puedes
descargar un archivo ZIP ``artificial'' que contiene las salidas de un
proyecto de Network Canvas
\href{data-raw/networkCanvasExport-fake.zip}{aquí}\footnote{Agradezco a
  \href{https://scholar.google.com/citations?user=Uht4YbkAAAAJ}{Jacqueline
  M. Kent-Marvick}, quien me proporcionó lo que usé como línea base para
  generar la exportación artificial de Network Canvas.}. Asumimos que el
archivo ZIP fue extraído a la carpeta \texttt{data-raw/egonets}. Puedes
proceder y extraer el ZIP por punto y clic o usar el siguiente código R
para automatizar el proceso:

\begin{verbatim}
[1] FALSE
\end{verbatim}

\begin{Shaded}
\begin{Highlighting}[]
\FunctionTok{unzip}\NormalTok{(}
  \AttributeTok{zipfile =} \StringTok{"data{-}raw/networkCanvasExport{-}fake.zip"}\NormalTok{,}
  \AttributeTok{exdir   =} \StringTok{"data{-}raw/egonets"}
\NormalTok{  )}
\end{Highlighting}
\end{Shaded}

Esto extraerá todos los archivos en
\texttt{networkCanvasExport-fake.zip} a la subcarpeta \texttt{egonets}.
Echemos un vistazo a los primeros archivos:

\begin{Shaded}
\begin{Highlighting}[]
\FunctionTok{head}\NormalTok{(}\FunctionTok{list.files}\NormalTok{(}\AttributeTok{path =} \StringTok{"data{-}raw/egonets"}\NormalTok{))}
\DocumentationTok{\#\# [1] "I\_{-}59190\_BRB9111\_attributeList\_Person.csv"}
\DocumentationTok{\#\# [2] "I\_{-}59190\_BRB9111\_edgeList\_Knows.csv"      }
\DocumentationTok{\#\# [3] "I\_{-}59190\_BRB9111\_ego.csv"                 }
\DocumentationTok{\#\# [4] "I\_{-}59190\_BRB9111.graphml"                 }
\DocumentationTok{\#\# [5] "I{-}100BB\_00B95{-}90\_attributeList\_Person.csv"}
\DocumentationTok{\#\# [6] "I{-}100BB\_00B95{-}90\_edgeList\_Knows.csv"}
\end{Highlighting}
\end{Shaded}

Como puedes ver, para cada ego en el conjunto de datos, hay cuatro
archivos:

\begin{itemize}
\item
  \texttt{...attributeList\_Person.csv}: Atributos de los alters.
\item
  \texttt{...edgeList\_Knows.csv}: Lista de enlaces indicando los
  vínculos entre los alters.
\item
  \texttt{...ego.csv}: Información sobre los egos.
\item
  \texttt{...graphml}: Y un archivo \texttt{graphml} que contiene las
  redes egocéntricas.
\end{itemize}

Las siguientes secciones ilustrarán, archivo por archivo, cómo leer la
información en R, aplicar cualquier procesamiento requerido, y almacenar
la información para uso posterior. Comenzamos con los archivos
\texttt{graphml}.

\section{Archivos de red (graphml)}\label{archivos-de-red-graphml}

Los archivos \texttt{graphml} pueden leerse directamente con la función
\texttt{read\_graph} de \texttt{igraph}. La clave es aprovechar las
listas de R para evitar escribir una y otra vez el mismo bloque de
código, y, en su lugar, manejar los datos a través de listas.

Al igual que cualquier función de lectura de datos, la función
\texttt{read\_graph} requiere una ruta de archivo al archivo de red.
\textbf{La función que usaremos para listar los archivos requeridos es
\texttt{list.files()}}:

\begin{Shaded}
\begin{Highlighting}[]
\CommentTok{\# Comenzamos cargando igraph}
\FunctionTok{library}\NormalTok{(igraph)}

\CommentTok{\# Listando todos los archivos graphml}
\NormalTok{graph\_files }\OtherTok{\textless{}{-}} \FunctionTok{list.files}\NormalTok{(}
  \AttributeTok{path       =} \StringTok{"data{-}raw/egonets"}\NormalTok{, }\CommentTok{\# ¿Dónde están estos archivos?}
  \AttributeTok{pattern    =} \StringTok{"*.graphml"}\NormalTok{,        }\CommentTok{\# Especificar un patrón para solo listar graphml}
  \AttributeTok{full.names =} \ConstantTok{TRUE}                \CommentTok{\# Y nos aseguramos de usar el nombre completo}
                                   \CommentTok{\# (ruta.) De lo contrario, solo obtendríamos nombres.}
\NormalTok{  )}

\CommentTok{\# Echando un vistazo a los primeros tres archivos que obtuvimos}
\NormalTok{graph\_files[}\DecValTok{1}\SpecialCharTok{:}\DecValTok{3}\NormalTok{]}
\DocumentationTok{\#\# [1] "data{-}raw/egonets/I\_{-}59190\_BRB9111.graphml"}
\DocumentationTok{\#\# [2] "data{-}raw/egonets/I{-}100BB\_00B95{-}90.graphml"}
\DocumentationTok{\#\# [3] "data{-}raw/egonets/I{-}1BB79950{-}0{-}7.graphml"}

\CommentTok{\# Aplicando read\_graph de igraph}
\NormalTok{graphs }\OtherTok{\textless{}{-}} \FunctionTok{lapply}\NormalTok{(}
  \AttributeTok{X      =}\NormalTok{ graph\_files,       }\CommentTok{\# Lista de archivos a leer}
  \AttributeTok{FUN    =}\NormalTok{ read\_graph,        }\CommentTok{\# La función a aplicar}
  \AttributeTok{format =} \StringTok{"graphml"}          \CommentTok{\# Argumento pasado a read\_graph}
\NormalTok{  )}
\end{Highlighting}
\end{Shaded}

Si la operación tuvo éxito, el bloque de código anterior debería generar
una lista de objetos \texttt{igraph} llamada \texttt{graphs}. Echemos un
vistazo a los primeros dos:

\begin{Shaded}
\begin{Highlighting}[]
\NormalTok{graphs[[}\DecValTok{1}\NormalTok{]]}
\DocumentationTok{\#\# IGRAPH 6cfea10 U{-}{-}{-} 12 25 {-}{-} }
\DocumentationTok{\#\# + attr: age (v/n), healthy\_diet (v/n), gender\_1 (v/l), eat\_with\_2}
\DocumentationTok{\#\# | (v/l), id (v/c)}
\DocumentationTok{\#\# + edges from 6cfea10:}
\DocumentationTok{\#\#  [1] 1{-}{-} 3 1{-}{-} 2 1{-}{-} 6 1{-}{-} 5 1{-}{-} 4 1{-}{-} 8 1{-}{-}11 1{-}{-}10 2{-}{-} 3 3{-}{-} 7 3{-}{-} 4 3{-}{-} 5}
\DocumentationTok{\#\# [13] 3{-}{-} 6 2{-}{-} 7 2{-}{-} 4 2{-}{-} 5 2{-}{-} 6 5{-}{-} 6 6{-}{-}10 7{-}{-} 9 4{-}{-} 5 5{-}{-} 7 4{-}{-}11 6{-}{-} 7}
\DocumentationTok{\#\# [25] 4{-}{-} 7}
\NormalTok{graphs[[}\DecValTok{2}\NormalTok{]]}
\DocumentationTok{\#\# IGRAPH 2a65112 U{-}{-}{-} 16 47 {-}{-} }
\DocumentationTok{\#\# + attr: age (v/n), healthy\_diet (v/n), gender\_1 (v/l), eat\_with\_2}
\DocumentationTok{\#\# | (v/l), id (v/c)}
\DocumentationTok{\#\# + edges from 2a65112:}
\DocumentationTok{\#\#  [1]  7{-}{-}13  1{-}{-} 5  1{-}{-} 6  1{-}{-} 4  1{-}{-} 2  7{-}{-}15  1{-}{-} 3 11{-}{-}13  1{-}{-}10  1{-}{-}16}
\DocumentationTok{\#\# [11]  4{-}{-} 6  2{-}{-} 6  6{-}{-} 7  1{-}{-}11 11{-}{-}15  6{-}{-} 9  6{-}{-} 8  3{-}{-} 9  5{-}{-}15  4{-}{-} 5}
\DocumentationTok{\#\# [21]  2{-}{-} 5  5{-}{-} 8  5{-}{-} 7  5{-}{-}10  3{-}{-} 5  6{-}{-}14 12{-}{-}13  6{-}{-}13  3{-}{-}13  2{-}{-} 3}
\DocumentationTok{\#\# [31]  3{-}{-} 4  3{-}{-}16  3{-}{-}11 10{-}{-}14  7{-}{-}14  2{-}{-} 4  2{-}{-}10  2{-}{-}15 10{-}{-}12  4{-}{-} 7}
\DocumentationTok{\#\# [41]  6{-}{-}10  5{-}{-}11  9{-}{-}10  1{-}{-} 9  1{-}{-}12  3{-}{-}12  4{-}{-}14}
\end{Highlighting}
\end{Shaded}

Como siempre, una de las primeras cosas que hacemos con redes es
visualizarlas. Usaremos el paquete de R \texttt{netplot} (de un
servidor) para dibujar las figuras:

\begin{Shaded}
\begin{Highlighting}[]
\FunctionTok{library}\NormalTok{(netplot)}
\FunctionTok{library}\NormalTok{(gridExtra)}

\CommentTok{\# El diseño del grafo es aleatorio}
\FunctionTok{set.seed}\NormalTok{(}\DecValTok{1231}\NormalTok{)}

\CommentTok{\# grid.arrange permite poner múltiples gráficos netplot en la misma página}
\FunctionTok{grid.arrange}\NormalTok{(}
  \FunctionTok{nplot}\NormalTok{(graphs[[}\DecValTok{1}\NormalTok{]]),}
  \FunctionTok{nplot}\NormalTok{(graphs[[}\DecValTok{2}\NormalTok{]]),}
  \FunctionTok{nplot}\NormalTok{(graphs[[}\DecValTok{3}\NormalTok{]]),}
  \FunctionTok{nplot}\NormalTok{(graphs[[}\DecValTok{4}\NormalTok{]]),}
  \AttributeTok{ncol =} \DecValTok{2}\NormalTok{, }\AttributeTok{nrow =} \DecValTok{2}
\NormalTok{)}
\end{Highlighting}
\end{Shaded}

\pandocbounded{\includegraphics[keepaspectratio]{part-01-07-egonets_files/figure-pdf/plot-nets-1.pdf}}

¡Excelente! Dado que los nodos en nuestra red tienen características,
podemos agregar un poco de color. Usaremos la variable
\texttt{eat\_with\_2}, codificada como \texttt{TRUE} o \texttt{FALSE}.
Los colores de los vértices pueden especificarse usando el argumento
\texttt{vertex.color} de la función \texttt{nplot}. En nuestro caso,
especificaremos colores pasando un vector con longitud igual al número
de nodos en el grafo. Además, dado que haremos esto múltiples veces,
vale la pena escribir una función:

\begin{Shaded}
\begin{Highlighting}[]
\CommentTok{\# Una función para colorear por la variable come con}
\NormalTok{color\_it }\OtherTok{\textless{}{-}} \ControlFlowTok{function}\NormalTok{(net) \{}

  \CommentTok{\# Codificando eat\_with\_2 para ser 1 (FALSE) o 2 (TRUE)}
\NormalTok{  eatswith }\OtherTok{\textless{}{-}} \FunctionTok{V}\NormalTok{(net)}\SpecialCharTok{$}\NormalTok{eat\_with\_2}

  \CommentTok{\# Subconjuntando el color}
  \FunctionTok{ifelse}\NormalTok{(eatswith, }\StringTok{"purple"}\NormalTok{, }\StringTok{"darkgreen"}\NormalTok{)}

\NormalTok{\}}
\end{Highlighting}
\end{Shaded}

Esta función toma dos argumentos: una red y un vector de dos colores.
Los atributos de vértice en \texttt{igraph} pueden accederse a través de
la función \texttt{V(...)\$...}. Para este ejemplo, para acceder al
atributo \texttt{eat\_with\_2} en la red \texttt{net}, escribimos
\texttt{V(net)\$eat\_with\_2}. Finalmente, individuos con
\texttt{eat\_with\_2} igual a true serán coloreados \texttt{purple}; de
lo contrario, si es igual a \texttt{FALSE}, serán coloreados
\texttt{darkgreen}. Antes de graficar las redes, veamos qué obtenemos
cuando accedemos al atributo \texttt{eat\_with\_2} en el primer grafo:

\begin{Shaded}
\begin{Highlighting}[]
\FunctionTok{V}\NormalTok{(graphs[[}\DecValTok{1}\NormalTok{]])}\SpecialCharTok{$}\NormalTok{eat\_with\_2}
\DocumentationTok{\#\#  [1]  TRUE  TRUE FALSE FALSE  TRUE  TRUE FALSE FALSE  TRUE  TRUE FALSE FALSE}
\end{Highlighting}
\end{Shaded}

Un vector lógico. Ahora redibujemos las figuras:

\begin{Shaded}
\begin{Highlighting}[]
\FunctionTok{grid.arrange}\NormalTok{(}
  \FunctionTok{nplot}\NormalTok{(graphs[[}\DecValTok{1}\NormalTok{]], }\AttributeTok{vertex.color =} \FunctionTok{color\_it}\NormalTok{(graphs[[}\DecValTok{1}\NormalTok{]])),}
  \FunctionTok{nplot}\NormalTok{(graphs[[}\DecValTok{2}\NormalTok{]], }\AttributeTok{vertex.color =} \FunctionTok{color\_it}\NormalTok{(graphs[[}\DecValTok{2}\NormalTok{]])),}
  \FunctionTok{nplot}\NormalTok{(graphs[[}\DecValTok{3}\NormalTok{]], }\AttributeTok{vertex.color =} \FunctionTok{color\_it}\NormalTok{(graphs[[}\DecValTok{3}\NormalTok{]])),}
  \FunctionTok{nplot}\NormalTok{(graphs[[}\DecValTok{4}\NormalTok{]], }\AttributeTok{vertex.color =} \FunctionTok{color\_it}\NormalTok{(graphs[[}\DecValTok{4}\NormalTok{]])),}
  \AttributeTok{ncol =} \DecValTok{2}\NormalTok{, }\AttributeTok{nrow =} \DecValTok{2}
\NormalTok{)}
\end{Highlighting}
\end{Shaded}

\pandocbounded{\includegraphics[keepaspectratio]{part-01-07-egonets_files/figure-pdf/part-01-07-plot-nets-colored-1.pdf}}

Dado que la mayoría del tiempo, estaremos tratando con muchas redes
egocéntricas; puedes querer dibujar cada red independientemente; el
siguiente bloque de código hace eso. Primero, si es necesario, creará
una carpeta para almacenar las redes. Luego, usando la función
\texttt{lapply}, usará \texttt{netplot::nplot()} para dibujar las redes,
agregar una leyenda, y guardar el grafo como
\texttt{.../graphml\_{[}número{]}.png}, donde \texttt{{[}número{]}} irá
de \texttt{01} al número total de redes en \texttt{graphs}.

\begin{Shaded}
\begin{Highlighting}[]
\ControlFlowTok{if}\NormalTok{ (}\SpecialCharTok{!}\FunctionTok{dir.exists}\NormalTok{(}\StringTok{"egonets/figs/egonets"}\NormalTok{))}
  \FunctionTok{dir.create}\NormalTok{(}\StringTok{"egonets/figs/egonets"}\NormalTok{, }\AttributeTok{recursive =} \ConstantTok{TRUE}\NormalTok{)}

\FunctionTok{lapply}\NormalTok{(}\FunctionTok{seq\_along}\NormalTok{(graphs), }\ControlFlowTok{function}\NormalTok{(i) \{}
  
  \CommentTok{\# Creando el dispositivo }
  \FunctionTok{png}\NormalTok{(}\FunctionTok{sprintf}\NormalTok{(}\StringTok{"egonets/figs/egonets/graphml\_\%02i.png"}\NormalTok{, i))  }
  
  \CommentTok{\# Dibujando el gráfico}
\NormalTok{  p }\OtherTok{\textless{}{-}} \FunctionTok{nplot}\NormalTok{(}
\NormalTok{    graphs[[i]],}
    \AttributeTok{vertex.color =} \FunctionTok{color\_it}\NormalTok{(graphs[[i]])}
\NormalTok{    )}
  
  \CommentTok{\# Agregando una leyenda}
\NormalTok{  p }\OtherTok{\textless{}{-}} \FunctionTok{nplot\_legend}\NormalTok{(}
\NormalTok{    p,}
    \AttributeTok{labels =} \FunctionTok{c}\NormalTok{(}\StringTok{"come con: FALSE"}\NormalTok{, }\StringTok{"come con: TRUE"}\NormalTok{),}
    \AttributeTok{pch    =} \DecValTok{21}\NormalTok{,}
    \AttributeTok{packgrob.args =} \FunctionTok{list}\NormalTok{(}\AttributeTok{side =} \StringTok{"bottom"}\NormalTok{),}
    \AttributeTok{gp            =} \FunctionTok{gpar}\NormalTok{(}
      \AttributeTok{fill =} \FunctionTok{c}\NormalTok{(}\StringTok{"darkgreen"}\NormalTok{, }\StringTok{"purple"}\NormalTok{)}
\NormalTok{    ),}
    \AttributeTok{ncol =} \DecValTok{2}
\NormalTok{  )}
  
  \FunctionTok{print}\NormalTok{(p)}
  
  \CommentTok{\# Cerrando el dispositivo}
  \FunctionTok{dev.off}\NormalTok{()}
\NormalTok{\})}
\end{Highlighting}
\end{Shaded}

\section{Archivos de persona}\label{archivos-de-persona}

Como antes, listamos los archivos que terminan en \texttt{Person.csv}
(con la ruta completa,) y los leemos en R. Aunque R tiene la función
\texttt{read.csv}, aquí uso la función \texttt{fread} del paquete de R
\texttt{data.table}. Junto con \texttt{dplyr}, \texttt{data.table} es
una de las herramientas de manipulación de datos más populares en R.
Además de la sintaxis, la mayor diferencia entre las dos es el
rendimiento; \texttt{data.table} es significativamente más rápido que
cualquier otro paquete de manejo de datos en R, y es una gran
alternativa para manejar grandes conjuntos de datos. El siguiente bloque
de código carga el paquete, lista los archivos, y los lee en R.

\begin{Shaded}
\begin{Highlighting}[]
\CommentTok{\# Cargando data.table}
\FunctionTok{library}\NormalTok{(data.table)}

\CommentTok{\# Listando los archivos}
\NormalTok{person\_files }\OtherTok{\textless{}{-}} \FunctionTok{list.files}\NormalTok{(}
  \AttributeTok{path       =} \StringTok{"data{-}raw/egonets"}\NormalTok{,}
  \AttributeTok{pattern    =} \StringTok{"*Person.csv"}\NormalTok{,}
  \AttributeTok{full.names =} \ConstantTok{TRUE}
\NormalTok{  )}

\CommentTok{\# Cargando todos en una sola lista}
\NormalTok{persons }\OtherTok{\textless{}{-}} \FunctionTok{lapply}\NormalTok{(person\_files, fread)}

\CommentTok{\# Mirando el primer elemento}
\NormalTok{persons[[}\DecValTok{1}\NormalTok{]]}
\DocumentationTok{\#\#     nodeID   age}
\DocumentationTok{\#\#      \textless{}int\textgreater{} \textless{}int\textgreater{}}
\DocumentationTok{\#\#  1:      1    45}
\DocumentationTok{\#\#  2:      2    32}
\DocumentationTok{\#\#  3:      3    31}
\DocumentationTok{\#\#  4:      4    45}
\DocumentationTok{\#\#  5:      5    43}
\DocumentationTok{\#\#  6:      6    47}
\DocumentationTok{\#\#  7:      7    45}
\DocumentationTok{\#\#  8:      8    62}
\DocumentationTok{\#\#  9:      9    28}
\DocumentationTok{\#\# 10:     10    41}
\DocumentationTok{\#\# 11:     11    41}
\DocumentationTok{\#\# 12:     12    46}
\DocumentationTok{\#\# 13:     13    46}
\DocumentationTok{\#\# 14:     14    46}
\DocumentationTok{\#\# 15:     15    62}
\DocumentationTok{\#\# 16:     16    41}
\end{Highlighting}
\end{Shaded}

Una tarea común es agregar un identificador a cada conjunto de datos en
\texttt{persons} para que sepamos a qué ego pertenecen. De nuevo, la
función \texttt{lapply} es nuestra amiga:

\begin{Shaded}
\begin{Highlighting}[]
\NormalTok{persons }\OtherTok{\textless{}{-}} \FunctionTok{lapply}\NormalTok{(}\FunctionTok{seq\_along}\NormalTok{(persons), }\ControlFlowTok{function}\NormalTok{(i) \{}
\NormalTok{  persons[[i]][, dataset\_num }\SpecialCharTok{:}\ErrorTok{=}\NormalTok{ i]}
\NormalTok{\})}
\end{Highlighting}
\end{Shaded}

En \texttt{data.table}, las variables se crean usando el símbolo
\texttt{:=}. El fragmento de código anterior es equivalente a esto:

\begin{Shaded}
\begin{Highlighting}[]
\ControlFlowTok{for}\NormalTok{ (i }\ControlFlowTok{in} \DecValTok{1}\SpecialCharTok{:}\FunctionTok{length}\NormalTok{(persons)) \{}
\NormalTok{  persons[[i]]}\SpecialCharTok{$}\NormalTok{dataset\_num }\OtherTok{\textless{}{-}}\NormalTok{ i}
\NormalTok{\}}
\end{Highlighting}
\end{Shaded}

Si es necesario, podemos transformar la lista \texttt{persons} en un
objeto \texttt{data.table} (es decir, un solo \texttt{data.frame})
usando la función \texttt{rbindlist}\footnote{Aunque no es lo mismo,
  \texttt{rbindlist} (casi siempre) produce el mismo resultado que
  llamar la función \texttt{do.call}. En particular, en lugar de
  ejecutar la llamada \texttt{rbindlist(persons)}, podríamos haber usado
  \texttt{do.call(rbind,\ persons)}.}. El siguiente bloque de código usa
esa función para combinar los \texttt{data.table}s en un solo conjunto
de datos.

\begin{Shaded}
\begin{Highlighting}[]
\CommentTok{\# Combinando los conjuntos de datos}
\NormalTok{persons }\OtherTok{\textless{}{-}} \FunctionTok{rbindlist}\NormalTok{(persons)}
\NormalTok{persons}
\DocumentationTok{\#\#      nodeID   age dataset\_num}
\DocumentationTok{\#\#       \textless{}int\textgreater{} \textless{}int\textgreater{}       \textless{}int\textgreater{}}
\DocumentationTok{\#\#   1:      1    45           1}
\DocumentationTok{\#\#   2:      2    32           1}
\DocumentationTok{\#\#   3:      3    31           1}
\DocumentationTok{\#\#   4:      4    45           1}
\DocumentationTok{\#\#   5:      5    43           1}
\DocumentationTok{\#\#  {-}{-}{-}                         }
\DocumentationTok{\#\# 271:      7    43          19}
\DocumentationTok{\#\# 272:      8    48          19}
\DocumentationTok{\#\# 273:      9    70          19}
\DocumentationTok{\#\# 274:     10    46          19}
\DocumentationTok{\#\# 275:     11    50          19}
\end{Highlighting}
\end{Shaded}

Ahora que tenemos un solo conjunto de datos, podemos hacer algo de
exploración de datos. Por ejemplo, podemos usar el paquete
\texttt{ggplot2} para dibujar un histograma de las edades de los alters.

\begin{Shaded}
\begin{Highlighting}[]
\CommentTok{\# Cargando el paquete ggplot2}
\FunctionTok{library}\NormalTok{(ggplot2)}

\CommentTok{\# Histograma de edad}
\FunctionTok{ggplot}\NormalTok{(persons, }\FunctionTok{aes}\NormalTok{(}\AttributeTok{x =}\NormalTok{ age)) }\SpecialCharTok{+}            \CommentTok{\# Iniciando el gráfico}
  \FunctionTok{geom\_histogram}\NormalTok{(}\AttributeTok{fill =} \StringTok{"purple"}\NormalTok{) }\SpecialCharTok{+}      \CommentTok{\# Agregando un histograma}
  \FunctionTok{labs}\NormalTok{(}\AttributeTok{x =} \StringTok{"Edad"}\NormalTok{, }\AttributeTok{y =} \StringTok{"Frecuencia"}\NormalTok{) }\SpecialCharTok{+}       \CommentTok{\# Cambiando las etiquetas del eje x/y}
  \FunctionTok{labs}\NormalTok{(}\AttributeTok{title =} \StringTok{"Distribución de Edad de Alters"}\NormalTok{) }\CommentTok{\# Agregando un título}
\DocumentationTok{\#\# \textasciigrave{}stat\_bin()\textasciigrave{} using \textasciigrave{}bins = 30\textasciigrave{}. Pick better value with \textasciigrave{}binwidth\textasciigrave{}.}
\end{Highlighting}
\end{Shaded}

\pandocbounded{\includegraphics[keepaspectratio]{part-01-07-egonets_files/figure-pdf/part-01-07-ggplot-ages-1.pdf}}

\section{Archivos de ego}\label{archivos-de-ego}

Los archivos de ego contienen información sobre los egos (¡obvio!.) De
nuevo, los leeremos todos a la vez usando \texttt{list.files} +
\texttt{lapply}:

\begin{Shaded}
\begin{Highlighting}[]
\CommentTok{\# Listando archivos que terminan con *ego.csv}
\NormalTok{ego\_files }\OtherTok{\textless{}{-}} \FunctionTok{list.files}\NormalTok{(}
  \AttributeTok{path       =} \StringTok{"data{-}raw/egonets"}\NormalTok{,}
  \AttributeTok{pattern    =} \StringTok{"*ego.csv"}\NormalTok{,}
  \AttributeTok{full.names =} \ConstantTok{TRUE}
\NormalTok{  )}

\CommentTok{\# Leyendo los archivos con fread}
\NormalTok{egos }\OtherTok{\textless{}{-}} \FunctionTok{lapply}\NormalTok{(ego\_files, fread)}

\CommentTok{\# Combinándolos}
\NormalTok{egos }\OtherTok{\textless{}{-}} \FunctionTok{rbindlist}\NormalTok{(egos)}
\FunctionTok{head}\NormalTok{(egos)}
\DocumentationTok{\#\#                    networkCanvasEgoUUID networkCanvasCaseID}
\DocumentationTok{\#\#                                  \textless{}char\textgreater{}              \textless{}char\textgreater{}}
\DocumentationTok{\#\# 1: I{-}11ca3a78c{-}62f131f37169{-}c139217a1f6    I\_{-}59190\_BRB9111}
\DocumentationTok{\#\# 2: I{-}fef{-}ab{-}4{-}5a{-}{-}7{-}35c4f23{-}96eb32{-}34ea    I{-}100BB\_00B95{-}90}
\DocumentationTok{\#\# 3: I2f1bd0b6d{-}f71f4664cf{-}d{-}26{-}97408f22d      I{-}1BB79950{-}0{-}7}
\DocumentationTok{\#\# 4: Id36bb{-}3b2bcbd2a6239b1103134c6b3d1d6    I000091I\_RB010B5}
\DocumentationTok{\#\# 5: I436d32fc67fb5c6{-}23{-}244f353849b120cd    I019051R0\_RRR0{-}0}
\DocumentationTok{\#\# 6: Ibf1f{-}2{-}34162bb5f2c36b8241{-}{-}316a{-}fff    I01B11{-}I1101\_44R}
\DocumentationTok{\#\#                  networkCanvasSessionID}
\DocumentationTok{\#\#                                  \textless{}char\textgreater{}}
\DocumentationTok{\#\# 1: I612b7a1af{-}{-}{-}0880b{-}70698204{-}b{-}8dbf09}
\DocumentationTok{\#\# 2: If5e0{-}f{-}26cbec070760f{-}e6b6d26ebfb06f}
\DocumentationTok{\#\# 3: I825c293a1304{-}e5{-}cbea8a80aae05b305fa}
\DocumentationTok{\#\# 4: I1b8a7d0f6b4{-}8298c9{-}848{-}9186d68a7f3c}
\DocumentationTok{\#\# 5: Ie620be37b75983c49ac63{-}38{-}425227c959}
\DocumentationTok{\#\# 6: Ie3{-}134323ed40{-}0e{-}d954b3d{-}febbcb9363}
\DocumentationTok{\#\#                       networkCanvasProtocolName        sessionStart}
\DocumentationTok{\#\#                                          \textless{}char\textgreater{}              \textless{}POSc\textgreater{}}
\DocumentationTok{\#\# 1: Postpartum social networks with sociogram\_V5 2023{-}02{-}22 23:41:59}
\DocumentationTok{\#\# 2: Postpartum social networks with sociogram\_V5 2023{-}02{-}10 21:46:02}
\DocumentationTok{\#\# 3: Postpartum social networks with sociogram\_V5 2023{-}03{-}01 16:52:09}
\DocumentationTok{\#\# 4: Postpartum social networks with sociogram\_V5 2023{-}01{-}26 20:38:07}
\DocumentationTok{\#\# 5: Postpartum social networks with sociogram\_V5 2023{-}02{-}06 14:55:57}
\DocumentationTok{\#\# 6: Postpartum social networks with sociogram\_V5 2023{-}03{-}16 18:20:02}
\DocumentationTok{\#\#          sessionFinish     sessionExported}
\DocumentationTok{\#\#                 \textless{}POSc\textgreater{}              \textless{}POSc\textgreater{}}
\DocumentationTok{\#\# 1: 2023{-}02{-}23 01:47:00 2023{-}02{-}23 01:47:08}
\DocumentationTok{\#\# 2: 2023{-}02{-}11 01:29:32 2023{-}02{-}11 01:34:12}
\DocumentationTok{\#\# 3: 2023{-}03{-}02 16:51:20 2023{-}03{-}02 17:04:42}
\DocumentationTok{\#\# 4: 2023{-}01{-}26 22:03:20 2023{-}01{-}26 22:03:34}
\DocumentationTok{\#\# 5: 2023{-}02{-}06 15:49:38 2023{-}02{-}06 15:56:42}
\DocumentationTok{\#\# 6: 2023{-}03{-}17 21:11:09 2023{-}03{-}17 21:16:15}
\end{Highlighting}
\end{Shaded}

Algo genial sobre \texttt{data.table} es que, dentro de corchetes
cuadrados, podemos manipular los datos refiriéndonos a las variables
directamente. Por ejemplo, si quisiéramos calcular la diferencia entre
\texttt{sessionFinish} y \texttt{sessionStart}, usando R base haríamos
lo siguiente:

\begin{Shaded}
\begin{Highlighting}[]
\NormalTok{egos}\SpecialCharTok{$}\NormalTok{total\_time }\OtherTok{\textless{}{-}}\NormalTok{ egos}\SpecialCharTok{$}\NormalTok{sessionFinish }\SpecialCharTok{{-}}\NormalTok{ egos}\SpecialCharTok{$}\NormalTok{sessionStart}
\end{Highlighting}
\end{Shaded}

Mientras que con \texttt{data.table}, la creación de variables es mucho
más directa (nota que en lugar de usar \texttt{\textless{}-} o
\texttt{=} para asignar una variable, usamos el operador \texttt{:=}):

\begin{Shaded}
\begin{Highlighting}[]
\CommentTok{\# ¿Cuánto tiempo?}
\NormalTok{egos[, total\_time }\SpecialCharTok{:}\ErrorTok{=}\NormalTok{ sessionFinish }\SpecialCharTok{{-}}\NormalTok{ sessionStart]}
\end{Highlighting}
\end{Shaded}

También podemos visualizar esto usando \texttt{ggplot2}:

\begin{Shaded}
\begin{Highlighting}[]
\FunctionTok{ggplot}\NormalTok{(egos, }\FunctionTok{aes}\NormalTok{(}\AttributeTok{x =}\NormalTok{ total\_time)) }\SpecialCharTok{+}
  \FunctionTok{geom\_histogram}\NormalTok{() }\SpecialCharTok{+}
  \FunctionTok{labs}\NormalTok{(}\AttributeTok{x =} \StringTok{"Tiempo en minutos"}\NormalTok{, }\AttributeTok{y =} \StringTok{"Conteo"}\NormalTok{) }\SpecialCharTok{+}
  \FunctionTok{labs}\NormalTok{(}\AttributeTok{title =} \StringTok{"Tiempo total gastado por egos"}\NormalTok{)}
\DocumentationTok{\#\# Don\textquotesingle{}t know how to automatically pick scale for object of type \textless{}difftime\textgreater{}.}
\DocumentationTok{\#\# Defaulting to continuous.}
\DocumentationTok{\#\# \textasciigrave{}stat\_bin()\textasciigrave{} using \textasciigrave{}bins = 30\textasciigrave{}. Pick better value with \textasciigrave{}binwidth\textasciigrave{}.}
\end{Highlighting}
\end{Shaded}

\pandocbounded{\includegraphics[keepaspectratio]{part-01-07-egonets_files/figure-pdf/egos-time-plot-1.pdf}}

\section{Archivos de lista de
enlaces}\label{archivos-de-lista-de-enlaces}

Como mencioné antes, dado que estamos leyendo los archivos
\texttt{graphml}, usar la lista de enlaces puede no ser necesario. Sin
embargo, el proceso para importar el archivo de lista de enlaces a R es
el mismo que hemos estado aplicando: listar los archivos y leerlos todos
a la vez usando \texttt{lapply}:

\begin{Shaded}
\begin{Highlighting}[]
\CommentTok{\# Listando todos los archivos que terminan en Knows.csv}
\NormalTok{edgelist\_files }\OtherTok{\textless{}{-}} \FunctionTok{list.files}\NormalTok{(}
  \AttributeTok{path =} \StringTok{"data{-}raw/egonets"}\NormalTok{,}
  \AttributeTok{pattern =} \StringTok{"*Knows.csv"}\NormalTok{,}
  \AttributeTok{full.names =} \ConstantTok{TRUE}
\NormalTok{  )}

\CommentTok{\# Leyendo todos los archivos a la vez}
\NormalTok{edgelists }\OtherTok{\textless{}{-}} \FunctionTok{lapply}\NormalTok{(edgelist\_files, fread)}
\end{Highlighting}
\end{Shaded}

Para evitar confusión, también podemos agregar ids correspondientes al
número de archivo. Una vez que hagamos eso, podemos combinar todos los
archivos en un solo objeto \texttt{data.table} usando
\texttt{rbindlist}:

\begin{Shaded}
\begin{Highlighting}[]
\NormalTok{edgelists }\OtherTok{\textless{}{-}} \FunctionTok{lapply}\NormalTok{(}\FunctionTok{seq\_along}\NormalTok{(edgelists), }\ControlFlowTok{function}\NormalTok{(i) \{}
\NormalTok{  edgelists[[i]][, dataset\_num }\SpecialCharTok{:}\ErrorTok{=}\NormalTok{ i]}
\NormalTok{\})}

\NormalTok{edgelists }\OtherTok{\textless{}{-}} \FunctionTok{rbindlist}\NormalTok{(edgelists)}

\FunctionTok{head}\NormalTok{(edgelists)}
\DocumentationTok{\#\#    edgeID  from    to                 networkCanvasEgoUUID}
\DocumentationTok{\#\#     \textless{}int\textgreater{} \textless{}int\textgreater{} \textless{}int\textgreater{}                               \textless{}char\textgreater{}}
\DocumentationTok{\#\# 1:      1     1     5 I839f{-}8fa8f8aeb8{-}eaf{-}{-}{-}ba8{-}cf3908f3a}
\DocumentationTok{\#\# 2:      2     1    10 If81a9c0f{-}9f4f28ccf{-}c4c923a{-}8{-}0f5fce}
\DocumentationTok{\#\# 3:      3     1     9 I899ffe{-}27{-}3{-}a3{-}ca2fb7f7{-}ca8e7715ce9}
\DocumentationTok{\#\# 4:      4     1    10 I814efaba88cbb02caa8c89790{-}83beeaf9{-}}
\DocumentationTok{\#\# 5:      5     7     6 Ifd{-}0eec2e08974eaf2b79f{-}9efb7e3{-}8998}
\DocumentationTok{\#\# 6:      6     2     6 I{-}28fe89cc{-}fc5db3825b92{-}ae87c{-}c18e3d}
\DocumentationTok{\#\#                       networkCanvasUUID              networkCanvasSourceUUID}
\DocumentationTok{\#\#                                  \textless{}char\textgreater{}                               \textless{}char\textgreater{}}
\DocumentationTok{\#\# 1: I720400eb19bccce{-}77cee773289b02{-}fe7e I4d5{-}{-}16a08f8ba463c6458f8979e{-}65fa9d}
\DocumentationTok{\#\# 2: I{-}b469c0{-}60f8bbb543{-}32{-}628{-}216f9{-}038 I{-}6cf8{-}f3da{-}4{-}96{-}87efaf5daaa48ba5e5c}
\DocumentationTok{\#\# 3: Ifa4933{-}9baaf5fc{-}f{-}e4f5c5e5{-}ff34{-}f{-}f I5{-}f69a6eaa{-}5956e8897ca999{-}ffb6ed{-}e1}
\DocumentationTok{\#\# 4: I4cb{-}904496b1{-}6194bcb51b58444b40{-}ef8 I3e5{-}6c8d5e0f086{-}{-}e{-}5ab45{-}4{-}5aaa5{-}0e}
\DocumentationTok{\#\# 5: I0ab7{-}{-}b7a0ee71e54c1e93cdb{-}4ca5ab1{-}b I5{-}b{-}9{-}7eca5ab5{-}91915ba9b6565a6e42cc}
\DocumentationTok{\#\# 6: Ic80142fc4c431009e84b3{-}ab3f{-}9b0eab03 Ie0a24eea4e01a4340343a0{-}66723{-}a{-}9970}
\DocumentationTok{\#\#                 networkCanvasTargetUUID dataset\_num}
\DocumentationTok{\#\#                                  \textless{}char\textgreater{}       \textless{}int\textgreater{}}
\DocumentationTok{\#\# 1: Id1c8befd46bdd195c{-}ce91a8{-}bc0{-}{-}{-}4f0e           1}
\DocumentationTok{\#\# 2: I757b4a{-}3ea4d95{-}{-}b9ebb9db3d55dcbaf{-}c           1}
\DocumentationTok{\#\# 3: I92a62925ff9{-}e2f27{-}6ef97d{-}29fb729624           1}
\DocumentationTok{\#\# 4: I7f{-}{-}da48{-}46a64{-}b972c{-}ef6bbec{-}{-}64cb4           1}
\DocumentationTok{\#\# 5: I{-}eaa7e95659{-}9cf01a4f5fd69af54e6{-}d60           1}
\DocumentationTok{\#\# 6: I69060e8a{-}454609{-}faa04cd3eeb{-}5{-}9550{-}           1}
\end{Highlighting}
\end{Shaded}

\section{Juntando todo}\label{juntando-todo}

En esta última parte del capítulo, usaremos los paquetes \texttt{igraph}
y \texttt{ergm} para generar características (covariables, controles,
variables independientes, o como las llames) a nivel de red egocéntrica.
Una vez más, la función \texttt{lapply} es nuestra amiga

\subsection{Generando estadísticas usando
igraph}\label{generando-estaduxedsticas-usando-igraph}

El paquete de R \texttt{igraph} tiene múltiples rutinas de alto
rendimiento para calcular estadísticas a nivel de grafo. Por ahora, nos
enfocaremos en las siguientes estadísticas: conteo de vértices, conteo
de enlaces, número de aislados, transitividad, y modularidad basada en
centralidad de intermediación:

\begin{Shaded}
\begin{Highlighting}[]
\NormalTok{net\_stats }\OtherTok{\textless{}{-}} \FunctionTok{lapply}\NormalTok{(graphs, }\ControlFlowTok{function}\NormalTok{(g) \{}
  
  \CommentTok{\# Calculando modularidad}
\NormalTok{  groups }\OtherTok{\textless{}{-}} \FunctionTok{cluster\_edge\_betweenness}\NormalTok{(g)}
  
  \CommentTok{\# Calculando las estadísticas}
  \FunctionTok{data.table}\NormalTok{(}
    \AttributeTok{size      =} \FunctionTok{vcount}\NormalTok{(g),}
    \AttributeTok{edges     =} \FunctionTok{ecount}\NormalTok{(g),}
    \AttributeTok{nisolates =} \FunctionTok{sum}\NormalTok{(}\FunctionTok{degree}\NormalTok{(g) }\SpecialCharTok{==} \DecValTok{0}\NormalTok{),}
    \AttributeTok{transit   =} \FunctionTok{transitivity}\NormalTok{(g, }\AttributeTok{type =} \StringTok{"global"}\NormalTok{),}
    \AttributeTok{modular   =} \FunctionTok{modularity}\NormalTok{(groups)}
\NormalTok{  )}
\NormalTok{\})}
\end{Highlighting}
\end{Shaded}

Observa que contamos aislados usando la función \texttt{degree()}.
Podemos combinar las estadísticas en un solo \texttt{data.table} usando
la función \texttt{rbindlist}:

\begin{Shaded}
\begin{Highlighting}[]
\NormalTok{net\_stats }\OtherTok{\textless{}{-}} \FunctionTok{rbindlist}\NormalTok{(net\_stats)}

\FunctionTok{head}\NormalTok{(net\_stats)}
\DocumentationTok{\#\#     size edges nisolates   transit     modular}
\DocumentationTok{\#\#    \textless{}num\textgreater{} \textless{}num\textgreater{}     \textless{}int\textgreater{}     \textless{}num\textgreater{}       \textless{}num\textgreater{}}
\DocumentationTok{\#\# 1:    12    25         1 0.6750000 0.012000000}
\DocumentationTok{\#\# 2:    16    47         0 0.4332130 0.003395201}
\DocumentationTok{\#\# 3:    16    58         0 0.5612009 0.002675386}
\DocumentationTok{\#\# 4:    15    75         0 0.8515112 0.000000000}
\DocumentationTok{\#\# 5:    15    52         0 0.5780488 0.000000000}
\DocumentationTok{\#\# 6:    17    68         0 0.6291161 0.025735294}
\end{Highlighting}
\end{Shaded}

\subsection{Generando estadísticas basadas en
ergm}\label{generando-estaduxedsticas-basadas-en-ergm}

El paquete de R \texttt{ergm} tiene un conjunto mucho más grande de
estadísticas a nivel de grafo que podemos agregar a nuestros
modelos.\footnote{¡Hay una razón obvia, los ERGMs son todo sobre
  estadísticas a nivel de grafo!} La clave para generar estadísticas
basadas en el paquete \texttt{ergm} es la función
\texttt{summary\_formula}. Antes de empezar a usar esa función, primero
necesitamos convertir las redes \texttt{igraph} a objetos
\texttt{network}, que son la clase de objeto nativa para el paquete
\texttt{ergm}. Usamos el paquete de R \texttt{intergraph} para eso, y en
particular, la función \texttt{asNetwork}:

\begin{Shaded}
\begin{Highlighting}[]
\CommentTok{\# Cargando los paquetes requeridos}
\FunctionTok{library}\NormalTok{(intergraph)}
\FunctionTok{library}\NormalTok{(ergm)}
\DocumentationTok{\#\# Loading required package: network}
\DocumentationTok{\#\# }
\DocumentationTok{\#\# \textquotesingle{}network\textquotesingle{} 1.19.0 (2024{-}12{-}08), part of the Statnet Project}
\DocumentationTok{\#\# * \textquotesingle{}news(package="network")\textquotesingle{} for changes since last version}
\DocumentationTok{\#\# * \textquotesingle{}citation("network")\textquotesingle{} for citation information}
\DocumentationTok{\#\# * \textquotesingle{}https://statnet.org\textquotesingle{} for help, support, and other information}
\DocumentationTok{\#\# }
\DocumentationTok{\#\# Attaching package: \textquotesingle{}network\textquotesingle{}}
\DocumentationTok{\#\# The following objects are masked from \textquotesingle{}package:igraph\textquotesingle{}:}
\DocumentationTok{\#\# }
\DocumentationTok{\#\#     \%c\%, \%s\%, add.edges, add.vertices, delete.edges, delete.vertices,}
\DocumentationTok{\#\#     get.edge.attribute, get.edges, get.vertex.attribute, is.bipartite,}
\DocumentationTok{\#\#     is.directed, list.edge.attributes, list.vertex.attributes,}
\DocumentationTok{\#\#     set.edge.attribute, set.vertex.attribute}
\DocumentationTok{\#\# }
\DocumentationTok{\#\# \textquotesingle{}ergm\textquotesingle{} 4.10.1 (2025{-}08{-}26), part of the Statnet Project}
\DocumentationTok{\#\# * \textquotesingle{}news(package="ergm")\textquotesingle{} for changes since last version}
\DocumentationTok{\#\# * \textquotesingle{}citation("ergm")\textquotesingle{} for citation information}
\DocumentationTok{\#\# * \textquotesingle{}https://statnet.org\textquotesingle{} for help, support, and other information}
\DocumentationTok{\#\# \textquotesingle{}ergm\textquotesingle{} 4 is a major update that introduces some backwards{-}incompatible}
\DocumentationTok{\#\# changes. Please type \textquotesingle{}news(package="ergm")\textquotesingle{} for a list of major}
\DocumentationTok{\#\# changes.}

\CommentTok{\# Convirtiendo todos los objetos "igraph" en graphs a objetos "network"}
\NormalTok{graphs\_network }\OtherTok{\textless{}{-}} \FunctionTok{lapply}\NormalTok{(graphs, asNetwork)}
\end{Highlighting}
\end{Shaded}

Con los objetos de red listos, podemos proceder a calcular estadísticas
a nivel de grafo usando la función \texttt{summary\_formula}. Aquí solo
veremos: el número de triángulos, homofilia de género, y homofilia de
dieta saludable:

\begin{Shaded}
\begin{Highlighting}[]
\NormalTok{net\_stats\_ergm }\OtherTok{\textless{}{-}} \FunctionTok{lapply}\NormalTok{(graphs\_network, }\ControlFlowTok{function}\NormalTok{(n) \{}
  
  \CommentTok{\# Calculando las estadísticas}
\NormalTok{  s }\OtherTok{\textless{}{-}} \FunctionTok{summary\_formula}\NormalTok{(}
\NormalTok{    n }\SpecialCharTok{\textasciitilde{}}\NormalTok{ triangles }\SpecialCharTok{+}
      \FunctionTok{nodematch}\NormalTok{(}\StringTok{"gender\_1"}\NormalTok{) }\SpecialCharTok{+}
      \FunctionTok{nodematch}\NormalTok{(}\StringTok{"healthy\_diet"}\NormalTok{)}
\NormalTok{    )}
  
  \CommentTok{\# Guardándolas como un objeto data.table}
  \FunctionTok{data.table}\NormalTok{(}
    \AttributeTok{triangles       =}\NormalTok{ s[}\DecValTok{1}\NormalTok{],}
    \AttributeTok{gender\_homoph   =}\NormalTok{ s[}\DecValTok{2}\NormalTok{],}
    \AttributeTok{healthyd\_homoph =}\NormalTok{ s[}\DecValTok{3}\NormalTok{]}
\NormalTok{  )}
\NormalTok{\})}
\end{Highlighting}
\end{Shaded}

Una vez más, usamos \texttt{rbindlist} para combinar todas las
estadísticas de red en un solo objeto \texttt{data.table}:

\begin{Shaded}
\begin{Highlighting}[]
\NormalTok{net\_stats\_ergm }\OtherTok{\textless{}{-}} \FunctionTok{rbindlist}\NormalTok{(net\_stats\_ergm)}
\FunctionTok{head}\NormalTok{(net\_stats\_ergm)}
\DocumentationTok{\#\#    triangles gender\_homoph healthyd\_homoph}
\DocumentationTok{\#\#        \textless{}num\textgreater{}         \textless{}num\textgreater{}           \textless{}num\textgreater{}}
\DocumentationTok{\#\# 1:        27            11               3}
\DocumentationTok{\#\# 2:        40            30              20}
\DocumentationTok{\#\# 3:        81            40              29}
\DocumentationTok{\#\# 4:       216            33              38}
\DocumentationTok{\#\# 5:        79            44              19}
\DocumentationTok{\#\# 6:       121            38              16}
\end{Highlighting}
\end{Shaded}

\section{Guardando los datos}\label{guardando-los-datos}

Terminamos el capítulo guardando todo nuestro trabajo en cuatro
conjuntos de datos:

\begin{itemize}
\item
  Estadísticas de red (como un archivo csv)
\item
  Objetos igraph (como un archivo rda, que podemos leer de vuelta usando
  \texttt{read.rds})
\item
  Objetos de red (ídem)
\item
  Archivos de persona (información de alters, como un archivo csv.)
\end{itemize}

Los archivos CSV pueden guardarse usando \texttt{write.csv} o, como
hacemos aquí, \texttt{fwrite} del paquete \texttt{data.table}:

\begin{Shaded}
\begin{Highlighting}[]
\CommentTok{\# Verificando que el directorio existe}
\ControlFlowTok{if}\NormalTok{ (}\SpecialCharTok{!}\FunctionTok{dir.exists}\NormalTok{(}\StringTok{"data"}\NormalTok{))}
  \FunctionTok{dir.create}\NormalTok{(}\StringTok{"data"}\NormalTok{)}

\CommentTok{\# Atributos de red}
\NormalTok{master }\OtherTok{\textless{}{-}} \FunctionTok{cbind}\NormalTok{(egos, net\_stats, net\_stats\_ergm)}
\FunctionTok{fwrite}\NormalTok{(master, }\AttributeTok{file =} \StringTok{"data/network\_stats.csv"}\NormalTok{)}

\CommentTok{\# Redes}
\FunctionTok{saveRDS}\NormalTok{(graphs, }\AttributeTok{file =} \StringTok{"data/networks\_igraph.rds"}\NormalTok{)}
\FunctionTok{saveRDS}\NormalTok{(graphs\_network, }\AttributeTok{file =} \StringTok{"data/networks\_network.rds"}\NormalTok{)}

\CommentTok{\# Atributos}
\FunctionTok{fwrite}\NormalTok{(persons, }\AttributeTok{file =} \StringTok{"data/persons.csv"}\NormalTok{)}
\end{Highlighting}
\end{Shaded}

\chapter{TBD}\label{tbd-1}

\begin{tcolorbox}[enhanced jigsaw, colback=white, opacityback=0, coltitle=black, title=\textcolor{quarto-callout-warning-color}{\faExclamationTriangle}\hspace{0.5em}{Nota de Traducción}, bottomrule=.15mm, colbacktitle=quarto-callout-warning-color!10!white, toptitle=1mm, colframe=quarto-callout-warning-color-frame, titlerule=0mm, rightrule=.15mm, leftrule=.75mm, breakable, bottomtitle=1mm, left=2mm, arc=.35mm, toprule=.15mm, opacitybacktitle=0.6]

Esta versión del capítulo fue traducida de manera automática utilizando
IA. El capítulo aún no ha sido revisado por un humano.

\end{tcolorbox}

\emph{Este capítulo está pendiente de traducción.}

\part{\textbf{Inferencia estadística}}

\chapter{Comportamiento y
coevolución}\label{comportamiento-y-coevoluciuxf3n}

\begin{tcolorbox}[enhanced jigsaw, colback=white, opacityback=0, coltitle=black, title=\textcolor{quarto-callout-warning-color}{\faExclamationTriangle}\hspace{0.5em}{Nota de Traducción}, bottomrule=.15mm, colbacktitle=quarto-callout-warning-color!10!white, toptitle=1mm, colframe=quarto-callout-warning-color-frame, titlerule=0mm, rightrule=.15mm, leftrule=.75mm, breakable, bottomtitle=1mm, left=2mm, arc=.35mm, toprule=.15mm, opacitybacktitle=0.6]

Esta versión del capítulo fue traducida de manera automática utilizando
IA. El capítulo aún no ha sido revisado por un humano.

\end{tcolorbox}

\begin{tcolorbox}[enhanced jigsaw, colback=white, opacityback=0, coltitle=black, title=\textcolor{quarto-callout-note-color}{\faInfo}\hspace{0.5em}{Note}, bottomrule=.15mm, colbacktitle=quarto-callout-note-color!10!white, toptitle=1mm, colframe=quarto-callout-note-color-frame, titlerule=0mm, rightrule=.15mm, leftrule=.75mm, breakable, bottomtitle=1mm, left=2mm, arc=.35mm, toprule=.15mm, opacitybacktitle=0.6]

Todo el contenido de esta sección fue presentado durante la Escuela de
Verano Sistemas Complejos 2024 en la Universidad del Desarrollo. La
versión original se puede encontrar
\href{https://gvegayon.github.io/networks-udd2024/}{aquí}.

\end{tcolorbox}

\newcommand{\tpose}[1]{#1^{\mathbf{t}}}
\newcommand{\Prergm}[0]{P_{\mathcal{Y}, \bm{{\theta}}}}
\newcommand{\Y}[0]{\bm{{Y}}}
\newcommand{\y}[0]{\bm{{y}}}
\newcommand{\X}[0]{\bm{{X}}}
\newcommand{\x}[0]{\bm{{x}}}
\newcommand{\thetas}[0]{\bm{{\theta}}}

\section{Introducción}\label{introducciuxf3n-1}

Esta sección se enfoca en la inferencia que involucra redes y un
resultado secundario. Aunque hay muchas formas de estudiar la
coevolución o dependencia entre red y comportamiento, esta sección se
enfoca en dos clases de análisis: cuando la red es fija y cuando tanto
la red como el comportamiento se influyen mutuamente.

Si tratamos la red como dada o endógena establece la complejidad de
realizar inferencia estadística. El análisis de datos se vuelve mucho
más directo si nuestra investigación se enfoca en resultados a nivel
individual embebidos en una red y no en la red misma. Aquí, trataremos
con tres casos particulares: (1) cuando los efectos de red son
rezagados, (2) redes egocéntricas, y (3) cuando los efectos de red son
contemporáneos.

\section{Exposición rezagada}\label{exposiciuxf3n-rezagada}

Si asumimos que la influencia de la red en forma de exposición está
rezagada, tenemos uno de los casos más directos para la inferencia de
redes (Haye et al. 2019; Valente and Vega Yon 2020; Valente, Wipfli, and
Vega Yon 2019). Aquí, en lugar de tratar con modelos estadísticos
complicados, el problema se reduce a estimar un modelo de regresión
lineal simple. Generalmente, los efectos de exposición rezagada se ven
así:

\[
y_{it} = \rho \left(\sum_{j\neq i}X_{ij}\right)^{-1}\left(\sum_{j\neq i}y_{jt-1} X_{ij}\right) + {\bm{{\theta}}}^{\mathbf{t}}\bm{{w_i}} + \varepsilon,\quad \varepsilon \sim \text{N}(0, \sigma^2)
\]

donde \(y_{it}\) es el resultado del individuo \(i\) en el tiempo \(t\),
\(X_{ij}\) es la entrada \(ij\)-ésima de la matriz de adyacencia,
\(\bm{{\theta}}\) es un vector de coeficientes, \(\bm{{w_i}}\) es un
vector de características/covariables del individuo \(i\), y
\(\varepsilon_i\) es un error distribuido normalmente. Aquí, el
componente clave es \(\rho\): el coeficiente asociado con el efecto de
exposición de red.

La estadística de exposición,
\(\left(\sum_{j\neq i}X_{ij}\right)^{-1}\left(\sum_{j\neq i}y_{jt-1} X_{ij}\right)\),
es el promedio ponderado de los resultados de los vecinos de \(i\) en el
tiempo \(t-1\).

\section{Ejemplo de código: Exposición
rezagada}\label{ejemplo-de-cuxf3digo-exposiciuxf3n-rezagada}

El siguiente ejemplo de código muestra cómo estimar un efecto de
exposición rezagada usando la función \texttt{glm} en R. El modelo que
simularemos y estimaremos presenta un grafo de Bernoulli con 1,000 nodos
y una densidad de 0.01.

\[
y_{it} = \theta_1 + \rho \text{Exposure}_{it} + \theta_2 w_i + \varepsilon
\]

donde \(\text{Exposure}_{it}\) es la estadística de exposición definida
arriba, y \(w_i\) es un vector de covariables.

\begin{Shaded}
\begin{Highlighting}[]
\DocumentationTok{\#\# Simulando datos}
\NormalTok{n }\OtherTok{\textless{}{-}} \DecValTok{1000}
\NormalTok{time }\OtherTok{\textless{}{-}} \DecValTok{2}
\NormalTok{theta }\OtherTok{\textless{}{-}} \FunctionTok{c}\NormalTok{(}\SpecialCharTok{{-}}\DecValTok{1}\NormalTok{, }\DecValTok{3}\NormalTok{)}

\DocumentationTok{\#\# Muestreando una red de Bernoulli}
\FunctionTok{set.seed}\NormalTok{(}\DecValTok{3132}\NormalTok{)}
\NormalTok{p }\OtherTok{\textless{}{-}} \FloatTok{0.01}
\NormalTok{X }\OtherTok{\textless{}{-}} \FunctionTok{matrix}\NormalTok{(}\FunctionTok{rbinom}\NormalTok{(n}\SpecialCharTok{\^{}}\DecValTok{2}\NormalTok{, }\DecValTok{1}\NormalTok{, p), }\AttributeTok{nrow =}\NormalTok{ n)}
\FunctionTok{diag}\NormalTok{(X) }\OtherTok{\textless{}{-}} \DecValTok{0}

\DocumentationTok{\#\# Covariable}
\NormalTok{W }\OtherTok{\textless{}{-}} \FunctionTok{matrix}\NormalTok{(}\FunctionTok{rnorm}\NormalTok{(n), }\AttributeTok{nrow =}\NormalTok{ n)}

\DocumentationTok{\#\# Simulando el resultado}
\NormalTok{rho }\OtherTok{\textless{}{-}} \FloatTok{0.5}
\NormalTok{Y0 }\OtherTok{\textless{}{-}} \FunctionTok{cbind}\NormalTok{(}\FunctionTok{rnorm}\NormalTok{(n))}

\DocumentationTok{\#\# La exposición rezagada}
\NormalTok{expo }\OtherTok{\textless{}{-}}\NormalTok{ (X }\SpecialCharTok{\%*\%}\NormalTok{ Y0)}\SpecialCharTok{/}\FunctionTok{rowSums}\NormalTok{(X)}
\NormalTok{Y1 }\OtherTok{\textless{}{-}}\NormalTok{ theta[}\DecValTok{1}\NormalTok{] }\SpecialCharTok{+}\NormalTok{ rho }\SpecialCharTok{*}\NormalTok{ expo }\SpecialCharTok{+}\NormalTok{ W }\SpecialCharTok{*}\NormalTok{ theta[}\DecValTok{2}\NormalTok{] }\SpecialCharTok{+} \FunctionTok{rnorm}\NormalTok{(n)}
\end{Highlighting}
\end{Shaded}

Ahora ajustamos el modelo usando GLM, en este caso, regresión lineal

\begin{Shaded}
\begin{Highlighting}[]
\NormalTok{fit }\OtherTok{\textless{}{-}} \FunctionTok{glm}\NormalTok{(Y1 }\SpecialCharTok{\textasciitilde{}}\NormalTok{ expo }\SpecialCharTok{+}\NormalTok{ W, }\AttributeTok{family =} \StringTok{"gaussian"}\NormalTok{)}
\FunctionTok{summary}\NormalTok{(fit)}
\end{Highlighting}
\end{Shaded}

\begin{verbatim}

Call:
glm(formula = Y1 ~ expo + W, family = "gaussian")

Coefficients:
            Estimate Std. Error t value Pr(>|t|)    
(Intercept) -1.07187    0.03284 -32.638  < 2e-16 ***
expo         0.61170    0.10199   5.998  2.8e-09 ***
W            3.00316    0.03233  92.891  < 2e-16 ***
---
Signif. codes:  0 '***' 0.001 '**' 0.01 '*' 0.05 '.' 0.1 ' ' 1

(Dispersion parameter for gaussian family taken to be 1.071489)

    Null deviance: 10319.3  on 999  degrees of freedom
Residual deviance:  1068.3  on 997  degrees of freedom
AIC: 2911.9

Number of Fisher Scoring iterations: 2
\end{verbatim}

\section{Redes egocéntricas}\label{redes-egocuxe9ntricas-1}

Generalmente, cuando usamos redes egocéntricas y resultados de los egos,
estamos pensando en un modelo donde una observación es el par
\((y_i, X_i)\), esto es, el resultado del individuo \(i\) y la red
egocéntrica del individuo \(i\). Cuando tal es el caso, dado que (a) las
redes son independientes entre egos y (b) las redes son fijas, como el
caso anterior, un modelo de regresión lineal simple es suficiente para
realizar los análisis. Un modelo típico se ve así:

\[
\bm{{y}} = \bm{{\theta}}_{x}^{\mathbf{t}}s(\bm{{X}}) + \bm{{\theta}}^{\mathbf{t}}\bm{{w}} + \varepsilon,\quad \varepsilon \sim \text{N}(0, \sigma^2)
\]

Donde \(\bm{{y}}\) es un vector de resultados, \(\bm{{X}}\) es una
matriz de redes egocéntricas, \(\bm{{w}}\) es un vector de covariables,
\(\bm{{\theta}}\) es un vector de coeficientes, y \(\varepsilon\) es un
vector de errores. El componente clave aquí es \(s(\bm{{X}})\), que es
un vector de estadísticas suficientes de las redes egocéntricas. Por
ejemplo, si estamos interesados en el número de vínculos,
\(s(\bm{{X}})\) es un vector del número de vínculos de cada ego.

\section{Ejemplo de código: Redes
egocéntricas}\label{ejemplo-de-cuxf3digo-redes-egocuxe9ntricas}

Para este ejemplo, simularemos un flujo de 1,000 grafos de Bernoulli
analizando la probabilidad de deserción escolar. Cada red tendrá entre 4
y 10 nodos y tendrá una densidad de 0.4. El proceso de generación de
datos es el siguiente:

\[
{\Pr{}}_{\bm{{\theta}}}\left(Y_i=1\right) = \text{logit}^{-1}\left(\bm{{\theta}}_x s(\bm{{X}}_i) \right)
\]

Donde
\(s(X) \equiv \left(\text{densidad}, \text{n vínculos mutuos}\right)\),
y \(\bm{{\theta}}_x = (0.5, -1)\). Este modelo solo presenta
estadísticas suficientes. Comenzamos simulando las redes

\begin{Shaded}
\begin{Highlighting}[]
\FunctionTok{set.seed}\NormalTok{(}\DecValTok{331}\NormalTok{)}
\NormalTok{n }\OtherTok{\textless{}{-}} \DecValTok{1000}
\NormalTok{sizes }\OtherTok{\textless{}{-}} \FunctionTok{sample}\NormalTok{(}\DecValTok{4}\SpecialCharTok{:}\DecValTok{10}\NormalTok{, n, }\AttributeTok{replace =} \ConstantTok{TRUE}\NormalTok{)}

\DocumentationTok{\#\# Simulando las redes}
\NormalTok{X }\OtherTok{\textless{}{-}} \FunctionTok{lapply}\NormalTok{(sizes, }\ControlFlowTok{function}\NormalTok{(x) }\FunctionTok{matrix}\NormalTok{(}\FunctionTok{rbinom}\NormalTok{(x}\SpecialCharTok{\^{}}\DecValTok{2}\NormalTok{, }\DecValTok{1}\NormalTok{, }\FloatTok{0.4}\NormalTok{), }\AttributeTok{nrow =}\NormalTok{ x))}
\NormalTok{X }\OtherTok{\textless{}{-}} \FunctionTok{lapply}\NormalTok{(X, \textbackslash{}(x) \{}\FunctionTok{diag}\NormalTok{(x) }\OtherTok{\textless{}{-}} \DecValTok{0}\NormalTok{; x\})}

\DocumentationTok{\#\# Inspeccionando las primeras 5}
\FunctionTok{head}\NormalTok{(X, }\DecValTok{5}\NormalTok{)}
\end{Highlighting}
\end{Shaded}

\begin{verbatim}
[[1]]
     [,1] [,2] [,3] [,4] [,5]
[1,]    0    1    1    1    0
[2,]    0    0    0    0    0
[3,]    0    1    0    0    0
[4,]    0    0    0    0    0
[5,]    1    0    0    1    0

[[2]]
     [,1] [,2] [,3] [,4]
[1,]    0    0    0    0
[2,]    0    0    0    0
[3,]    0    0    0    0
[4,]    1    0    1    0

[[3]]
     [,1] [,2] [,3] [,4] [,5] [,6]
[1,]    0    1    0    1    0    0
[2,]    0    0    0    0    0    0
[3,]    0    1    0    0    0    1
[4,]    0    0    0    0    1    0
[5,]    0    0    0    0    0    0
[6,]    0    0    0    0    0    0

[[4]]
     [,1] [,2] [,3] [,4] [,5]
[1,]    0    1    0    1    0
[2,]    0    0    0    0    1
[3,]    0    1    0    0    0
[4,]    0    1    1    0    1
[5,]    1    0    1    0    0

[[5]]
     [,1] [,2] [,3] [,4] [,5]
[1,]    0    0    0    0    0
[2,]    1    0    0    0    0
[3,]    1    0    0    0    0
[4,]    0    0    0    0    0
[5,]    1    0    0    0    0
\end{verbatim}

Usando el paquete de R \texttt{ergm} (Handcock et al. 2023; David R.
Hunter et al. 2008), podemos extraer las estadísticas suficientes
asociadas de las redes egocéntricas:

\begin{Shaded}
\begin{Highlighting}[]
\FunctionTok{library}\NormalTok{(ergm)}
\NormalTok{stats }\OtherTok{\textless{}{-}} \FunctionTok{lapply}\NormalTok{(X, \textbackslash{}(x) }\FunctionTok{summary\_formula}\NormalTok{(x }\SpecialCharTok{\textasciitilde{}}\NormalTok{ density }\SpecialCharTok{+}\NormalTok{ mutual))}

\DocumentationTok{\#\# Convirtiendo la lista en una matriz}
\NormalTok{stats }\OtherTok{\textless{}{-}} \FunctionTok{do.call}\NormalTok{(rbind, stats)}

\DocumentationTok{\#\# Inspeccionando las primeras 5}
\FunctionTok{head}\NormalTok{(stats, }\DecValTok{5}\NormalTok{)}
\end{Highlighting}
\end{Shaded}

\begin{verbatim}
       density mutual
[1,] 0.3000000      0
[2,] 0.1666667      0
[3,] 0.1666667      0
[4,] 0.4500000      0
[5,] 0.1500000      0
\end{verbatim}

Ahora simulamos los resultados

\begin{Shaded}
\begin{Highlighting}[]
\NormalTok{y }\OtherTok{\textless{}{-}} \FunctionTok{rbinom}\NormalTok{(n, }\DecValTok{1}\NormalTok{, }\FunctionTok{plogis}\NormalTok{(stats }\SpecialCharTok{\%*\%} \FunctionTok{c}\NormalTok{(}\FloatTok{0.5}\NormalTok{, }\SpecialCharTok{{-}}\DecValTok{1}\NormalTok{)))}
\FunctionTok{glm}\NormalTok{(y }\SpecialCharTok{\textasciitilde{}}\NormalTok{ stats, }\AttributeTok{family =} \FunctionTok{binomial}\NormalTok{(}\AttributeTok{link =} \StringTok{"logit"}\NormalTok{)) }\SpecialCharTok{|\textgreater{}}
  \FunctionTok{summary}\NormalTok{()}
\end{Highlighting}
\end{Shaded}

\begin{verbatim}

Call:
glm(formula = y ~ stats, family = binomial(link = "logit"))

Coefficients:
             Estimate Std. Error z value Pr(>|z|)    
(Intercept)   0.07319    0.41590   0.176    0.860    
statsdensity  0.42568    1.26942   0.335    0.737    
statsmutual  -1.14804    0.12166  -9.436   <2e-16 ***
---
Signif. codes:  0 '***' 0.001 '**' 0.01 '*' 0.05 '.' 0.1 ' ' 1

(Dispersion parameter for binomial family taken to be 1)

    Null deviance: 768.96  on 999  degrees of freedom
Residual deviance: 518.78  on 997  degrees of freedom
AIC: 524.78

Number of Fisher Scoring iterations: 7
\end{verbatim}

\section{Los efectos de red son
endógenos}\label{los-efectos-de-red-son-enduxf3genos}

Aquí tenemos dos enfoques diferentes: Autocorrelación Espacial
{[}SAR{]}, y el modelo de atributo de actor autologístico {[}ALAAM{]}
(Robins, Pattison, and Elliott 2001). El primero es una generalización
del modelo de regresión lineal que considera la dependencia espacial. El
segundo es un pariente cercano de los ERGMs que trata las covariables
como endógenas y la red como exógena. En general, los ALAAMs son más
flexibles que los SARs, pero los SARs son más fáciles de estimar.

\textbf{SAR} Formalmente, los modelos SAR (ver LeSage 2008) pueden
usarse para estimar efectos de exposición de red. La forma general es:

\[
\bm{{y}} = \rho \bm{{W}} \bm{{y}} + \bm{{\theta}}^{\mathbf{t}} \bm{{X}} + \epsilon,\quad \epsilon \sim \text{MVN}(0, \Sigma)
\]

donde \(\bm{{y}}\equiv \{y_i\}\) es un vector de resultados, \(\rho\) es
un coeficiente de autocorrelación, \(\bm{{W}} \in \{w_{ij}\}\) es una
matriz cuadrada estocástica por filas de tamaño \(n\), \(\bm{{\theta}}\)
es un vector de parámetros del modelo, \(\bm{{X}}\) es la matriz
correspondiente con variables exógenas, y \(\epsilon\) es un vector de
errores que se distribuye normal multivariado con media 0 y covarianza
\(\Sigma\).{[}\^{}notation{]} El modelo SAR es una generalización del
modelo de regresión lineal que considera la dependencia espacial. El
modelo SAR puede estimarse usando el paquete \texttt{spatialreg} en R
(Roger Bivand 2022).

\begin{tcolorbox}[enhanced jigsaw, colback=white, opacityback=0, coltitle=black, title=\textcolor{quarto-callout-tip-color}{\faLightbulb}\hspace{0.5em}{Tip}, bottomrule=.15mm, colbacktitle=quarto-callout-tip-color!10!white, toptitle=1mm, colframe=quarto-callout-tip-color-frame, titlerule=0mm, rightrule=.15mm, leftrule=.75mm, breakable, bottomtitle=1mm, left=2mm, arc=.35mm, toprule=.15mm, opacitybacktitle=0.6]

¿Cuál es la red apropiada para usar en el modelo SAR? Según LeSage and
Pace (2014), no es muy importante. Dado que
\((I_n - \rho \mathbf{W})^{-1} = \rho \mathbf{W} + \rho^2 \mathbf{W}^2 + \dots\).

\end{tcolorbox}

Aunque el modelo SAR fue desarrollado para datos espaciales, es fácil
aplicarlo a datos de red. Además, cada entrada del vector \(\bm{{Wy}}\)
tiene la misma definición que la exposición de red, es decir

\[
\bm{{Wy}} \equiv \left\{\sum_{j}y_j w_{ij}\right\}_i
\]

Dado que \(\bm{{W}}\) es estocástica por filas, \(\bm{{Wy}}\) es un
promedio ponderado del resultado de los vecinos de \(i\), \emph{es
decir}, un vector de exposiciones de red.

\textbf{ALAAM} La forma más simple en que podemos pensar sobre esta
clase de modelos es como si una covariable dada intercambiara lugares
con la red en un ERGM, entonces la red ahora es fija y la covariable es
la variable aleatoria. Aunque los ALAAMs también pueden estimar efectos
de exposición de red, podemos usarlos para construir modelos más
complejos más allá de la exposición. La forma general es:

\[
{\mathbb{P}\left({\bm{{Y}} = \bm{{y}}}\vphantom{\bm{{W}},\bm{{X}}}\;\right|\left.\vphantom{\bm{{Y}} = \bm{{y}}}{\bm{{W}},\bm{{X}}}\right)} = \text{exp}\left\{\left(\bm{{\theta}}^{\mathbf{t}}s(\bm{{y}},\bm{{W}}, \bm{{X}})\right)\right\}\times\eta(\bm{{\theta}})^{-1}
\]

\[
\eta(\bm{{\theta}}) = \sum_{\bm{{y}}}\text{exp}\left\{\left(\bm{{\theta}}^{\mathbf{t}}s(\bm{{y}},\bm{{W}}, \bm{{X}})\right)\right\}
\]

Donde \(\bm{{Y}}\equiv \{y_i \in (0, 1)\}\) es un vector de resultados
individuales binarios, \(\bm{{W}}\) denota la red social, \(\bm{{X}}\)
es una matriz de variables exógenas, \(\bm{{\theta}}\) es un vector de
parámetros del modelo, \(s(\bm{{y}},\bm{{W}}, \bm{{X}})\) es un vector
de estadísticas suficientes, y \(\eta(\bm{{\theta}})\) es una constante
normalizadora.

\section{Ejemplo de código: SAR}\label{ejemplo-de-cuxf3digo-sar}

La simulación de modelos SAR puede hacerse usando la siguiente
observación: Aunque el resultado aparece en ambos lados de la ecuación,
podemos aislarlo en un lado y resolverlo; formalmente:

\[
\bm{{y}} = \rho \bm{{X}} \bm{{y}} + \bm{{\theta}}^{\mathbf{t}}\bm{{W}} + \varepsilon \implies \bm{{y}} = \left(\bm{{I}} - \rho \bm{{X}}\right)^{-1}\bm{{\theta}}^{\mathbf{t}}\bm{{W}} + \left(\bm{{I}} - \rho \bm{{X}}\right)^{-1}\varepsilon
\]

El siguiente fragmento de código simula un modelo SAR con un grafo de
Bernoulli con 1,000 nodos y una densidad de 0.01. El proceso de
generación de datos es el siguiente:

\begin{Shaded}
\begin{Highlighting}[]
\FunctionTok{set.seed}\NormalTok{(}\DecValTok{4114}\NormalTok{)}
\NormalTok{n }\OtherTok{\textless{}{-}} \DecValTok{1000}

\DocumentationTok{\#\# Simulando la red}
\NormalTok{p }\OtherTok{\textless{}{-}} \FloatTok{0.01}
\NormalTok{X }\OtherTok{\textless{}{-}} \FunctionTok{matrix}\NormalTok{(}\FunctionTok{rbinom}\NormalTok{(n}\SpecialCharTok{\^{}}\DecValTok{2}\NormalTok{, }\DecValTok{1}\NormalTok{, p), }\AttributeTok{nrow =}\NormalTok{ n)}

\DocumentationTok{\#\# Covariable}
\NormalTok{W }\OtherTok{\textless{}{-}} \FunctionTok{matrix}\NormalTok{(}\FunctionTok{rnorm}\NormalTok{(n), }\AttributeTok{nrow =}\NormalTok{ n)}

\DocumentationTok{\#\# Simulando el resultado}
\NormalTok{rho }\OtherTok{\textless{}{-}} \FloatTok{0.5}
\FunctionTok{library}\NormalTok{(MASS) }\CommentTok{\# Para la función mvrnorm}

\DocumentationTok{\#\# Identidad menos rho * X}
\NormalTok{X\_rowstoch }\OtherTok{\textless{}{-}}\NormalTok{ X }\SpecialCharTok{/} \FunctionTok{rowSums}\NormalTok{(X)}
\NormalTok{I }\OtherTok{\textless{}{-}} \FunctionTok{diag}\NormalTok{(n) }\SpecialCharTok{{-}}\NormalTok{ rho }\SpecialCharTok{*}\NormalTok{ X\_rowstoch}

\DocumentationTok{\#\# El resultado}
\NormalTok{Y }\OtherTok{\textless{}{-}} \FunctionTok{solve}\NormalTok{(I) }\SpecialCharTok{\%*\%}\NormalTok{ (}\DecValTok{2} \SpecialCharTok{*}\NormalTok{ W) }\SpecialCharTok{+} \FunctionTok{solve}\NormalTok{(I) }\SpecialCharTok{\%*\%} \FunctionTok{mvrnorm}\NormalTok{(}\DecValTok{1}\NormalTok{, }\FunctionTok{rep}\NormalTok{(}\DecValTok{0}\NormalTok{, n), }\FunctionTok{diag}\NormalTok{(n))}
\end{Highlighting}
\end{Shaded}

Usando el paquete de R \texttt{spatialreg}, podemos ajustar el modelo
usando la función \texttt{lagsarlm}:

\begin{Shaded}
\begin{Highlighting}[]
\FunctionTok{library}\NormalTok{(spdep) }\CommentTok{\# para la función mat2listw}
\FunctionTok{library}\NormalTok{(spatialreg)}
\NormalTok{fit }\OtherTok{\textless{}{-}} \FunctionTok{lagsarlm}\NormalTok{(}
\NormalTok{  Y }\SpecialCharTok{\textasciitilde{}}\NormalTok{ W,}
  \AttributeTok{data  =} \FunctionTok{as.data.frame}\NormalTok{(X),}
  \AttributeTok{listw =} \FunctionTok{mat2listw}\NormalTok{(X\_rowstoch)}
\NormalTok{  )}
\end{Highlighting}
\end{Shaded}

\begin{Shaded}
\begin{Highlighting}[]
\DocumentationTok{\#\# Usando texreg para obtener una impresión bonita}
\NormalTok{texreg}\SpecialCharTok{::}\FunctionTok{screenreg}\NormalTok{(fit, }\AttributeTok{single.row =} \ConstantTok{TRUE}\NormalTok{)}
\end{Highlighting}
\end{Shaded}

\begin{verbatim}

========================================
                     Model 1            
----------------------------------------
(Intercept)             -0.01 (0.03)    
W                        1.97 (0.03) ***
rho                      0.54 (0.04) ***
----------------------------------------
Num. obs.             1000              
Parameters               4              
Log Likelihood       -1373.02           
AIC (Linear model)    2920.37           
AIC (Spatial model)   2754.05           
LR test: statistic     168.32           
LR test: p-value         0.00           
========================================
*** p < 0.001; ** p < 0.01; * p < 0.05
\end{verbatim}

La interpretación de este modelo es casi la misma que una regresión
lineal, con la diferencia de que tenemos el efecto de autocorrelación
(\texttt{rho}). Como se esperaba, el modelo obtuvo una estimación lo
suficientemente cercana al parámetro poblacional: \(\rho = 0.5\).

\section{Ejemplo de código: ALAAM}\label{ejemplo-de-cuxf3digo-alaam}

Hasta la fecha, no hay un paquete de R que implemente el marco ALAAM.
Sin embargo, puedes ajustar ALAAMs usando el software PNet desarrollado
por el grupo Melnet de la Universidad de Melbourne (haz clic
\href{https://www.melnet.org.au/pnet/}{aquí}).

Debido a las similitudes, los ALAAMs pueden implementarse usando ERGMs.
Debido a la novedad de esto, el ejemplo de código se dejará como un
posible proyecto de clase. Publicaremos un ejemplo completo después del
taller.

\section{Coevolución}\label{coevoluciuxf3n}

Finalmente, discutimos la coevolución cuando tanto la red como el
comportamiento están embebidos en un bucle de retroalimentación. La
coevolución debería ser la suposición predeterminada cuando se trata de
redes sociales. Sin embargo, los modelos capaces de capturar la
coevolución son difíciles de estimar. Aquí, discutiremos dos de tales
modelos: Modelos Estocásticos Orientados al Actor (o Modelos Siena)
(introducidos por primera vez en T. a. B. Snijders (1996); ver también
Tom A. B. Snijders (2017)) y Modelos de Red Exponencial Aleatorios de
familia exponencial {[}ERNMs,{]} una generalización de ERGMs (Z. Wang,
Fellows, and Handcock 2023; Fellows 2012).

\textbf{Siena} Los Modelos Estocásticos Orientados al Actor {[}SOAMs{]}
o Modelos Siena son modelos dinámicos de red y comportamiento que
describen la transición de un sistema de red dentro de dos o más puntos
de tiempo.

\textbf{ERNMs} Este modelo está estrechamente relacionado con los ERGMs,
con la diferencia de que incorporan una salida a nivel de vértice.
Conceptualmente, se está moviendo de tener una red aleatoria, a un
modelo donde una característica de vértice dada y la red son aleatorias:

\[
P_{\mathcal{Y}, \bm{{\theta}}}(\bm{{Y}}=\bm{{y}}|\bm{{X}}=\bm{{x}}) \to P_{\mathcal{Y}, \bm{{\theta}}}(\bm{{Y}}=\bm{{y}}, \bm{{X}}=\bm{{x}})
\]

\section{Ejemplo de código: Siena}\label{ejemplo-de-cuxf3digo-siena}

Este ejemplo fue adaptado del paquete de R \texttt{RSiena} (ver página
\texttt{?sienaGOF-auxiliary}). Comenzamos cargando el paquete y echando
un vistazo a los datos que usaremos:

\begin{Shaded}
\begin{Highlighting}[]
\FunctionTok{library}\NormalTok{(RSiena)}

\DocumentationTok{\#\# Visualizando la matriz de adyacencia y comportamiento}
\NormalTok{op }\OtherTok{\textless{}{-}} \FunctionTok{par}\NormalTok{(}\AttributeTok{mfrow=}\FunctionTok{c}\NormalTok{(}\DecValTok{2}\NormalTok{, }\DecValTok{2}\NormalTok{))}
\FunctionTok{image}\NormalTok{(s501, }\AttributeTok{main =} \StringTok{"Red: s501"}\NormalTok{)}
\FunctionTok{image}\NormalTok{(s502, }\AttributeTok{main =} \StringTok{"Red: s502"}\NormalTok{)}
\FunctionTok{hist}\NormalTok{(s50a[,}\DecValTok{1}\NormalTok{], }\AttributeTok{main =} \StringTok{"Comp1"}\NormalTok{)}
\FunctionTok{hist}\NormalTok{(s50a[,}\DecValTok{2}\NormalTok{], }\AttributeTok{main =} \StringTok{"Comp2"}\NormalTok{)}
\end{Highlighting}
\end{Shaded}

\pandocbounded{\includegraphics[keepaspectratio]{part-01-03-behavior_files/figure-pdf/siena-data-struct-1.pdf}}

\begin{Shaded}
\begin{Highlighting}[]
\FunctionTok{par}\NormalTok{(op)}
\end{Highlighting}
\end{Shaded}

El siguiente paso es el proceso de preparación de datos. \texttt{RSiena}
no recibe datos crudos tal como están. Necesitamos declarar
explícitamente las redes y la variable de resultado. Los modelos Siena
también pueden modelar cambios de red

\begin{Shaded}
\begin{Highlighting}[]
\DocumentationTok{\#\# Inicializando la variable dependiente (red)}
\NormalTok{mynet1 }\OtherTok{\textless{}{-}} \FunctionTok{sienaDependent}\NormalTok{(}\FunctionTok{array}\NormalTok{(}\FunctionTok{c}\NormalTok{(s501, s502), }\AttributeTok{dim=}\FunctionTok{c}\NormalTok{(}\DecValTok{50}\NormalTok{, }\DecValTok{50}\NormalTok{, }\DecValTok{2}\NormalTok{)))}
\NormalTok{mynet1}
\end{Highlighting}
\end{Shaded}

\begin{verbatim}
Type         oneMode             
Observations 2                   
Nodeset      Actors (50 elements)
\end{verbatim}

\begin{Shaded}
\begin{Highlighting}[]
\NormalTok{mybeh  }\OtherTok{\textless{}{-}} \FunctionTok{sienaDependent}\NormalTok{(s50a[,}\DecValTok{1}\SpecialCharTok{:}\DecValTok{2}\NormalTok{], }\AttributeTok{type=}\StringTok{"behavior"}\NormalTok{)}
\NormalTok{mybeh}
\end{Highlighting}
\end{Shaded}

\begin{verbatim}
Type         behavior            
Observations 2                   
Nodeset      Actors (50 elements)
\end{verbatim}

\begin{Shaded}
\begin{Highlighting}[]
\DocumentationTok{\#\# Covariables a nivel de nodo (artificiales)}
\NormalTok{mycov  }\OtherTok{\textless{}{-}} \FunctionTok{c}\NormalTok{(}\FunctionTok{rep}\NormalTok{(}\DecValTok{1}\SpecialCharTok{:}\DecValTok{3}\NormalTok{,}\DecValTok{16}\NormalTok{),}\DecValTok{1}\NormalTok{,}\DecValTok{2}\NormalTok{)}

\DocumentationTok{\#\# Covariables a nivel de enlace (también artificiales)}
\NormalTok{mydycov }\OtherTok{\textless{}{-}} \FunctionTok{matrix}\NormalTok{(}\FunctionTok{rep}\NormalTok{(}\DecValTok{1}\SpecialCharTok{:}\DecValTok{5}\NormalTok{, }\DecValTok{500}\NormalTok{), }\DecValTok{50}\NormalTok{, }\DecValTok{50}\NormalTok{) }
\end{Highlighting}
\end{Shaded}

\begin{Shaded}
\begin{Highlighting}[]
\DocumentationTok{\#\# Creando el objeto de datos}
\NormalTok{mydata }\OtherTok{\textless{}{-}} \FunctionTok{sienaDataCreate}\NormalTok{(mynet1, mybeh)}

\DocumentationTok{\#\# Agregando los efectos (¡primero obtenerlos!)}
\NormalTok{myeff }\OtherTok{\textless{}{-}} \FunctionTok{getEffects}\NormalTok{(mydata)}

\DocumentationTok{\#\# Nota que Siena agrega algunos efectos predeterminados}
\NormalTok{myeff}
\DocumentationTok{\#\#   name   effectName                  include fix   test  initialValue parm}
\DocumentationTok{\#\# 1 mynet1 basic rate parameter mynet1 TRUE    FALSE FALSE    4.69604   0   }
\DocumentationTok{\#\# 2 mynet1 outdegree (density)         TRUE    FALSE FALSE   {-}1.48852   0   }
\DocumentationTok{\#\# 3 mynet1 reciprocity                 TRUE    FALSE FALSE    0.00000   0   }
\DocumentationTok{\#\# 4 mybeh  rate mybeh period 1         TRUE    FALSE FALSE    0.70571   0   }
\DocumentationTok{\#\# 5 mybeh  mybeh linear shape          TRUE    FALSE FALSE    0.32247   0   }
\DocumentationTok{\#\# 6 mybeh  mybeh quadratic shape       TRUE    FALSE FALSE    0.00000   0}

\DocumentationTok{\#\# Agregando algunos efectos extra (automáticamente los imprime)}
\NormalTok{myeff }\OtherTok{\textless{}{-}} \FunctionTok{includeEffects}\NormalTok{(myeff, transTies, cycle3)}
\DocumentationTok{\#\#   effectNumber effectName      shortName include fix   test  initialValue parm}
\DocumentationTok{\#\# 1 41           3{-}cycles        cycle3    TRUE    FALSE FALSE          0   0   }
\DocumentationTok{\#\# 2 44           transitive ties transTies TRUE    FALSE FALSE          0   0}
\end{Highlighting}
\end{Shaded}

Para agregar más efectos, primero, llama a la función
\texttt{effectsDocumentation(myeff)}. Te mostrará explícitamente cómo
agregar un efecto particular. Por ejemplo, si quisiéramos agregar
exposición de red (\texttt{avExposure},) bajo la documentación de
\texttt{effectsDocumentation(myeff)} necesitamos pasar los siguientes
argumentos:

\begin{Shaded}
\begin{Highlighting}[]
\DocumentationTok{\#\# Y ahora, efecto de exposición}
\NormalTok{myeff }\OtherTok{\textless{}{-}} \FunctionTok{includeEffects}\NormalTok{(}
\NormalTok{  myeff,}
\NormalTok{  avExposure,}
  \CommentTok{\# Estos últimos tres son especificados por effectsDocum...}
  \AttributeTok{name         =} \StringTok{"mybeh"}\NormalTok{,}
  \AttributeTok{interaction1 =} \StringTok{"mynet1"}\NormalTok{,}
  \AttributeTok{type         =} \StringTok{"rate"}
\NormalTok{  )}
\end{Highlighting}
\end{Shaded}

\begin{verbatim}
  effectNumber effectName                            shortName  include fix  
1 462          average exposure effect on rate mybeh avExposure TRUE    FALSE
  test  initialValue parm
1 FALSE          0   0   
\end{verbatim}

El siguiente paso involucra crear el modelo con
(\texttt{sienaAlgorithmCreate},) donde especificamos todos los
parámetros para ajustar el modelo (ej., pasos MCMC.) Aquí, modificamos
los valores de \texttt{n3} y \texttt{nsub} a la mitad de los valores
predeterminados para reducir el tiempo que tomaría ajustar el modelo;
sin embargo esto degrada la calidad del ajuste.

\begin{Shaded}
\begin{Highlighting}[]
\DocumentationTok{\#\# Fases 2 y 3 más cortas, solo para ejemplo:}
\NormalTok{myalgorithm }\OtherTok{\textless{}{-}} \FunctionTok{sienaAlgorithmCreate}\NormalTok{(}
  \AttributeTok{nsub =} \DecValTok{2}\NormalTok{, }\AttributeTok{n3 =} \DecValTok{500}\NormalTok{, }\AttributeTok{seed =} \DecValTok{122}\NormalTok{, }\AttributeTok{projname =} \ConstantTok{NULL}
\NormalTok{  )}
\DocumentationTok{\#\# If you use this algorithm object, siena07 will create/use an output file /tmp/RtmpvM5MaK/Siena81965c168ec.txt .}
\DocumentationTok{\#\# This is a temporary file for this R session.}
 
\DocumentationTok{\#\# Ajustando e imprimiendo el modelo}
\NormalTok{ans }\OtherTok{\textless{}{-}} \FunctionTok{siena07}\NormalTok{(}
\NormalTok{  myalgorithm,}
  \AttributeTok{data =}\NormalTok{ mydata, }\AttributeTok{effects =}\NormalTok{ myeff,}
  \AttributeTok{returnDeps =} \ConstantTok{TRUE}\NormalTok{, }\AttributeTok{batch =} \ConstantTok{TRUE}
\NormalTok{  )}
\DocumentationTok{\#\# }
\DocumentationTok{\#\# Start phase 0 }
\DocumentationTok{\#\# theta:  4.696 {-}1.489  0.000  0.000  0.000  0.706  0.000  0.322  0.000 }
\DocumentationTok{\#\# }
\DocumentationTok{\#\# Start phase 1 }
\DocumentationTok{\#\# Phase 1 Iteration 1 Progress: 0\%}
\DocumentationTok{\#\# Phase 1 Iteration 2 Progress: 0\%}
\DocumentationTok{\#\# Phase 1 Iteration 3 Progress: 0\%}
\DocumentationTok{\#\# Phase 1 Iteration 4 Progress: 0\%}
\DocumentationTok{\#\# Phase 1 Iteration 5 Progress: 0\%}
\DocumentationTok{\#\# Phase 1 Iteration 10 Progress: 1\%}
\DocumentationTok{\#\# Phase 1 Iteration 15 Progress: 1\%}
\DocumentationTok{\#\# Phase 1 Iteration 20 Progress: 1\%}
\DocumentationTok{\#\# Phase 1 Iteration 25 Progress: 2\%}
\DocumentationTok{\#\# Phase 1 Iteration 30 Progress: 2\%}
\DocumentationTok{\#\# Phase 1 Iteration 35 Progress: 2\%}
\DocumentationTok{\#\# Phase 1 Iteration 40 Progress: 3\%}
\DocumentationTok{\#\# Phase 1 Iteration 45 Progress: 3\%}
\DocumentationTok{\#\# Phase 1 Iteration 50 Progress: 3\%}
\DocumentationTok{\#\# theta:  5.380 {-}1.734  0.481  0.147  0.166  1.168 {-}0.340  0.250  0.140 }
\DocumentationTok{\#\# }
\DocumentationTok{\#\# Start phase 2.1}
\DocumentationTok{\#\# Phase 2 Subphase 1 Iteration 1 Progress: 33\%}
\DocumentationTok{\#\# Phase 2 Subphase 1 Iteration 2 Progress: 33\%}
\DocumentationTok{\#\# theta  6.044 {-}1.877  0.840  0.300  0.473  1.096 {-}0.332  0.112  0.191 }
\DocumentationTok{\#\# ac  0.375 {-}0.183  3.835  4.574 23.412  0.922  0.908  0.626  3.302 }
\DocumentationTok{\#\# Phase 2 Subphase 1 Iteration 3 Progress: 33\%}
\DocumentationTok{\#\# Phase 2 Subphase 1 Iteration 4 Progress: 33\%}
\DocumentationTok{\#\# theta  6.1075 {-}2.0372  1.2512  0.0341  0.5207  1.2593 {-}0.1534  0.3018  0.1928 }
\DocumentationTok{\#\# ac  0.9859  0.2280 {-}0.0703 {-}2.2973 {-}2.7455  1.1518  1.0749  0.8732  0.5399 }
\DocumentationTok{\#\# Phase 2 Subphase 1 Iteration 5 Progress: 33\%}
\DocumentationTok{\#\# Phase 2 Subphase 1 Iteration 6 Progress: 33\%}
\DocumentationTok{\#\# theta  6.8650 {-}2.3185  1.7577  0.2694  0.7636  1.1997 {-}0.0433  0.3000  0.1762 }
\DocumentationTok{\#\# ac  0.4789  0.4883  0.0199 {-}1.9449 {-}2.2609  1.1444  1.0397  0.9282  0.6470 }
\DocumentationTok{\#\# Phase 2 Subphase 1 Iteration 7 Progress: 33\%}
\DocumentationTok{\#\# Phase 2 Subphase 1 Iteration 8 Progress: 33\%}
\DocumentationTok{\#\# theta  6.801140 {-}2.485273  1.966471  0.301197  0.817324  1.228010  0.000627  0.377719  0.072274 }
\DocumentationTok{\#\# ac  0.4538  0.4933 {-}0.0158 {-}1.9326 {-}2.2588  1.1102  1.0395  0.9239  0.6905 }
\DocumentationTok{\#\# Phase 2 Subphase 1 Iteration 9 Progress: 33\%}
\DocumentationTok{\#\# Phase 2 Subphase 1 Iteration 10 Progress: 33\%}
\DocumentationTok{\#\# theta  6.1258 {-}2.5124  1.8704  0.1206  0.6054  1.4538 {-}0.0145  0.5091 {-}0.0671 }
\DocumentationTok{\#\# ac  0.472  0.103 {-}0.330 {-}1.625 {-}1.991  1.152  1.090  0.767  0.514 }
\DocumentationTok{\#\# Phase 2 Subphase 1 Iteration 1 Progress: 33\%}
\DocumentationTok{\#\# Phase 2 Subphase 1 Iteration 2 Progress: 33\%}
\DocumentationTok{\#\# Phase 2 Subphase 1 Iteration 3 Progress: 33\%}
\DocumentationTok{\#\# Phase 2 Subphase 1 Iteration 4 Progress: 33\%}
\DocumentationTok{\#\# Phase 2 Subphase 1 Iteration 5 Progress: 33\%}
\DocumentationTok{\#\# Phase 2 Subphase 1 Iteration 6 Progress: 34\%}
\DocumentationTok{\#\# Phase 2 Subphase 1 Iteration 7 Progress: 34\%}
\DocumentationTok{\#\# Phase 2 Subphase 1 Iteration 8 Progress: 34\%}
\DocumentationTok{\#\# Phase 2 Subphase 1 Iteration 9 Progress: 34\%}
\DocumentationTok{\#\# Phase 2 Subphase 1 Iteration 10 Progress: 34\%}
\DocumentationTok{\#\# theta  6.4976 {-}2.5900  1.9265  0.2750  0.8543  1.0420  0.0446  0.3234 {-}0.0686 }
\DocumentationTok{\#\# ac {-}0.212 {-}0.697 {-}0.818 {-}0.831 {-}0.765 {-}0.442 {-}0.268 {-}0.240 {-}0.267 }
\DocumentationTok{\#\# theta:  6.4976 {-}2.5900  1.9265  0.2750  0.8543  1.0420  0.0446  0.3234 {-}0.0686 }
\DocumentationTok{\#\# }
\DocumentationTok{\#\# Start phase 2.2}
\DocumentationTok{\#\# Phase 2 Subphase 2 Iteration 1 Progress: 48\%}
\DocumentationTok{\#\# Phase 2 Subphase 2 Iteration 2 Progress: 48\%}
\DocumentationTok{\#\# Phase 2 Subphase 2 Iteration 3 Progress: 48\%}
\DocumentationTok{\#\# Phase 2 Subphase 2 Iteration 4 Progress: 48\%}
\DocumentationTok{\#\# Phase 2 Subphase 2 Iteration 5 Progress: 48\%}
\DocumentationTok{\#\# Phase 2 Subphase 2 Iteration 6 Progress: 48\%}
\DocumentationTok{\#\# Phase 2 Subphase 2 Iteration 7 Progress: 49\%}
\DocumentationTok{\#\# Phase 2 Subphase 2 Iteration 8 Progress: 49\%}
\DocumentationTok{\#\# Phase 2 Subphase 2 Iteration 9 Progress: 49\%}
\DocumentationTok{\#\# Phase 2 Subphase 2 Iteration 10 Progress: 49\%}
\DocumentationTok{\#\# theta  6.7988 {-}2.5810  1.9335  0.3941  0.7843  1.1111  0.0358  0.3252 {-}0.0415 }
\DocumentationTok{\#\# ac {-}0.0757 {-}0.3672 {-}0.4025 {-}0.3676 {-}0.4157 {-}0.0166  0.0222 {-}0.0874  0.0651 }
\DocumentationTok{\#\# theta:  6.7988 {-}2.5810  1.9335  0.3941  0.7843  1.1111  0.0358  0.3252 {-}0.0415 }
\DocumentationTok{\#\# }
\DocumentationTok{\#\# Start phase 3 }
\DocumentationTok{\#\# Phase 3 Iteration 500 Progress 100\%}

\NormalTok{ans}
\DocumentationTok{\#\# Estimates, standard errors and convergence t{-}ratios}
\DocumentationTok{\#\# }
\DocumentationTok{\#\#                                                 Estimate   Standard   Convergence }
\DocumentationTok{\#\#                                                              Error      t{-}ratio   }
\DocumentationTok{\#\# Network Dynamics }
\DocumentationTok{\#\#   1. rate basic rate parameter mynet1            6.7988  ( 1.2496   )   {-}0.0460   }
\DocumentationTok{\#\#   2. eval outdegree (density)                   {-}2.5810  ( 0.1505   )    0.0113   }
\DocumentationTok{\#\#   3. eval reciprocity                            1.9335  ( 0.2627   )    0.0205   }
\DocumentationTok{\#\#   4. eval 3{-}cycles                               0.3941  ( 0.2742   )   {-}0.0658   }
\DocumentationTok{\#\#   5. eval transitive ties                        0.7843  ( 0.2379   )   {-}0.0014   }
\DocumentationTok{\#\# }
\DocumentationTok{\#\# Behavior Dynamics}
\DocumentationTok{\#\#   6. rate rate mybeh period 1                    1.1111  ( 1.3096   )   {-}0.0199   }
\DocumentationTok{\#\#   7. rate average exposure effect on rate mybeh  0.0358  ( 0.4373   )   {-}0.0190   }
\DocumentationTok{\#\#   8. eval mybeh linear shape                     0.3252  ( 0.2329   )   {-}0.0048   }
\DocumentationTok{\#\#   9. eval mybeh quadratic shape                 {-}0.0415  ( 0.1139   )    0.0992   }
\DocumentationTok{\#\# }
\DocumentationTok{\#\# Overall maximum convergence ratio:    0.2163 }
\DocumentationTok{\#\# }
\DocumentationTok{\#\# }
\DocumentationTok{\#\# Total of 940 iteration steps.}
\end{Highlighting}
\end{Shaded}

Como regla general, valores t absolutos por debajo de 0.1 muestran buena
convergencia, por debajo de 0.2 decimos ``razonablemente bien,'' y por
encima no hay convergencia. Resaltemos dos de los efectos que tenemos en
nuestro modelo

\begin{enumerate}
\def\labelenumi{\arabic{enumi}.}
\item
  Los vínculos transitivos (número cinco) son positivos 0.78 con un
  valor t menor que 0.01. Por lo tanto, decimos que la red tiene una
  tendencia hacia la transitividad (equilibrio) que es significativa.
\item
  El efecto de exposición (número siete) también es positivo, pero
  pequeño, 0.03, pero aún significativo (valor t de -0.01)
\end{enumerate}

Como con ERGMs, también hacemos bondad de ajuste:

\begin{Shaded}
\begin{Highlighting}[]
\FunctionTok{sienaGOF}\NormalTok{(ans, OutdegreeDistribution, }\AttributeTok{varName=}\StringTok{"mynet1"}\NormalTok{) }\SpecialCharTok{|\textgreater{}}
  \FunctionTok{plot}\NormalTok{()}
\end{Highlighting}
\end{Shaded}

\pandocbounded{\includegraphics[keepaspectratio]{part-01-03-behavior_files/figure-pdf/siena-gof-1.pdf}}

\begin{Shaded}
\begin{Highlighting}[]
\FunctionTok{sienaGOF}\NormalTok{(ans, BehaviorDistribution, }\AttributeTok{varName =} \StringTok{"mybeh"}\NormalTok{) }\SpecialCharTok{|\textgreater{}} 
  \FunctionTok{plot}\NormalTok{()}
\end{Highlighting}
\end{Shaded}

\begin{verbatim}
Note: some statistics are not plotted because their variance is 0.
This holds for the statistic: 5.
\end{verbatim}

\pandocbounded{\includegraphics[keepaspectratio]{part-01-03-behavior_files/figure-pdf/siena-gof-2.pdf}}

\section{Ejemplo de código: ERNM}\label{ejemplo-de-cuxf3digo-ernm}

Hasta la fecha, no hay un lanzamiento CRAN para el modelo ERNM. La única
implementación de la que estoy al tanto es de uno de los autores
principales, que está disponible en GitHub:
\url{https://github.com/fellstat/ernm}. Desafortunadamente, la versión
actual del paquete parece estar rota.

Al igual que el caso ALAAM, como los ERNMs están estrechamente
relacionados con los ERGMS, ¡construir un ejemplo usando el paquete ERGM
podría ser una gran oportunidad para un proyecto de clase!

\chapter{Modelos de Grafos Aleatorios
Exponenciales}\label{modelos-de-grafos-aleatorios-exponenciales}

\begin{tcolorbox}[enhanced jigsaw, colback=white, opacityback=0, coltitle=black, title=\textcolor{quarto-callout-warning-color}{\faExclamationTriangle}\hspace{0.5em}{Nota de Traducción}, bottomrule=.15mm, colbacktitle=quarto-callout-warning-color!10!white, toptitle=1mm, colframe=quarto-callout-warning-color-frame, titlerule=0mm, rightrule=.15mm, leftrule=.75mm, breakable, bottomtitle=1mm, left=2mm, arc=.35mm, toprule=.15mm, opacitybacktitle=0.6]

Esta versión del capítulo fue traducida de manera automática utilizando
IA. El capítulo aún no ha sido revisado por un humano.

\end{tcolorbox}

Recomiendo encarecidamente leer la viñeta del paquete de R
\texttt{ergm}.

\begin{Shaded}
\begin{Highlighting}[]
\FunctionTok{vignette}\NormalTok{(}\StringTok{"ergm"}\NormalTok{, }\AttributeTok{package=}\StringTok{"ergm"}\NormalTok{)}
\end{Highlighting}
\end{Shaded}

\begin{quote}
El propósito de los ERGMs, en pocas palabras, es describir de manera
parsimoniosa las fuerzas de selección local que dan forma a la
estructura global de una red. Para este fin, un conjunto de datos de
red, como los que se muestran en la Figura 1, puede ser considerado como
la respuesta en un modelo de regresión, donde los predictores son cosas
como ``propensión de individuos del mismo sexo a formar asociaciones'' o
``propensión de individuos a formar triángulos de asociaciones''. En la
Figura 1(b), por ejemplo, es evidente que los nodos individuales parecen
agruparse en grupos de las mismas etiquetas numéricas (que resultan ser
las calificaciones de los estudiantes, del 7 al 12); por lo tanto, un
ERGM puede ayudarnos a cuantificar la fuerza de este efecto intra-grupo.

--- (David R. Hunter et al. 2008)
\end{quote}

\begin{figure}[H]

{\centering \pandocbounded{\includegraphics[keepaspectratio]{img/hunter2008.png}}

}

\caption{Fuente: Hunter et al.~(2008)}

\end{figure}%

En pocas palabras, usamos ERGMs como una interpretación paramétrica de
la distribución de \(\mathbf{Y}\), que toma la forma canónica:

\begin{equation}\phantomsection\label{eq-main-ergm}{
{\mathbb{P}_{\mathcal{Y}}\left(\mathbf{Y}=\mathbf{y};\mathbf{\theta}\right) } = \frac{\text{exp}\left\{\theta^{\text{T}}\mathbf{g}(\mathbf{y})\right\}}{\kappa\left(\theta, \mathcal{Y}\right)},\quad\mathbf{y}\in\mathcal{Y} 
}\end{equation}

Donde \(\theta\in\Omega\subset\mathbb{R}^q\) es el vector de
coeficientes del modelo y \(\mathbf{g}(\mathbf{y})\) es un
\emph{q}-vector de estadísticas basadas en la matriz de adyacencia
\(\mathbf{y}\).

El modelo Equation~\ref{eq-main-ergm} puede expandirse reemplazando
\(\mathbf{g}(\mathbf{y})\) con \(\mathbf{g}(\mathbf{y}, \mathbf{X})\)
para permitir información adicional de covariables \(\mathbf{X}\) sobre
la red. El denominador
\(\kappa\left(\theta,\mathcal{Y}\right) = \sum_{\mathbf{y}\in\mathcal{Y}}\text{exp}\left\{\theta^{\text{T}}\mathbf{g}(\mathbf{y})\right\}\)
es el factor normalizador que asegura que la ecuación
Equation~\ref{eq-main-ergm} sea una distribución de probabilidad
legítima. Incluso después de fijar \(\mathcal{Y}\) para que sean todas
las redes que tienen tamaño \(n\), el tamaño de \(\mathcal{Y}\) hace que
este tipo de modelo estadístico sea difícil de estimar ya que hay
\(N = 2^{n(n-1)}\) redes posibles! (David R. Hunter et al. 2008)

Desarrollos posteriores incluyen nuevas estructuras de dependencia para
considerar efectos de vecindario más generales. Estos modelos relajan
las suposiciones de dependencia markoviana de un paso, permitiendo la
investigación de configuraciones de mayor alcance, como rutas más largas
en la red o ciclos más grandes (Pattison y Robins 2002). Se han
desarrollado modelos para estructuras de red bipartitas (Faust y
Skvoretz 1999) y tripartitas (Mische y Robins 2000). (David R. Hunter et
al. 2008, 9)

\section{Un ejemplo ingenuo}\label{un-ejemplo-ingenuo}

En el caso más simple, los ERGMs equivalen a una regresión logística.
Por simple, me refiero a casos sin términos markovianos--motivos que
involucran más de un enlace--por ejemplo, el grafo de Bernoulli. En el
grafo de Bernoulli, los vínculos son independientes, por lo que la
presencia/ausencia de un vínculo entre los nodos \(i\) y \(j\) no
afectará la presencia/ausencia de un vínculo entre los nodos \(k\) y
\(l\).

Ajustemos un ERGM usando el conjunto de datos \texttt{sampson} en el
paquete \texttt{ergm}.

\begin{Shaded}
\begin{Highlighting}[]
\FunctionTok{library}\NormalTok{(ergm)}
\FunctionTok{library}\NormalTok{(netplot)}
\FunctionTok{data}\NormalTok{(}\StringTok{"sampson"}\NormalTok{)}
\FunctionTok{nplot}\NormalTok{(samplike)}
\end{Highlighting}
\end{Shaded}

\pandocbounded{\includegraphics[keepaspectratio]{part-01-04-ergms_files/figure-pdf/part-01-04-loading-data-1.pdf}}

Usar \texttt{ergm} para ajustar un grafo de Bernoulli requiere usar el
término \texttt{edges}, que cuenta cuántos vínculos hay en el grafo:

\begin{Shaded}
\begin{Highlighting}[]
\NormalTok{ergm\_fit }\OtherTok{\textless{}{-}} \FunctionTok{ergm}\NormalTok{(samplike }\SpecialCharTok{\textasciitilde{}}\NormalTok{ edges)}
\DocumentationTok{\#\# Starting maximum pseudolikelihood estimation (MPLE):}
\DocumentationTok{\#\# Obtaining the responsible dyads.}
\DocumentationTok{\#\# Evaluating the predictor and response matrix.}
\DocumentationTok{\#\# Maximizing the pseudolikelihood.}
\DocumentationTok{\#\# Finished MPLE.}
\DocumentationTok{\#\# Evaluating log{-}likelihood at the estimate.}
\end{Highlighting}
\end{Shaded}

Dado que esto es equivalente a una regresión logística, podemos usar la
función \texttt{glm} para ajustar el mismo modelo. Primero, necesitamos
preparar los datos para que podamos pasarlos a \texttt{glm}:

\begin{Shaded}
\begin{Highlighting}[]
\NormalTok{y }\OtherTok{\textless{}{-}} \FunctionTok{sort}\NormalTok{(}\FunctionTok{as.vector}\NormalTok{(}\FunctionTok{as.matrix}\NormalTok{(samplike)))}
\NormalTok{y }\OtherTok{\textless{}{-}}\NormalTok{ y[}\SpecialCharTok{{-}}\FunctionTok{c}\NormalTok{(}\DecValTok{1}\SpecialCharTok{:}\DecValTok{18}\NormalTok{)] }\CommentTok{\# Eliminamos la diagonal del modelo, que es todo 0.}
\NormalTok{y}
\DocumentationTok{\#\#   [1] 0 0 0 0 0 0 0 0 0 0 0 0 0 0 0 0 0 0 0 0 0 0 0 0 0 0 0 0 0 0 0 0 0 0 0 0 0}
\DocumentationTok{\#\#  [38] 0 0 0 0 0 0 0 0 0 0 0 0 0 0 0 0 0 0 0 0 0 0 0 0 0 0 0 0 0 0 0 0 0 0 0 0 0}
\DocumentationTok{\#\#  [75] 0 0 0 0 0 0 0 0 0 0 0 0 0 0 0 0 0 0 0 0 0 0 0 0 0 0 0 0 0 0 0 0 0 0 0 0 0}
\DocumentationTok{\#\# [112] 0 0 0 0 0 0 0 0 0 0 0 0 0 0 0 0 0 0 0 0 0 0 0 0 0 0 0 0 0 0 0 0 0 0 0 0 0}
\DocumentationTok{\#\# [149] 0 0 0 0 0 0 0 0 0 0 0 0 0 0 0 0 0 0 0 0 0 0 0 0 0 0 0 0 0 0 0 0 0 0 0 0 0}
\DocumentationTok{\#\# [186] 0 0 0 0 0 0 0 0 0 0 0 0 0 0 0 0 0 0 0 0 0 0 0 0 0 0 0 0 0 0 0 0 0 1 1 1 1}
\DocumentationTok{\#\# [223] 1 1 1 1 1 1 1 1 1 1 1 1 1 1 1 1 1 1 1 1 1 1 1 1 1 1 1 1 1 1 1 1 1 1 1 1 1}
\DocumentationTok{\#\# [260] 1 1 1 1 1 1 1 1 1 1 1 1 1 1 1 1 1 1 1 1 1 1 1 1 1 1 1 1 1 1 1 1 1 1 1 1 1}
\DocumentationTok{\#\# [297] 1 1 1 1 1 1 1 1 1 1}
\end{Highlighting}
\end{Shaded}

Ahora podemos ajustar el modelo GLM:

\begin{Shaded}
\begin{Highlighting}[]
\NormalTok{glm\_fit }\OtherTok{\textless{}{-}} \FunctionTok{glm}\NormalTok{(y}\SpecialCharTok{\textasciitilde{}}\DecValTok{1}\NormalTok{, }\AttributeTok{family=}\FunctionTok{binomial}\NormalTok{(}\StringTok{"logit"}\NormalTok{))}
\end{Highlighting}
\end{Shaded}

Los coeficientes de ambos ERGM y GLM deberían coincidir:

\begin{Shaded}
\begin{Highlighting}[]
\NormalTok{glm\_fit}
\DocumentationTok{\#\# }
\DocumentationTok{\#\# Call:  glm(formula = y \textasciitilde{} 1, family = binomial("logit"))}
\DocumentationTok{\#\# }
\DocumentationTok{\#\# Coefficients:}
\DocumentationTok{\#\# (Intercept)  }
\DocumentationTok{\#\#     {-}0.9072  }
\DocumentationTok{\#\# }
\DocumentationTok{\#\# Degrees of Freedom: 305 Total (i.e. Null);  305 Residual}
\DocumentationTok{\#\# Null Deviance:       367.2 }
\DocumentationTok{\#\# Residual Deviance: 367.2     AIC: 369.2}
\NormalTok{ergm\_fit}
\DocumentationTok{\#\# }
\DocumentationTok{\#\# Call:}
\DocumentationTok{\#\# ergm(formula = samplike \textasciitilde{} edges)}
\DocumentationTok{\#\# }
\DocumentationTok{\#\# Maximum Likelihood Coefficients:}
\DocumentationTok{\#\#   edges  }
\DocumentationTok{\#\# {-}0.9072}
\end{Highlighting}
\end{Shaded}

Además, en el caso del grafo de Bernoulli, podemos obtener la estimación
usando la función Logit:

\begin{Shaded}
\begin{Highlighting}[]
\NormalTok{pr }\OtherTok{\textless{}{-}} \FunctionTok{mean}\NormalTok{(y)}
\CommentTok{\# Función Logit:}
\CommentTok{\# Alternativamente podríamos haber usado log(pr) {-} log(1{-}pr)}
\FunctionTok{qlogis}\NormalTok{(pr)}
\DocumentationTok{\#\# [1] {-}0.9071582}
\end{Highlighting}
\end{Shaded}

De nuevo, el mismo resultado. El grafo de Bernoulli no es el único
modelo ERGM que puede ajustarse usando una regresión logística. Además,
si todos los términos del modelo son términos no-Markov, \texttt{ergm}
automáticamente usa por defecto una regresión logística.

\section{Estimación de ERGMs}\label{estimaciuxf3n-de-ergms}

El objetivo final es realizar inferencia estadística sobre el modelo
propuesto. En un entorno \emph{estándar}, podríamos usar Estimación de
Máxima Verosimilitud (MLE), que consiste en encontrar los parámetros del
modelo \(\theta\) que, dados los datos observados, maximicen la
verosimilitud del modelo. Para esto último, generalmente usamos
\href{https://en.wikipedia.org/wiki/Newton\%27s_method_in_optimization}{el
método de Newton}. El método de Newton requiere calcular la
log-verosimilitud del modelo, lo que puede ser desafiante en ERGMs.

Para ERGMs, dado que parte de la verosimilitud involucra una constante
normalizadora que es una función de todas las redes posibles, esto no es
tan directo como en el entorno regular. Debido a esto, la mayoría de los
métodos de estimación se basan en simulaciones.

En \texttt{statnet}, el método de estimación predeterminado se basa en
un método propuesto por (Geyer and Thompson 1992), MLE de Cadena de
Markov, que usa Monte Carlo de Cadena de Markov para simular redes y una
versión modificada del algoritmo Newton-Raphson para estimar los
parámetros.

La idea del MC-MLE para esta familia de modelos estadísticos es
aproximar la expectativa de las razones de constantes normalizadoras
usando la ley de los grandes números. En particular, lo siguiente:

\begin{align*}
\frac{\kappa\left(\theta,\mathcal{Y}\right)}{\kappa\left(\theta_0,\mathcal{Y}\right)} & = 
  \frac{
    \sum_{\mathbf{y}\in\mathcal{Y}}\text{exp}\left\{\theta^{\text{T}}\mathbf{g}(\mathbf{y})\right\}}{ 
    \sum_{\mathbf{y}\in\mathcal{Y}}\text{exp}\left\{\theta_0^{\text{T}}\mathbf{g}(\mathbf{y})\right\} 
  } \\
& = \sum_{\mathbf{y}\in\mathcal{Y}}\left( %
  \frac{1}{%
    \sum_{\mathbf{y}\in\mathcal{Y}\text{exp}\left\{\theta_0^{\text{T}}\mathbf{g}(\mathbf{y})\right\}}%
  } \times %
  \text{exp}\left\{\theta^{\text{T}}\mathbf{g}(\mathbf{y})\right\} %
  \right) \\
& = \sum_{\mathbf{y}\in\mathcal{Y}}\left( %
  \frac{\text{exp}\left\{\theta_0^{\text{T}}\mathbf{g}(\mathbf{y})\right\}}{%
    \sum_{\mathbf{y}\in\mathcal{Y}\text{exp}\left\{\theta_0^{\text{T}}\mathbf{g}(\mathbf{y})\right\}}%
  } \times %
  \text{exp}\left\{(\theta - \theta_0)^{\text{T}}\mathbf{g}(\mathbf{y})\right\} %
  \right) \\
& = \sum_{\mathbf{y}\in\mathcal{Y}}\left( %
  {\mathbb{P}_{\times }\left(Y = y|\mathcal{Y}, \theta_0\right) }%
  \text{exp}\left\{(\theta - \theta_0)^{\text{T}}\mathbf{g}(\mathbf{y})\right\} %
  \right) \\
& = \text{E}_{\theta_0}\left(\text{exp}\left\{(\theta - \theta_0)^{\text{T}}\mathbf{g}(\mathbf{y})\right\} \right)
\end{align*}

En particular, el algoritmo MC-MLE usa este hecho para maximizar la
razón de log-verosimilitud. La función objetivo misma puede aproximarse
simulando \(m\) redes de la distribución con parámetro \(\theta_0\):

\[
l(\theta) - l(\theta_0) \approx (\theta - \theta_0)^{\text{T}}\mathbf{g}(\mathbf{y}_{obs}) - 
\text{log}{\left[\frac{1}{m}\sum_{i = 1}^m\text{exp}\left\{(\theta-\theta_0)^{\text{T}}\right\}\mathbf{g}(\mathbf{Y}_i)\right]}
\]

Para más detalles, ver (David R. Hunter et al. 2008). Un bosquejo del
algoritmo sigue:

\begin{enumerate}
\def\labelenumi{\arabic{enumi}.}
\item
  Inicializar el algoritmo con una conjetura inicial de \(\theta\),
  llamarlo \(\theta^{(t)}\) (debe ser una conjetura bastante buena)
\item
  Mientras (no haya convergencia) hacer:
\end{enumerate}

\begin{enumerate}
\def\labelenumi{\alph{enumi}.}
\item
  Usando \(\theta^{(t)}\), simular \(M\) redes por medio de pequeños
  cambios en la \(\mathbf{Y}_{obs}\) (la red observada). Esta parte se
  hace usando un método de muestreo de importancia que pondera cada red
  propuesta por su verosimilitud condicional en \(\theta^{(t)}\)
\item
  Con las redes simuladas, podemos hacer el paso de Newton para
  actualizar el parámetro \(\theta^{(t)}\) (esta es la parte de
  iteración en el paquete \texttt{ergm}):
  \(\theta^{(t)}\to\theta^{(t+1)}\).
\item
  Si se ha alcanzado la convergencia (lo que usualmente significa que
  \(\theta^{(t)}\) y \(\theta^{(t + 1)}\) no son muy diferentes),
  entonces parar; de lo contrario, ir al paso a.
\end{enumerate}

Lusher, Koskinen, and Robins (2013);Admiraal and Handcock (2006);T. A.
Snijders (2002);P. Wang et al. (2009) proporcionan detalles sobre el
algoritmo usado por PNet (el mismo que el usado en \texttt{RSiena}), y
Lusher, Koskinen, and Robins (2013) proporciona una breve discusión
sobre las diferencias entre \texttt{ergm} y \texttt{PNet}.

\section{\texorpdfstring{El paquete
\texttt{ergm}}{El paquete ergm}}\label{el-paquete-ergm}

El paquete de R \texttt{ergm} (Handcock et al. 2023)

De la sección anterior:\footnote{Puedes descargar el archivo 03.rda
  desde \href{https://github.com/gvegayon/appliedsnar}{este enlace}.}

\begin{Shaded}
\begin{Highlighting}[]
\FunctionTok{library}\NormalTok{(igraph)}

\FunctionTok{library}\NormalTok{(dplyr)}

\FunctionTok{load}\NormalTok{(}\StringTok{"03.rda"}\NormalTok{)}
\end{Highlighting}
\end{Shaded}

En esta sección, usaremos el paquete \texttt{ergm} (del conjunto de
paquetes \texttt{statnet} Handcock et al. (2023),) y el paquete
\texttt{intergraph} (Bojanowski 2023). Este último proporciona funciones
para ir y venir entre objetos \texttt{igraph} y \texttt{network} de los
paquetes \texttt{igraph} y \texttt{network} respectivamente\footnote{Sí,
  las clases tienen el mismo nombre que los paquetes.}

\begin{Shaded}
\begin{Highlighting}[]
\FunctionTok{library}\NormalTok{(ergm)}
\FunctionTok{library}\NormalTok{(intergraph)}
\end{Highlighting}
\end{Shaded}

Como una nota lateral bastante importante, el orden en que se cargan los
paquetes de R importa. ¿Por qué es importante mencionarlo ahora? Bueno,
resulta que al menos un par de funciones en el paquete \texttt{network}
tienen el mismo nombre que algunas funciones en el paquete
\texttt{igraph}. Cuando se carga el paquete \texttt{ergm}, dado que
depende de \texttt{network}, cargará el paquete \texttt{network}
primero, lo que \emph{enmascarará} algunas funciones en \texttt{igraph}.
Esto se vuelve evidente una vez que cargas \texttt{ergm} después de
cargar \texttt{igraph}:

\begin{verbatim}
Los siguientes objetos están enmascarados desde 'package:igraph':

  add.edges, add.vertices, %c%, delete.edges, delete.vertices, get.edge.attribute, get.edges,
  get.vertex.attribute, is.bipartite, is.directed, list.edge.attributes, list.vertex.attributes, %s%,
  set.edge.attribute, set.vertex.attribute
\end{verbatim}

¿Cuáles son las implicaciones de esto? Si llamas la función
\texttt{list.edge.attributes} para un objeto de clase \texttt{igraph} R
devolverá un error ya que la primera función que coincide con ese nombre
viene del paquete \texttt{network}! Para evitar esto puedes usar la
notación de doble dos puntos:

\begin{Shaded}
\begin{Highlighting}[]
\NormalTok{igraph}\SpecialCharTok{::}\FunctionTok{list.edge.attributes}\NormalTok{(my\_igraph\_object)}
\NormalTok{network}\SpecialCharTok{::}\FunctionTok{list.edge.attributes}\NormalTok{(my\_network\_object)}
\end{Highlighting}
\end{Shaded}

De todos modos\ldots{} Usando la función \texttt{asNetwork}, podemos
coercionar el objeto \texttt{igraph} en un objeto network para que
podamos usarlo con la función \texttt{ergm}:

\begin{Shaded}
\begin{Highlighting}[]
\CommentTok{\# Creando la nueva red}
\NormalTok{network\_111 }\OtherTok{\textless{}{-}}\NormalTok{ intergraph}\SpecialCharTok{::}\FunctionTok{asNetwork}\NormalTok{(ig\_year1\_111)}

\CommentTok{\# Ejecutando un ergm simple (solo ajustando cuenta de enlaces)}
\FunctionTok{ergm}\NormalTok{(network\_111 }\SpecialCharTok{\textasciitilde{}}\NormalTok{ edges)}
\DocumentationTok{\#\# Warning in ergm.getnetwork(formula): This network contains loops}
\DocumentationTok{\#\# Starting maximum pseudolikelihood estimation (MPLE):}
\DocumentationTok{\#\# Obtaining the responsible dyads.}
\DocumentationTok{\#\# Evaluating the predictor and response matrix.}
\DocumentationTok{\#\# Maximizing the pseudolikelihood.}
\DocumentationTok{\#\# Finished MPLE.}
\DocumentationTok{\#\# Evaluating log{-}likelihood at the estimate.}
\DocumentationTok{\#\# }
\DocumentationTok{\#\# Call:}
\DocumentationTok{\#\# ergm(formula = network\_111 \textasciitilde{} edges)}
\DocumentationTok{\#\# }
\DocumentationTok{\#\# Maximum Likelihood Coefficients:}
\DocumentationTok{\#\#  edges  }
\DocumentationTok{\#\# {-}4.734}
\end{Highlighting}
\end{Shaded}

Entonces, ¿qué pasó aquí? Obtuvimos una advertencia. Resulta que nuestra
red tiene bucles (¡no pensé en eso antes!). Echemos un vistazo a eso con
la función \texttt{which\_loop}

\begin{Shaded}
\begin{Highlighting}[]
\FunctionTok{E}\NormalTok{(ig\_year1\_111)[}\FunctionTok{which\_loop}\NormalTok{(ig\_year1\_111)]}
\DocumentationTok{\#\# + 1/2638 edge from a9038c0 (vertex names):}
\DocumentationTok{\#\# [1] 1110111{-}\textgreater{}1110111}
\end{Highlighting}
\end{Shaded}

Podemos deshacernos de estos usando el \texttt{igraph::-.igraph}.
Eliminemos los aislados usando el mismo operador

\begin{Shaded}
\begin{Highlighting}[]
\CommentTok{\# Creando la nueva red}
\NormalTok{network\_111 }\OtherTok{\textless{}{-}}\NormalTok{ ig\_year1\_111}

\CommentTok{\# Eliminando bucles}
\NormalTok{network\_111 }\OtherTok{\textless{}{-}}\NormalTok{ network\_111 }\SpecialCharTok{{-}} \FunctionTok{E}\NormalTok{(network\_111)[}\FunctionTok{which}\NormalTok{(}\FunctionTok{which\_loop}\NormalTok{(network\_111))]}

\CommentTok{\# Eliminando aislados}
\NormalTok{network\_111 }\OtherTok{\textless{}{-}}\NormalTok{ network\_111 }\SpecialCharTok{{-}} \FunctionTok{which}\NormalTok{(}\FunctionTok{degree}\NormalTok{(network\_111, }\AttributeTok{mode =} \StringTok{"all"}\NormalTok{) }\SpecialCharTok{==} \DecValTok{0}\NormalTok{)}

\CommentTok{\# Convirtiendo la red}
\NormalTok{network\_111 }\OtherTok{\textless{}{-}}\NormalTok{ intergraph}\SpecialCharTok{::}\FunctionTok{asNetwork}\NormalTok{(network\_111)}
\end{Highlighting}
\end{Shaded}

\texttt{asNetwork(simplify(ig\_year1\_111))}
\texttt{ig\_year1\_111\ \textbar{}\textgreater{}\ simplify()\ \textbar{}\textgreater{}\ asNetwork()}

Un problema que tenemos con estos datos es el hecho de que algunos
vértices tienen valores faltantes en las variables \texttt{hispanic},
\texttt{female1}, y \texttt{eversmk1}. Por ahora, procederemos imputando
valores basados en los promedios:

\begin{Shaded}
\begin{Highlighting}[]
\ControlFlowTok{for}\NormalTok{ (v }\ControlFlowTok{in} \FunctionTok{c}\NormalTok{(}\StringTok{"hispanic"}\NormalTok{, }\StringTok{"female1"}\NormalTok{, }\StringTok{"eversmk1"}\NormalTok{)) \{}
\NormalTok{  tmpv }\OtherTok{\textless{}{-}}\NormalTok{ network\_111 }\SpecialCharTok{\%v\%}\NormalTok{ v}
\NormalTok{  tmpv[}\FunctionTok{is.na}\NormalTok{(tmpv)] }\OtherTok{\textless{}{-}} \FunctionTok{mean}\NormalTok{(tmpv, }\AttributeTok{na.rm =} \ConstantTok{TRUE}\NormalTok{) }\SpecialCharTok{\textgreater{}}\NormalTok{ .}\DecValTok{5}
\NormalTok{  network\_111 }\SpecialCharTok{\%v\%}\NormalTok{ v }\OtherTok{\textless{}{-}}\NormalTok{ tmpv}
\NormalTok{\}}
\end{Highlighting}
\end{Shaded}

Echemos un vistazo a la red

\begin{Shaded}
\begin{Highlighting}[]
\FunctionTok{nplot}\NormalTok{(}
\NormalTok{  network\_111,}
  \AttributeTok{vertex.color =} \SpecialCharTok{\textasciitilde{}}\NormalTok{ hispanic}
\NormalTok{  )}
\end{Highlighting}
\end{Shaded}

\begin{figure}[H]

\centering{

\pandocbounded{\includegraphics[keepaspectratio]{part-01-04-ergms_files/figure-pdf/fig-before-big-fit-1.pdf}}

}

\caption{\label{fig-before-big-fit}}

\end{figure}%

\section{Ejecutando ERGMs}\label{ejecutando-ergms}

Flujo de trabajo propuesto:

\begin{enumerate}
\def\labelenumi{\arabic{enumi}.}
\item
  Estimar el modelo más simple, agregando una variable a la vez.
\item
  Después de cada estimación, ejecutar la función
  \texttt{mcmc.diagnostics} para ver qué tan bien (o mal) se comportaron
  las cadenas.
\item
  Ejecutar la función \texttt{gof} y verificar qué tan bien el modelo
  coincide con las estadísticas estructurales de la red.
\end{enumerate}

Qué usar:

\begin{enumerate}
\def\labelenumi{\arabic{enumi}.}
\item
  \texttt{control.ergms}: Número máximo de iteraciones, semilla para
  Pseudo-RNG, cuántos núcleos
\item
  \texttt{ergm.constraints}: De dónde muestrear la red. Da estabilidad y
  (en algunos casos) convergencia más rápida ya que al restringir el
  modelo estás reduciendo el tamaño de la muestra.
\end{enumerate}

Aquí hay un ejemplo de un par de modelos que podríamos
comparar\footnote{Nota que este documento puede no incluir los mensajes
  usuales que el comando \texttt{ergm} genera durante el procedimiento
  de estimación. Esto es solo para hacerlo más amigable para imprimir.}

\begin{Shaded}
\begin{Highlighting}[]
\NormalTok{ans0 }\OtherTok{\textless{}{-}} \FunctionTok{ergm}\NormalTok{(}
\NormalTok{  network\_111 }\SpecialCharTok{\textasciitilde{}}
\NormalTok{    edges }\SpecialCharTok{+}
    \FunctionTok{nodematch}\NormalTok{(}\StringTok{"hispanic"}\NormalTok{) }\SpecialCharTok{+}
    \FunctionTok{nodematch}\NormalTok{(}\StringTok{"female1"}\NormalTok{) }\SpecialCharTok{+}
    \FunctionTok{nodematch}\NormalTok{(}\StringTok{"eversmk1"}\NormalTok{) }\SpecialCharTok{+}
\NormalTok{    mutual,}
  \AttributeTok{constraints =} \SpecialCharTok{\textasciitilde{}}\FunctionTok{bd}\NormalTok{(}\AttributeTok{maxout =} \DecValTok{19}\NormalTok{),}
  \AttributeTok{control =} \FunctionTok{control.ergm}\NormalTok{(}
    \AttributeTok{seed        =} \DecValTok{1}\NormalTok{,}
    \AttributeTok{MCMLE.maxit =} \DecValTok{10}\NormalTok{,}
    \AttributeTok{parallel    =} \DecValTok{4}\NormalTok{,}
    \AttributeTok{CD.maxit    =} \DecValTok{10}
\NormalTok{    )}
\NormalTok{  )}
\DocumentationTok{\#\# Warning: \textquotesingle{}glpk\textquotesingle{} selected as the solver, but package \textquotesingle{}Rglpk\textquotesingle{} is not available;}
\DocumentationTok{\#\# falling back to \textquotesingle{}lpSolveAPI\textquotesingle{}. This should be fine unless the sample size and/or}
\DocumentationTok{\#\# the number of parameters is very big.}
\DocumentationTok{\#\# Warning in nobs.ergm(object, ...): The number of observed dyads in this network}
\DocumentationTok{\#\# is ill{-}defined due to complex constraints on the sample space. Disable this}
\DocumentationTok{\#\# warning with \textquotesingle{}options(ergm.loglik.warn\_dyads=FALSE)\textquotesingle{}.}
\DocumentationTok{\#\# Warning in nobs.ergm(object, ..., verbose = verbose): The number of observed}
\DocumentationTok{\#\# dyads in this network is ill{-}defined due to complex constraints on the sample}
\DocumentationTok{\#\# space. Disable this warning with \textquotesingle{}options(ergm.loglik.warn\_dyads=FALSE)\textquotesingle{}.}
\end{Highlighting}
\end{Shaded}

Entonces, ¿qué estamos haciendo aquí:

\begin{enumerate}
\def\labelenumi{\arabic{enumi}.}
\item
  El modelo está controlando por:

  \begin{enumerate}
  \def\labelenumii{\alph{enumii}.}
  \item
    \texttt{edges} Número de enlaces en la red (en oposición a su
    densidad)
  \item
    \texttt{nodematch("algún-nombre-de-variable-aquí")} Incluye un
    término que controla por homofilia/heterofilia
  \item
    \texttt{mutual} Número de conexiones mutuas entre
    \((i, j), (j, i)\). Esto puede estar relacionado con, por ejemplo,
    cierre triádico.
  \end{enumerate}
\end{enumerate}

Para más sobre parámetros de control, ver (Morris, Handcock, and Hunter
2008).

\begin{Shaded}
\begin{Highlighting}[]
\NormalTok{ans1 }\OtherTok{\textless{}{-}} \FunctionTok{ergm}\NormalTok{(}
\NormalTok{  network\_111 }\SpecialCharTok{\textasciitilde{}}
\NormalTok{    edges }\SpecialCharTok{+}
    \FunctionTok{nodematch}\NormalTok{(}\StringTok{"hispanic"}\NormalTok{) }\SpecialCharTok{+}
    \FunctionTok{nodematch}\NormalTok{(}\StringTok{"female1"}\NormalTok{) }\SpecialCharTok{+}
    \FunctionTok{nodematch}\NormalTok{(}\StringTok{"eversmk1"}\NormalTok{)}
\NormalTok{    ,}
  \AttributeTok{constraints =} \SpecialCharTok{\textasciitilde{}}\FunctionTok{bd}\NormalTok{(}\AttributeTok{maxout =} \DecValTok{19}\NormalTok{),}
  \AttributeTok{control =} \FunctionTok{control.ergm}\NormalTok{(}
    \AttributeTok{seed        =} \DecValTok{1}\NormalTok{,}
    \AttributeTok{MCMLE.maxit =} \DecValTok{10}\NormalTok{,}
    \AttributeTok{parallel    =} \DecValTok{4}\NormalTok{,}
    \AttributeTok{CD.maxit    =} \DecValTok{10}
\NormalTok{    )}
\NormalTok{  )}
\end{Highlighting}
\end{Shaded}

Este ejemplo toma más tiempo para calcular

\begin{Shaded}
\begin{Highlighting}[]
\NormalTok{ans2 }\OtherTok{\textless{}{-}} \FunctionTok{ergm}\NormalTok{(}
\NormalTok{  network\_111 }\SpecialCharTok{\textasciitilde{}}
\NormalTok{    edges }\SpecialCharTok{+}
    \FunctionTok{nodematch}\NormalTok{(}\StringTok{"hispanic"}\NormalTok{) }\SpecialCharTok{+}
    \FunctionTok{nodematch}\NormalTok{(}\StringTok{"female1"}\NormalTok{) }\SpecialCharTok{+}
    \FunctionTok{nodematch}\NormalTok{(}\StringTok{"eversmk1"}\NormalTok{) }\SpecialCharTok{+} 
\NormalTok{    mutual }\SpecialCharTok{+}
\NormalTok{    balance}
\NormalTok{    ,}
  \AttributeTok{constraints =} \SpecialCharTok{\textasciitilde{}}\FunctionTok{bd}\NormalTok{(}\AttributeTok{maxout =} \DecValTok{19}\NormalTok{),}
  \AttributeTok{control =} \FunctionTok{control.ergm}\NormalTok{(}
    \AttributeTok{seed        =} \DecValTok{1}\NormalTok{,}
    \AttributeTok{MCMLE.maxit =} \DecValTok{10}\NormalTok{,}
    \AttributeTok{parallel    =} \DecValTok{4}\NormalTok{,}
    \AttributeTok{CD.maxit    =} \DecValTok{10}
\NormalTok{    )}
\NormalTok{  )}
\end{Highlighting}
\end{Shaded}

Ahora, un truco agradable para ver todas las regresiones en la misma
tabla, podemos usar el paquete \texttt{texreg} (Leifeld 2013) que
soporta salidas de \texttt{ergm}!

\begin{Shaded}
\begin{Highlighting}[]
\FunctionTok{library}\NormalTok{(texreg)}
\DocumentationTok{\#\# Version:  1.39.4}
\DocumentationTok{\#\# Date:     2024{-}07{-}23}
\DocumentationTok{\#\# Author:   Philip Leifeld (University of Manchester)}
\DocumentationTok{\#\# }
\DocumentationTok{\#\# Consider submitting praise using the praise or praise\_interactive functions.}
\DocumentationTok{\#\# Please cite the JSS article in your publications {-}{-} see citation("texreg").}
\FunctionTok{screenreg}\NormalTok{(}\FunctionTok{list}\NormalTok{(ans0, ans1, ans2))}
\DocumentationTok{\#\# Warning in nobs.ergm(object): The number of observed dyads in this network is}
\DocumentationTok{\#\# ill{-}defined due to complex constraints on the sample space. Disable this}
\DocumentationTok{\#\# warning with \textquotesingle{}options(ergm.loglik.warn\_dyads=FALSE)\textquotesingle{}.}
\DocumentationTok{\#\# Warning: This object was fit with \textquotesingle{}ergm\textquotesingle{} version 4.1.2 or earlier. Summarizing}
\DocumentationTok{\#\# it with version 4.2 or later may return incorrect results or fail.}
\DocumentationTok{\#\# Warning in nobs.ergm(object): The number of observed dyads in this network is}
\DocumentationTok{\#\# ill{-}defined due to complex constraints on the sample space. Disable this}
\DocumentationTok{\#\# warning with \textquotesingle{}options(ergm.loglik.warn\_dyads=FALSE)\textquotesingle{}.}
\DocumentationTok{\#\# Warning: This object was fit with \textquotesingle{}ergm\textquotesingle{} version 4.1.2 or earlier. Summarizing}
\DocumentationTok{\#\# it with version 4.2 or later may return incorrect results or fail.}
\DocumentationTok{\#\# Warning in nobs.ergm(object): The number of observed dyads in this network is}
\DocumentationTok{\#\# ill{-}defined due to complex constraints on the sample space. Disable this}
\DocumentationTok{\#\# warning with \textquotesingle{}options(ergm.loglik.warn\_dyads=FALSE)\textquotesingle{}.}
\DocumentationTok{\#\# }
\DocumentationTok{\#\# ===============================================================}
\DocumentationTok{\#\#                     Model 1        Model 2        Model 3      }
\DocumentationTok{\#\# {-}{-}{-}{-}{-}{-}{-}{-}{-}{-}{-}{-}{-}{-}{-}{-}{-}{-}{-}{-}{-}{-}{-}{-}{-}{-}{-}{-}{-}{-}{-}{-}{-}{-}{-}{-}{-}{-}{-}{-}{-}{-}{-}{-}{-}{-}{-}{-}{-}{-}{-}{-}{-}{-}{-}{-}{-}{-}{-}{-}{-}{-}{-}}
\DocumentationTok{\#\# edges                   {-}5.62 ***      {-}5.49 ***      {-}5.60 ***}
\DocumentationTok{\#\#                         (0.06)         (0.06)         (0.06)   }
\DocumentationTok{\#\# nodematch.hispanic       0.22 ***       0.30 ***       0.22 ***}
\DocumentationTok{\#\#                         (0.04)         (0.05)         (0.04)   }
\DocumentationTok{\#\# nodematch.female1        0.87 ***       1.17 ***       0.87 ***}
\DocumentationTok{\#\#                         (0.04)         (0.05)         (0.04)   }
\DocumentationTok{\#\# nodematch.eversmk1       0.33 ***       0.45 ***       0.34 ***}
\DocumentationTok{\#\#                         (0.04)         (0.04)         (0.04)   }
\DocumentationTok{\#\# mutual                   4.12 ***                      1.75 ***}
\DocumentationTok{\#\#                         (0.07)                        (0.14)   }
\DocumentationTok{\#\# balance                                                0.01 ***}
\DocumentationTok{\#\#                                                       (0.00)   }
\DocumentationTok{\#\# {-}{-}{-}{-}{-}{-}{-}{-}{-}{-}{-}{-}{-}{-}{-}{-}{-}{-}{-}{-}{-}{-}{-}{-}{-}{-}{-}{-}{-}{-}{-}{-}{-}{-}{-}{-}{-}{-}{-}{-}{-}{-}{-}{-}{-}{-}{-}{-}{-}{-}{-}{-}{-}{-}{-}{-}{-}{-}{-}{-}{-}{-}{-}}
\DocumentationTok{\#\# AIC                 {-}39978.24      {-}37511.87      {-}39989.59    }
\DocumentationTok{\#\# BIC                 {-}39927.89      {-}37471.60      {-}39929.18    }
\DocumentationTok{\#\# Log Likelihood       19994.12       18759.94       20000.79    }
\DocumentationTok{\#\# ===============================================================}
\DocumentationTok{\#\# *** p \textless{} 0.001; ** p \textless{} 0.01; * p \textless{} 0.05}
\end{Highlighting}
\end{Shaded}

O, si estás usando rmarkdown, puedes exportar los resultados usando
LaTeX o html, intentemos este último para ver cómo se ve aquí:

\begin{Shaded}
\begin{Highlighting}[]
\FunctionTok{library}\NormalTok{(texreg)}
\FunctionTok{texreg}\NormalTok{(}\FunctionTok{list}\NormalTok{(ans0, ans1, ans2))}
\DocumentationTok{\#\# Warning in nobs.ergm(object): The number of observed dyads in this network is}
\DocumentationTok{\#\# ill{-}defined due to complex constraints on the sample space. Disable this}
\DocumentationTok{\#\# warning with \textquotesingle{}options(ergm.loglik.warn\_dyads=FALSE)\textquotesingle{}.}
\DocumentationTok{\#\# Warning: This object was fit with \textquotesingle{}ergm\textquotesingle{} version 4.1.2 or earlier. Summarizing}
\DocumentationTok{\#\# it with version 4.2 or later may return incorrect results or fail.}
\DocumentationTok{\#\# Warning in nobs.ergm(object): The number of observed dyads in this network is}
\DocumentationTok{\#\# ill{-}defined due to complex constraints on the sample space. Disable this}
\DocumentationTok{\#\# warning with \textquotesingle{}options(ergm.loglik.warn\_dyads=FALSE)\textquotesingle{}.}
\DocumentationTok{\#\# Warning: This object was fit with \textquotesingle{}ergm\textquotesingle{} version 4.1.2 or earlier. Summarizing}
\DocumentationTok{\#\# it with version 4.2 or later may return incorrect results or fail.}
\DocumentationTok{\#\# Warning in nobs.ergm(object): The number of observed dyads in this network is}
\DocumentationTok{\#\# ill{-}defined due to complex constraints on the sample space. Disable this}
\DocumentationTok{\#\# warning with \textquotesingle{}options(ergm.loglik.warn\_dyads=FALSE)\textquotesingle{}.}
\end{Highlighting}
\end{Shaded}

\begin{table}
\begin{center}
\begin{tabular}{l c c c}
\hline
 & Model 1 & Model 2 & Model 3 \\
\hline
edges              & $-5.62^{***}$ & $-5.49^{***}$ & $-5.60^{***}$ \\
                   & $(0.06)$      & $(0.06)$      & $(0.06)$      \\
nodematch.hispanic & $0.22^{***}$  & $0.30^{***}$  & $0.22^{***}$  \\
                   & $(0.04)$      & $(0.05)$      & $(0.04)$      \\
nodematch.female1  & $0.87^{***}$  & $1.17^{***}$  & $0.87^{***}$  \\
                   & $(0.04)$      & $(0.05)$      & $(0.04)$      \\
nodematch.eversmk1 & $0.33^{***}$  & $0.45^{***}$  & $0.34^{***}$  \\
                   & $(0.04)$      & $(0.04)$      & $(0.04)$      \\
mutual             & $4.12^{***}$  &               & $1.75^{***}$  \\
                   & $(0.07)$      &               & $(0.14)$      \\
balance            &               &               & $0.01^{***}$  \\
                   &               &               & $(0.00)$      \\
\hline
AIC                & $-39978.24$   & $-37511.87$   & $-39989.59$   \\
BIC                & $-39927.89$   & $-37471.60$   & $-39929.18$   \\
Log Likelihood     & $19994.12$    & $18759.94$    & $20000.79$    \\
\hline
\multicolumn{4}{l}{\scriptsize{$^{***}p<0.001$; $^{**}p<0.01$; $^{*}p<0.05$}}
\end{tabular}
\caption{Statistical models}

\end{center}
\end{table}

\section{Bondad de Ajuste del Modelo}\label{bondad-de-ajuste-del-modelo}

En términos brutos, una vez que cada cadena ha alcanzado la distribución
estacionaria, podemos decir que no hay problemas con la autocorrelación
y que cada punto de muestra es iid. Esto último implica que, dado que
estamos ejecutando el modelo con más de una cadena, podemos usar todas
las muestras (cadenas) como un solo conjunto de datos.

\begin{quote}
Cambios recientes en el algoritmo de estimación de ergm significan que
estos gráficos ya no pueden usarse para asegurar que las estadísticas
medias del modelo coincidan con las estadísticas de red observadas. Para
esa funcionalidad, por favor usa el comando GOF:
\texttt{gof(object,\ GOF=\textasciitilde{}model)}.

---?ergm::mcmc.diagnostics
\end{quote}

Dado que \texttt{ans0} es el mejor modelo, veamos las estadísticas GOF.
Primero, veamos cómo se comportó el MCMC. Podemos usar la función
\texttt{mcmc.diagnostics} incluida en el paquete. La función es un
envoltorio de un par de funciones del paquete \texttt{coda} (Plummer et
al. 2006), que son llamadas sobre el objeto \texttt{\$sample} que
contiene las estadísticas \emph{centradas} de las redes muestreadas. Al
principio, puede ser confuso mirar el objeto \texttt{\$sample}; no
coincide ni con las estadísticas observadas ni con los coeficientes.

Cuando se llama \texttt{mcmc.diagnostics(ans0,\ centered\ =\ FALSE)},
verás muchas salidas, incluyendo un par de gráficos mostrando la traza y
distribución posterior de las estadísticas \emph{no centradas}
(\texttt{centered\ =\ FALSE}). Los siguientes fragmentos de código
reproducirán la salida de la función \texttt{mcmc.diagnostics} paso a
paso usando el paquete coda. Primero, necesitamos \emph{descentrar} el
objeto de muestra:

\begin{Shaded}
\begin{Highlighting}[]
\CommentTok{\# Obteniendo la muestra centrada}
\NormalTok{sample\_centered }\OtherTok{\textless{}{-}}\NormalTok{ ans0}\SpecialCharTok{$}\NormalTok{sample}

\CommentTok{\# Obteniendo las estadísticas observadas y convirtiéndolas en una matriz para que podamos agregarlas}
\CommentTok{\# a las muestras}
\NormalTok{observed }\OtherTok{\textless{}{-}} \FunctionTok{summary}\NormalTok{(ans0}\SpecialCharTok{$}\NormalTok{formula)}
\NormalTok{observed }\OtherTok{\textless{}{-}} \FunctionTok{matrix}\NormalTok{(}
\NormalTok{  observed,}
  \AttributeTok{nrow  =} \FunctionTok{nrow}\NormalTok{(sample\_centered[[}\DecValTok{1}\NormalTok{]]),}
  \AttributeTok{ncol  =} \FunctionTok{length}\NormalTok{(observed),}
  \AttributeTok{byrow =} \ConstantTok{TRUE}
\NormalTok{  )}

\CommentTok{\# Ahora descentramos la muestra}
\NormalTok{sample\_uncentered }\OtherTok{\textless{}{-}} \FunctionTok{lapply}\NormalTok{(sample\_centered, }\ControlFlowTok{function}\NormalTok{(x) \{}
\NormalTok{  x }\SpecialCharTok{+}\NormalTok{ observed}
\NormalTok{\})}

\CommentTok{\# Tenemos que hacer de esto un objeto mcmc.list}
\NormalTok{sample\_uncentered }\OtherTok{\textless{}{-}}\NormalTok{ coda}\SpecialCharTok{::}\FunctionTok{mcmc.list}\NormalTok{(sample\_uncentered)}
\end{Highlighting}
\end{Shaded}

Bajo el capó:

\begin{enumerate}
\def\labelenumi{\arabic{enumi}.}
\tightlist
\item
  \emph{Medias empíricas y sd, y cuantiles}:
\end{enumerate}

\begin{Shaded}
\begin{Highlighting}[]
\FunctionTok{summary}\NormalTok{(sample\_uncentered)}
\DocumentationTok{\#\# }
\DocumentationTok{\#\# Iterations = 262144:4489216}
\DocumentationTok{\#\# Thinning interval = 32768 }
\DocumentationTok{\#\# Number of chains = 4 }
\DocumentationTok{\#\# Sample size per chain = 130 }
\DocumentationTok{\#\# }
\DocumentationTok{\#\# 1. Empirical mean and standard deviation for each variable,}
\DocumentationTok{\#\#    plus standard error of the mean:}
\DocumentationTok{\#\# }
\DocumentationTok{\#\#                      Mean    SD Naive SE Time{-}series SE}
\DocumentationTok{\#\# edges              2467.8 53.94   2.3653          4.089}
\DocumentationTok{\#\# nodematch.hispanic 1824.4 46.49   2.0385          3.420}
\DocumentationTok{\#\# nodematch.female1  1874.1 49.78   2.1831          4.310}
\DocumentationTok{\#\# nodematch.eversmk1 1756.9 45.67   2.0029          4.226}
\DocumentationTok{\#\# mutual              482.2 20.34   0.8921          1.922}
\DocumentationTok{\#\# }
\DocumentationTok{\#\# 2. Quantiles for each variable:}
\DocumentationTok{\#\# }
\DocumentationTok{\#\#                    2.5\%  25\%    50\%  75\% 97.5\%}
\DocumentationTok{\#\# edges              2371 2431 2466.5 2507  2574}
\DocumentationTok{\#\# nodematch.hispanic 1736 1792 1823.0 1853  1920}
\DocumentationTok{\#\# nodematch.female1  1773 1842 1871.0 1908  1976}
\DocumentationTok{\#\# nodematch.eversmk1 1669 1725 1755.0 1786  1848}
\DocumentationTok{\#\# mutual              447  467  480.5  497   522}
\end{Highlighting}
\end{Shaded}

\begin{enumerate}
\def\labelenumi{\arabic{enumi}.}
\setcounter{enumi}{1}
\tightlist
\item
  \emph{Correlación cruzada}:
\end{enumerate}

\begin{Shaded}
\begin{Highlighting}[]
\NormalTok{coda}\SpecialCharTok{::}\FunctionTok{crosscorr}\NormalTok{(sample\_uncentered)}
\DocumentationTok{\#\#                        edges nodematch.hispanic nodematch.female1}
\DocumentationTok{\#\# edges              1.0000000          0.8533639         0.8720593}
\DocumentationTok{\#\# nodematch.hispanic 0.8533639          1.0000000         0.7515793}
\DocumentationTok{\#\# nodematch.female1  0.8720593          0.7515793         1.0000000}
\DocumentationTok{\#\# nodematch.eversmk1 0.8300531          0.7016987         0.7066999}
\DocumentationTok{\#\# mutual             0.7135837          0.6062160         0.6822046}
\DocumentationTok{\#\#                    nodematch.eversmk1    mutual}
\DocumentationTok{\#\# edges                       0.8300531 0.7135837}
\DocumentationTok{\#\# nodematch.hispanic          0.7016987 0.6062160}
\DocumentationTok{\#\# nodematch.female1           0.7066999 0.6822046}
\DocumentationTok{\#\# nodematch.eversmk1          1.0000000 0.6073621}
\DocumentationTok{\#\# mutual                      0.6073621 1.0000000}
\end{Highlighting}
\end{Shaded}

\begin{enumerate}
\def\labelenumi{\arabic{enumi}.}
\setcounter{enumi}{2}
\tightlist
\item
  \emph{Autocorrelación}: Por ahora, solo veremos la autocorrelación
  para la cadena uno. La autocorrelación debería ser pequeña (en un
  entorno MCMC general). Si la autocorrelación es alta, entonces
  significa que tu muestra no es idd (no hay propiedad de Markov). Una
  forma de resolver esto es \emph{adelgazar} la muestra.
\end{enumerate}

\begin{Shaded}
\begin{Highlighting}[]
\NormalTok{coda}\SpecialCharTok{::}\FunctionTok{autocorr}\NormalTok{(sample\_uncentered)[[}\DecValTok{1}\NormalTok{]]}
\DocumentationTok{\#\# , , edges}
\DocumentationTok{\#\# }
\DocumentationTok{\#\#                   edges nodematch.hispanic nodematch.female1 nodematch.eversmk1}
\DocumentationTok{\#\# Lag 0        1.00000000         0.87038563        0.85695797         0.84964840}
\DocumentationTok{\#\# Lag 32768    0.38083146         0.32819125        0.45842435         0.37125033}
\DocumentationTok{\#\# Lag 163840   0.06146600         0.10051676        0.05202905         0.12024660}
\DocumentationTok{\#\# Lag 327680  {-}0.16795243        {-}0.08613342       {-}0.20728173         0.04309449}
\DocumentationTok{\#\# Lag 1638400 {-}0.03699504        {-}0.02361089       {-}0.01486841        {-}0.04258415}
\DocumentationTok{\#\#                  mutual}
\DocumentationTok{\#\# Lag 0        0.73428636}
\DocumentationTok{\#\# Lag 32768    0.53707416}
\DocumentationTok{\#\# Lag 163840   0.14828707}
\DocumentationTok{\#\# Lag 327680  {-}0.19975218}
\DocumentationTok{\#\# Lag 1638400 {-}0.04516715}
\DocumentationTok{\#\# }
\DocumentationTok{\#\# , , nodematch.hispanic}
\DocumentationTok{\#\# }
\DocumentationTok{\#\#                   edges nodematch.hispanic nodematch.female1 nodematch.eversmk1}
\DocumentationTok{\#\# Lag 0        0.87038563         1.00000000       0.753630265         0.78730768}
\DocumentationTok{\#\# Lag 32768    0.33154979         0.36721111       0.421877139         0.37735773}
\DocumentationTok{\#\# Lag 163840   0.02567941         0.07375327      {-}0.004869182         0.10131007}
\DocumentationTok{\#\# Lag 327680  {-}0.13032837        {-}0.07619828      {-}0.180415625         0.01336572}
\DocumentationTok{\#\# Lag 1638400 {-}0.03446213        {-}0.03416479      {-}0.036769263        {-}0.03703742}
\DocumentationTok{\#\#                  mutual}
\DocumentationTok{\#\# Lag 0        0.63956209}
\DocumentationTok{\#\# Lag 32768    0.44464140}
\DocumentationTok{\#\# Lag 163840   0.09038778}
\DocumentationTok{\#\# Lag 327680  {-}0.18811250}
\DocumentationTok{\#\# Lag 1638400 {-}0.06331191}
\DocumentationTok{\#\# }
\DocumentationTok{\#\# , , nodematch.female1}
\DocumentationTok{\#\# }
\DocumentationTok{\#\#                   edges nodematch.hispanic nodematch.female1 nodematch.eversmk1}
\DocumentationTok{\#\# Lag 0        0.85695797        0.753630265      1.0000000000         0.70666301}
\DocumentationTok{\#\# Lag 32768    0.41643123        0.375629107      0.5873756954         0.38187993}
\DocumentationTok{\#\# Lag 163840   0.08616473        0.152305781      0.0451469853         0.16697786}
\DocumentationTok{\#\# Lag 327680  {-}0.04471012       {-}0.013620051     {-}0.1066084116         0.12406968}
\DocumentationTok{\#\# Lag 1638400 {-}0.03037027       {-}0.004581187      0.0006103092        {-}0.05585196}
\DocumentationTok{\#\#                  mutual}
\DocumentationTok{\#\# Lag 0        0.72487538}
\DocumentationTok{\#\# Lag 32768    0.60015684}
\DocumentationTok{\#\# Lag 163840   0.15686542}
\DocumentationTok{\#\# Lag 327680  {-}0.09927182}
\DocumentationTok{\#\# Lag 1638400 {-}0.04211718}
\DocumentationTok{\#\# }
\DocumentationTok{\#\# , , nodematch.eversmk1}
\DocumentationTok{\#\# }
\DocumentationTok{\#\#                   edges nodematch.hispanic nodematch.female1 nodematch.eversmk1}
\DocumentationTok{\#\# Lag 0        0.84964840         0.78730768        0.70666301         1.00000000}
\DocumentationTok{\#\# Lag 32768    0.30380777         0.30835275        0.35806615         0.44526722}
\DocumentationTok{\#\# Lag 163840  {-}0.04267252         0.01091272       {-}0.06985454         0.04014145}
\DocumentationTok{\#\# Lag 327680  {-}0.07736289        {-}0.02746901       {-}0.08808037         0.02538263}
\DocumentationTok{\#\# Lag 1638400 {-}0.06772906        {-}0.05544144       {-}0.05717397        {-}0.01199472}
\DocumentationTok{\#\#                  mutual}
\DocumentationTok{\#\# Lag 0        0.63019406}
\DocumentationTok{\#\# Lag 32768    0.44209175}
\DocumentationTok{\#\# Lag 163840  {-}0.02147483}
\DocumentationTok{\#\# Lag 327680  {-}0.09510289}
\DocumentationTok{\#\# Lag 1638400 {-}0.05166925}
\DocumentationTok{\#\# }
\DocumentationTok{\#\# , , mutual}
\DocumentationTok{\#\# }
\DocumentationTok{\#\#                   edges nodematch.hispanic nodematch.female1 nodematch.eversmk1}
\DocumentationTok{\#\# Lag 0        0.73428636         0.63956209        0.72487538         0.63019406}
\DocumentationTok{\#\# Lag 32768    0.50331299         0.42783685        0.56116190         0.48334529}
\DocumentationTok{\#\# Lag 163840  {-}0.01972873         0.04386899       {-}0.04270727         0.12235016}
\DocumentationTok{\#\# Lag 327680  {-}0.15294595        {-}0.08223582       {-}0.19226572        {-}0.02075609}
\DocumentationTok{\#\# Lag 1638400 {-}0.02524845        {-}0.01458091        0.02415611        {-}0.03288693}
\DocumentationTok{\#\#                  mutual}
\DocumentationTok{\#\# Lag 0        1.00000000}
\DocumentationTok{\#\# Lag 32768    0.68168001}
\DocumentationTok{\#\# Lag 163840   0.05146068}
\DocumentationTok{\#\# Lag 327680  {-}0.18299386}
\DocumentationTok{\#\# Lag 1638400 {-}0.05254958}
\end{Highlighting}
\end{Shaded}

\begin{enumerate}
\def\labelenumi{\arabic{enumi}.}
\setcounter{enumi}{3}
\tightlist
\item
  \emph{Diagnóstico de Geweke}: Del archivo de ayuda de la función:
\end{enumerate}

\begin{quote}
``Si las muestras se extraen de la distribución estacionaria de la
cadena, las dos medias son iguales y la estadística de Geweke tiene una
distribución normal estándar asintóticamente. {[}\ldots{]} El Z-score se
calcula bajo la suposición de que las dos partes de la cadena son
asintóticamente independientes, lo que requiere que la suma de frac1 y
frac2 sea estrictamente menor que 1.''

---?coda::geweke.diag
\end{quote}

Echemos un vistazo a una sola cadena:

\begin{Shaded}
\begin{Highlighting}[]
\NormalTok{coda}\SpecialCharTok{::}\FunctionTok{geweke.diag}\NormalTok{(sample\_uncentered)[[}\DecValTok{1}\NormalTok{]]}
\DocumentationTok{\#\# }
\DocumentationTok{\#\# Fraction in 1st window = 0.1}
\DocumentationTok{\#\# Fraction in 2nd window = 0.5 }
\DocumentationTok{\#\# }
\DocumentationTok{\#\#              edges nodematch.hispanic  nodematch.female1 nodematch.eversmk1 }
\DocumentationTok{\#\#            0.55240            0.11024           {-}0.05328            0.17691 }
\DocumentationTok{\#\#             mutual }
\DocumentationTok{\#\#           {-}0.21020}
\end{Highlighting}
\end{Shaded}

\begin{enumerate}
\def\labelenumi{\arabic{enumi}.}
\setcounter{enumi}{4}
\tightlist
\item
  \emph{(no incluido) Diagnóstico de Gelman}: Del archivo de ayuda de la
  función:
\end{enumerate}

\begin{quote}
Gelman y Rubin (1992) proponen un enfoque general para monitorear la
convergencia de salida MCMC en el que se ejecutan m \textgreater{} 1
cadenas paralelas con valores iniciales que están sobre-dispersos
relativo a la distribución posterior. La convergencia se diagnostica
cuando las cadenas han `olvidado' sus valores iniciales, y la salida de
todas las cadenas es indistinguible. El diagnóstico gelman.diag se
aplica a una sola variable de la cadena. Se basa en una comparación de
varianzas dentro de cadena y entre cadenas, y es similar a un análisis
de varianza clásico. ---?coda::gelman.diag
\end{quote}

Como diferencia del estadístico de diagnóstico anterior, este usa todas
las cadenas simultáneamente:

\begin{Shaded}
\begin{Highlighting}[]
\NormalTok{coda}\SpecialCharTok{::}\FunctionTok{gelman.diag}\NormalTok{(sample\_uncentered)}
\DocumentationTok{\#\# Potential scale reduction factors:}
\DocumentationTok{\#\# }
\DocumentationTok{\#\#                    Point est. Upper C.I.}
\DocumentationTok{\#\# edges                    1.02       1.07}
\DocumentationTok{\#\# nodematch.hispanic       1.02       1.07}
\DocumentationTok{\#\# nodematch.female1        1.02       1.08}
\DocumentationTok{\#\# nodematch.eversmk1       1.02       1.06}
\DocumentationTok{\#\# mutual                   1.03       1.10}
\DocumentationTok{\#\# }
\DocumentationTok{\#\# Multivariate psrf}
\DocumentationTok{\#\# }
\DocumentationTok{\#\# 1.08}
\end{Highlighting}
\end{Shaded}

Como regla general, valores en el \([.9,1.1]\) son buenos.

Una característica agradable de la función \texttt{mcmc.diagnostics} son
los gráficos bonitos de traza y distribución posterior que genera. Si
tienes el paquete de R \texttt{latticeExtra} (Sarkar and Andrews 2022),
la función anulará los gráficos predeterminados usados por
\texttt{coda::plot.mcmc} y usará lattice en su lugar, creando gráficos
de mejor apariencia. El siguiente fragmento de código llama la función
\texttt{mcmc.diagnostic}, pero suprimimos el resto de la salida (ver
figura \textbf{?@fig-coda-plots}).

\begin{Shaded}
\begin{Highlighting}[]
\CommentTok{\# [2022{-}03{-}13] Esta línea está fallando por lo que podría ser un error de ergm}
\CommentTok{\# mcmc.diagnostics(ans0, center = FALSE) \# Suprimiendo toda la salida}
\end{Highlighting}
\end{Shaded}

Si llamamos la función \texttt{mcmc.diagnostics}, este mensaje aparece
al final:

\begin{quote}
Los diagnósticos MCMC mostrados aquí son de la última ronda de
simulación, previo al cálculo de las estimaciones finales de parámetros.
Porque las estimaciones finales son refinamientos de aquellas usadas
para esta ejecución de simulación, estos diagnósticos pueden subestimar
el rendimiento del modelo. Para evaluar directamente el rendimiento del
modelo final en estadísticas del modelo, por favor usa el comando GOF:
gof(ergmFitObject, GOF=\textasciitilde model).

---\texttt{mcmc.diagnostics(ans0)}
\end{quote}

¡No está tan mal (aunque el término \texttt{mutual} podría hacerlo
mejor)!\footnote{El sitio web wiki de statnet tiene un ejemplo muy
  agradable de gráficos de diagnóstico MCMC (muy) malos y buenos
  \href{https://statnet.org/trac/raw-attachment/wiki/Resources/ergm.fit.diagnostics.pdf}{aquí}.}
Primero, observa que en la figura, vemos cuatro líneas diferentes; ¿por
qué es eso? Dado que estábamos ejecutando en paralelo usando cuatro
núcleos, el algoritmo ejecutó cuatro cadenas del algoritmo MCMC. Una
prueba visual es ver si todas las cadenas se movieron más o menos al
mismo lugar; en tal caso, podemos empezar a pensar sobre convergencia
del modelo desde la perspectiva MCMC.

Una vez que estamos seguros de haber alcanzado convergencia en el
algoritmo MCMC, podemos empezar a pensar sobre qué tan bien nuestro
modelo predice las propiedades de la red observada. Además de las
estadísticas que definen nuestro ERGM, el comportamiento predeterminado
de la función \texttt{gof} muestra GOF para:

\begin{enumerate}
\def\labelenumi{\alph{enumi}.}
\tightlist
\item
  Distribución de grado de entrada,
\item
  Distribución de grado de salida,
\item
  Socios compartidos por enlace, y
\item
  Geodésicas
\end{enumerate}

Echemos un vistazo

\begin{Shaded}
\begin{Highlighting}[]
\CommentTok{\# Calculando e imprimiendo estadísticas GOF}
\NormalTok{ans\_gof }\OtherTok{\textless{}{-}} \FunctionTok{gof}\NormalTok{(ans0)}
\DocumentationTok{\#\# Best valid proposal \textquotesingle{}BDStratTNT\textquotesingle{} cannot take into account hint(s) \textquotesingle{}triadic\textquotesingle{}.}
\NormalTok{ans\_gof}
\DocumentationTok{\#\# }
\DocumentationTok{\#\# Goodness{-}of{-}fit for model statistics }
\DocumentationTok{\#\# }
\DocumentationTok{\#\#                     obs  min    mean  max MC p{-}value}
\DocumentationTok{\#\# edges              2475 2346 2480.36 2588       1.00}
\DocumentationTok{\#\# nodematch.hispanic 1832 1739 1821.73 1934       0.82}
\DocumentationTok{\#\# nodematch.female1  1879 1729 1870.76 1954       0.92}
\DocumentationTok{\#\# nodematch.eversmk1 1755 1660 1768.64 1861       0.74}
\DocumentationTok{\#\# mutual              486  458  487.12  526       1.00}
\DocumentationTok{\#\# }
\DocumentationTok{\#\# Goodness{-}of{-}fit for minimum geodesic distance }
\DocumentationTok{\#\# }
\DocumentationTok{\#\#       obs   min     mean   max MC p{-}value}
\DocumentationTok{\#\# 1    2475  2346  2480.36  2588          1}
\DocumentationTok{\#\# 2   10672 12760 13903.79 15184          0}
\DocumentationTok{\#\# 3   31134 51937 56593.70 62000          0}
\DocumentationTok{\#\# 4   50673 77198 80049.56 82576          0}
\DocumentationTok{\#\# 5   42563 13157 19027.36 24137          0}
\DocumentationTok{\#\# 6   18719   409  1031.41  2116          0}
\DocumentationTok{\#\# 7    4808     0    28.92   171          0}
\DocumentationTok{\#\# 8     822     0     0.35     6          0}
\DocumentationTok{\#\# 9     100     0     0.01     1          0}
\DocumentationTok{\#\# 10      7     0     0.00     0          0}
\DocumentationTok{\#\# Inf 12333     0  1190.54  3323          0}
\DocumentationTok{\#\# }
\DocumentationTok{\#\# Goodness{-}of{-}fit for out{-}degree }
\DocumentationTok{\#\# }
\DocumentationTok{\#\#    obs min  mean max MC p{-}value}
\DocumentationTok{\#\# 0    4   0  1.34   5       0.04}
\DocumentationTok{\#\# 1   28   1  7.27  14       0.00}
\DocumentationTok{\#\# 2   45  12 21.05  38       0.00}
\DocumentationTok{\#\# 3   50  25 39.20  59       0.10}
\DocumentationTok{\#\# 4   54  43 56.97  81       0.70}
\DocumentationTok{\#\# 5   62  48 66.10  80       0.64}
\DocumentationTok{\#\# 6   40  48 66.74  85       0.00}
\DocumentationTok{\#\# 7   28  39 54.33  73       0.00}
\DocumentationTok{\#\# 8   13  29 41.01  54       0.00}
\DocumentationTok{\#\# 9   16  14 28.30  41       0.02}
\DocumentationTok{\#\# 10  20  10 17.29  26       0.64}
\DocumentationTok{\#\# 11   8   3  9.42  18       0.80}
\DocumentationTok{\#\# 12  11   1  4.90  11       0.02}
\DocumentationTok{\#\# 13  13   0  2.23   6       0.00}
\DocumentationTok{\#\# 14   6   0  0.99   4       0.00}
\DocumentationTok{\#\# 15   6   0  0.55   2       0.00}
\DocumentationTok{\#\# 16   7   0  0.22   2       0.00}
\DocumentationTok{\#\# 17   4   0  0.08   1       0.00}
\DocumentationTok{\#\# 18   3   0  0.01   1       0.00}
\DocumentationTok{\#\# }
\DocumentationTok{\#\# Goodness{-}of{-}fit for in{-}degree }
\DocumentationTok{\#\# }
\DocumentationTok{\#\#    obs min  mean max MC p{-}value}
\DocumentationTok{\#\# 0   13   0  1.48   5       0.00}
\DocumentationTok{\#\# 1   34   3  7.42  16       0.00}
\DocumentationTok{\#\# 2   37  11 20.32  28       0.00}
\DocumentationTok{\#\# 3   48  26 40.56  59       0.22}
\DocumentationTok{\#\# 4   37  37 56.37  71       0.02}
\DocumentationTok{\#\# 5   47  50 65.75  86       0.00}
\DocumentationTok{\#\# 6   42  49 65.13  83       0.00}
\DocumentationTok{\#\# 7   39  38 54.98  69       0.04}
\DocumentationTok{\#\# 8   35  30 42.20  56       0.24}
\DocumentationTok{\#\# 9   21  17 28.35  43       0.16}
\DocumentationTok{\#\# 10  12   9 17.37  31       0.24}
\DocumentationTok{\#\# 11  19   3  9.42  19       0.02}
\DocumentationTok{\#\# 12   4   0  4.55  10       1.00}
\DocumentationTok{\#\# 13   7   0  2.39   7       0.04}
\DocumentationTok{\#\# 14   6   0  1.02   4       0.00}
\DocumentationTok{\#\# 15   3   0  0.46   3       0.02}
\DocumentationTok{\#\# 16   4   0  0.12   2       0.00}
\DocumentationTok{\#\# 17   3   0  0.08   1       0.00}
\DocumentationTok{\#\# 18   3   0  0.02   1       0.00}
\DocumentationTok{\#\# 19   2   0  0.01   1       0.00}
\DocumentationTok{\#\# 20   1   0  0.00   0       0.00}
\DocumentationTok{\#\# 22   1   0  0.00   0       0.00}
\DocumentationTok{\#\# }
\DocumentationTok{\#\# Goodness{-}of{-}fit for edgewise shared partner }
\DocumentationTok{\#\# }
\DocumentationTok{\#\#        obs  min    mean  max MC p{-}value}
\DocumentationTok{\#\# .OTP0 1032 1949 2219.89 2336          0}
\DocumentationTok{\#\# .OTP1  755  154  242.59  455          0}
\DocumentationTok{\#\# .OTP2  352    5   16.73   81          0}
\DocumentationTok{\#\# .OTP3  202    0    1.13   10          0}
\DocumentationTok{\#\# .OTP4   79    0    0.02    1          0}
\DocumentationTok{\#\# .OTP5   36    0    0.00    0          0}
\DocumentationTok{\#\# .OTP6   14    0    0.00    0          0}
\DocumentationTok{\#\# .OTP7    4    0    0.00    0          0}
\DocumentationTok{\#\# .OTP8    1    0    0.00    0          0}

\CommentTok{\# Graficando estadísticas GOF}
\FunctionTok{plot}\NormalTok{(ans\_gof)}
\end{Highlighting}
\end{Shaded}

\pandocbounded{\includegraphics[keepaspectratio]{part-01-04-ergms_files/figure-pdf/checking-gof-1.pdf}}

\pandocbounded{\includegraphics[keepaspectratio]{part-01-04-ergms_files/figure-pdf/checking-gof-2.pdf}}

\pandocbounded{\includegraphics[keepaspectratio]{part-01-04-ergms_files/figure-pdf/checking-gof-3.pdf}}

\pandocbounded{\includegraphics[keepaspectratio]{part-01-04-ergms_files/figure-pdf/checking-gof-4.pdf}}

\pandocbounded{\includegraphics[keepaspectratio]{part-01-04-ergms_files/figure-pdf/checking-gof-5.pdf}}

Prueba la siguiente configuración en su lugar

\begin{Shaded}
\begin{Highlighting}[]
\NormalTok{ans0\_bis }\OtherTok{\textless{}{-}} \FunctionTok{ergm}\NormalTok{(}
\NormalTok{  network\_111 }\SpecialCharTok{\textasciitilde{}}
\NormalTok{    edges }\SpecialCharTok{+}
    \FunctionTok{nodematch}\NormalTok{(}\StringTok{"hispanic"}\NormalTok{) }\SpecialCharTok{+}
    \FunctionTok{nodematch}\NormalTok{(}\StringTok{"female1"}\NormalTok{) }\SpecialCharTok{+}
\NormalTok{    mutual }\SpecialCharTok{+} 
    \FunctionTok{esp}\NormalTok{(}\DecValTok{0}\SpecialCharTok{:}\DecValTok{3}\NormalTok{) }\SpecialCharTok{+} 
    \FunctionTok{idegree}\NormalTok{(}\DecValTok{0}\SpecialCharTok{:}\DecValTok{10}\NormalTok{)}
\NormalTok{    ,}
  \AttributeTok{constraints =} \SpecialCharTok{\textasciitilde{}}\FunctionTok{bd}\NormalTok{(}\AttributeTok{maxout =} \DecValTok{19}\NormalTok{),}
  \AttributeTok{control =} \FunctionTok{control.ergm}\NormalTok{(}
    \AttributeTok{seed        =} \DecValTok{1}\NormalTok{,}
    \AttributeTok{MCMLE.maxit =} \DecValTok{15}\NormalTok{,}
    \AttributeTok{parallel    =} \DecValTok{4}\NormalTok{,}
    \AttributeTok{CD.maxit    =} \DecValTok{15}\NormalTok{,}
    \AttributeTok{MCMC.samplesize =} \DecValTok{2048}\SpecialCharTok{*}\DecValTok{4}\NormalTok{,}
    \AttributeTok{MCMC.burnin =} \DecValTok{30000}\NormalTok{,}
    \AttributeTok{MCMC.interval =} \DecValTok{2048}\SpecialCharTok{*}\DecValTok{4}
\NormalTok{    )}
\NormalTok{  )}
\end{Highlighting}
\end{Shaded}

Aumentar el tamaño de muestra, para que las curvas sean más suaves,
intervalos más largos (adelgazamiento), lo que reduce la
autocorrelación, y un quemado más grande. Todo esto junto para mejorar
la estadística de prueba de Gelman. También agregamos idegree del 0 al
10, y esp del 0 al 3 para coincidir explícitamente con esas estadísticas
en nuestro modelo.

\begin{figure}[H]

{\centering \pandocbounded{\includegraphics[keepaspectratio]{img/awful-chains.png}}

}

\caption{Un ejemplo de un ERGM terrible (no hay convergencia en
absoluto). También, un buen ejemplo de por qué ejecutar múltiples
cadenas puede ser útil}

\end{figure}%

\section{Más sobre convergencia
MCMC}\label{muxe1s-sobre-convergencia-mcmc}

Para más sobre este tema, recomiendo revisar
\href{http://www.mcmchandbook.net/HandbookChapter1.pdf}{capítulo 1} y
\href{http://www.mcmchandbook.net/HandbookChapter6.pdf}{capítulo 6} del
Manual de MCMC (Brooks et al. 2011). Ambos capítulos están disponibles
para descarga gratuita desde el
\href{http://www.mcmchandbook.net/HandbookSampleChapters.html}{sitio web
del libro}.

Para GOF echa un vistazo a la sección 6 del
\href{https://statnet.csde.washington.edu/trac/raw-attachment/wiki/Sunbelt2016/ergm_tutorial.html}{tutorial
ERGM 2016 Sunbelt}, y para una revisión más técnica, puedes echar un
vistazo a (David R. Hunter, Goodreau, and Handcock 2008).

\section{Interpretación
Matemática}\label{interpretaciuxf3n-matemuxe1tica}

Una de las partes más críticas del modelado estadístico es interpretar
los resultados, si no la más importante. En el caso de ERGMs, un aspecto
clave se basa en estadísticas de cambio. Supón que nos gustaría saber
qué tan probable es que el vínculo \(y_{ij}\) ocurra, dada el resto de
la red. Podemos calcular tales probabilidades usando lo que la
literatura a veces describe como el muestreador de Gibbs.

En particular, las log-odds del vínculo \(ij\) ocurriendo condicional en
el resto de la red pueden escribirse como:

\begin{equation}
    \text{logit}\left({\Pr{y_{ij} = 1|y_{-ij}}}\right) = {\theta}^\mathbf{t}\Delta\delta\left(y_{ij}:0\to 1\right),
\end{equation}

\noindent con
\(\delta\left(y_{ij}:0\to 1\right)\equiv s\left(\mathbf{y}\right)_{\text{ij}}^+ - s\left(\mathbf{y}\right)_{\text{ij}}^-\)
como el vector de estadísticas de cambio, en otras palabras, la
diferencia entre las estadísticas suficientes cuando \(y_{ij}=1\) y su
valor cuando \(y_{ij} = 0\). Para mostrar esto, escribimos lo siguiente:

\begin{align*}
    \Pr{y_{ij} = 1|y_{-ij}} & = %
        \frac{\Pr{y_{ij} = 1, x_{-ij}}}{%
            {\mathbb{P}_{+}\left(y_{ij} = 1, y_{-ij}\right) } \Pr{y_{ij} = 0, y_{-ij}}} \\
        & = \frac{\text{exp}\left\{{\theta}^\mathbf{t}s\left(\mathbf{y}\right)^+_{ij}\right\}}{%
            \text{exp}\left\{{\theta}^\mathbf{t}s\left(\mathbf{y}\right)^+_{ij}\right\} + \text{exp}\left\{{\theta}^\mathbf{t}s\left(\mathbf{y}\right)^-_{ij}\right\}}
\end{align*}

Aplicando la función logit a la ecuación anterior, obtenemos:

\begin{align*}
& = \text{log}\left\{\frac{\text{exp}\left\{{\theta}^\mathbf{t}s\left(\mathbf{y}\right)^+_{ij}\right\}}{%
        \text{exp}\left\{{\theta}^\mathbf{t}s\left(\mathbf{y}\right)^+_{ij}\right\} + %
        \text{exp}\left\{{\theta}^\mathbf{t}s\left(\mathbf{y}\right)^-_{ij}\right\}}\right\} - %
    \text{log}\left\{ %
        \frac{\text{exp}\left\{{\theta}^\mathbf{t}s\left(\mathbf{y}\right)^-_{ij}\right\}}{%
            \text{exp}\left\{{\theta}^\mathbf{t}s\left(\mathbf{y}\right)^+_{ij}\right\} + \text{exp}\left\{{\theta}^\mathbf{t}s\left(\mathbf{y}\right)^-_{ij}\right\}}%
     \right\} \\
 & = \text{log}\left\{\text{exp}\left\{{\theta}^\mathbf{t}s\left(\mathbf{y}\right)^+_{ij}\right\}\right\} - \text{log}\left\{\text{exp}\left\{{\theta}^\mathbf{t}s\left(\mathbf{y}\right)^-_{ij}\right\}\right\} \\
 & = {\theta}^\mathbf{t}\left(s\left(\mathbf{y}\right)^+_{ij} - s\left(\mathbf{y}\right)^-_{ij}\right) \\
 & = {\theta}^\mathbf{t}\Delta\delta\left(y_{ij}:0\to 1\right)
\end{align*} \noindent Por lo tanto, la probabilidad condicional del
nodo \(n\) ganando función \(k\) puede escribirse como:

\begin{equation}
    {\mathbb{P}_{=}\left(y_{ij} = 1|y_{-ij}\right) } \frac{1}{1 + \text{exp}\left\{-{\theta}^\mathbf{t}\Delta\delta\left(y_{ij}:0\to 1\right)\right\}}
\end{equation}

\noindent es decir, una probabilidad logística.

\section{Independencia de Markov}\label{independencia-de-markov}

El desafío de analizar redes es su naturaleza interdependiente. No
obstante, en ausencia de tal interdependencia, los ERGMs son
equivalentes a regresión logística. Conceptualmente, si todas las
estadísticas incluidas en el modelo no involucran dos o más díadas,
entonces el modelo es no-Markoviano en el sentido de grafos de Markov.

Matemáticamente, para ver esto, es suficiente mostrar que la
probabilidad ERGM puede escribirse como el producto de las
probabilidades de cada díada.

\begin{equation*}
{\mathbb{P}_{=}\left(\mathbf{y} | \theta\right) } \frac{\text{exp}\left\{{\theta}^\mathbf{t}s\left(\mathbf{y}\right)\right\}}{\sum_{y}\text{exp}\left\{{\theta}^\mathbf{t}s\left(\mathbf{y}\right)\right\}} 
= \frac{\prod_{ij}\text{exp}\left\{{\theta}^\mathbf{t}s\left(\mathbf{y}\right)_{ij}\right\}}{\sum_{y}\text{exp}\left\{{\theta}^\mathbf{t}s\left(\mathbf{y}\right)\right\}}
\end{equation*}

Donde \(s\left(\right)_{ij}\) es una función tal que
\(s\left(\mathbf{y}\right) = \sum_{ij}{s\left(\mathbf{y}\right)_{ij}}\).
Ahora necesitamos tratar con la constante normalizadora. Para ver cómo
eso puede separarse, comencemos desde el resultado:

\newcommand{\thetaS}[1]{\exp{\transpose{\theta}\s{\mathbf{y}}_{#1}}}

\begin{align*}
& =\prod_{ij}\left(1 + \text{exp}\left\{{\theta}^\mathbf{t}s\left(\mathbf{y}\right)_{ij}\right\}\right) \\
& = \left(1 + \text{exp}\left\{{\theta}^\mathbf{t}s\left(\mathbf{y}\right)_{11}\right\}\right)\left(1 + \text{exp}\left\{{\theta}^\mathbf{t}s\left(\mathbf{y}\right)_{12}\right\}\right)\dots\left(1 + \text{exp}\left\{{\theta}^\mathbf{t}s\left(\mathbf{y}\right)_{nn}\right\}\right) \\
& = 1 + \text{exp}\left\{{\theta}^\mathbf{t}s\left(\mathbf{y}\right)_{11}\right\} + \text{exp}\left\{{\theta}^\mathbf{t}s\left(\mathbf{y}\right)_{11}\right\}\text{exp}\left\{{\theta}^\mathbf{t}s\left(\mathbf{y}\right)_{12}\right\} + \dots + \prod_{ij}\text{exp}\left\{{\theta}^\mathbf{t}s\left(\mathbf{y}\right)_{ij}\right\} \\
& = 1 + \text{exp}\left\{{\theta}^\mathbf{t}s\left(\mathbf{y}\right)_{11}\right\} + \text{exp}\left\{{\theta}^\mathbf{t}\left(s\left(\mathbf{y}\right)_{11} + s\left(\mathbf{y}\right)_{12}\right)\right\} + \dots + \prod_{ij}\text{exp}\left\{{\theta}^\mathbf{t}s\left(\mathbf{y}\right)_{ij}\right\} \\
& = \sum_{\mathbf{y}\in\mathcal{Y}}\text{exp}\left\{{\theta}^\mathbf{t}s\left(\mathbf{y}\right)\right\}
\end{align*}

Donde la última igualdad sigue del hecho de que la suma \emph{es} la
suma sobre todas las combinaciones posibles de redes, comenzando desde
\(exp(0) = 1\), hasta \(exp(all)\). De esta manera, ahora podemos
escribir:

\begin{equation}
\frac{\prod_{ij}\text{exp}\left\{{\theta}^\mathbf{t}s\left(\mathbf{y}\right)_{ij}\right\}}{\sum_{y}\text{exp}\left\{{\theta}^\mathbf{t}s\left(\mathbf{y}\right)\right\}} = 
\prod_{ij}\frac{\text{exp}\left\{{\theta}^\mathbf{t}s\left(\mathbf{y}\right)_{ij}\right\}}{1 + \text{exp}\left\{{\theta}^\mathbf{t}s\left(\mathbf{y}\right)_{ij}\right\}}
\end{equation}

Relacionado con esto, los ERGMs bloque-diagonales pueden estimarse como
modelos independientes, uno por bloque. Para ver más sobre esto, lee
(SNIJDERS 2010). De manera similar, dado que la independencia
depende--juego de palabras intencionado--de particionar la función
objetivo, como señala Snijders, las funciones no lineales hacen que el
modelo sea dependiente, ej.,
\(s\left(\mathbf{y}\right) = \sqrt{\sum_{ij}y_{ij}}\), la raíz cuadrada
del conteo de enlaces ya no es un grafo de Bernoulli.

\renewcommand{\Pr}[1]{\mathbb{P}{#1}}

\chapter{Usando restricciones en
ERGMs}\label{usando-restricciones-en-ergms}

\begin{tcolorbox}[enhanced jigsaw, colback=white, opacityback=0, coltitle=black, title=\textcolor{quarto-callout-warning-color}{\faExclamationTriangle}\hspace{0.5em}{Nota de Traducción}, bottomrule=.15mm, colbacktitle=quarto-callout-warning-color!10!white, toptitle=1mm, colframe=quarto-callout-warning-color-frame, titlerule=0mm, rightrule=.15mm, leftrule=.75mm, breakable, bottomtitle=1mm, left=2mm, arc=.35mm, toprule=.15mm, opacitybacktitle=0.6]

Esta versión del capítulo fue traducida de manera automática utilizando
IA. El capítulo aún no ha sido revisado por un humano.

\end{tcolorbox}

Los Modelos Exponenciales de Grafos Aleatorios {[}ERGMs{]} pueden
representar una variedad de clases de redes. A menudo observamos redes
sociales ``regulares'' como estudiantes en escuelas, colegas en el lugar
de trabajo, o familias. No obstante, algunas redes sociales que
estudiamos tienen características que restringen cómo pueden ocurrir las
conexiones. Ejemplos típicos son
\href{https://en.wikipedia.org/wiki/Bipartite_graph}{grafos bi-partitos}
y
\href{https://cran.r-project.org/web/packages/mlergm/vignettes/mlergm_tutorial.html}{redes
multinivel}. Hay dos clases de vértices en redes bi-partitas, y los
vínculos solo pueden ocurrir entre clases. Por otro lado, las redes
multinivel pueden presentar múltiples clases con vínculos entre clases
algo restringidos. En ambos casos, existen restricciones estructurales,
lo que significa que algunas configuraciones pueden no ser plausibles.

Matemáticamente, lo que estamos tratando de hacer es, en lugar de asumir
que todas las configuraciones de red son posibles:

\[
\left\{\mathbf{y} \in \mathcal{Y}: y_{ij} = 0, \forall i = j\right\}
\]

\noindent queremos ir un poco más allá evitando bucles, a saber:

\[
\left\{\mathbf{y} \in \mathcal{Y}: y_{ij} = 0, \forall i = j; \mathbf{y} \in C\right\}
\],

\noindent donde \(C\) es una restricción, por ejemplo, solo redes sin
triángulos. El paquete de R \texttt{ergm} tiene capacidades incorporadas
para lidiar con algunos de estos casos. No obstante, podemos especificar
modelos con restricciones estructurales arbitrarias incorporadas en el
modelo. La clave está en usar términos de offset.

\section{Ejemplo 1: Egos entrelazados y alters
desconectados}\label{ejemplo-1-egos-entrelazados-y-alters-desconectados}

Imagina que tenemos dos conjuntos de vértices. El primero, grupo
\texttt{E}, son egos parte de un estudio egocéntrico. El segundo grupo,
llamado \texttt{A}, está compuesto por personas mencionadas por egos en
\texttt{E} pero que no fueron encuestadas. Asume que individuos en
\texttt{A} solo pueden conectarse a individuos en \texttt{E}; además,
individuos en \texttt{E} no tienen restricciones para conectarse. En
otras palabras, solo existen dos tipos de vínculos: \texttt{E-E} y
\texttt{A-E}. La pregunta es ahora, ¿cómo podemos aplicar tal
restricción en un ERGM?

Usar offsets, y en particular, establecer coeficientes a \texttt{-Inf}
proporciona una forma fácil de restringir el conjunto de soporte de
ERGMs. Por ejemplo, si quisiéramos restringir el soporte para incluir
redes sin triángulos, agregaríamos el término \texttt{offset(triangle)}
y usaríamos la opción \texttt{offset.coef\ =\ -Inf} para indicar que las
realizaciones que incluyen triángulos no son posibles. Usando R:

\begin{Shaded}
\begin{Highlighting}[]
\FunctionTok{ergm}\NormalTok{(net }\SpecialCharTok{\textasciitilde{}}\NormalTok{ edges }\SpecialCharTok{+} \FunctionTok{offset}\NormalTok{(triangle), }\AttributeTok{offset.coef =} \SpecialCharTok{{-}}\ConstantTok{Inf}\NormalTok{)}
\end{Highlighting}
\end{Shaded}

En este modelo, un grafo de Bernoulli, reducimos el espacio muestral a
redes sin triángulos. En nuestro ejemplo, tal estadística solo debería
tomar valores no cero cuando los vínculos dentro de la clase \texttt{A}
ocurran. Podemos usar el término \texttt{nodematch()} para hacer eso.
Formalmente

\[
\text{NodeMatch}(x) = \sum_{i,j} y_{ij} \mathbf{1}({x_{i} = x_{j}})
\]

Esta estadística sumará sobre todos los vínculos en los que el atributo
\(X\) de la fuente (\(i\)) y el objetivo (\(j\)) son iguales. Una forma
de hacer que esto suceda es creando una variable auxiliar que sea igual
a, p.~ej., 0 para todos los vértices en \texttt{A}, y un valor único
diferente de cero en caso contrario. Por ejemplo, si tuviéramos 2
\texttt{A}s y tres \texttt{E}s, los datos se verían algo así:
\(\{0,0,1,2,3\}\). El siguiente bloque de código crea un grafo vacío con
50 nodos, 10 de los cuales están en el grupo \texttt{E} (ego).

\begin{Shaded}
\begin{Highlighting}[]
\FunctionTok{library}\NormalTok{(ergm, }\AttributeTok{quietly =}  \ConstantTok{TRUE}\NormalTok{)}
\FunctionTok{library}\NormalTok{(sna, }\AttributeTok{quietly =}  \ConstantTok{TRUE}\NormalTok{)}

\NormalTok{n }\OtherTok{\textless{}{-}} \DecValTok{50}
\NormalTok{n\_egos }\OtherTok{\textless{}{-}} \DecValTok{10}
\NormalTok{net }\OtherTok{\textless{}{-}} \FunctionTok{as.network}\NormalTok{(}\FunctionTok{matrix}\NormalTok{(}\DecValTok{0}\NormalTok{, }\AttributeTok{ncol =}\NormalTok{ n, }\AttributeTok{nrow =}\NormalTok{ n), }\AttributeTok{directed =} \ConstantTok{TRUE}\NormalTok{)}

\CommentTok{\# Asignemos los grupos}
\NormalTok{net }\SpecialCharTok{\%v\%} \StringTok{"is.ego"} \OtherTok{\textless{}{-}} \FunctionTok{c}\NormalTok{(}\FunctionTok{rep}\NormalTok{(}\ConstantTok{TRUE}\NormalTok{, n\_egos), }\FunctionTok{rep}\NormalTok{(}\ConstantTok{FALSE}\NormalTok{, n }\SpecialCharTok{{-}}\NormalTok{ n\_egos))}
\NormalTok{net }\SpecialCharTok{\%v\%} \StringTok{"is.ego"}
\end{Highlighting}
\end{Shaded}

\begin{verbatim}
 [1]  TRUE  TRUE  TRUE  TRUE  TRUE  TRUE  TRUE  TRUE  TRUE  TRUE FALSE FALSE
[13] FALSE FALSE FALSE FALSE FALSE FALSE FALSE FALSE FALSE FALSE FALSE FALSE
[25] FALSE FALSE FALSE FALSE FALSE FALSE FALSE FALSE FALSE FALSE FALSE FALSE
[37] FALSE FALSE FALSE FALSE FALSE FALSE FALSE FALSE FALSE FALSE FALSE FALSE
[49] FALSE FALSE
\end{verbatim}

\begin{Shaded}
\begin{Highlighting}[]
\FunctionTok{gplot}\NormalTok{(net, }\AttributeTok{vertex.col =}\NormalTok{ net }\SpecialCharTok{\%v\%} \StringTok{"is.ego"}\NormalTok{)}
\end{Highlighting}
\end{Shaded}

\pandocbounded{\includegraphics[keepaspectratio]{part-01-05-ergms-constrains_files/figure-pdf/unnamed-chunk-1-1.pdf}}

Para crear la variable auxiliar, usaremos la siguiente función:

\begin{Shaded}
\begin{Highlighting}[]
\CommentTok{\# Función que crea una variable auxiliar para el modelo ergm}
\NormalTok{make\_aux\_var }\OtherTok{\textless{}{-}} \ControlFlowTok{function}\NormalTok{(my\_net, is\_ego\_dummy) \{}
  
\NormalTok{  n\_vertex }\OtherTok{\textless{}{-}} \FunctionTok{length}\NormalTok{(my\_net }\SpecialCharTok{\%v\%}\NormalTok{ is\_ego\_dummy)}
\NormalTok{  n\_ego\_   }\OtherTok{\textless{}{-}} \FunctionTok{sum}\NormalTok{(my\_net }\SpecialCharTok{\%v\%}\NormalTok{ is\_ego\_dummy)}
  
  \CommentTok{\# Creando una variable auxiliar para identificar los vínculos no{-}informante no{-}informante}
\NormalTok{  my\_net }\SpecialCharTok{\%v\%} \StringTok{"aux\_var"} \OtherTok{\textless{}{-}} \FunctionTok{ifelse}\NormalTok{(}
    \SpecialCharTok{!}\NormalTok{my\_net }\SpecialCharTok{\%v\%}\NormalTok{ is\_ego\_dummy, }\DecValTok{0}\NormalTok{, }\DecValTok{1}\SpecialCharTok{:}\NormalTok{(n\_vertex }\SpecialCharTok{{-}}\NormalTok{ n\_ego\_)}
\NormalTok{    )}

\NormalTok{  my\_net}
\NormalTok{\}}
\end{Highlighting}
\end{Shaded}

Llamar la función en nuestros datos resulta en lo siguiente:

\begin{Shaded}
\begin{Highlighting}[]
\NormalTok{net }\OtherTok{\textless{}{-}} \FunctionTok{make\_aux\_var}\NormalTok{(net, }\StringTok{"is.ego"}\NormalTok{)}

\CommentTok{\# Echando un vistazo a las primeras 15 filas de datos}
\FunctionTok{cbind}\NormalTok{(}
  \AttributeTok{Is\_Ego =}\NormalTok{ net }\SpecialCharTok{\%v\%} \StringTok{"is.ego"}\NormalTok{,}
  \AttributeTok{Aux    =}\NormalTok{ net }\SpecialCharTok{\%v\%} \StringTok{"aux\_var"}  
\NormalTok{) }\SpecialCharTok{|\textgreater{}} \FunctionTok{head}\NormalTok{(}\AttributeTok{n =} \DecValTok{15}\NormalTok{)}
\end{Highlighting}
\end{Shaded}

\begin{verbatim}
      Is_Ego Aux
 [1,]      1   1
 [2,]      1   2
 [3,]      1   3
 [4,]      1   4
 [5,]      1   5
 [6,]      1   6
 [7,]      1   7
 [8,]      1   8
 [9,]      1   9
[10,]      1  10
[11,]      0   0
[12,]      0   0
[13,]      0   0
[14,]      0   0
[15,]      0   0
\end{verbatim}

Ahora podemos usar estos datos para simular una red en la cual los
vínculos entre vértices de clase \texttt{A} no son posibles:

\begin{Shaded}
\begin{Highlighting}[]
\FunctionTok{set.seed}\NormalTok{(}\DecValTok{2828}\NormalTok{)}
\NormalTok{net\_sim }\OtherTok{\textless{}{-}} \FunctionTok{simulate}\NormalTok{(net }\SpecialCharTok{\textasciitilde{}}\NormalTok{ edges }\SpecialCharTok{+} \FunctionTok{nodematch}\NormalTok{(}\StringTok{"aux\_var"}\NormalTok{), }\AttributeTok{coef =} \FunctionTok{c}\NormalTok{(}\SpecialCharTok{{-}}\FloatTok{3.0}\NormalTok{, }\SpecialCharTok{{-}}\ConstantTok{Inf}\NormalTok{))}
\FunctionTok{gplot}\NormalTok{(net\_sim, }\AttributeTok{vertex.col =}\NormalTok{ net\_sim }\SpecialCharTok{\%v\%} \StringTok{"is.ego"}\NormalTok{)}
\end{Highlighting}
\end{Shaded}

\pandocbounded{\includegraphics[keepaspectratio]{part-01-05-ergms-constrains_files/figure-pdf/unnamed-chunk-4-1.pdf}}

Como puedes ver, esta red solo tiene vínculos del tipo \texttt{E-E} y
\texttt{A-E}. Podemos verificar (i) viendo los conteos y (ii)
visualizando cada subgrafo inducido por separado:

\begin{Shaded}
\begin{Highlighting}[]
\FunctionTok{summary}\NormalTok{(net\_sim }\SpecialCharTok{\textasciitilde{}}\NormalTok{ edges }\SpecialCharTok{+} \FunctionTok{nodematch}\NormalTok{(}\StringTok{"aux\_var"}\NormalTok{))}
\end{Highlighting}
\end{Shaded}

\begin{verbatim}
            edges nodematch.aux_var 
               39                 0 
\end{verbatim}

\begin{Shaded}
\begin{Highlighting}[]
\NormalTok{net\_of\_alters }\OtherTok{\textless{}{-}} \FunctionTok{get.inducedSubgraph}\NormalTok{(}
\NormalTok{  net\_sim, }\FunctionTok{which}\NormalTok{((net\_sim }\SpecialCharTok{\%v\%} \StringTok{"aux\_var"}\NormalTok{) }\SpecialCharTok{==} \DecValTok{0}\NormalTok{)}
\NormalTok{  )}

\NormalTok{net\_of\_egos }\OtherTok{\textless{}{-}} \FunctionTok{get.inducedSubgraph}\NormalTok{(}
\NormalTok{  net\_sim, }\FunctionTok{which}\NormalTok{((net\_sim }\SpecialCharTok{\%v\%} \StringTok{"aux\_var"}\NormalTok{) }\SpecialCharTok{!=} \DecValTok{0}\NormalTok{)}
\NormalTok{  )}

\CommentTok{\# Conteos}
\FunctionTok{summary}\NormalTok{(net\_of\_alters }\SpecialCharTok{\textasciitilde{}}\NormalTok{ edges }\SpecialCharTok{+} \FunctionTok{nodematch}\NormalTok{(}\StringTok{"aux\_var"}\NormalTok{))}
\end{Highlighting}
\end{Shaded}

\begin{verbatim}
            edges nodematch.aux_var 
                0                 0 
\end{verbatim}

\begin{Shaded}
\begin{Highlighting}[]
\FunctionTok{summary}\NormalTok{(net\_of\_egos }\SpecialCharTok{\textasciitilde{}}\NormalTok{ edges }\SpecialCharTok{+} \FunctionTok{nodematch}\NormalTok{(}\StringTok{"aux\_var"}\NormalTok{))}
\end{Highlighting}
\end{Shaded}

\begin{verbatim}
            edges nodematch.aux_var 
                5                 0 
\end{verbatim}

\begin{Shaded}
\begin{Highlighting}[]
\CommentTok{\# Figuras}
\NormalTok{op }\OtherTok{\textless{}{-}} \FunctionTok{par}\NormalTok{(}\AttributeTok{mfcol =} \FunctionTok{c}\NormalTok{(}\DecValTok{1}\NormalTok{, }\DecValTok{2}\NormalTok{))}
\FunctionTok{gplot}\NormalTok{(net\_of\_alters, }\AttributeTok{vertex.col =}\NormalTok{ net\_of\_alters }\SpecialCharTok{\%v\%} \StringTok{"is.ego"}\NormalTok{, }\AttributeTok{main =} \StringTok{"A"}\NormalTok{)}
\FunctionTok{gplot}\NormalTok{(net\_of\_egos, }\AttributeTok{vertex.col =}\NormalTok{ net\_of\_egos }\SpecialCharTok{\%v\%} \StringTok{"is.ego"}\NormalTok{, }\AttributeTok{main =} \StringTok{"E"}\NormalTok{)}
\end{Highlighting}
\end{Shaded}

\pandocbounded{\includegraphics[keepaspectratio]{part-01-05-ergms-constrains_files/figure-pdf/unnamed-chunk-5-1.pdf}}

\begin{Shaded}
\begin{Highlighting}[]
\FunctionTok{par}\NormalTok{(op)}
\end{Highlighting}
\end{Shaded}

Ahora, para ajustar un ERGM con esta restricción, simplemente
necesitamos hacer uso de los términos offset. Aquí hay un ejemplo:

\begin{Shaded}
\begin{Highlighting}[]
\NormalTok{ans }\OtherTok{\textless{}{-}} \FunctionTok{ergm}\NormalTok{(}
\NormalTok{  net\_sim }\SpecialCharTok{\textasciitilde{}}\NormalTok{ edges }\SpecialCharTok{+} \FunctionTok{offset}\NormalTok{(}\FunctionTok{nodematch}\NormalTok{(}\StringTok{"aux\_var"}\NormalTok{)), }\CommentTok{\# El modelo (nota el offset)}
  \AttributeTok{offset.coef =} \SpecialCharTok{{-}}\ConstantTok{Inf}                              \CommentTok{\# El coeficiente offset}
\NormalTok{  )}
\DocumentationTok{\#\# Starting maximum pseudolikelihood estimation (MPLE):}
\DocumentationTok{\#\# Obtaining the responsible dyads.}
\DocumentationTok{\#\# Evaluating the predictor and response matrix.}
\DocumentationTok{\#\# Maximizing the pseudolikelihood.}
\DocumentationTok{\#\# Finished MPLE.}
\DocumentationTok{\#\# Evaluating log{-}likelihood at the estimate.}
\FunctionTok{summary}\NormalTok{(ans)}
\DocumentationTok{\#\# Call:}
\DocumentationTok{\#\# ergm(formula = net\_sim \textasciitilde{} edges + offset(nodematch("aux\_var")), }
\DocumentationTok{\#\#     offset.coef = {-}Inf)}
\DocumentationTok{\#\# }
\DocumentationTok{\#\# Maximum Likelihood Results:}
\DocumentationTok{\#\# }
\DocumentationTok{\#\#                           Estimate Std. Error MCMC \% z value Pr(\textgreater{}|z|)    }
\DocumentationTok{\#\# edges                      {-}3.0829     0.1638      0  {-}18.83   \textless{}1e{-}04 ***}
\DocumentationTok{\#\# offset(nodematch.aux\_var)     {-}Inf     0.0000      0    {-}Inf   \textless{}1e{-}04 ***}
\DocumentationTok{\#\# {-}{-}{-}}
\DocumentationTok{\#\# Signif. codes:  0 \textquotesingle{}***\textquotesingle{} 0.001 \textquotesingle{}**\textquotesingle{} 0.01 \textquotesingle{}*\textquotesingle{} 0.05 \textquotesingle{}.\textquotesingle{} 0.1 \textquotesingle{} \textquotesingle{} 1}
\DocumentationTok{\#\# }
\DocumentationTok{\#\#      Null Deviance: 3396.4  on 2450  degrees of freedom}
\DocumentationTok{\#\#  Residual Deviance:  320.2  on 2448  degrees of freedom}
\DocumentationTok{\#\#  }
\DocumentationTok{\#\# AIC: 322.2  BIC: 327  (Smaller is better. MC Std. Err. = 0)}
\DocumentationTok{\#\# }
\DocumentationTok{\#\#  The following terms are fixed by offset and are not estimated:}
\DocumentationTok{\#\#   offset(nodematch.aux\_var)}
\end{Highlighting}
\end{Shaded}

Este modelo ERGM--que por cierto solo presentaba términos diádicos
independientes, y por lo tanto puede reducirse a una regresión
logística--restringe el soporte excluyendo todas las redes en las que
existen vínculos dentro de la clase \texttt{A}. Para finalizar, veamos
algunas simulaciones basadas en este modelo:

\begin{Shaded}
\begin{Highlighting}[]
\FunctionTok{set.seed}\NormalTok{(}\DecValTok{1323}\NormalTok{)}
\NormalTok{op }\OtherTok{\textless{}{-}} \FunctionTok{par}\NormalTok{(}\AttributeTok{mfcol =} \FunctionTok{c}\NormalTok{(}\DecValTok{2}\NormalTok{,}\DecValTok{2}\NormalTok{), }\AttributeTok{mar =} \FunctionTok{rep}\NormalTok{(}\DecValTok{1}\NormalTok{, }\DecValTok{4}\NormalTok{))}
\ControlFlowTok{for}\NormalTok{ (i }\ControlFlowTok{in} \DecValTok{1}\SpecialCharTok{:}\DecValTok{4}\NormalTok{) \{}
  \FunctionTok{gplot}\NormalTok{(}\FunctionTok{simulate}\NormalTok{(ans), }\AttributeTok{vertex.col =}\NormalTok{ net }\SpecialCharTok{\%v\%} \StringTok{"is.ego"}\NormalTok{, }\AttributeTok{vertex.cex =} \DecValTok{2}\NormalTok{)}
  \FunctionTok{box}\NormalTok{()}
\NormalTok{\}}
\end{Highlighting}
\end{Shaded}

\pandocbounded{\includegraphics[keepaspectratio]{part-01-05-ergms-constrains_files/figure-pdf/unnamed-chunk-7-1.pdf}}

\begin{Shaded}
\begin{Highlighting}[]
\FunctionTok{par}\NormalTok{(op)}
\end{Highlighting}
\end{Shaded}

Todas las redes sin vínculos entre nodos \texttt{A}.

\section{Ejemplo 2: Redes
bi-partitas}\label{ejemplo-2-redes-bi-partitas}

En el caso de redes bipartitas (a veces llamadas redes de afiliación,)
podemos usar los términos de \texttt{ergm} para grafos bipartitos para
corroborar lo que discutimos aquí. Por ejemplo, el término de dos
estrellas. Comencemos simulando una red bipartita usando los parámetros
\texttt{edges} y \texttt{two-star}. Dado que el término \texttt{k-star}
usualmente es complejo de ajustar (tiende a generar modelos
degenerados,) aprovecharemos la función de transformación \texttt{Log()}
en el paquete \texttt{ergm} para suavizar el término.\footnote{Después
  de escribir este ejemplo, se hizo aparente que el uso de la función de
  transformación \texttt{Log()} puede no ser ideal. Dado que muchos
  términos usados en ERGMs pueden ser cero, p.~ej., triángulos, el
  término \texttt{Log(\textasciitilde{}\ ostar(2))} está indefinido
  cuando \texttt{ostar(2)\ =\ 0}. En la práctica, el paquete ERGM
  establece un límite inferior para el log de 0, así que, en lugar de
  tener \texttt{Log(0)\ \textasciitilde{}\ -Inf}, lo establecen como un
  número negativo realmente grande. Esto causa todo tipo de problemas a
  las estimaciones; en nuestro ejemplo, una sobreestimación del
  parámetro poblacional y una log-verosimilitud positiva. Por lo tanto,
  no recomendaría usar esta transformación muy a menudo.}

La red bipartita que estaremos simulando tendrá 100 actores y 50
entidades. Los actores, que mapearemos al primer nivel de los términos
\texttt{ergm}, esto es, \texttt{b1star} \texttt{b1nodematch}, etc.
enviarán vínculos a las entidades, el segundo nivel del ERGM bipartito.
Para crear una red bipartita, crearemos una matriz vacía de tamaño
\texttt{nactors\ x\ nentitites}; por lo tanto, los actores están
representados por filas y las entidades por columnas.

\begin{Shaded}
\begin{Highlighting}[]
\CommentTok{\# Parámetros para la simulación}
\NormalTok{nactors   }\OtherTok{\textless{}{-}} \DecValTok{100}
\NormalTok{nentities }\OtherTok{\textless{}{-}} \FunctionTok{floor}\NormalTok{(nactors}\SpecialCharTok{/}\DecValTok{2}\NormalTok{)}
\NormalTok{n         }\OtherTok{\textless{}{-}}\NormalTok{ nactors }\SpecialCharTok{+}\NormalTok{ nentities}

\CommentTok{\# Creando una red bipartita vacía (línea base)}
\NormalTok{net\_b }\OtherTok{\textless{}{-}} \FunctionTok{network}\NormalTok{(}
  \FunctionTok{matrix}\NormalTok{(}\DecValTok{0}\NormalTok{, }\AttributeTok{nrow =}\NormalTok{ nactors, }\AttributeTok{ncol =}\NormalTok{ nentities), }\AttributeTok{bipartite =} \ConstantTok{TRUE}
\NormalTok{)}

\CommentTok{\# Simulando el ERGM bipartito,}
\NormalTok{net\_b }\OtherTok{\textless{}{-}} \FunctionTok{simulate}\NormalTok{(net\_b }\SpecialCharTok{\textasciitilde{}}\NormalTok{ edges }\SpecialCharTok{+} \FunctionTok{Log}\NormalTok{(}\SpecialCharTok{\textasciitilde{}}\FunctionTok{b1star}\NormalTok{(}\DecValTok{2}\NormalTok{)), }\AttributeTok{coef =} \FunctionTok{c}\NormalTok{(}\SpecialCharTok{{-}}\DecValTok{3}\NormalTok{, }\FloatTok{1.5}\NormalTok{), }\AttributeTok{seed =} \DecValTok{55}\NormalTok{)}
\end{Highlighting}
\end{Shaded}

Veamos qué obtuvimos aquí:

\begin{Shaded}
\begin{Highlighting}[]
\FunctionTok{summary}\NormalTok{(net\_b }\SpecialCharTok{\textasciitilde{}}\NormalTok{ edges }\SpecialCharTok{+} \FunctionTok{Log}\NormalTok{(}\SpecialCharTok{\textasciitilde{}}\FunctionTok{b1star}\NormalTok{(}\DecValTok{2}\NormalTok{)))}
\end{Highlighting}
\end{Shaded}

\begin{verbatim}
      edges Log~b1star2 
 245.000000    5.746203 
\end{verbatim}

\begin{Shaded}
\begin{Highlighting}[]
\NormalTok{netplot}\SpecialCharTok{::}\FunctionTok{nplot}\NormalTok{(net\_b, }\AttributeTok{vertex.col =}\NormalTok{ (}\DecValTok{1}\SpecialCharTok{:}\NormalTok{n }\SpecialCharTok{\textless{}=}\NormalTok{ nactors) }\SpecialCharTok{+} \DecValTok{1}\NormalTok{)}
\end{Highlighting}
\end{Shaded}

\pandocbounded{\includegraphics[keepaspectratio]{part-01-05-ergms-constrains_files/figure-pdf/05-example2-simulated-graph-1.pdf}}

Nota que los primeros \texttt{nactors} vértices en la red son los
actores, y los restantes son las entidades. Ahora, aunque el paquete
\texttt{ergm} presenta términos de red bipartita, aún podemos ajustar un
ERGM bipartito sin declarar explícitamente el grafo como tal. En tal
caso, el término \texttt{b1star(2)} de una red bipartita es equivalente
a un \texttt{ostar(2)} en un grafo dirigido. De manera similar,
\texttt{b2star(2)} en un grafo bipartito coincide con el término
\texttt{istar(2)} en un grafo dirigido. Esta información será relevante
al ajustar el ERGM. Transformemos la red bipartita en un grafo dirigido.
El siguiente bloque de código hace eso:

\begin{Shaded}
\begin{Highlighting}[]
\CommentTok{\# Identificando los enlaces}
\NormalTok{net\_not\_b }\OtherTok{\textless{}{-}} \FunctionTok{which}\NormalTok{(}\FunctionTok{as.matrix}\NormalTok{(net\_b) }\SpecialCharTok{!=} \DecValTok{0}\NormalTok{, }\AttributeTok{arr.ind =} \ConstantTok{TRUE}\NormalTok{)}

\CommentTok{\# Necesitamos compensar el punto final de los vínculos por nactors}
\CommentTok{\# para que los ids vayan de 1 hasta (nactors + nentitites)}
\NormalTok{net\_not\_b[,}\DecValTok{2}\NormalTok{] }\OtherTok{\textless{}{-}}\NormalTok{ net\_not\_b[,}\DecValTok{2}\NormalTok{] }\SpecialCharTok{+}\NormalTok{ nactors}

\CommentTok{\# El grafo resultante es una red dirigida}
\NormalTok{net\_not\_b }\OtherTok{\textless{}{-}} \FunctionTok{network}\NormalTok{(net\_not\_b, }\AttributeTok{directed =} \ConstantTok{TRUE}\NormalTok{)}
\end{Highlighting}
\end{Shaded}

Ahora casi hemos terminado. Como antes, necesitamos usar covariables a
nivel de nodo para poner las restricciones en nuestro modelo. Para que
este ERGM refleje un ERGM en una red bipartita, necesitamos dos
restricciones:

\begin{enumerate}
\def\labelenumi{\arabic{enumi}.}
\tightlist
\item
  Solo se permiten vínculos de actores a entidades, y
\item
  las entidades solo pueden recibir vínculos.
\end{enumerate}

Los términos offset correspondientes para este modelo son:
\texttt{nodematch("is.actor")\ \textasciitilde{}\ -Inf}, y
\texttt{nodeocov("isnot.actor")\ \textasciitilde{}\ -Inf}.
Matemáticamente:

\begin{align*}
\text{NodeMatch(x = "is.actor")} &= \sum_{i<j} y_{ij}\mathbb{1}\left(x_i = x_j\right) \\
\text{NodeOCov(x = "isnot.actor")} &= \sum_{i} x_i \times \sum_{j<i} y_{ij} 
\end{align*}

En otras palabras, estamos estableciendo que los vínculos entre nodos de
la misma clase están prohibidos, y los vínculos salientes están
prohibidos para las entidades. Creemos los atributos de vértice
necesarios para usar los términos mencionados:

\begin{Shaded}
\begin{Highlighting}[]
\NormalTok{net\_not\_b }\SpecialCharTok{\%v\%} \StringTok{"is.actor"} \OtherTok{\textless{}{-}} \FunctionTok{as.integer}\NormalTok{(}\DecValTok{1}\SpecialCharTok{:}\NormalTok{n }\SpecialCharTok{\textless{}=}\NormalTok{ nactors)}
\NormalTok{net\_not\_b }\SpecialCharTok{\%v\%} \StringTok{"isnot.actor"} \OtherTok{\textless{}{-}} \FunctionTok{as.integer}\NormalTok{(}\DecValTok{1}\SpecialCharTok{:}\NormalTok{n }\SpecialCharTok{\textgreater{}}\NormalTok{ nactors)}
\end{Highlighting}
\end{Shaded}

Finalmente, para asegurarnos de que hemos hecho todo bien, veamos cómo
ambas redes--bipartita y unimodal--se ven lado a lado:

\begin{Shaded}
\begin{Highlighting}[]
\CommentTok{\# Primero, obtengamos el diseño}
\NormalTok{fig }\OtherTok{\textless{}{-}}\NormalTok{ netplot}\SpecialCharTok{::}\FunctionTok{nplot}\NormalTok{(net\_b, }\AttributeTok{vertex.col =}\NormalTok{ (}\DecValTok{1}\SpecialCharTok{:}\NormalTok{n }\SpecialCharTok{\textless{}=}\NormalTok{ nactors) }\SpecialCharTok{+} \DecValTok{1}\NormalTok{)}
\NormalTok{gridExtra}\SpecialCharTok{::}\FunctionTok{grid.arrange}\NormalTok{(}
\NormalTok{  fig,}
\NormalTok{  netplot}\SpecialCharTok{::}\FunctionTok{nplot}\NormalTok{(}
\NormalTok{    net\_not\_b, }\AttributeTok{vertex.col =}\NormalTok{ (}\DecValTok{1}\SpecialCharTok{:}\NormalTok{n }\SpecialCharTok{\textless{}=}\NormalTok{ nactors) }\SpecialCharTok{+} \DecValTok{1}\NormalTok{,}
    \AttributeTok{layout =}\NormalTok{ fig}\SpecialCharTok{$}\NormalTok{.layout}
\NormalTok{     ),}
  \AttributeTok{ncol =} \DecValTok{2}\NormalTok{, }\AttributeTok{nrow =} \DecValTok{1}
\NormalTok{)}
\end{Highlighting}
\end{Shaded}

\pandocbounded{\includegraphics[keepaspectratio]{part-01-05-ergms-constrains_files/figure-pdf/05-example2-side-by-side-1.pdf}}

\begin{Shaded}
\begin{Highlighting}[]
\CommentTok{\# Viendo los conteos}
\FunctionTok{summary}\NormalTok{(net\_b }\SpecialCharTok{\textasciitilde{}}\NormalTok{ edges }\SpecialCharTok{+} \FunctionTok{b1star}\NormalTok{(}\DecValTok{2}\NormalTok{) }\SpecialCharTok{+} \FunctionTok{b2star}\NormalTok{(}\DecValTok{2}\NormalTok{))}
\end{Highlighting}
\end{Shaded}

\begin{verbatim}
  edges b1star2 b2star2 
    245     313     645 
\end{verbatim}

\begin{Shaded}
\begin{Highlighting}[]
\FunctionTok{summary}\NormalTok{(net\_not\_b }\SpecialCharTok{\textasciitilde{}}\NormalTok{ edges }\SpecialCharTok{+} \FunctionTok{ostar}\NormalTok{(}\DecValTok{2}\NormalTok{) }\SpecialCharTok{+} \FunctionTok{istar}\NormalTok{(}\DecValTok{2}\NormalTok{))}
\end{Highlighting}
\end{Shaded}

\begin{verbatim}
 edges ostar2 istar2 
   245    313    645 
\end{verbatim}

Con las dos redes coincidiendo, ahora podemos ajustar los ERGMs con y
sin términos offset y comparar los resultados entre los dos modelos:

\begin{Shaded}
\begin{Highlighting}[]
\CommentTok{\# ERGM con un grafo bipartito}
\NormalTok{res\_b     }\OtherTok{\textless{}{-}} \FunctionTok{ergm}\NormalTok{(}
  \CommentTok{\# Fórmula principal}
\NormalTok{  net\_b }\SpecialCharTok{\textasciitilde{}}\NormalTok{ edges }\SpecialCharTok{+} \FunctionTok{Log}\NormalTok{(}\SpecialCharTok{\textasciitilde{}}\FunctionTok{b1star}\NormalTok{(}\DecValTok{2}\NormalTok{)),}

  \CommentTok{\# Parámetros de control}
  \AttributeTok{control =} \FunctionTok{control.ergm}\NormalTok{(}\AttributeTok{seed =} \DecValTok{1}\NormalTok{)}
\NormalTok{  )}
\end{Highlighting}
\end{Shaded}

\begin{verbatim}
Warning: 'glpk' selected as the solver, but package 'Rglpk' is not available;
falling back to 'lpSolveAPI'. This should be fine unless the sample size and/or
the number of parameters is very big.
\end{verbatim}

\begin{Shaded}
\begin{Highlighting}[]
\CommentTok{\# ERGM con un digrafo con restricciones}
\NormalTok{res\_not\_b }\OtherTok{\textless{}{-}} \FunctionTok{ergm}\NormalTok{(}
  \CommentTok{\# Fórmula principal}
\NormalTok{  net\_not\_b }\SpecialCharTok{\textasciitilde{}}\NormalTok{ edges }\SpecialCharTok{+} \FunctionTok{Log}\NormalTok{(}\SpecialCharTok{\textasciitilde{}}\FunctionTok{ostar}\NormalTok{(}\DecValTok{2}\NormalTok{)) }\SpecialCharTok{+}

  \CommentTok{\# Términos offset }
  \FunctionTok{offset}\NormalTok{(}\FunctionTok{nodematch}\NormalTok{(}\StringTok{"is.actor"}\NormalTok{)) }\SpecialCharTok{+} \FunctionTok{offset}\NormalTok{(}\FunctionTok{nodeocov}\NormalTok{(}\StringTok{"isnot.actor"}\NormalTok{)),}
  \AttributeTok{offset.coef =} \FunctionTok{c}\NormalTok{(}\SpecialCharTok{{-}}\ConstantTok{Inf}\NormalTok{, }\SpecialCharTok{{-}}\ConstantTok{Inf}\NormalTok{),}

  \CommentTok{\# Parámetros de control}
  \AttributeTok{control =} \FunctionTok{control.ergm}\NormalTok{(}\AttributeTok{seed =} \DecValTok{1}\NormalTok{)}
\NormalTok{  )}
\end{Highlighting}
\end{Shaded}

Aquí están las estimaciones (usando el paquete de R \texttt{texreg} para
una salida más bonita):

\begin{Shaded}
\begin{Highlighting}[]
\NormalTok{texreg}\SpecialCharTok{::}\FunctionTok{screenreg}\NormalTok{(}\FunctionTok{list}\NormalTok{(}\AttributeTok{Bipartite =}\NormalTok{ res\_b, }\AttributeTok{Directed =}\NormalTok{ res\_not\_b))}
\end{Highlighting}
\end{Shaded}

\begin{verbatim}

======================================================
                              Bipartite    Directed   
------------------------------------------------------
edges                           -3.14 ***    -3.14 ***
                                (0.15)       (0.15)   
Log~b1star2                     21.89                 
                               (17.13)                
Log~ostar2                                   22.40    
                                            (16.39)   
offset(nodematch.is.actor)                    -Inf    
                                                      
offset(nodeocov.isnot.actor)                  -Inf    
                                                      
------------------------------------------------------
AIC                           1958.00      1957.57    
BIC                           1971.03      1973.60    
Log Likelihood                -977.00      -976.78    
======================================================
*** p < 0.001; ** p < 0.01; * p < 0.05
\end{verbatim}

Como se esperaba, ambos modelos producen la ``misma'' estimación. Las
diferencias menores observadas entre los modelos son cómo el paquete
\texttt{ergm} realiza el muestreo. En particular, en el caso bipartito,
\texttt{ergm} tiene rutinas especiales para hacer el muestreo más
eficiente, teniendo una tasa de aceptación más alta que la del modelo en
el que el grafo bipartito no fue explícitamente declarado. Podemos decir
esto inspeccionando las tasas de rechazo:

\begin{Shaded}
\begin{Highlighting}[]
\FunctionTok{data.frame}\NormalTok{(}
  \AttributeTok{Bipartite =}\NormalTok{ coda}\SpecialCharTok{::}\FunctionTok{rejectionRate}\NormalTok{(res\_b}\SpecialCharTok{$}\NormalTok{sample[[}\DecValTok{1}\NormalTok{]]) }\SpecialCharTok{*} \DecValTok{100}\NormalTok{,}
  \AttributeTok{Directed  =}\NormalTok{ coda}\SpecialCharTok{::}\FunctionTok{rejectionRate}\NormalTok{(res\_not\_b}\SpecialCharTok{$}\NormalTok{sample[[}\DecValTok{1}\NormalTok{]][, }\SpecialCharTok{{-}}\FunctionTok{c}\NormalTok{(}\DecValTok{3}\NormalTok{,}\DecValTok{4}\NormalTok{)]) }\SpecialCharTok{*} \DecValTok{100}
\NormalTok{) }\SpecialCharTok{|\textgreater{}}\NormalTok{ knitr}\SpecialCharTok{::}\FunctionTok{kable}\NormalTok{(}\AttributeTok{digits =} \DecValTok{2}\NormalTok{, }\AttributeTok{caption =} \StringTok{"Tasa de rechazo (porcentaje)"}\NormalTok{)}
\end{Highlighting}
\end{Shaded}

\begin{longtable}[]{@{}lrr@{}}
\caption{Tasa de rechazo (porcentaje)}\tabularnewline
\toprule\noalign{}
& Bipartite & Directed \\
\midrule\noalign{}
\endfirsthead
\toprule\noalign{}
& Bipartite & Directed \\
\midrule\noalign{}
\endhead
\bottomrule\noalign{}
\endlastfoot
edges & 2.48 & 4.47 \\
Log\textasciitilde b1star2 & 1.24 & 1.68 \\
\end{longtable}

El ERGM ajustado con los términos offset tiene una tasa de rechazo mucho
más alta que la del ERGM ajustado con el ERGM bipartito.

Finalmente, el hecho de que podamos ajustar ERGMs usando offset no
significa que necesitemos usarlo TODO el tiempo. A menos que haya una
muy buena razón para evitar las capacidades de \texttt{ergm}, no
recomendaría ajustar ERGMs bipartitos como acabamos de hacer, ya que los
autores del paquete han incluido (MUCHAS) características para hacer
nuestro trabajo más fácil.

\chapter{Modelos de Grafos Aleatorios de Familia Exponencial
Temporal}\label{modelos-de-grafos-aleatorios-de-familia-exponencial-temporal}

\begin{tcolorbox}[enhanced jigsaw, colback=white, opacityback=0, coltitle=black, title=\textcolor{quarto-callout-warning-color}{\faExclamationTriangle}\hspace{0.5em}{Nota de Traducción}, bottomrule=.15mm, colbacktitle=quarto-callout-warning-color!10!white, toptitle=1mm, colframe=quarto-callout-warning-color-frame, titlerule=0mm, rightrule=.15mm, leftrule=.75mm, breakable, bottomtitle=1mm, left=2mm, arc=.35mm, toprule=.15mm, opacitybacktitle=0.6]

Esta versión del capítulo fue traducida de manera automática utilizando
IA. El capítulo aún no ha sido revisado por un humano.

\end{tcolorbox}

¡Este tutorial es genial!
\url{https://statnet.org/trac/raw-attachment/wiki/Sunbelt2016/tergm_tutorial.pdf}

\chapter{Pruebas de hipótesis en
redes}\label{pruebas-de-hipuxf3tesis-en-redes}

\begin{tcolorbox}[enhanced jigsaw, colback=white, opacityback=0, coltitle=black, title=\textcolor{quarto-callout-warning-color}{\faExclamationTriangle}\hspace{0.5em}{Nota de Traducción}, bottomrule=.15mm, colbacktitle=quarto-callout-warning-color!10!white, toptitle=1mm, colframe=quarto-callout-warning-color-frame, titlerule=0mm, rightrule=.15mm, leftrule=.75mm, breakable, bottomtitle=1mm, left=2mm, arc=.35mm, toprule=.15mm, opacitybacktitle=0.6]

Esta versión del capítulo fue traducida de manera automática utilizando
IA. El capítulo aún no ha sido revisado por un humano.

\end{tcolorbox}

En general, hay muchas formas en las que podemos ver las pruebas de
hipótesis dentro del contexto de redes:

\begin{enumerate}
\def\labelenumi{\arabic{enumi}.}
\item
  \textbf{Comparar dos o más redes}, p.~ej., queremos ver si la densidad
  de dos redes son \emph{iguales}.
\item
  \textbf{Prevalencia de un motivo/patrón}, p.~ej., verificar si el
  número observado de tríadas transitivas es diferente del esperado por
  casualidad.
\item
  \textbf{Multivariado usando ERGMs}, p.~ej., probar conjuntamente si la
  homofilia y las estrellas de dos puntos son los motivos que impulsan
  la estructura de red.
\end{enumerate}

Esta última ya la revisamos en el capítulo de ERGM. En esta parte,
veremos los tipos uno y dos; ambos usando métodos no paramétricos.

\section{Comparando redes}\label{comparando-redes}

Imagina que tenemos dos grafos, \((G_1,G_2) \in \mathcal{G}\), y nos
gustaría evaluar si una estadística dada \(s(\cdot)\), p.~ej., densidad,
es igual en ambos. Formalmente, nos gustaría evaluar si
\(H_0: s(G_1) - s(G_2) = k\) vs \(H_a: s(G_1) - s(G_2) \neq k\).

Como es usual, la distribución verdadera de \(s(\cdot)\) es desconocida,
por lo tanto, un enfoque que podríamos usar es una prueba bootstrap no
paramétrica.

\subsection{Bootstrap de redes}\label{bootstrap-de-redes}

Los métodos bootstrap no paramétrico y jackknife para redes sociales
fueron introducidos por (Tom A. B. Snijders and Borgatti 1999). El
método mismo se usa para generar errores estándar para estadísticas a
nivel de red. Ambos métodos están implementados en el paquete de R
\href{https://cran.r-project.org/package=netdiffuseR}{\texttt{netdiffuseR}}.

\subsection{Cuando la estadística es
normal}\label{cuando-la-estaduxedstica-es-normal}

Cuando tratamos con cosas que están distribuidas normalmente, p.~ej.,
medias muestrales como la densidad\footnote{La densidad es en efecto una
  media muestral ya que estamos, en principio calculando el promedio de
  una secuencia de variables de Bernoulli. Formalmente:
  \(\text{densidad}(G) = \frac{1}{n(n-1)}\sum_{ij}A_{ij}\).}, podemos
hacer uso de la distribución de Student para hacer inferencia. En
particular, podemos usar Bootstrap/Jackknife para aproximar los errores
estándar de la estadística para cada red:

\begin{enumerate}
\def\labelenumi{\arabic{enumi}.}
\item
  Dado que \(s(G_i)\sim \text{N}(\mu_i,\sigma_i^2/m_i)\) para
  \(i\in\{1,2\}\), en el caso de la densidad, \(m_i = n_i * (n_i - 1)\).
  La estadística es entonces:

  \[
  s(G_1) - s(G_0)\sim \text{N}(\mu_1-\mu_0, \sigma_1^2/m_1 + \sigma_1^2/m_2)
  \]

  Por lo tanto

  \[
  \frac{s(G_1) - s(G_0) - \mu_1 + \mu_2}{\sqrt{\sigma_1^2/{m_1} + \sigma_1^2/{m_2}}} \sim t_{m_1 + m_2 - 2}
  \] Pero, si estamos probando \(H_0: \mu_1 - \mu_2 = k\), entonces,
  bajo la nula

  \[
  \frac{s(G_1) - s(G_0) - k}{\sqrt{\sigma_1^2/{m_1} + \sigma_1^2/{m_2}}} \sim t_{m_1 + m_2 - 2}
  \] Donde ahora procedemos a aproximar las varianzas.
\item
  Usando el \emph{principio plugin} (Efron and Tibshirani 1994), podemos
  aproximar las varianzas usando Bootstrap/Jackknife, es decir, calcular
  \(\hat\sigma_1^2\approx\sigma_1^2/m_1\) y
  \(\hat\sigma_2^2\approx\sigma_2^2/m_2\). Usando netdiffuseR

\begin{Shaded}
\begin{Highlighting}[]
\FunctionTok{library}\NormalTok{(netdiffuseR)}

\CommentTok{\# Obtener 100 réplicas}
\NormalTok{sg1 }\OtherTok{\textless{}{-}} \FunctionTok{bootnet}\NormalTok{(g1, }\ControlFlowTok{function}\NormalTok{(i, ...) }\FunctionTok{sum}\NormalTok{(i)}\SpecialCharTok{/}\NormalTok{(}\FunctionTok{nnodes}\NormalTok{(i) }\SpecialCharTok{*}\NormalTok{ (}\FunctionTok{nnodes}\NormalTok{(i) }\SpecialCharTok{{-}} \DecValTok{1}\NormalTok{)), }\AttributeTok{R =} \DecValTok{100}\NormalTok{)}
\NormalTok{sg2 }\OtherTok{\textless{}{-}} \FunctionTok{bootnet}\NormalTok{(g2, }\ControlFlowTok{function}\NormalTok{(i, ...) }\FunctionTok{sum}\NormalTok{(i)}\SpecialCharTok{/}\NormalTok{(}\FunctionTok{nnodes}\NormalTok{(i) }\SpecialCharTok{*}\NormalTok{ (}\FunctionTok{nnodes}\NormalTok{(i) }\SpecialCharTok{{-}} \DecValTok{1}\NormalTok{)), }\AttributeTok{R =} \DecValTok{100}\NormalTok{)}

\CommentTok{\# Recuperando las varianzas}
\NormalTok{hat\_sigma1 }\OtherTok{\textless{}{-}}\NormalTok{ sg1}\SpecialCharTok{$}\NormalTok{var\_t}
\NormalTok{hat\_sigma2 }\OtherTok{\textless{}{-}}\NormalTok{ sg2}\SpecialCharTok{$}\NormalTok{var\_t}

\CommentTok{\# Y los valores reales}
\NormalTok{sg1 }\OtherTok{\textless{}{-}}\NormalTok{ sg1}\SpecialCharTok{$}\NormalTok{t0}
\NormalTok{sg2 }\OtherTok{\textless{}{-}}\NormalTok{ sg2}\SpecialCharTok{$}\NormalTok{t0}
\end{Highlighting}
\end{Shaded}
\item
  Con las aproximaciones en mano, podemos entonces usar la ``tabla de
  prueba t'' para recuperar el valor correspondiente, en R:

\begin{Shaded}
\begin{Highlighting}[]
\CommentTok{\# Construyendo la estadística}
\NormalTok{k }\OtherTok{\textless{}{-}} \DecValTok{0} \CommentTok{\# Para varianzas iguales}
\NormalTok{tstat }\OtherTok{\textless{}{-}}\NormalTok{ (sg1 }\SpecialCharTok{{-}}\NormalTok{ sg2 }\SpecialCharTok{{-}}\NormalTok{ k)}\SpecialCharTok{/}\NormalTok{(}\FunctionTok{sqrt}\NormalTok{(hat\_sigma1 }\SpecialCharTok{+}\NormalTok{ hat\_sigma2))}

\CommentTok{\# Calculando el valor p}
\NormalTok{m1 }\OtherTok{\textless{}{-}} \FunctionTok{nnodes}\NormalTok{(g1)}\SpecialCharTok{*}\NormalTok{(}\FunctionTok{nnodes}\NormalTok{(g1) }\SpecialCharTok{{-}} \DecValTok{1}\NormalTok{)}
\NormalTok{m2 }\OtherTok{\textless{}{-}} \FunctionTok{nnodes}\NormalTok{(g2)}\SpecialCharTok{*}\NormalTok{(}\FunctionTok{nnodes}\NormalTok{(g2) }\SpecialCharTok{{-}} \DecValTok{1}\NormalTok{)}
\FunctionTok{pt}\NormalTok{(tstat, }\AttributeTok{df =}\NormalTok{ m1 }\SpecialCharTok{+}\NormalTok{ m2 }\SpecialCharTok{{-}} \DecValTok{2}\NormalTok{)}
\end{Highlighting}
\end{Shaded}
\end{enumerate}

\subsection{Cuando la estadística NO es
normal}\label{cuando-la-estaduxedstica-no-es-normal}

En el caso de que la estadística no esté distribuida normalmente, ya no
podemos usar la estadística t. Sin embargo, el Bootstrap puede venir a
ayudar. Aunque en general es mejor usar distribuciones de estadísticas
pivote (ver (Efron and Tibshirani 1994)), aún podemos aprovechar el
poder de este método para hacer inferencias. Para este ejemplo,
\(s(\cdot)\) será el rango del umbral en un grafo de difusión.

Como antes, imagina que estamos tratando con una estadística
\(s(\cdot)\) para dos redes diferentes, y nos gustaría evaluar si
podemos rechazar \(H_0\) o
\href{https://www.thoughtco.com/fail-to-reject-in-a-hypothesis-test-3126424}{fallar
en rechazarla}. El procedimiento es muy similar:

\begin{enumerate}
\def\labelenumi{\arabic{enumi}.}
\item
  Un enfoque que podemos probar es si
  \(k \in \text{ConfInt}(s(G_1) - s(G_2))\). Construir intervalos de
  confianza con bootstrap podría ser más intuitivo.
\item
  Como antes, usamos bootstrap para generar una distribución de
  \(s(G_1)\) y \(s(G_2)\), en R:

\begin{Shaded}
\begin{Highlighting}[]
\CommentTok{\# Obtener 1000 réplicas}
\NormalTok{sg1 }\OtherTok{\textless{}{-}} \FunctionTok{bootnet}\NormalTok{(g1, }\ControlFlowTok{function}\NormalTok{(i, ...) }\FunctionTok{range}\NormalTok{(}\FunctionTok{threshold}\NormalTok{(i)), }\AttributeTok{R =} \DecValTok{1000}\NormalTok{)}
\NormalTok{sg2 }\OtherTok{\textless{}{-}} \FunctionTok{bootnet}\NormalTok{(g2, }\ControlFlowTok{function}\NormalTok{(i, ...) }\FunctionTok{range}\NormalTok{(}\FunctionTok{threshold}\NormalTok{(i)), }\AttributeTok{R =} \DecValTok{1000}\NormalTok{)}

\CommentTok{\# Recuperando las distribuciones}
\NormalTok{sg1 }\OtherTok{\textless{}{-}}\NormalTok{ sg1}\SpecialCharTok{$}\NormalTok{boot}\SpecialCharTok{$}\NormalTok{t}
\NormalTok{sg2 }\OtherTok{\textless{}{-}}\NormalTok{ sg2}\SpecialCharTok{$}\NormalTok{boot}\SpecialCharTok{$}\NormalTok{t}

\CommentTok{\# Definir la estadística}
\NormalTok{sdiff }\OtherTok{\textless{}{-}}\NormalTok{ sg1 }\SpecialCharTok{{-}}\NormalTok{ sg2}
\end{Highlighting}
\end{Shaded}
\item
  Una vez que tenemos \texttt{sdiff}, podemos proceder y calcular el,
  por ejemplo, 95\% intervalo de confianza, y evaluar si \(k\) cae
  dentro. En R:

\begin{Shaded}
\begin{Highlighting}[]
\NormalTok{diff\_ci }\OtherTok{\textless{}{-}} \FunctionTok{quantile}\NormalTok{(sdiff, }\AttributeTok{probs =} \FunctionTok{c}\NormalTok{(}\FloatTok{0.025}\NormalTok{, .}\DecValTok{975}\NormalTok{))}
\end{Highlighting}
\end{Shaded}
\end{enumerate}

Esto corresponde a lo que Efron y Tibshirani llaman ``intervalo de
percentil.'' Esto es fácil de calcular, pero un mejor enfoque es usar el
método ``BCa'', ``Corregido por Sesgo y Acelerado.'' (TBD)

\section{Ejemplos}\label{ejemplos}

\subsection{Promedio de estadísticas a nivel de
nodo}\label{promedio-de-estaduxedsticas-a-nivel-de-nodo}

Supón que nos gustaría comparar algo como el grado de entrada promedio.
En particular, para ambas redes, \(G_1\) y \(G_2\), calculamos el grado
de entrada promedio por nodo:

\[
s(G_1) = \text{GradoEntProm}(G_1) = \frac{1}{n}\sum_{i}\sum_{j\neq i}A^1_{ji}
\]

\noindent donde \(A^1_{ji}\) es igual a uno si el vértice \(j\) envía un
vínculo a \(i\). En este caso, dado que estamos viendo un promedio,
tenemos que \(\text{GradoEntProm}(G_1) \sim N(\mu_1, \sigma^2_1/n)\).
Por lo tanto, aprovechando la normalidad de la estadística, podemos
construir una estadística de prueba como sigue:

\[
\frac{s(G_1) - s(G_2) - k}{\sqrt{\hat\sigma_{1}^2 + \hat\sigma_{2}^2}} \sim t_{n_1 + n_2 - 2}
\] Donde \(\hat\sigma_i\) es el error estándar bootstrap, y \(k = 0\)
cuando estamos probando igualdad. Esto se distribuye \(t\) con
\(n_1+n_2-2\) grados de libertad. Como diferencia del ejemplo anterior
usando densidad, los grados de libertad para esta prueba son menores ya
que, en lugar de tener un promedio a través de todas las entradas de la
matriz de adyacencia, tenemos un promedio a través de todos los
vértices.

\chapter{Modelos Estocásticos Orientados al
Actor}\label{modelos-estocuxe1sticos-orientados-al-actor}

\begin{tcolorbox}[enhanced jigsaw, colback=white, opacityback=0, coltitle=black, title=\textcolor{quarto-callout-warning-color}{\faExclamationTriangle}\hspace{0.5em}{Nota de Traducción}, bottomrule=.15mm, colbacktitle=quarto-callout-warning-color!10!white, toptitle=1mm, colframe=quarto-callout-warning-color-frame, titlerule=0mm, rightrule=.15mm, leftrule=.75mm, breakable, bottomtitle=1mm, left=2mm, arc=.35mm, toprule=.15mm, opacitybacktitle=0.6]

Esta versión del capítulo fue traducida de manera automática utilizando
IA. El capítulo aún no ha sido revisado por un humano.

\end{tcolorbox}

Los Modelos Estocásticos Orientados al Actor (SOAM), también conocidos
como modelos Siena fueron introducidos por CITATION NEEDED.

Como diferencia de los ERGMs, los modelos Siena observan el proceso de
generación de datos desde el punto de vista de los individuos. Basado en
las ideas de McFadden sobre elección probabilística, el modelo se
fundamenta en la siguiente ecuación

\[
U_i(x) - U_i(x') \sim \text{Distribución de Valor Extremo}
\]

En otras palabras, los individuos eligen entre estados \(x\) y \(x'\) de
manera probabilística (con algo de ruido),

\[
\frac{\text{exp}\left\{f_i^Z(\beta^z,x, z)\right\}}{\sum_{Z'\in\mathcal{C}}\text{exp}\left\{f_i^{Z}(\beta, x, z')\right\}}
\]

snijders\_(sociological methodology 2001)

Ripley et al. (2011)

\chapter{Cálculo de poder en estudios de
redes}\label{cuxe1lculo-de-poder-en-estudios-de-redes}

\begin{tcolorbox}[enhanced jigsaw, colback=white, opacityback=0, coltitle=black, title=\textcolor{quarto-callout-warning-color}{\faExclamationTriangle}\hspace{0.5em}{Nota de Traducción}, bottomrule=.15mm, colbacktitle=quarto-callout-warning-color!10!white, toptitle=1mm, colframe=quarto-callout-warning-color-frame, titlerule=0mm, rightrule=.15mm, leftrule=.75mm, breakable, bottomtitle=1mm, left=2mm, arc=.35mm, toprule=.15mm, opacitybacktitle=0.6]

Esta versión del capítulo fue traducida de manera automática utilizando
IA. El capítulo aún no ha sido revisado por un humano.

\end{tcolorbox}

En el diseño de encuestas y estudios, calcular el tamaño de muestra
requerido es crítico. Hoy en día, abundan las herramientas y métodos
para calcular el tamaño de muestra requerido; no obstante, el cálculo de
tamaño de muestra para estudios que involucran redes sociales aún está
subdesarrollado. Este capítulo ilustrará cómo podemos usar simulaciones
por computadora para estimar el tamaño de muestra requerido. El Capítulo
Chapter~\ref{sec-part2-power} proporciona una vista general del análisis
de poder.

\section[Ejemplo 1: Efectos de derrame en estudios
egocéntricos]{\texorpdfstring{Ejemplo 1: Efectos de derrame en estudios
egocéntricos\footnote{El problema original fue planteado por
  \href{https://faculty.utah.edu/u6037777-SHINDUK_LEE/hm/index.hml}{Dr.~Shinduk
  Lee} de la Escuela de Enfermería de la Universidad de Utah.}}{Ejemplo 1: Efectos de derrame en estudios egocéntricos}}\label{ejemplo-1-efectos-de-derrame-en-estudios-egocuxe9ntricoscredit-ego-power}

Supón que queremos ejecutar una intervención sobre una población
particular, y estamos interesados en los efectos de tal intervención en
los alters de los egos. En economía, este problema, que llaman el
``efecto de derrame,'' se estudia activamente.

Asumimos que los alters solo se exponen si los egos adquieren el
comportamiento para el cálculo de poder. Además, para esta primera
ejecución, asumiremos que no hay refuerzo social o influencia entre
alters. Posteriormente relajaremos esta suposición. Para calcular el
poder, haremos lo siguiente:

\begin{enumerate}
\def\labelenumi{\arabic{enumi}.}
\item
  Simular el comportamiento de los egos siguiendo una distribución
  logit.
\item
  Eliminar aleatoriamente algunos egos como resultado de la deserción.
\item
  Simular el comportamiento de los alters usando sus egos como el
  tratamiento.
\item
  Ajustar una regresión logística basada en el modelo anterior.
\item
  Aceptar/rechazar la nula y almacenar el resultado.
\end{enumerate}

Los pasos anteriores se repetirán 500 veces para cada valor de \(n\) que
analicemos. Finalizaremos graficando el poder contra los tamaños de
muestra. Comencemos primero escribiendo los parámetros de simulación:

\begin{Shaded}
\begin{Highlighting}[]
\CommentTok{\# Diseño}
\NormalTok{n\_sims    }\OtherTok{\textless{}{-}} \DecValTok{500} \CommentTok{\# Número de simulaciones}
\NormalTok{n\_a       }\OtherTok{\textless{}{-}} \DecValTok{4}   \CommentTok{\# Número de alters}
\NormalTok{sizes     }\OtherTok{\textless{}{-}}     \CommentTok{\# Tamaños a probar}
  \FunctionTok{seq}\NormalTok{(}\AttributeTok{from =} \DecValTok{100}\NormalTok{, }\AttributeTok{to =} \DecValTok{200}\NormalTok{, }\AttributeTok{by =} \DecValTok{25}\NormalTok{)}

\CommentTok{\# Suposiciones}
\NormalTok{odds\_h\_1  }\OtherTok{\textless{}{-}} \FloatTok{2.0} \CommentTok{\# Odds de Aumento/}
\NormalTok{attrition }\OtherTok{\textless{}{-}}\NormalTok{ .}\DecValTok{3}
\NormalTok{baseline  }\OtherTok{\textless{}{-}}\NormalTok{ .}\DecValTok{2}  \CommentTok{\# Baja prevalencia en 1s}

\CommentTok{\# Parámetros}
\NormalTok{alpha    }\OtherTok{\textless{}{-}}\NormalTok{ .}\DecValTok{05}
\NormalTok{beta\_pow }\OtherTok{\textless{}{-}} \FloatTok{0.2}
\end{Highlighting}
\end{Shaded}

Como discutimos en Chapter~\ref{sec-part2-power}, siempre es una buena
idea encapsular la simulación en una función:

\begin{Shaded}
\begin{Highlighting}[]
\CommentTok{\# Los odds convertidos a una prob}
\NormalTok{theta\_h\_1 }\OtherTok{\textless{}{-}} \FunctionTok{plogis}\NormalTok{(}\FunctionTok{log}\NormalTok{(odds\_h\_1))}

\CommentTok{\# Función de simulación}
\NormalTok{sim\_data }\OtherTok{\textless{}{-}} \ControlFlowTok{function}\NormalTok{(n) \{}

  \CommentTok{\# Asignación de tratamiento}
\NormalTok{  tr  }\OtherTok{\textless{}{-}} \FunctionTok{c}\NormalTok{(}\FunctionTok{rep}\NormalTok{(}\DecValTok{1}\NormalTok{, n}\SpecialCharTok{/}\DecValTok{2}\NormalTok{), }\FunctionTok{rep}\NormalTok{(}\DecValTok{0}\NormalTok{, n}\SpecialCharTok{/}\DecValTok{2}\NormalTok{))}

  \CommentTok{\# Paso 1: Muestreando población de egos}
\NormalTok{  y\_ego }\OtherTok{\textless{}{-}} \FunctionTok{runif}\NormalTok{(n) }\SpecialCharTok{\textless{}} \FunctionTok{c}\NormalTok{(}
    \FunctionTok{rep}\NormalTok{(theta\_h\_1, n}\SpecialCharTok{/}\DecValTok{2}\NormalTok{),}
    \FunctionTok{rep}\NormalTok{(}\FloatTok{0.5}\NormalTok{, n}\SpecialCharTok{/}\DecValTok{2}\NormalTok{)}
\NormalTok{  )}

  \CommentTok{\# Paso 2: Simulando deserción}
\NormalTok{  todrop }\OtherTok{\textless{}{-}} \FunctionTok{order}\NormalTok{(}\FunctionTok{runif}\NormalTok{(n))[}\DecValTok{1}\SpecialCharTok{:}\NormalTok{(n }\SpecialCharTok{*}\NormalTok{ attrition)]}
\NormalTok{  y\_ego  }\OtherTok{\textless{}{-}}\NormalTok{ y\_ego[}\SpecialCharTok{{-}}\NormalTok{todrop]}
\NormalTok{  tr     }\OtherTok{\textless{}{-}}\NormalTok{ tr[}\SpecialCharTok{{-}}\NormalTok{todrop]}
\NormalTok{  n      }\OtherTok{\textless{}{-}}\NormalTok{ n }\SpecialCharTok{{-}} \FunctionTok{length}\NormalTok{(todrop)}

  \CommentTok{\# Paso 3: Simulando efecto del alter. Asumimos lo mismo que en}
  \CommentTok{\# ego}
\NormalTok{  tr\_alter }\OtherTok{\textless{}{-}} \FunctionTok{rep}\NormalTok{(y\_ego }\SpecialCharTok{*}\NormalTok{ tr, n\_a)}
\NormalTok{  y\_alter  }\OtherTok{\textless{}{-}} \FunctionTok{runif}\NormalTok{(n }\SpecialCharTok{*}\NormalTok{ n\_a) }\SpecialCharTok{\textless{}} \FunctionTok{ifelse}\NormalTok{(tr\_alter, theta\_h\_1, }\FloatTok{0.5}\NormalTok{)}

  \CommentTok{\# Paso 4: Calculando estadística de prueba}
\NormalTok{  res\_ego   }\OtherTok{\textless{}{-}} \FunctionTok{tryCatch}\NormalTok{(}\FunctionTok{glm}\NormalTok{(y\_ego }\SpecialCharTok{\textasciitilde{}}\NormalTok{ tr, }\AttributeTok{family =} \FunctionTok{binomial}\NormalTok{(}\StringTok{"logit"}\NormalTok{)), }\AttributeTok{error =} \ControlFlowTok{function}\NormalTok{(e) e)}
\NormalTok{  res\_alter }\OtherTok{\textless{}{-}} \FunctionTok{tryCatch}\NormalTok{(}\FunctionTok{glm}\NormalTok{(y\_alter }\SpecialCharTok{\textasciitilde{}}\NormalTok{ tr\_alter, }\AttributeTok{family =} \FunctionTok{binomial}\NormalTok{(}\StringTok{"logit"}\NormalTok{)), }\AttributeTok{error =} \ControlFlowTok{function}\NormalTok{(e) e)}

  \ControlFlowTok{if}\NormalTok{ (}\FunctionTok{inherits}\NormalTok{(res\_ego, }\StringTok{"error"}\NormalTok{) }\SpecialCharTok{|} \FunctionTok{inherits}\NormalTok{(res\_alter, }\StringTok{"error"}\NormalTok{))}
    \FunctionTok{return}\NormalTok{(}\FunctionTok{c}\NormalTok{(}\AttributeTok{ego =}  \ConstantTok{NA}\NormalTok{, }\AttributeTok{alter =} \ConstantTok{NA}\NormalTok{))}
  
  \CommentTok{\# Paso 5: ¿Rechazar?}
  \FunctionTok{c}\NormalTok{(}
    \AttributeTok{ego   =} \FunctionTok{summary}\NormalTok{(res\_ego)}\SpecialCharTok{$}\NormalTok{coefficients[}\StringTok{"tr"}\NormalTok{, }\StringTok{"Pr(\textgreater{}|z|)"}\NormalTok{] }\SpecialCharTok{\textless{}}\NormalTok{ alpha,}
    \AttributeTok{alter =} \FunctionTok{summary}\NormalTok{(res\_alter)}\SpecialCharTok{$}\NormalTok{coefficients[}\StringTok{"tr\_alter"}\NormalTok{, }\StringTok{"Pr(\textgreater{}|z|)"}\NormalTok{] }\SpecialCharTok{\textless{}}\NormalTok{ alpha}
\NormalTok{  )}
  

\NormalTok{\}}
\end{Highlighting}
\end{Shaded}

Ahora que tenemos la función de generación de datos, podemos ejecutar
las simulaciones para aproximar el poder estadístico dado el tamaño de
muestra. Los resultados se almacenarán en la matriz \texttt{spower}.
Dado que estamos simulando datos, es crucial establecer la semilla para
que podamos reproducir los resultados.

\begin{Shaded}
\begin{Highlighting}[]
\CommentTok{\# Siempre establecemos la semilla}
\FunctionTok{set.seed}\NormalTok{(}\DecValTok{88}\NormalTok{) }

\CommentTok{\# Haciendo espacio, ¡y ejecutando!}
\NormalTok{spower }\OtherTok{\textless{}{-}} \ConstantTok{NULL}
\ControlFlowTok{for}\NormalTok{ (s }\ControlFlowTok{in}\NormalTok{ sizes) \{}

  \CommentTok{\# Ejecutar la simulación para el tamaño s}
\NormalTok{  simres }\OtherTok{\textless{}{-}} \FunctionTok{rowMeans}\NormalTok{(}\FunctionTok{replicate}\NormalTok{(n\_sims, }\FunctionTok{sim\_data}\NormalTok{(s)), }\AttributeTok{na.rm =} \ConstantTok{TRUE}\NormalTok{)}

  \CommentTok{\# Y almacenar los resultados}
\NormalTok{  spower }\OtherTok{\textless{}{-}} \FunctionTok{rbind}\NormalTok{(spower, simres)}

\NormalTok{\}}
\end{Highlighting}
\end{Shaded}

La siguiente figura muestra el poder aproximado para encontrar efectos
en ambos niveles, ego y alter:

\begin{Shaded}
\begin{Highlighting}[]
\FunctionTok{library}\NormalTok{(ggplot2)}

\NormalTok{spower }\OtherTok{\textless{}{-}} \FunctionTok{rbind}\NormalTok{(}
  \FunctionTok{data.frame}\NormalTok{(}\AttributeTok{size =}\NormalTok{ sizes, }\AttributeTok{power =}\NormalTok{ spower[,}\StringTok{"ego"}\NormalTok{], }\AttributeTok{type =}  \StringTok{"ego"}\NormalTok{),}
  \FunctionTok{data.frame}\NormalTok{(}\AttributeTok{size =}\NormalTok{ sizes, }\AttributeTok{power =}\NormalTok{ spower[,}\StringTok{"alter"}\NormalTok{], }\AttributeTok{type =}  \StringTok{"alter"}\NormalTok{)}
\NormalTok{)}

\NormalTok{spower }\SpecialCharTok{|\textgreater{}}
  \FunctionTok{ggplot}\NormalTok{(}\FunctionTok{aes}\NormalTok{(}\AttributeTok{x =}\NormalTok{ size, }\AttributeTok{y =}\NormalTok{ power, }\AttributeTok{colour =}\NormalTok{ type)) }\SpecialCharTok{+}
  \FunctionTok{geom\_point}\NormalTok{() }\SpecialCharTok{+}
  \FunctionTok{geom\_smooth}\NormalTok{(}\AttributeTok{method =} \StringTok{"loess"}\NormalTok{, }\AttributeTok{formula =}\NormalTok{ y }\SpecialCharTok{\textasciitilde{}}\NormalTok{ x) }\SpecialCharTok{+}
  \FunctionTok{labs}\NormalTok{(}\AttributeTok{x =} \StringTok{"Número de Egos"}\NormalTok{, }\AttributeTok{y =} \StringTok{"Poder Aprox."}\NormalTok{, }\AttributeTok{colour =} \StringTok{"Tipo de nodo"}\NormalTok{) }\SpecialCharTok{+}
  \FunctionTok{geom\_hline}\NormalTok{(}\AttributeTok{yintercept =} \DecValTok{1} \SpecialCharTok{{-}}\NormalTok{ beta\_pow)}
\end{Highlighting}
\end{Shaded}

\pandocbounded{\includegraphics[keepaspectratio]{part-01-11-power_files/figure-pdf/part-01-power-plot1-1.pdf}}

Como se muestra en el Capítulo Chapter~\ref{sec-part2-power}, podemos
usar un modelo de regresión lineal para predecir el tamaño de muestra
como una función del poder estadístico:

\begin{Shaded}
\begin{Highlighting}[]
\CommentTok{\# Ajustando el modelo}
\NormalTok{power\_model }\OtherTok{\textless{}{-}} \FunctionTok{glm}\NormalTok{(}
\NormalTok{  size }\SpecialCharTok{\textasciitilde{}}\NormalTok{ power }\SpecialCharTok{+} \FunctionTok{I}\NormalTok{(power}\SpecialCharTok{\^{}}\DecValTok{2}\NormalTok{),}
  \AttributeTok{data =}\NormalTok{ spower, }\AttributeTok{family =} \FunctionTok{gaussian}\NormalTok{(), }\AttributeTok{subset =}\NormalTok{ type }\SpecialCharTok{==} \StringTok{"alter"}
\NormalTok{)}

\FunctionTok{summary}\NormalTok{(power\_model)}
\end{Highlighting}
\end{Shaded}

\begin{verbatim}

Call:
glm(formula = size ~ power + I(power^2), family = gaussian(), 
    data = spower, subset = type == "alter")

Coefficients:
            Estimate Std. Error t value Pr(>|t|)
(Intercept)     1460       1342   1.088    0.390
power          -3532       3124  -1.131    0.376
I(power^2)      2293       1805   1.270    0.332

(Dispersion parameter for gaussian family taken to be 317.0856)

    Null deviance: 6250.00  on 4  degrees of freedom
Residual deviance:  634.17  on 2  degrees of freedom
AIC: 46.404

Number of Fisher Scoring iterations: 2
\end{verbatim}

\begin{Shaded}
\begin{Highlighting}[]
\CommentTok{\# Predecir}
\FunctionTok{predict}\NormalTok{(power\_model, }\AttributeTok{newdata =} \FunctionTok{data.frame}\NormalTok{(}\AttributeTok{power =}\NormalTok{ .}\DecValTok{8}\NormalTok{), }\AttributeTok{type =} \StringTok{"response"}\NormalTok{) }\SpecialCharTok{|\textgreater{}}
  \FunctionTok{ceiling}\NormalTok{()}
\end{Highlighting}
\end{Shaded}

\begin{verbatim}
  1 
102 
\end{verbatim}

De la figura, se hace evidente que, aunque no hay suficiente poder para
identificar efectos a nivel ego, porque cada ego trae cinco alters, el
tamaño de muestra de alter es lo suficientemente alto como para que
podamos alcanzar por encima de 0.8 de poder estadístico con un tamaño de
muestra relativamente pequeño.

\section{Ejemplo 2: Efectos de derrame efecto
pre-post}\label{ejemplo-2-efectos-de-derrame-efecto-pre-post}

Ahora las dinámicas son diferentes. En lugar de tener un grupo tratado y
de control, tenemos un solo grupo sobre el cual mediremos el cambio de
comportamiento. Simularemos individuos en su estado inicial, aún 0/1, y
luego simularemos que la intervención los hará más propensos a tener
\(y = 1.\) También asumiremos que los sujetos generalmente no cambian su
comportamiento y que la prevalencia basal de ceros es más alta. Los
pasos de simulación son los siguientes:

\begin{enumerate}
\def\labelenumi{\arabic{enumi}.}
\item
  Para cada individuo en la población, extraer la probabilidad
  subyacente de que \(y = 1\). Con esa probabilidad, asignar el valor de
  \(y\). Esto aplica tanto para ego como para alter.
\item
  Eliminar aleatoriamente algunos egos, y sus alters correspondientes
  debido a la deserción.
\item
  Simular el comportamiento de los alters usando sus egos como el
  tratamiento. Tanto la probabilidad subyacente de ego como de alter se
  incrementan por los odds elegidos.
\item
  Para controlar la probabilidad subyacente de que un individuo tenga
  \(y = 1\), usamos regresión logística condicional (también conocida
  como logit de caso-control pareado,) para estimar los efectos del
  tratamiento.
\item
  Aceptar/rechazar la nula y almacenar el resultado.
\end{enumerate}

\begin{Shaded}
\begin{Highlighting}[]
\NormalTok{beta\_pars }\OtherTok{\textless{}{-}} \FunctionTok{c}\NormalTok{(}\DecValTok{4}\NormalTok{, }\DecValTok{6}\NormalTok{)}
\NormalTok{odds\_h\_1  }\OtherTok{\textless{}{-}} \FloatTok{2.0}
\end{Highlighting}
\end{Shaded}

\begin{Shaded}
\begin{Highlighting}[]
\CommentTok{\# Función de simulación}
\FunctionTok{library}\NormalTok{(survival)}
\NormalTok{sim\_data\_prepost }\OtherTok{\textless{}{-}} \ControlFlowTok{function}\NormalTok{(n) \{}


  \CommentTok{\# Paso 1: Muestreando población de egos}
\NormalTok{  y\_ego\_star }\OtherTok{\textless{}{-}} \FunctionTok{rbeta}\NormalTok{(n, beta\_pars[}\DecValTok{1}\NormalTok{], beta\_pars[}\DecValTok{2}\NormalTok{])}
\NormalTok{  y\_ego\_0    }\OtherTok{\textless{}{-}} \FunctionTok{runif}\NormalTok{(n) }\SpecialCharTok{\textless{}}\NormalTok{ y\_ego\_star}

  \CommentTok{\# Paso 2: Simulando deserción}
\NormalTok{  todrop     }\OtherTok{\textless{}{-}} \FunctionTok{order}\NormalTok{(}\FunctionTok{runif}\NormalTok{(n))[}\DecValTok{1}\SpecialCharTok{:}\NormalTok{(n }\SpecialCharTok{*}\NormalTok{ attrition)]}
\NormalTok{  y\_ego\_0    }\OtherTok{\textless{}{-}}\NormalTok{ y\_ego\_0[}\SpecialCharTok{{-}}\NormalTok{todrop]}
\NormalTok{  n          }\OtherTok{\textless{}{-}}\NormalTok{ n }\SpecialCharTok{{-}} \FunctionTok{length}\NormalTok{(todrop)}
\NormalTok{  y\_ego\_star }\OtherTok{\textless{}{-}}\NormalTok{ y\_ego\_star[}\SpecialCharTok{{-}}\NormalTok{todrop]}

  \CommentTok{\# Paso 3: Simulando efecto del alter. Asumimos lo mismo que en}
  \CommentTok{\# ego}
\NormalTok{  y\_alter\_star }\OtherTok{\textless{}{-}} \FunctionTok{rbeta}\NormalTok{(n }\SpecialCharTok{*}\NormalTok{ n\_a, beta\_pars[}\DecValTok{1}\NormalTok{], beta\_pars[}\DecValTok{2}\NormalTok{])}
\NormalTok{  y\_alter\_0    }\OtherTok{\textless{}{-}} \FunctionTok{runif}\NormalTok{(n }\SpecialCharTok{*}\NormalTok{ n\_a) }\SpecialCharTok{\textless{}}\NormalTok{ y\_alter\_star}

  \CommentTok{\# Simulando post}
\NormalTok{  y\_ego\_1   }\OtherTok{\textless{}{-}} \FunctionTok{runif}\NormalTok{(n) }\SpecialCharTok{\textless{}} \FunctionTok{plogis}\NormalTok{(}\FunctionTok{qlogis}\NormalTok{(y\_ego\_star) }\SpecialCharTok{+} \FunctionTok{log}\NormalTok{(odds\_h\_1))}
\NormalTok{  tr\_alter  }\OtherTok{\textless{}{-}} \FunctionTok{as.integer}\NormalTok{(}\FunctionTok{rep}\NormalTok{(y\_ego\_1, n\_a))}
\NormalTok{  y\_alter\_1 }\OtherTok{\textless{}{-}} \FunctionTok{runif}\NormalTok{(n }\SpecialCharTok{*}\NormalTok{ n\_a) }\SpecialCharTok{\textless{}} \FunctionTok{plogis}\NormalTok{(}\FunctionTok{qlogis}\NormalTok{(y\_alter\_star) }\SpecialCharTok{+} \FunctionTok{log}\NormalTok{(odds\_h\_1) }\SpecialCharTok{*}\NormalTok{ tr\_alter) }\CommentTok{\# Así que solo si ego hizo algo}

  \CommentTok{\# Paso 4: Calculando estadística de prueba}
\NormalTok{  y\_ego\_0 }\OtherTok{\textless{}{-}} \FunctionTok{as.integer}\NormalTok{(y\_ego\_0)}
\NormalTok{  y\_ego\_1 }\OtherTok{\textless{}{-}} \FunctionTok{as.integer}\NormalTok{(y\_ego\_1)}
\NormalTok{  y\_alter\_0 }\OtherTok{\textless{}{-}} \FunctionTok{as.integer}\NormalTok{(y\_alter\_0)}
\NormalTok{  y\_alter\_1 }\OtherTok{\textless{}{-}} \FunctionTok{as.integer}\NormalTok{(y\_alter\_1)}

\NormalTok{  d }\OtherTok{\textless{}{-}} \FunctionTok{data.frame}\NormalTok{(}
    \AttributeTok{y  =} \FunctionTok{c}\NormalTok{(y\_ego\_0, y\_ego\_1),}
    \AttributeTok{tr =} \FunctionTok{c}\NormalTok{(}\FunctionTok{rep}\NormalTok{(}\DecValTok{0}\NormalTok{, n), }\FunctionTok{rep}\NormalTok{(}\DecValTok{1}\NormalTok{, n)),}
    \AttributeTok{g  =} \FunctionTok{c}\NormalTok{(}\DecValTok{1}\SpecialCharTok{:}\NormalTok{n, }\DecValTok{1}\SpecialCharTok{:}\NormalTok{n)}
\NormalTok{  )}

\NormalTok{  res\_ego   }\OtherTok{\textless{}{-}} \FunctionTok{tryCatch}\NormalTok{(}
    \FunctionTok{clogit}\NormalTok{(y }\SpecialCharTok{\textasciitilde{}}\NormalTok{ tr }\SpecialCharTok{+} \FunctionTok{strata}\NormalTok{(g), }\AttributeTok{data =}\NormalTok{ d, }\AttributeTok{method =} \StringTok{"exact"}\NormalTok{),}
    \AttributeTok{error =} \ControlFlowTok{function}\NormalTok{(e) e}
\NormalTok{    )}

\NormalTok{  d }\OtherTok{\textless{}{-}} \FunctionTok{data.frame}\NormalTok{(}
    \AttributeTok{y  =} \FunctionTok{c}\NormalTok{(y\_alter\_0, y\_alter\_1),}
    \AttributeTok{tr =} \FunctionTok{c}\NormalTok{(}\FunctionTok{rep}\NormalTok{(}\DecValTok{0}\NormalTok{, n }\SpecialCharTok{*}\NormalTok{ n\_a), tr\_alter),}
    \AttributeTok{g  =} \FunctionTok{c}\NormalTok{(}\DecValTok{1}\SpecialCharTok{:}\NormalTok{(n }\SpecialCharTok{*}\NormalTok{ n\_a), }\DecValTok{1}\SpecialCharTok{:}\NormalTok{(n }\SpecialCharTok{*}\NormalTok{ n\_a))}
\NormalTok{  )}

\NormalTok{  res\_alter }\OtherTok{\textless{}{-}} \FunctionTok{tryCatch}\NormalTok{(}
    \FunctionTok{clogit}\NormalTok{(y }\SpecialCharTok{\textasciitilde{}}\NormalTok{ tr }\SpecialCharTok{+} \FunctionTok{strata}\NormalTok{(g), }\AttributeTok{data =}\NormalTok{ d, }\AttributeTok{method =} \StringTok{"exact"}\NormalTok{),}
    \AttributeTok{error =} \ControlFlowTok{function}\NormalTok{(e) e}
\NormalTok{    )}

  \ControlFlowTok{if}\NormalTok{ (}\FunctionTok{inherits}\NormalTok{(res\_ego, }\StringTok{"error"}\NormalTok{) }\SpecialCharTok{|} \FunctionTok{inherits}\NormalTok{(res\_alter, }\StringTok{"error"}\NormalTok{))}
    \FunctionTok{return}\NormalTok{(}\FunctionTok{c}\NormalTok{(}\AttributeTok{ego =}  \ConstantTok{NA}\NormalTok{, }\AttributeTok{alter =} \ConstantTok{NA}\NormalTok{))}
  
  \CommentTok{\# Paso 5: ¿Rechazar?}
  \FunctionTok{c}\NormalTok{(}
    \CommentTok{\# ego        = res\_ego$p.value \textless{} alpha,}
    \AttributeTok{ego        =} \FunctionTok{summary}\NormalTok{(res\_ego)}\SpecialCharTok{$}\NormalTok{coefficients[}\StringTok{"tr"}\NormalTok{, }\StringTok{"Pr(\textgreater{}|z|)"}\NormalTok{] }\SpecialCharTok{\textless{}}\NormalTok{ alpha,}
    \AttributeTok{alter      =} \FunctionTok{summary}\NormalTok{(res\_alter)}\SpecialCharTok{$}\NormalTok{coefficients[}\StringTok{"tr"}\NormalTok{, }\StringTok{"Pr(\textgreater{}|z|)"}\NormalTok{] }\SpecialCharTok{\textless{}}\NormalTok{ alpha,}
    \AttributeTok{ego\_test   =} \FunctionTok{coef}\NormalTok{(res\_ego),}
    \AttributeTok{alter\_glm  =} \FunctionTok{coef}\NormalTok{(res\_alter)}
\NormalTok{  )}
  

\NormalTok{\}}
\end{Highlighting}
\end{Shaded}

\begin{Shaded}
\begin{Highlighting}[]
\CommentTok{\# Siempre establecemos la semilla }
\FunctionTok{set.seed}\NormalTok{(}\DecValTok{88}\NormalTok{)}

\CommentTok{\# ¡Haciendo espacio y ejecutando!}
\NormalTok{spower }\OtherTok{\textless{}{-}} \ConstantTok{NULL}
\ControlFlowTok{for}\NormalTok{ (s }\ControlFlowTok{in}\NormalTok{ sizes) \{}

  \CommentTok{\# Ejecutar la simulación para el tamaño s}
\NormalTok{  simres }\OtherTok{\textless{}{-}} \FunctionTok{rowMeans}\NormalTok{(}
    \FunctionTok{replicate}\NormalTok{(n\_sims, }\FunctionTok{sim\_data\_prepost}\NormalTok{(s)),}
    \AttributeTok{na.rm =} \ConstantTok{TRUE}
\NormalTok{    )}

  \CommentTok{\# Y almacenar los resultados}
\NormalTok{  spower }\OtherTok{\textless{}{-}} \FunctionTok{rbind}\NormalTok{(spower, simres)}

\NormalTok{\}}
\end{Highlighting}
\end{Shaded}

\begin{Shaded}
\begin{Highlighting}[]
\FunctionTok{library}\NormalTok{(ggplot2)}

\NormalTok{spowerd }\OtherTok{\textless{}{-}} \FunctionTok{rbind}\NormalTok{(}
  \FunctionTok{data.frame}\NormalTok{(}\AttributeTok{size =}\NormalTok{ sizes, }\AttributeTok{power =}\NormalTok{ spower[,}\StringTok{"ego"}\NormalTok{], }\AttributeTok{type =}  \StringTok{"ego"}\NormalTok{),}
  \FunctionTok{data.frame}\NormalTok{(}\AttributeTok{size =}\NormalTok{ sizes, }\AttributeTok{power =}\NormalTok{ spower[,}\StringTok{"alter"}\NormalTok{], }\AttributeTok{type =}  \StringTok{"alter"}\NormalTok{)}
\NormalTok{)}

\NormalTok{spowerd }\SpecialCharTok{|\textgreater{}}
  \FunctionTok{ggplot}\NormalTok{(}\FunctionTok{aes}\NormalTok{(}\AttributeTok{x =}\NormalTok{ size, }\AttributeTok{y =}\NormalTok{ power, }\AttributeTok{colour =}\NormalTok{ type)) }\SpecialCharTok{+}
  \FunctionTok{geom\_point}\NormalTok{() }\SpecialCharTok{+}
  \FunctionTok{geom\_smooth}\NormalTok{(}\AttributeTok{method =} \StringTok{"loess"}\NormalTok{, }\AttributeTok{formula =}\NormalTok{ y }\SpecialCharTok{\textasciitilde{}}\NormalTok{ x) }\SpecialCharTok{+}
  \FunctionTok{labs}\NormalTok{(}\AttributeTok{x =} \StringTok{"Número de Egos"}\NormalTok{, }\AttributeTok{y =} \StringTok{"Poder Aprox."}\NormalTok{, }\AttributeTok{colour =} \StringTok{"Tipo de nodo"}\NormalTok{) }\SpecialCharTok{+}
  \FunctionTok{geom\_hline}\NormalTok{(}\AttributeTok{yintercept =} \DecValTok{1} \SpecialCharTok{{-}}\NormalTok{ beta\_pow)}
\end{Highlighting}
\end{Shaded}

\pandocbounded{\includegraphics[keepaspectratio]{part-01-11-power_files/figure-pdf/part-01-power-plot2-1.pdf}}

Como se muestra en el Capítulo Chapter~\ref{sec-part2-power}, podemos
usar un modelo de regresión lineal para predecir el tamaño de muestra
como una función del poder estadístico:

\begin{Shaded}
\begin{Highlighting}[]
\CommentTok{\# Ajustando el modelo}
\NormalTok{power\_model }\OtherTok{\textless{}{-}} \FunctionTok{glm}\NormalTok{(}
\NormalTok{  size }\SpecialCharTok{\textasciitilde{}}\NormalTok{ power }\SpecialCharTok{+} \FunctionTok{I}\NormalTok{(power}\SpecialCharTok{\^{}}\DecValTok{2}\NormalTok{),}
  \AttributeTok{data =}\NormalTok{ spowerd, }\AttributeTok{family =} \FunctionTok{gaussian}\NormalTok{(), }\AttributeTok{subset =}\NormalTok{ type }\SpecialCharTok{==} \StringTok{"alter"}
\NormalTok{)}

\FunctionTok{summary}\NormalTok{(power\_model)}
\end{Highlighting}
\end{Shaded}

\begin{verbatim}

Call:
glm(formula = size ~ power + I(power^2), family = gaussian(), 
    data = spowerd, subset = type == "alter")

Coefficients:
            Estimate Std. Error t value Pr(>|t|)
(Intercept)    611.4      666.3   0.918    0.456
power        -1553.8     1504.7  -1.033    0.410
I(power^2)    1147.9      844.8   1.359    0.307

(Dispersion parameter for gaussian family taken to be 52.1536)

    Null deviance: 6250.00  on 4  degrees of freedom
Residual deviance:  104.31  on 2  degrees of freedom
AIC: 37.379

Number of Fisher Scoring iterations: 2
\end{verbatim}

\begin{Shaded}
\begin{Highlighting}[]
\CommentTok{\# Predecir}
\FunctionTok{predict}\NormalTok{(power\_model, }\AttributeTok{newdata =} \FunctionTok{data.frame}\NormalTok{(}\AttributeTok{power =}\NormalTok{ .}\DecValTok{8}\NormalTok{), }\AttributeTok{type =} \StringTok{"response"}\NormalTok{) }\SpecialCharTok{|\textgreater{}}
  \FunctionTok{ceiling}\NormalTok{()}
\end{Highlighting}
\end{Shaded}

\begin{verbatim}
  1 
104 
\end{verbatim}

\section{Ejemplo 3: Primera
diferencia}\label{ejemplo-3-primera-diferencia}

Ahora, en lugar de mirar un resultado dicotómico, evaluemos qué pasa si
la variable es continua. Los efectos que estamos interesados en
identificar son el efecto ego y alter, \(\gamma_{ego}\) y
\(\gamma_{alter}\), respectivamente. Además, el proceso de generación de
datos es

\begin{align*}
y_{itg} & = \alpha_i + \kappa_g + X_i\beta + \varepsilon_{itg} \\
y_{itg} & = \alpha_i + \kappa_g + X_i\beta + D_{i}^{ego}\gamma_{ego} + D_i^{alter}\gamma_{alter} + \varepsilon_{itg}
\end{align*}

Donde \(D_i^{ego/alter}\) es una variable indicadora. Aquí, el
comportamiento de ego y alter están correlacionados a través de un
efecto fijo. En otras palabras, dentro de cada grupo, estamos asumiendo
que hay una prevalencia basal compartida del resultado. La diferencia
principal es que ego y alter pueden tener diferentes resultados con
respecto al tamaño del efecto del tratamiento. Otra forma de abordar la
correlación a nivel de grupo podría ser a través de un proceso de
autocorrelación, como en un modelo autocorrelacionado espacial; no
obstante, estimar tales modelos es computacionalmente costoso, así que
optamos por usar el anterior.

Para simplicidad, asumimos que no hay efecto de tiempo. Dos componentes
esenciales aquí, \(\alpha_i\) y \(\kappa_g\) son efectos fijos no
observados a nivel individual y de grupo. El enfoque más directo aquí es
usar un estimador de primera diferencia:

\[
(y_{it+1g} - y_{itg}) = D_{i}^{ego}\gamma_{ego} + D_i^{alter}\gamma_{alter}  + \varepsilon'_i, \quad \varepsilon'_i = \varepsilon_{it+1g} - \varepsilon_{itg}
\]

Al tomar la primera diferencia, los efectos fijos se eliminan de la
ecuación, y podemos proceder con un modelo lineal regular.

\begin{Shaded}
\begin{Highlighting}[]
\NormalTok{effect\_size\_ego   }\OtherTok{\textless{}{-}} \FloatTok{0.5}
\NormalTok{effect\_size\_alter }\OtherTok{\textless{}{-}} \FloatTok{0.25}
\NormalTok{sizes }\OtherTok{\textless{}{-}} \FunctionTok{seq}\NormalTok{(}\DecValTok{10}\NormalTok{, }\DecValTok{100}\NormalTok{, }\AttributeTok{by =} \DecValTok{10}\NormalTok{)}
\end{Highlighting}
\end{Shaded}

\begin{Shaded}
\begin{Highlighting}[]
\CommentTok{\# Función de simulación}
\NormalTok{sim\_data\_prepost }\OtherTok{\textless{}{-}} \ControlFlowTok{function}\NormalTok{(n) \{}

  \CommentTok{\# Aplicando deserción}
\NormalTok{  n }\OtherTok{\textless{}{-}} \FunctionTok{floor}\NormalTok{(n }\SpecialCharTok{*}\NormalTok{ (}\DecValTok{1} \SpecialCharTok{{-}}\NormalTok{ attrition))}

  \CommentTok{\# Paso 1: Muestreando efectos fijos}
\NormalTok{  alpha\_i }\OtherTok{\textless{}{-}} \FunctionTok{rnorm}\NormalTok{(n }\SpecialCharTok{*}\NormalTok{ (n\_a }\SpecialCharTok{+} \DecValTok{1}\NormalTok{))}
\NormalTok{  kappa\_g }\OtherTok{\textless{}{-}} \FunctionTok{rep}\NormalTok{(}\FunctionTok{rnorm}\NormalTok{(n\_a }\SpecialCharTok{+} \DecValTok{1}\NormalTok{), n)}

  \CommentTok{\# Paso 2: Generando el resultado en t = 1}
\NormalTok{  is\_ego   }\OtherTok{\textless{}{-}} \FunctionTok{rep}\NormalTok{(}\FunctionTok{c}\NormalTok{(}\DecValTok{1}\NormalTok{, }\FunctionTok{rep}\NormalTok{(}\DecValTok{0}\NormalTok{, n\_a)), n)}
\NormalTok{  is\_alter }\OtherTok{\textless{}{-}} \DecValTok{1} \SpecialCharTok{{-}}\NormalTok{ is\_ego}
\NormalTok{  y\_0      }\OtherTok{\textless{}{-}}\NormalTok{ alpha\_i }\SpecialCharTok{+}\NormalTok{ kappa\_g }\SpecialCharTok{+} \FunctionTok{rnorm}\NormalTok{(n }\SpecialCharTok{*}\NormalTok{ (n\_a }\SpecialCharTok{+} \DecValTok{1}\NormalTok{))}
\NormalTok{  y\_1      }\OtherTok{\textless{}{-}}\NormalTok{ alpha\_i }\SpecialCharTok{+}\NormalTok{ kappa\_g }\SpecialCharTok{+}
\NormalTok{    is\_ego }\SpecialCharTok{*}\NormalTok{ effect\_size\_ego }\SpecialCharTok{+}
\NormalTok{    is\_alter }\SpecialCharTok{*}\NormalTok{ effect\_size\_alter }\SpecialCharTok{+} 
    \FunctionTok{rnorm}\NormalTok{(n }\SpecialCharTok{*}\NormalTok{ (n\_a }\SpecialCharTok{+} \DecValTok{1}\NormalTok{)) }

  \CommentTok{\# Paso 4: Calculando estadística de prueba}
\NormalTok{  res }\OtherTok{\textless{}{-}} \FunctionTok{tryCatch}\NormalTok{(}
    \FunctionTok{glm}\NormalTok{(}\FunctionTok{I}\NormalTok{(y\_1 }\SpecialCharTok{{-}}\NormalTok{ y\_0) }\SpecialCharTok{\textasciitilde{}} \SpecialCharTok{{-}}\DecValTok{1} \SpecialCharTok{+}\NormalTok{ is\_ego }\SpecialCharTok{+}\NormalTok{ is\_alter, }\AttributeTok{family =} \FunctionTok{gaussian}\NormalTok{(}\StringTok{"identity"}\NormalTok{)),}
    \AttributeTok{error =} \ControlFlowTok{function}\NormalTok{(e) e}
\NormalTok{  )}

  \ControlFlowTok{if}\NormalTok{ (}\FunctionTok{inherits}\NormalTok{(res, }\StringTok{"error"}\NormalTok{))}
    \FunctionTok{return}\NormalTok{(}\FunctionTok{c}\NormalTok{(}\AttributeTok{ego =}  \ConstantTok{NA}\NormalTok{, }\AttributeTok{alter =} \ConstantTok{NA}\NormalTok{))}
  
  \CommentTok{\# Paso 5: ¿Rechazar?}
  \FunctionTok{c}\NormalTok{(}
    \CommentTok{\# ego      = res\_ego$p.value \textless{} alpha,}
    \AttributeTok{ego        =} \FunctionTok{summary}\NormalTok{(res)}\SpecialCharTok{$}\NormalTok{coefficients[}\StringTok{"is\_ego"}\NormalTok{, }\StringTok{"Pr(\textgreater{}|t|)"}\NormalTok{] }\SpecialCharTok{\textless{}}\NormalTok{ alpha,}
    \AttributeTok{alter      =} \FunctionTok{summary}\NormalTok{(res)}\SpecialCharTok{$}\NormalTok{coefficients[}\StringTok{"is\_alter"}\NormalTok{, }\StringTok{"Pr(\textgreater{}|t|)"}\NormalTok{] }\SpecialCharTok{\textless{}}\NormalTok{ alpha,}
    \FunctionTok{coef}\NormalTok{(res)[}\DecValTok{1}\NormalTok{],}
    \FunctionTok{coef}\NormalTok{(res)[}\DecValTok{2}\NormalTok{]}
\NormalTok{  )}
  

\NormalTok{\}}
\end{Highlighting}
\end{Shaded}

\begin{Shaded}
\begin{Highlighting}[]
\CommentTok{\# Siempre establecemos la semilla }
\FunctionTok{set.seed}\NormalTok{(}\DecValTok{88}\NormalTok{)}

\CommentTok{\# ¡Haciendo espacio y ejecutando!}
\NormalTok{spower }\OtherTok{\textless{}{-}} \ConstantTok{NULL}
\ControlFlowTok{for}\NormalTok{ (s }\ControlFlowTok{in}\NormalTok{ sizes) \{}

  \CommentTok{\# Ejecutar la simulación para el tamaño s}
\NormalTok{  simres }\OtherTok{\textless{}{-}} \FunctionTok{rowMeans}\NormalTok{(}
    \FunctionTok{replicate}\NormalTok{(n\_sims, }\FunctionTok{sim\_data\_prepost}\NormalTok{(s)),}
    \AttributeTok{na.rm =} \ConstantTok{TRUE}
\NormalTok{    )}

  \CommentTok{\# Y almacenar los resultados}
\NormalTok{  spower }\OtherTok{\textless{}{-}} \FunctionTok{rbind}\NormalTok{(spower, simres)}

\NormalTok{\}}
\end{Highlighting}
\end{Shaded}

\begin{Shaded}
\begin{Highlighting}[]
\FunctionTok{library}\NormalTok{(ggplot2)}

\NormalTok{spowerd }\OtherTok{\textless{}{-}} \FunctionTok{rbind}\NormalTok{(}
  \FunctionTok{data.frame}\NormalTok{(}\AttributeTok{size =}\NormalTok{ sizes, }\AttributeTok{power =}\NormalTok{ spower[,}\StringTok{"ego"}\NormalTok{], }\AttributeTok{type =}  \StringTok{"ego"}\NormalTok{),}
  \FunctionTok{data.frame}\NormalTok{(}\AttributeTok{size =}\NormalTok{ sizes, }\AttributeTok{power =}\NormalTok{ spower[,}\StringTok{"alter"}\NormalTok{], }\AttributeTok{type =}  \StringTok{"alter"}\NormalTok{)}
\NormalTok{)}

\NormalTok{spowerd }\SpecialCharTok{|\textgreater{}}
  \FunctionTok{ggplot}\NormalTok{(}\FunctionTok{aes}\NormalTok{(}\AttributeTok{x =}\NormalTok{ size, }\AttributeTok{y =}\NormalTok{ power, }\AttributeTok{colour =}\NormalTok{ type)) }\SpecialCharTok{+}
  \FunctionTok{geom\_point}\NormalTok{() }\SpecialCharTok{+}
  \FunctionTok{geom\_smooth}\NormalTok{(}\AttributeTok{method =} \StringTok{"loess"}\NormalTok{, }\AttributeTok{formula =}\NormalTok{ y }\SpecialCharTok{\textasciitilde{}}\NormalTok{ x) }\SpecialCharTok{+}
  \FunctionTok{labs}\NormalTok{(}\AttributeTok{x =} \StringTok{"Número de Egos"}\NormalTok{, }\AttributeTok{y =} \StringTok{"Poder Aprox."}\NormalTok{, }\AttributeTok{colour =} \StringTok{"Tipo de nodo"}\NormalTok{) }\SpecialCharTok{+}
  \FunctionTok{geom\_hline}\NormalTok{(}\AttributeTok{yintercept =} \DecValTok{1} \SpecialCharTok{{-}}\NormalTok{ beta\_pow) }\SpecialCharTok{+}
  \FunctionTok{labs}\NormalTok{(}
    \AttributeTok{caption =} \FunctionTok{sprintf}\NormalTok{(}
      \StringTok{"Efecto Ego: \%.2f; Efecto Alter: \%.2f"}\NormalTok{, effect\_size\_ego, effect\_size\_alter)}
\NormalTok{      )}
\end{Highlighting}
\end{Shaded}

\pandocbounded{\includegraphics[keepaspectratio]{part-01-11-power_files/figure-pdf/part-01-power-plot3-1.pdf}}

Desde el punto de vista inferencial, aún podríamos usar un operador
demean para estimar efectos a nivel individual. En particular,
necesitaríamos usar el operador demean a nivel de grupo y luego ajustar
un modelo de efecto fijo para estimar parámetros a nivel individual.

\part{\textbf{Fundamentos}}

\chapter{Regla de Bayes}\label{regla-de-bayes}

\begin{tcolorbox}[enhanced jigsaw, colback=white, opacityback=0, coltitle=black, title=\textcolor{quarto-callout-warning-color}{\faExclamationTriangle}\hspace{0.5em}{Nota de Traducción}, bottomrule=.15mm, colbacktitle=quarto-callout-warning-color!10!white, toptitle=1mm, colframe=quarto-callout-warning-color-frame, titlerule=0mm, rightrule=.15mm, leftrule=.75mm, breakable, bottomtitle=1mm, left=2mm, arc=.35mm, toprule=.15mm, opacitybacktitle=0.6]

Esta versión del capítulo fue traducida de manera automática utilizando
IA. El capítulo aún no ha sido revisado por un humano.

\end{tcolorbox}

La Regla de Bayes es una ecuación fundamental en estadística bayesiana.
Con ella, podemos reformular problemas inferenciales escribiendo
probabilidades en términos de cantidades conocidas. La regla de Bayes
puede enunciarse como sigue:

\begin{equation}
\mathbb{P}{\left(X=x|Y=y\right)} = \frac{\mathbb{P}{\left(Y=y|X=y\right)}\mathbb{P}{\left(X=x\right)}}{\mathbb{P}{\left(Y=y\right)}}
\end{equation}

Aquí, decimos que la probabilidad condicional de \(X\) dado \(Y\) puede
expresarse en términos de la probabilidad condicional de \(Y\) dado
\(X\). Por ejemplo, sea \(X\) un vector desconocido de parámetros
\(\theta\in\Theta\) y \(Y\) un conjunto de datos \(D \sim f(\theta)\)
cuyo proceso de generación de datos depende del \(\theta\) no observado.
Como la distribución posterior de los parámetros del modelo es en
general, elusiva, en su lugar, usamos la regla de Bayes para reformular
el problema:

\begin{equation*}
\mathbb{P}{\left(\theta|D\right)} = \frac{\mathbb{P}{\left(\theta|D\right)}\mathbb{P}{\left(\theta\right)}}{\mathbb{P}{\left(D\right)}}
\end{equation*}

Dado que el denominador de la ecuación no depende de \(\theta\),
podemos, en su lugar, escribir

\begin{equation*}
\mathbb{P}{\left(\theta|D\right)} \propto \mathbb{P}{\left(\theta|D\right)}\mathbb{P}{\left(\theta\right)}
\end{equation*}

En el mundo bayesiano, se asume que la distribución incondicional de los
parámetros del modelo proviene de una distribución particular, mientras
que en el mundo frecuentista, no se hacen suposiciones distribucionales
sobre los parámetros del modelo. Lo último es entonces equivalente a
decir que \(\theta\sim \text{Uniforme}(-\infty, +\infty)\); por lo
tanto, ¡incluso los frecuentistas asumen algo sobre los parámetros del
modelo!\footnote{La discusión sobre diferencias y similitudes entre
  frecuentistas y bayesianos tiene una larga tradición. En conclusión,
  nadie puede decir 100\% que son una cosa u otra. En rigor, los
  frecuentistas dicen que los parámetros del modelo no son aleatorios
  sino determinísticos.}

La regla de Bayes puede derivarse usando probabilidades condicionales.
En particular, \(\mathbb{P}{\left(x=x|Y=y\right)}\) se define como
\(\mathbb{P}{\left(x=x, Y=y\right)}/Pr{\left(Y=y\right)}\). De manera
similar, \(\mathbb{P}{\left(y=y|X=x\right)}\) se define como
\(\mathbb{P}{\left(y=y, X=x\right)}/Pr{\left(X=x\right)}\), que puede
reescribirse como
\(\mathbb{P}{\left(x=x, Y=y\right)} = \mathbb{P}{\left(y=y|X=x\right)}Pr{\left(X=x\right)}\).
Reemplazando la última igualdad en la primera ecuación, obtenemos

\begin{align*}
\mathbb{P}{\left(x=x|Y=y\right)} & = \frac{\mathbb{P}{\left(x=x, Y=y\right)}}{Pr{\left(Y=y\right)}} \\
& \frac{\mathbb{P}{\left(y=y|X=x\right)}Pr{\left(X=x\right)}}{Pr{\left(Y=y\right)}}
\end{align*}

\chapter{Cadena de Markov}\label{cadena-de-markov}

Una Cadena de Markov es una secuencia de variables aleatorias en la cual
la distribución condicional del \(n\)-ésimo elemento solo depende de
\(n-1\).

\section{Algoritmo de Metropolis}\label{algoritmo-de-metropolis}

El Algoritmo de Metropolis, o MCMC de Metropolis, construye una Cadena
de Markov que, bajo ciertas condiciones, converge a la distribución
objetivo. La clave está en aceptar un movimiento propuesto de \(\theta\)
a \(\theta'\) con probabilidad igual a:

\begin{equation}
r = \min\left(1, \frac{\mathbb{P}{\left(\theta'|D\right)}}{\mathbb{P}{\left(\theta|D\right)}}\right)
\end{equation}

La secuencia resultante converge a la distribución objetivo. Podemos
probar convergencia mostrando que (a) la secuencia es ergódica y (b) la
distribución posterior coincide con la distribución objetivo. La
ergodicidad describe tres propiedades de una cadena:

\begin{itemize}
\item
  Irreductibilidad: No hay probabilidad cero de transición entre
  cualquier par de estados.
\item
  Aperiodicidad: Como el término sugiere, la cadena no tiene
  períodos/secuencias repetitivos.
\item
  No transitoria: Transitoria se refiere a una cadena que tiene
  probabilidad no-cero de nunca regresar a un estado inicial.
\end{itemize}

Las tres propiedades son alcanzadas por cualquier caminata aleatoria
basada en una distribución de probabilidad bien definida, así que nos
enfocaremos en mostrar que la posterior coincide con la distribución
objetivo.

\section{Metropolis-Hastings}\label{metropolis-hastings}

\[
\min\left(1, \frac{\mathbb{P}{\left(d|\theta'\right)}\mathbb{P}{\left(\theta'\right)}\mathbb{P}{\left(\theta'|\theta\right)}}{\mathbb{P}{\left(d|\theta\right)}\mathbb{P}{\left(\theta\right)}\mathbb{P}{\left(\theta|\theta'\right)}}\right)
\]

Si la probabilidad de transición es simétrica, entonces la ecuación
anterior se reduce al probabilidad de Metropolis.

\section{MCMC libre de verosimilitud}\label{mcmc-libre-de-verosimilitud}

\begin{enumerate}
\def\labelenumi{\arabic{enumi}.}
\item
  Inicializar el algoritmo con \(\theta_0\), \(\theta^* =\theta_0\)--el
  estado aceptado actual,--y estadística de resumen observada
  \(s_0 = S(D_{observados})\):
\item
  Para \(t = 1\) hasta \(T\) hacer:

  \begin{enumerate}
  \def\labelenumii{\alph{enumii}.}
  \item
    Extraer \(\theta_t\) de la distribución de propuesta
    \(J(\theta_t|\theta^*)\)
  \item
    Extraer datos simulados \(D_t\) del modelo \(M(\theta_t)\)
  \item
    Calcular las estadísticas de resumen \(s_t = S(D_t)\)
  \item
    Aceptar el estado propuesto con probabilidad
  \end{enumerate}

  Si se acepta, establecer \(\theta^* = \theta_t\).

  \begin{enumerate}
  \def\labelenumii{\alph{enumii}.}
  \setcounter{enumii}{4}
  \tightlist
  \item
    Siguiente \(t\)
  \end{enumerate}
\end{enumerate}

\chapter{Poder y tamaño de muestra}\label{sec-part2-power}

\begin{tcolorbox}[enhanced jigsaw, colback=white, opacityback=0, coltitle=black, title=\textcolor{quarto-callout-warning-color}{\faExclamationTriangle}\hspace{0.5em}{Nota de Traducción}, bottomrule=.15mm, colbacktitle=quarto-callout-warning-color!10!white, toptitle=1mm, colframe=quarto-callout-warning-color-frame, titlerule=0mm, rightrule=.15mm, leftrule=.75mm, breakable, bottomtitle=1mm, left=2mm, arc=.35mm, toprule=.15mm, opacitybacktitle=0.6]

Esta versión del capítulo fue traducida de manera automática utilizando
IA. El capítulo aún no ha sido revisado por un humano.

\end{tcolorbox}

Calcular el poder y el tamaño de muestra son tareas comunes en el diseño
de estudios. Este capítulo te guiará a través del análisis de poder para
estudios de redes. Primero, comenzaremos con algunos preliminares sobre
tipos de error y poder estadístico.

\section{Tipos de error}\label{tipos-de-error}

Una de las tablas más importantes que veremos es la tabla de
contingencia de aceptar/rechazar la hipótesis nula condicional en el
estado verdadero:

\begin{longtable}[]{@{}lll@{}}
\toprule\noalign{}
& Aceptar H0 & Rechazar H0 \\
\midrule\noalign{}
\endhead
\bottomrule\noalign{}
\endlastfoot
H0 es verdadera & Verdadero positivo & Falso negativo \\
H1 es verdadera & Falso positivo & Verdadero negativo \\
\end{longtable}

Una mejor manera, versión más estadísticamente precisa de esta tabla
sería

\begin{longtable}[]{@{}lll@{}}
\toprule\noalign{}
& Aceptar H0 & Rechazar H0 \\
\midrule\noalign{}
\endhead
\bottomrule\noalign{}
\endlastfoot
H0 es verdadera & Inferencia correcta & Error Tipo I \\
H1 es verdadera & Error Tipo II & Inferencia Correcta \\
\end{longtable}

Con \(\mathbb{P}{(\text{Error Tipo I})} = \alpha\) y
\(\mathbb{P}{(\text{Error Tipo II})} = \beta\). De esta manera, el poder
puede definirse como la probabilidad de rechazar la nula dado que la
alternativa es verdadera,
\(\mathbb{P}{(\text{Rechazar H0}|\text{H1 es verdadera})} = 1-\beta\).

\section{Ejemplo 1: Tamaño de muestra para una
proporción}\label{ejemplo-1-tamauxf1o-de-muestra-para-una-proporciuxf3n}

Imaginemos que estamos preparando un estudio en el cual nos gustaría
estimar la proporción de individuos con un estado dado. Formalmente,
entonces decimos que la variable \(Y\sim\text{Bernoulli}(p)\). Para
hacerlo, necesitaremos encuestar \(n\) individuos y estimar tal número
tomando el promedio muestral. Además, hipotetizamos que bajo la nula la
proporción es \(H_0: p = p_0\).

La clave aquí es pensar en una regla de rechazo simple. De nuevo, el
poder es la probabilidad de \textbf{rechazar la nula} dado que
\textbf{la alternativa es verdadera}. Así que, para escribir la
ecuación, necesitamos pensar en regiones de aceptación y rechazo. Sea
\(\hat p\) nuestro estimado para el parámetro poblacional, además,
\(\hat p = n^{-1}\sum_i y_i\). Nuestra estadística de prueba puede
ser--y será, la mayoría de los casos--estandarizada para aprovechar la
ley de los grandes números; bajo la nula, escribimos lo siguiente:

\begin{align*}
\mathbb{E}(\hat p) & = p_0 \\
\mathbf{Var}(\hat p) & = \sqrt{p_0(1-p_0)/n}
\end{align*}

Por lo tanto, la estadística:

\begin{equation*}
\frac{\hat p - p_0}{\sqrt{p_0(1-p_0)/n}} = \frac{\sqrt{n}(\hat p - p_0)}{\sqrt{p_0(1-p_0)}} \sim \text{N}(0, 1)
\end{equation*}

Dado que la estadística está distribuida normalmente, podemos entonces
decir cuándo rechazaremos la nula. Para este caso, eso depende del valor
crítico, que la mayoría de las veces se define en términos de la tasa de
error tipo I. Formalmente, rechazamos la nula si

\begin{equation*}
\frac{\sqrt{n}(\hat p - p_0)}{\sqrt{p_0(1-p_0)}} > Z_{1-\alpha/2}
\end{equation*}

Esto es equivalente a decir que la \textbf{estadística de prueba cayó en
la región de rechazo}. Con esto en mano, ahora podemos escribir la
ecuación que usaremos para calcular el tamaño de muestra. Volviendo a la
definición de poder:

\begin{align*}
\mathbb{P}{(\text{Rechazar H0}|\text{H1 es verdadera})} & = 1-\beta \\
\mathbb{P}{\left(\frac{\sqrt{n}(\hat p - p_0)}{\sqrt{p_0(1-p_0)}} > Z_{1-\alpha/2}\right.\left|\vphantom{\frac{1}{2}}p = p_1\right)} & = 1 - \beta
\end{align*}

Observa que no podemos calcular el poder para todo \(p\neq p_0\); en su
lugar, vemos un valor de parámetro dado. Una buena idea es empezar desde
uno previamente conocido o identificado en otros estudios. La idea clave
aquí es poder manipular el argumento de la probabilidad para convertirlo
en una distribución conocida, por ejemplo, la distribución normal:

Para un Tipo I dado de 0.05 y poder de 0.8, el tamaño de muestra
requerido puede calcularse como sigue:

\begin{align*}
1 - \beta & = \mathbb{P}{\left(\frac{\sqrt{n}(\hat p - p_0)}{\sqrt{p_0(1-p_0)}} > Z_{1-\alpha/2}\right.\left|\vphantom{\frac{1}{2}}p = p_1\right)} \\
& = \mathbb{P}{\left(\frac{\sqrt{n}(\hat p - p_0)}{\sqrt{p_0(1-p_0)}} < Z_{\alpha/2}\right.\left|\vphantom{\frac{1}{2}}p = p_1\right)} \\
& = \mathbb{P}{\left(\frac{\sqrt{n}(\hat p - p_0)}{\sqrt{p_1(1-p_1)}} < \frac{Z_{\alpha/2}\sqrt{p_0(1-p_0)}}{\sqrt{p_1(1-p_1)}}\right.\left|\vphantom{\frac{1}{2}}p = p_1\right)} \\
& = \mathbb{P}{\left(\frac{\sqrt{n}(\hat p - p_0 + p_0 - p_1)}{\sqrt{p_1(1-p_1)}} < \frac{Z_{\alpha/2}\sqrt{p_0(1-p_0)} + \sqrt{n}(p_0 - p_1)}{\sqrt{p_1(1-p_1)}}\right.\left|\vphantom{\frac{1}{2}}p = p_1\right)} \\
& = \mathbb{P}{\left(\frac{\sqrt{n}(\hat p - p_1)}{\sqrt{p_1(1-p_1)}} < \frac{Z_{\alpha/2}\sqrt{p_0(1-p_0)} + \sqrt{n}(p_0 - p_1)}{\sqrt{p_1(1-p_1)}}\right.\left|\vphantom{\frac{1}{2}}p = p_1\right)} \\
& = \Phi\left(\frac{Z_{\alpha/2}\sqrt{p_0(1-p_0)} + \sqrt{n}(p_0 - p_1)}{\sqrt{p_1(1-p_1)}}\right.\left|\vphantom{\frac{1}{2}}p = p_1\right) \\
\end{align*}

La última igualdad sigue de la cantidad
\(\frac{\sqrt{n}(\hat p - p_1)}{\sqrt{p_1(1-p_1)}}\) distribuyendo
normal estándar. Ahora podemos tomar la inversa de la función de
distribución acumulativa (cdf) para aislar el tamaño de muestra \(n\):

\begin{align*}
\Phi^{-1}(1 - \beta)& = \frac{Z_{\alpha/2}\sqrt{p_0(1-p_0)} + \sqrt{n}(p_0 - p_1)}{\sqrt{p_1(1-p_1)}} \\
Z_{1-\beta}\sqrt{p_1(1-p_1)}& = Z_{\alpha/2}\sqrt{p_0(1-p_0)} + \sqrt{n}(p_0 - p_1) \\
\frac{\left(Z_{1-\beta}\sqrt{p_1(1-p_1)} - Z_{\alpha/2}\sqrt{p_0(1-p_0)}\right)^2}{(p_0 - p_1)^2}& = n \\
\end{align*}

Por lo tanto, para los parámetros
\((1-\beta, \alpha, p_0, p_1) = (0.8, 0.05, 0.5, 0.6)\), el tamaño de
muestra requerido es 193.8473 \(\sim\) 194.

\section{Ejemplo 2: Tamaño de muestra para una proporción
(vis)}\label{ejemplo-2-tamauxf1o-de-muestra-para-una-proporciuxf3n-vis}

Ahora, ¿qué pasa si el modelo que estamos planeando estimar no tiene una
forma cerrada? Si las soluciones analíticas no están disponibles, las
simulaciones pueden ser una excelente alternativa para salvar el día.
Rehagamos el cálculo de tamaño de muestra usando simulaciones.

El procedimiento para calcular el tamaño de muestra basado en
simulaciones es computacionalmente intensivo. El concepto es directo,
escoger un conjunto de mejores conjeturas para el tamaño de muestra, y
para cada una de ellas, simular el sistema para estimar el poder. Ahora,
para un valor dado de \(n\), nosotros:

\begin{enumerate}
\def\labelenumi{\arabic{enumi}.}
\item
  Simulamos una muestra de tamaño \(n\) bajo la alternativa.
\item
  Calculamos la estadística de prueba correspondiente a la nula.
\item
  Aceptamos o rechazamos de acuerdo al \(\alpha\) seleccionado, y
  almacenamos el resultado.
\item
  Repetimos los pasos 1-3 muchas veces. El promedio obtenido es el poder
  correspondiente.
\end{enumerate}

Cuando ejecutamos simulaciones, es conveniente escribir una función para
el proceso de generación de datos. En nuestro caso, la función se
llamará \texttt{sim\_fun}. Las siguientes líneas de código logran
nuestro objetivo: aproximar el poder simulando 10,000 experimentos para
cada candidato de tamaño de muestra:

\begin{Shaded}
\begin{Highlighting}[]
\CommentTok{\# Parámetros del modelo}
\NormalTok{p0        }\OtherTok{\textless{}{-}}\NormalTok{ .}\DecValTok{5}
\NormalTok{p1        }\OtherTok{\textless{}{-}}\NormalTok{ .}\DecValTok{6}
\NormalTok{betapower }\OtherTok{\textless{}{-}} \DecValTok{1} \SpecialCharTok{{-}} \FloatTok{0.8}
\NormalTok{alpha     }\OtherTok{\textless{}{-}} \FloatTok{0.05}
\NormalTok{nsims     }\OtherTok{\textless{}{-}} \DecValTok{10000}

\CommentTok{\# Paso 1: Simular los datos bajo H1}
\NormalTok{z\_one\_minus\_alpha\_half }\OtherTok{\textless{}{-}} \FunctionTok{qnorm}\NormalTok{(}\DecValTok{1} \SpecialCharTok{{-}}\NormalTok{ alpha }\SpecialCharTok{/} \DecValTok{2}\NormalTok{)}
\NormalTok{sim\_fun }\OtherTok{\textless{}{-}} \ControlFlowTok{function}\NormalTok{(n) \{}

    \CommentTok{\# Generando los datos}
\NormalTok{    y }\OtherTok{\textless{}{-}} \FunctionTok{as.integer}\NormalTok{(}\FunctionTok{runif}\NormalTok{(n) }\SpecialCharTok{\textless{}}\NormalTok{ p1)}
\NormalTok{    phat }\OtherTok{\textless{}{-}} \FunctionTok{mean}\NormalTok{(y)}

    \CommentTok{\# ¿Aceptar o rechazar?}
    \FunctionTok{sqrt}\NormalTok{(n) }\SpecialCharTok{*}\NormalTok{ (phat }\SpecialCharTok{{-}}\NormalTok{ p0) }\SpecialCharTok{/} \FunctionTok{sqrt}\NormalTok{(p0 }\SpecialCharTok{*}\NormalTok{ (}\DecValTok{1} \SpecialCharTok{{-}}\NormalTok{ p0)) }\SpecialCharTok{\textgreater{}}
\NormalTok{        z\_one\_minus\_alpha\_half}

\NormalTok{\}}

\CommentTok{\# Paso 2: Para un arreglo de n, simular múltiples experimentos}
\NormalTok{n\_seq }\OtherTok{\textless{}{-}} \FunctionTok{seq}\NormalTok{(}\AttributeTok{from =} \DecValTok{150}\NormalTok{, }\AttributeTok{to =} \DecValTok{250}\NormalTok{, }\AttributeTok{by =} \DecValTok{10}\NormalTok{)}

\NormalTok{simulations }\OtherTok{\textless{}{-}} \ConstantTok{NULL}
\FunctionTok{set.seed}\NormalTok{(}\DecValTok{12312}\NormalTok{)}
\ControlFlowTok{for}\NormalTok{ (n }\ControlFlowTok{in}\NormalTok{ n\_seq) \{}

    \CommentTok{\# Ejecutar los nsims experimentos}
\NormalTok{    res }\OtherTok{\textless{}{-}} \FunctionTok{replicate}\NormalTok{(nsims, }\FunctionTok{sim\_fun}\NormalTok{(n))}

    \CommentTok{\# Calcular poder y almacenar el valor}
\NormalTok{    simulations }\OtherTok{\textless{}{-}} \FunctionTok{rbind}\NormalTok{(}
\NormalTok{        simulations,}
        \FunctionTok{data.frame}\NormalTok{(}\AttributeTok{size =}\NormalTok{ n, }\AttributeTok{power =} \FunctionTok{mean}\NormalTok{(res))}
\NormalTok{    )}
\NormalTok{\}}

\CommentTok{\# Descubriendo cuál es el valor más cercano}
\NormalTok{best }\OtherTok{\textless{}{-}} \FunctionTok{which.min}\NormalTok{(}
    \FunctionTok{abs}\NormalTok{((}\DecValTok{1} \SpecialCharTok{{-}}\NormalTok{ betapower) }\SpecialCharTok{{-}}\NormalTok{ simulations}\SpecialCharTok{$}\NormalTok{power)}
\NormalTok{    )}

\NormalTok{simulations[best,,drop}\OtherTok{=}\ConstantTok{FALSE}\NormalTok{]}
\end{Highlighting}
\end{Shaded}

\begin{verbatim}
  size  power
5  190 0.7952
\end{verbatim}

Visualicemos la curva de poder que generamos de esta simulación:

\begin{Shaded}
\begin{Highlighting}[]
\FunctionTok{library}\NormalTok{(ggplot2)}
\FunctionTok{ggplot}\NormalTok{(simulations, }\FunctionTok{aes}\NormalTok{(}\AttributeTok{x =}\NormalTok{ size, }\AttributeTok{y =}\NormalTok{ power)) }\SpecialCharTok{+}
    \FunctionTok{geom\_point}\NormalTok{() }\SpecialCharTok{+}
    \FunctionTok{geom\_smooth}\NormalTok{() }\SpecialCharTok{+}
    \FunctionTok{geom\_hline}\NormalTok{(}\AttributeTok{yintercept =} \DecValTok{1} \SpecialCharTok{{-}}\NormalTok{ betapower)}
\end{Highlighting}
\end{Shaded}

\begin{verbatim}
`geom_smooth()` using method = 'loess' and formula = 'y ~ x'
\end{verbatim}

\pandocbounded{\includegraphics[keepaspectratio]{part-02-11-power_files/figure-pdf/11-power-plot-1.pdf}}

Alternativamente, podemos ajustar un modelo de regresión lineal donde
predecimos el poder como una función del tamaño de muestra usando
efectos lineales y cuadráticos:

\[
n = \theta_0 + \theta_1 (1 - \beta) + \theta_2 (1 - \beta)^2
\]

\begin{Shaded}
\begin{Highlighting}[]
\CommentTok{\# Ajustando el modelo}
\NormalTok{power\_model }\OtherTok{\textless{}{-}} \FunctionTok{glm}\NormalTok{(}
\NormalTok{  size }\SpecialCharTok{\textasciitilde{}}\NormalTok{ power }\SpecialCharTok{+} \FunctionTok{I}\NormalTok{(power}\SpecialCharTok{\^{}}\DecValTok{2}\NormalTok{),}
  \AttributeTok{data =}\NormalTok{ simulations, }\AttributeTok{family =} \FunctionTok{gaussian}\NormalTok{()}
\NormalTok{)}

\CommentTok{\# Imprimiendo los resultados}
\FunctionTok{summary}\NormalTok{(power\_model)}
\end{Highlighting}
\end{Shaded}

\begin{verbatim}

Call:
glm(formula = size ~ power + I(power^2), family = gaussian(), 
    data = simulations)

Coefficients:
            Estimate Std. Error t value Pr(>|t|)   
(Intercept)    632.5      232.9   2.715  0.02644 * 
power        -1590.3      598.1  -2.659  0.02885 * 
I(power^2)    1301.0      381.6   3.410  0.00923 **
---
Signif. codes:  0 '***' 0.001 '**' 0.01 '*' 0.05 '.' 0.1 ' ' 1

(Dispersion parameter for gaussian family taken to be 34.83159)

    Null deviance: 11000.00  on 10  degrees of freedom
Residual deviance:   278.65  on  8  degrees of freedom
AIC: 74.769

Number of Fisher Scoring iterations: 2
\end{verbatim}

\begin{Shaded}
\begin{Highlighting}[]
\CommentTok{\# Predecir}
\FunctionTok{predict}\NormalTok{(power\_model, }\AttributeTok{newdata =} \FunctionTok{data.frame}\NormalTok{(}\AttributeTok{power =}\NormalTok{ .}\DecValTok{8}\NormalTok{), }\AttributeTok{type =} \StringTok{"response"}\NormalTok{) }\SpecialCharTok{|\textgreater{}}
  \FunctionTok{ceiling}\NormalTok{()}
\end{Highlighting}
\end{Shaded}

\begin{verbatim}
  1 
193 
\end{verbatim}

Según nuestro estudio de simulación, el más cercano a nuestro 80\% de
poder es usar un tamaño de muestra igual a 193, que está muy cerca de la
solución analítica de 194.

Como comentario final para este ejemplo, recuerda que mientras más
simulaciones mejor.

\part{\textbf{Apéndice}}

\chapter{Conjuntos de datos}\label{conjuntos-de-datos}

\begin{tcolorbox}[enhanced jigsaw, colback=white, opacityback=0, coltitle=black, title=\textcolor{quarto-callout-warning-color}{\faExclamationTriangle}\hspace{0.5em}{Nota de Traducción}, bottomrule=.15mm, colbacktitle=quarto-callout-warning-color!10!white, toptitle=1mm, colframe=quarto-callout-warning-color-frame, titlerule=0mm, rightrule=.15mm, leftrule=.75mm, breakable, bottomtitle=1mm, left=2mm, arc=.35mm, toprule=.15mm, opacitybacktitle=0.6]

Esta versión del capítulo fue traducida de manera automática utilizando
IA. El capítulo aún no ha sido revisado por un humano.

\end{tcolorbox}

\section{Datos SNS}\label{sns-data}

\subsection{About the data}\label{about-the-data}

\begin{itemize}
\item
  This data is part of the NIH Challenge grant \# RC 1RC1AA019239
  ``Social Networks and Networking That Puts Adolescents at High Risk''.
\item
  In general terms, the SNS's goal was(is) ``Understand the network
  effects on risk behaviors such as smoking initiation and substance
  use''.
\end{itemize}

\subsection{Variables}\label{variables}

The data has a \emph{wide} structure, which means that there is one row
per individual, and that dynamic attributes are represented as one
column per time.

\begin{itemize}
\item
  \texttt{photoid} Photo id at the school level (can be repeated across
  schools).
\item
  \texttt{school} School id.
\item
  \texttt{hispanic} Indicator variable that equals 1 if the indivual
  ever reported himself as hispanic.
\item
  \texttt{female1}, \ldots, \texttt{female4} Indicator variable that
  equals 1 if the individual reported to be female at the particular
  wave.
\item
  \texttt{grades1},\ldots, \texttt{grades4} Academic grades by wave.
  Values from 1 to 5, with 5 been the best.
\item
  \texttt{eversmk1}, \ldots, \texttt{eversmk4} Indicator variable of
  ever smoking by wave. A one indicated that the individual had smoked
  at the time of the survey.
\item
  \texttt{everdrk1}, \ldots, \texttt{everdrk4} Indicator variable of
  ever drinking by wave. A one indicated that the individual had drink
  at the time of the survey.
\item
  \texttt{home1}, \ldots, \texttt{home4} Factor variable for home status
  by wave. A one indicates home ownership, a 2 rent, and a 3 a ``I don't
  know''.
\end{itemize}

During the survey, participants were asked to name up to 19 of their
school friends:

\begin{itemize}
\item
  \texttt{sch\_friend11}, \ldots, \texttt{sch\_friend119} School friends
  nominations (19 in total) for wave 1. The codes are mapped to the
  variable \texttt{photoid}.
\item
  \texttt{sch\_friend21}, \ldots, \texttt{sch\_friend219} School friends
  nominations (19 in total) for wave 2. The codes are mapped to the
  variable \texttt{photoid}.
\item
  \texttt{sch\_friend31}, \ldots, \texttt{sch\_friend319} School friends
  nominations (19 in total) for wave 3. The codes are mapped to the
  variable \texttt{photoid}.
\item
  \texttt{sch\_friend41}, \ldots, \texttt{sch\_friend419} School friends
  nominations (19 in total) for wave 4. The codes are mapped to the
  variable \texttt{photoid}.
\end{itemize}

\bookmarksetup{startatroot}

\chapter*{Referencias}\label{referencias}
\addcontentsline{toc}{chapter}{Referencias}

\markboth{Referencias}{Referencias}

\begin{tcolorbox}[enhanced jigsaw, colback=white, opacityback=0, coltitle=black, title=\textcolor{quarto-callout-warning-color}{\faExclamationTriangle}\hspace{0.5em}{Nota de Traducción}, bottomrule=.15mm, colbacktitle=quarto-callout-warning-color!10!white, toptitle=1mm, colframe=quarto-callout-warning-color-frame, titlerule=0mm, rightrule=.15mm, leftrule=.75mm, breakable, bottomtitle=1mm, left=2mm, arc=.35mm, toprule=.15mm, opacitybacktitle=0.6]

Esta versión del capítulo fue traducida de manera automática utilizando
IA. El capítulo aún no ha sido revisado por un humano.

\end{tcolorbox}

\phantomsection\label{refs}
\begin{CSLReferences}{1}{0}
\bibitem[\citeproctext]{ref-admiraal2006}
Admiraal, Ryan, and Mark S Handcock. 2006. {``Sequential Importance
Sampling for Bipartite Graphs with Applications to Likelihood-Based
Inference.''} Department of Statistics, University of Washington.

\bibitem[\citeproctext]{ref-R-intergraph}
Bojanowski, Michał. 2023. \emph{{intergraph}: Coercion Routines for
Network Data Objects}. \url{https://mbojan.github.io/intergraph/}.

\bibitem[\citeproctext]{ref-brooks2011}
Brooks, Steve, Andrew Gelman, Galin Jones, and Xiao-Li Meng. 2011.
\emph{Handbook of Markov Chain Monte Carlo}. CRC press.

\bibitem[\citeproctext]{ref-R-igraph}
Csárdi, Gábor, Tamás Nepusz, Vincent Traag, Szabolcs Horvát, Fabio
Zanini, Daniel Noom, and Kirill Müller. 2024. \emph{{igraph}: Network
Analysis and Visualization in r}.
\url{https://doi.org/10.5281/zenodo.7682609}.

\bibitem[\citeproctext]{ref-Efron1994}
Efron, Bradley, and Robert J Tibshirani. 1994. \emph{An Introduction to
the Bootstrap}. CRC press.

\bibitem[\citeproctext]{ref-ernmsFellows2023}
Fellows, Ian E. 2012. {``Exponential Family Random Network Models.''}
\emph{ProQuest Dissertations and Theses}. PhD thesis.
\url{https://login.ezproxy.lib.utah.edu/login?url=https://www.proquest.com/dissertations-theses/exponential-family-random-network-models/docview/1221548720/se-2}.

\bibitem[\citeproctext]{ref-Geyer1992}
Geyer, Charles J., and Elizabeth A. Thompson. 1992. {``Constrained Monte
Carlo Maximum Likelihood for Dependent Data.''} \emph{Journal of the
Royal Statistical Society. Series B (Methodological)} 54 (3): 657--99.
\url{http://www.jstor.org/stable/2345852}.

\bibitem[\citeproctext]{ref-R-ergm}
Handcock, Mark S., David R. Hunter, Carter T. Butts, Steven M. Goodreau,
Pavel N. Krivitsky, and Martina Morris. 2023. \emph{Ergm: Fit, Simulate
and Diagnose Exponential-Family Models for Networks}. The Statnet
Project (\url{https://statnet.org}).
\url{https://CRAN.R-project.org/package=ergm}.

\bibitem[\citeproctext]{ref-hayeSmokingDiffusionNetworks2019}
Haye, Kayla de la, Heesung Shin, George G. Vega Yon, and Thomas W.
Valente. 2019. {``Smoking {Diffusion} Through {Networks} of {Diverse},
{Urban American Adolescents} over the {High School Period}.''}
\emph{Journal of Health and Social Behavior}.
\url{https://doi.org/10.1177/0022146519870521}.

\bibitem[\citeproctext]{ref-HunterJASA2008}
Hunter, David R, Steven M Goodreau, and Mark S Handcock. 2008.
{``Goodness of Fit of Social Network Models.''} \emph{Journal of the
American Statistical Association} 103 (481): 248--58.
\url{https://doi.org/10.1198/016214507000000446}.

\bibitem[\citeproctext]{ref-Hunter2008}
Hunter, David R., Mark S. Handcock, Carter T. Butts, Steven M. Goodreau,
and Martina Morris. 2008. {``{ergm : A Package to Fit, Simulate and
Diagnose Exponential-Family Models for Networks}.''} \emph{Journal of
Statistical Software} 24 (3).
\url{https://doi.org/10.18637/jss.v024.i03}.

\bibitem[\citeproctext]{ref-lazega2015}
Lazega, Emmanuel, and Tom AB Snijders. 2015. \emph{Multilevel Network
Analysis for the Social Sciences: Theory, Methods and Applications}.
Vol. 12. Springer.

\bibitem[\citeproctext]{ref-R-texreg}
Leifeld, Philip. 2013. {``{texreg}: Conversion of Statistical Model
Output in {R} to {LaTeX} and {HTML} Tables.''} \emph{Journal of
Statistical Software} 55 (8): 1--24.
\url{https://doi.org/10.18637/jss.v055.i08}.

\bibitem[\citeproctext]{ref-lesageIntroductionSpatialEconometrics2008}
LeSage, James P. 2008. {``An {Introduction} to {Spatial
Econometrics}.''} \emph{Revue d'{é}conomie Industrielle} 123 (123):
19--44. \url{https://doi.org/10.4000/rei.3887}.

\bibitem[\citeproctext]{ref-lesageBiggestMythSpatial2014}
LeSage, James P., and R. Kelley Pace. 2014. {``The {Biggest Myth} in
{Spatial Econometrics}.''} \emph{Econometrics} 2 (4): 217--49.
\url{https://doi.org/10.2139/ssrn.1725503}.

\bibitem[\citeproctext]{ref-lusherExponentialRandomGraph2013}
Lusher, Dean, Johan Koskinen, and Garry Robins. 2013. \emph{Exponential
{Random Graph Models} for {Social Networks}: {Theory}, {Methods}, and
{Applications}}. {Cambridge University Press}.

\bibitem[\citeproctext]{ref-Matloff2011}
Matloff, Norman. 2011. \emph{The Art of r Programming: A Tour of
Statistical Software Design}. No Starch Press.

\bibitem[\citeproctext]{ref-Morris2008}
Morris, Martina, Mark Handcock, and David Hunter. 2008. {``Specification
of Exponential-Family Random Graph Models: Terms and Computational
Aspects.''} \emph{Journal of Statistical Software, Articles} 24 (4):
1--24. \url{https://doi.org/10.18637/jss.v024.i04}.

\bibitem[\citeproctext]{ref-R-coda}
Plummer, Martyn, Nicky Best, Kate Cowles, and Karen Vines. 2006.
{``CODA: Convergence Diagnosis and Output Analysis for MCMC.''} \emph{R
News} 6 (1): 7--11. \url{https://journal.r-project.org/archive/}.

\bibitem[\citeproctext]{ref-R-foreign}
R Core Team. 2023. \emph{Foreign: Read Data Stored by 'Minitab', 's',
'SAS', 'SPSS', 'Stata', 'Systat', 'Weka', 'dBase', ...}
\url{https://svn.r-project.org/R-packages/trunk/foreign/}.

\bibitem[\citeproctext]{ref-R}
---------. 2024. \emph{R: A Language and Environment for Statistical
Computing}. Vienna, Austria: R Foundation for Statistical Computing.
\url{https://www.R-project.org/}.

\bibitem[\citeproctext]{ref-Ripley2011}
Ripley, Ruth M., Tom AB Snijders, Paulina Preciado, and Others. 2011.
{``{Manual for RSIENA}.''} \emph{University of Oxford: Department of
Statistics, Nuffield College}, no. 2007.
\url{https://www.uni-due.de/hummell/sna/R/RSiena\%7B/_\%7DManual.pdf}.

\bibitem[\citeproctext]{ref-robinsNetworkModelsSocial2001b}
Robins, Garry, Philippa Pattison, and Peter Elliott. 2001. {``Network
Models for Social Influence Processes.''} \emph{Psychometrika} 66 (2):
161--89. \url{https://doi.org/10.1007/BF02294834}.

\bibitem[\citeproctext]{ref-Bivand2022}
Roger Bivand. 2022. {``R Packages for Analyzing Spatial Data: A
Comparative Case Study with Areal Data.''} \emph{Geographical Analysis}
54 (3): 488--518. \url{https://doi.org/10.1111/gean.12319}.

\bibitem[\citeproctext]{ref-R-latticeExtra}
Sarkar, Deepayan, and Felix Andrews. 2022. \emph{latticeExtra: Extra
Graphical Utilities Based on Lattice}.
\url{http://latticeextra.r-forge.r-project.org/}.

\bibitem[\citeproctext]{ref-snijdersStochasticActorOriented1996}
Snijders, Tom a B. 1996. {``Stochastic Actor-Oriented Models for Network
Change.''} \emph{The Journal of Mathematical Sociology} 21 (1-2):
149--72. \url{https://doi.org/10.1080/0022250X.1996.9990178}.

\bibitem[\citeproctext]{ref-Snijders1999}
Snijders, Tom A B, and Stephen P Borgatti. 1999. {``{Non-Parametric
Standard Errors and Tests for Network Statistics}.''} \emph{Connections}
22 (2): 1--10.
\url{https://insna.org/PDF/Connections/v22/1999_I-2_61-70.pdf}.

\bibitem[\citeproctext]{ref-Snijders2010}
Snijders, Tom A B, Gerhard G. van de Bunt, and Christian E G Steglich.
2010. {``{Introduction to stochastic actor-based models for network
dynamics}.''} \emph{Social Networks} 32 (1): 44--60.
\url{https://doi.org/10.1016/j.socnet.2009.02.004}.

\bibitem[\citeproctext]{ref-Snijders2010margin}
SNIJDERS, TOM A. B. 2010. {``{Conditional Marginalization for
Exponential Random Graph Models}.''} \emph{The Journal of Mathematical
Sociology} 34 (4): 239--52.
\url{https://doi.org/10.1080/0022250X.2010.485707}.

\bibitem[\citeproctext]{ref-snijdersStochasticActorOrientedModels2017}
Snijders, Tom A. B. 2017. {``Stochastic {Actor-Oriented Models} for
{Network Dynamics}.''} \emph{Annual Review of Statistics and Its
Application} 4 (1): 343--63.
\url{https://doi.org/10.1146/annurev-statistics-060116-054035}.

\bibitem[\citeproctext]{ref-Snijders2002}
Snijders, Tom AB. 2002. {``Markov Chain Monte Carlo Estimation of
Exponential Random Graph Models.''} \emph{Journal of Social Structure}
3.

\bibitem[\citeproctext]{ref-R-rex}
Ushey, Kevin, Jim Hester, and Robert Krzyzanowski. 2021. \emph{Rex:
Friendly Regular Expressions}. \url{https://github.com/kevinushey/rex}.

\bibitem[\citeproctext]{ref-valenteDiffusionContagionProcesses2020}
Valente, Thomas W., and George G. Vega Yon. 2020.
{``Diffusion/{Contagion Processes} on {Social Networks}.''} \emph{Health
Education \& Behavior} 47 (2): 235--48.
\url{https://doi.org/10.1177/1090198120901497}.

\bibitem[\citeproctext]{ref-valenteNetworkInfluencesPolicy2019}
Valente, Thomas W., Heather Wipfli, and George G. Vega Yon. 2019.
{``Network Influences on Policy Implementation: {Evidence} from a Global
Health Treaty.''} \emph{Social Science and Medicine}.
\url{https://doi.org/10.1016/j.socscimed.2019.01.008}.

\bibitem[\citeproctext]{ref-Wang2009}
Wang, Peng, Ken Sharpe, Garry L. Robins, and Philippa E. Pattison. 2009.
{``Exponential Random Graph (p*) Models for Affiliation Networks.''}
\emph{Social Networks} 31 (1): 12--25.
https://doi.org/\url{https://doi.org/10.1016/j.socnet.2008.08.002}.

\bibitem[\citeproctext]{ref-wangUnderstandingNetworksExponentialfamily2023}
Wang, Zeyi, Ian E. Fellows, and Mark S. Handcock. 2023. {``Understanding
Networks with Exponential-Family Random Network Models.''} \emph{Social
Networks}, August, S0378873323000497.
\url{https://doi.org/10.1016/j.socnet.2023.07.003}.

\bibitem[\citeproctext]{ref-R-stringr}
Wickham, Hadley. 2023. \emph{Stringr: Simple, Consistent Wrappers for
Common String Operations}. \url{https://stringr.tidyverse.org}.

\bibitem[\citeproctext]{ref-R-readxl}
Wickham, Hadley, and Jennifer Bryan. 2023. \emph{Readxl: Read Excel
Files}. \url{https://readxl.tidyverse.org}.

\bibitem[\citeproctext]{ref-R-dplyr}
Wickham, Hadley, Romain François, Lionel Henry, Kirill Müller, and Davis
Vaughan. 2023. \emph{Dplyr: A Grammar of Data Manipulation}.
\url{https://dplyr.tidyverse.org}.

\bibitem[\citeproctext]{ref-R-readr}
Wickham, Hadley, Jim Hester, and Jennifer Bryan. 2024. \emph{Readr: Read
Rectangular Text Data}. \url{https://readr.tidyverse.org}.

\bibitem[\citeproctext]{ref-R-tidyr}
Wickham, Hadley, Davis Vaughan, and Maximilian Girlich. 2024.
\emph{Tidyr: Tidy Messy Data}. \url{https://tidyr.tidyverse.org}.

\end{CSLReferences}




\end{document}
